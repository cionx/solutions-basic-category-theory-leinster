\subsection{}



\subsubsection*{The functor \texorpdfstring{$D$}{D}}

We can regard every set~$A$ as a discrete category, which we then denote by~$D(A)$.
Every set-theoretic function
\[
	f \colon A \to B
\]
can then be regarded as a functor~$D(f)$ from~$D(A)$ to~$D(B)$.
We arrive in this way at a functor~$D$ from~$\Set$ to~$\Cat$.
This functor is left adjoint to the given functor~$O$.



\subsubsection*{The functor \texorpdfstring{$I$}{I}}

We can similarly regard every set~$A$ is an indiscrete category, which we will then denote by~$I(A)$.
The collection of objects of~$I(A)$ is the set~$A$, and there exists for any two elements~$a$ and~$a'$ of~$a$ precisely one morphism from~$a$ to~$a'$ in~$I(A)$.
The composition of morphisms in~$I(A)$ is defined in the only possible way.
(The category~$I(A)$ may also be constructed by endowing the set~$A$ with the preorder for which any two elements are comparable, and then take the category associated to this preordered set.)
Every set-theoretic function
\[
	f \colon A \to B
\]
can be uniquely extended to a functor~$I(f)$ from~$I(A)$ to~$I(B)$.
We arrive in this way at a functor~$I$ from~$\Set$ to~$\Cat$.
This functor is right adjoint to the given functor~$O$.



\subsubsection*{The functor \texorpdfstring{$C$}{C}}

Every category~$\cat{A}$ has an underlying undirected graph, which in turn has a set of connected components. We will call these the \defemph{connected components} of~$\cat{A}$.
If the category~$\cat{A}$ is small, then its collections of objects is a set, whence its collection of connected components is a set.
We denote this set by~$C(\cat{A})$.
Every functor
\[
	F \colon \cat{A} \to \cat{B}
\]
maps each connected components of~$\cat{A}$ into a connected component of~$\cat{B}$, and therefore induces a map
\[
	C(F) \colon C(\cat{A}) \to C(\cat{B}) \,.
\]
We arrive in this way at a functor~$C$ from~$\Cat$ to~$\Set$.
This functor is left adjoint to the previous functor~$D$.
