\subsection{}



\subsubsection{}

We denote the given subset of~$A$ by~$S$.
Suppose that there exists an element~$a$ of~$A$ with~$S = f(a)$.
We may wonder if the element~$a$ is contained in the set~$S$.
By the definition of~$S$, we have the equivalence
\[
	a ∈ S \iff a ∉ f(a) \,.
\]
By the choice of~$a$, we have the equivalence
\[
	a ∉ f(a) \iff a ∉ S \,.
\]
By combining both of these equivalences, we arrive at the contradiction
\[
	a ∈ S \iff a ∉ S \,.
\]
We hence find that the desired element~$a$ cannot exist.
Therefore,~$f$ cannot be surjective.



\subsubsection{}

We know that~$\card{A} ≤ \card{\Power(A)}$ because there exists an injective function from~$A$ to~$\Power(A)$, giving by the mapping~$a \mapsto \{ a \}$.

But we have seen in part~(a) of this exercise that there exists no surjective function from~$A$ to~$\Power(A)$.
This entails that there exists no bijection between~$A$ and~$\Power(A)$, so that~$\card{A} ≠ \card{\Power(A)}$.

By combining both of these observations, we find that~$\card{A} < \card{\Power(A)}$.
