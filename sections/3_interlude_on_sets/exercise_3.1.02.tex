\subsection{}

An object of~$\cat{C}$ is a triple~$(X, x_0, f)$ consisting of a set~$X$, an element~$x_0$ of~$X$, and a function~$f$ from~$X$ to~$X$.
Given two such objects~$(X, x_0, f)$ and~$(Y, y_0, g)$, a morphism from~$(X, x_0, f)$ to~$(Y, y_0, g)$ in~$\cat{C}$ is a set-theoretic map~$φ$ from~$X$ to~$Y$ that satisfies the condition
\[
	φ(x_0) = y_0
\]
and that makes the square diagram
\[
	\begin{tikzcd}
		X
		\arrow{r}[above]{f}
		\arrow{d}[left]{φ}
		&
		X
		\arrow{d}[right]{φ}
		\\
		Y
		\arrow{r}[above]{g}
		&
		Y
	\end{tikzcd}
\]
commute.
The composition of two such morphisms is the usual composition of functions.
