\subsection{}



\subsubsection{}

We replace the power set~$\Power(A)$ by an arbitrary complete lattice~$P$, i.e., by a partially ordered set in which every subset admits a supremum.

Let~$R$ be the set of all elements of~$P$ that are increased by~$θ$, i.e.,
\[
	R ≔ \{ r ∈ P \suchthat r ≤ θ(r) \} \,.
\]

We observe that the set~$R$ is closed under the action of~$θ$.
Indeed, let~$r$ be any element of~$R$.
It follows from the inequality~$r ≤ θ(r)$ that~$θ(r) ≤ θ(θ(r))$ because the map~$θ$ is order-preserving.
This inequality~$θ(r) ≤ θ(θ(r))$ tells us that the element~$θ(r)$ is again contained in the set~$R$.

Let~$s$ be the supremum of~$R$.
We show in the following that~$s$ is a fixed point of~$θ$.
That is, we  need to show that~$θ(s) = s$.
We will do so by showing that both~$s ≤ θ(s)$ and~$θ(s) ≤ s$.

We observe that the second inequality follows from the first.
Indeed, suppose for a moment that~$s ≤ θ(s)$.
This inequality tells us that the supremum~$s$ is itself again contained in the set~$R$.
It then follows that~$θ(s)$ is also contained in~$R$, because~$R$ is closed under~$θ$.
But~$s$ is the supremum of~$R$, so that~$r ≤ s$ for every element~$r$ of~$R$.
We may choose~$r$ as~$θ(s)$, and thus find that~$θ(s) ≤ s$.

We are left to show the inequality~$s ≤ θ(s)$.
We note that
\[
	\sup θ(X) ≤ θ( \sup X )
\]
for every subset~$X$ of~$P$ because the map~$θ$ is order-preserving.
By choosing for~$X$ the set~$R$, we find that
\[
	s
	=
	\sup R
	=
	\sup_{r ∈ R} r
	≤
	\sup_{r ∈ R} θ(r)
	=
	\sup θ(R)
	≤
	θ( \sup R )
	=
	θ(s) \,,
\]
where we used once again that the map~$θ$ is order-preserving and that the set~$R$ is closed under the action of~$θ$.



\subsubsection{}

We are tasked to show that there exists a subset~$S$ of~$A$ with
\[
	g(B ∖ f(S)) = A ∖ S \,.
\]
By taking the complements of both side of this equality, we can express the equality equivalently as
\[
	S = A ∖ g(B ∖ f(S)) \,.
\]
We hence need to show that the map
\[
	θ
	\colon
	\Power(A) \to \Power(A) \,,
	\quad
	S \mapsto A ∖ g(B ∖ f(S))
\]
admits a fixed point.
According to part~(a) of this exercise, it suffices to show that~$θ$ is order-preserving.
To this end, we note that~$θ$ is the composite of the four maps
\[
	S \mapsto f(S) \,,
	\quad
	T \mapsto B ∖ T \,,
	\quad
	U \mapsto g(U) \,,
	\quad
	V \mapsto A ∖V \,.
\]
The first and third of these maps are order-preserving, and the second and fourth are order-reversing.
It follows that~$θ$ is order-preserving, as desired.



\subsubsection{}

Let~$A$ and~$B$ be two sets with~$\card{A} ≤ \card{B}$ and~$\card{B} ≤ \card{A}$.
By definition, this means that there exist injective functions
\[
	f \colon A \to B
	\quad\text{and}\quad
	g \colon B \to A \,.
\]
There exists by the previous part of this exercise a subset~$S$ of~$A$ with
\begin{equation}
	\label{formula for cantor bernstein schroeder}
	g(B ∖ f(S)) = A ∖ S \,.
\end{equation}
Let~$T$ be the image of~$S$ under~$f$, i.e., let~$T ≔ f(S)$.
The injection~$f$ restricts to a bijection between the sets~$S$ and~$T$, and formula~\eqref{formula for cantor bernstein schroeder} tells us that the injection~$g$ restricts to a bijection between~$B ∖ T$ and~$A ∖ S$.
By combining these two bijections, we arrive at a bijection between the sets~$A$ and~$B$.

The existence of such a bijection shows that~$A$ and~$B$ have the same cardinality.
