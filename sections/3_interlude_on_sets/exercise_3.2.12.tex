\subsection{}



\subsubsection{}

We replace~$\Power(A)$ by a complete lattice~$P$, i.e., by a partially ordered set in which every subset admits a supremum in~$P$.
Let~$R$ be the set of all elements~$r$ of~$P$ that are increased by~$θ$, i.e.,
\[
	R ≔ \{ r ∈ P \suchthat r ≤ θ(r) \} \,.
\]
Let~$s$ be the supremum of~$R$.
We show in the following, that~$s$ is a fixed point of~$θ$.

We note that
\[
	\sup θ(X) ≤ θ( \sup X )
\]
for every subset~$X$ of~$P$ because~$θ$ is order-preserving.
This entails that
\[
	s
	=
	\sup R
	=
	\sup_{r ∈ R} r
	≤
	\sup_{r ∈ R} θ(r)
	=
	\sup θ(R)
	≤
	θ( \sup R )
	=
	θ(s) \,.
\]
We also note that~$R$ is closed under~$θ$:
for every element~$r$ of~$R$ we have~$r ≤ θ(r)$, and then also~$θ(r) ≤ θ(θ(r))$ because~$θ$ is order-preserving.

The already-proven inequality~$s ≤ θ(s)$ tells us that~$s$ is itself an element of~$R$.
We now find that~$θ(s)$ is again contained in~$R$, because~$R$ is closed under~$θ$.
It follows that also~$θ(s) ≤ \sup R = s$.



\subsubsection{}

We consider the map
\[
	θ
	\colon
	\Power(A) \to \Power(A) \,,
	\quad
	S \mapsto A ∖ g(B ∖ f(S)) \,.
\]
This map is order-preserving:
it is the composite of the four maps
\[
	S \mapsto f(S) \,,
	\quad
	T \mapsto B ∖ T \,,
	\quad
	U \mapsto g(U) \,,
	\quad
	V \mapsto A ∖V \,,
\]
the first and third of which are order-preserving, and the second and fourth of which are order-reversing.
It follows from part~(a) of this exercise that there exists an element~$S$ of~$\Power(A)$ that is a fixed point of~$θ$.
In other words, there exists a subset~$S$ of~$A$ with
\[
	S = A ∖ g(B ∖ f(S)) \,,
\]
and therefore equivalently
\[
	g(B ∖ f(S)) = A ∖ S \,.
\]



\subsubsection{}

Let~$A$ and~$B$ be two sets with~$\card{A} ≤ \card{B}$ and~$\card{B} ≤ \card{A}$.
This means that there exist injective functions
\[
	f \colon A \to B \,,
	\quad
	g \colon B \to A \,.
\]

There exist by the previous part of this exercise a subset~$S$ of~$A$ with
\begin{equation}
	\label{formula for cantor bernstein schroeder}
	g(B ∖ f(S)) = A ∖ S \,.
\end{equation}
Let~$T ≔ f(S)$.
The injection~$f$ restricts to a bijection between the sets~$S$ and~$T$, and formula~\eqref{formula for cantor bernstein schroeder} tells us that the injection~$g$ restricts to a bijection between~$B ∖ T$ and~$A ∖ S$.
By combining these two bijections, we get a bijection between~$A$ and~$B$.

The existence of this bijection shows that~$\card{A} = \card{B}$.
