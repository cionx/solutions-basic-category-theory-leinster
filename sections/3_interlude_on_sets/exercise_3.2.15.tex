\subsection{}



\subsubsection{}

We can use Exercise~2.3.14 to see that~$\Mon$ is not essentially small, and therefore also not small.



\subsubsection{}

The resulting category is small because its collection of morphisms is the underlying set of~$ℤ$.
This category is therefore also essentially small.



\subsubsection{}

The given category is locally small, since there exists at most one morphism between any two objects.
Its class of objects is the underlying set of~$ℤ$, and therefore a set.
It follows that the category is small, and therefore also essentially small.



\subsubsection{}

We can regard every set as a discrete category.
This allows us to regard~$\Set$ as a full subcategory of~$\Cat$.
Two sets are isomorphic if and only if they are isomorphic as discrete categories.
The inclusion from~$\Set$ and~$\Cat$ does therefore induce an injection
\[
	\Set / {≅} \to \Cat / {≅} \,.
\]
It follows that if~$\Cat$ were to be essentially small, then~$\Set$ would also be essentially small.
But we know that this is not the case.

We therefore find that~$\Cat$ is not essentially small, and hence also not small.



\subsubsection{}

There exists for every family of cardinal~$(κ_i)_{i ∈ I}$ a cardinal~$λ$ that is distinct to each~$κ_i$.
(One may take for~$λ$ the cardinal associated to the power set of the sum~$∑_{i ∈ I} κ_i$.)
The collection of all cardinals is therefore a proper class, and not a set.
It follows that the given category is not locally small.
It is therefore not essentially small, and hence also not small.
