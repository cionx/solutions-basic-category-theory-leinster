\subsection{}



\subsubsection{}

Let~$(A_i)_{i ∈ I}$ be a family of objects of~$\cat{A}$, where~$I$ is some index set.
We are tasked with finding an object~$A$ of~$\cat{A}$ that is isomorphic to none of the objects~$A_i$.

It suffices to find an object~$A$ of~$\cat{A}$ for which the set~$U(A)$ is non-isomorphic to each of sets~$U(A_i)$, since the functor~$U$ preserves isomorphisms.
We can ensure these non-isomorphisms by making the set~$U(A)$ have strictly larger cardinality than each of the sets~$U(A_i)$.
To ensure that~$U(A)$ has strictly larger cardinality than~$U(A_i)$ for every index~$i$ at the same time, we will choose the object~$A$ so that~$U(A)$ has strictly larger cardinality than the set~$∑_{i ∈ I} U(A_i)$.%
\footnote{
	Here we use that~$I$ is a set, to ensure that the sum~$∑_{i ∈ I} A_i$ exists.
}
To summarize our discussion:
we need to find an object~$A$ of~$\cat{A}$ such that
\[
	\card*{\sum_{i ∈ I} U(A_i)} < \card{U(A)} \,.
\]

The functor~$U$ is part of an adjunction~$(F, U, η, ε)$ between the categories~$\cat{A}$ and~$\Set$ (by assumption).
We know from Exercise~2.3.11 that under the given assumptions, the map
\[
	η_P \colon P \to UF(P)
\]
is injective for every set~$P$.
We have therefore the inequality
\[
	\card{P} ≤ \card{UF(P)}
\]
for every set~$P$.
If we choose the set~$P$ to be of strictly larger cardinality than the set~$∑_{i ∈ I} U(A_i)$ (e.g., its power set), then it follows for the object~$A ≔ F(P)$ of~$\cat{A}$ that
\[
	\card*{∑_{i ∈ I} U(A_i)}
	≤
	\card{P}
	≤
	\card{UF(P)}
	=
	\card{U(A)} \,.
\]
We have thus found the desired object~$A$.

Let us summarize our findings:
given a set~$P$ of strictly larger cardinality than the sum~$∑_{i ∈ I} U(A_i)$, we have for the object~$A ≔ F(P)$ the chain of inequalities
\[
	\card{ U(A_j) }
	≤
	\card*{ ∑_{i ∈ I} U(A_i) }
	<
	\card{ P }
	≤
	\card{ UF(P) }
	=
	\card{ U(A) }
\]
for every index~$j$.
These inequalities show that each set~$U(A_j)$ is non-isomorphic to the set~$U(A)$, whence each object~$A_j$ is non-isomorphic to the object~$A$.



\subsubsection{}

Suppose that the category~$\cat{A}$ is essentially small.
This means that there exists a small category~$\cat{B}$ equivalent to~$\cat{A}$.
Such an equivalence between~$\cat{A}$ and~$\cat{B}$ induces a bijection between the class of isomorphism classes of objects of~$\cat{A}$ and the class of isomorphism classes of objects of~$\cat{B}$.
(In other words, a bijection between the quotient classes~$\cat{A} / {≅}$ and~$\cat{B} / {≅}$.)
It follows that the category~$\cat{B}$ again satisfies the assumption of part~(a) of this exercise.
But the category~$\cat{B}$ is small, whence the family of objects~$(B)_{B ∈ \Ob(\cat{B})}$ is indexed by a set.
This family contains all objects of~$\cat{B}$, contradicting part~(a) of this exercise.



\subsubsection{}

We have for each of the given categories a forgetful functor to~$\Set$, which then admits a left adjoint:
\begin{itemize}

	\item
		The forgetful functor from~$\Set$ to~$\Set$ is the identity functor of~$\Set$, whose left adjoint is again the identity functor.

	\item
		For the algebraic categories~$\Vect{𝕜}$,~$\Grp$,~$\Ab$, and~$\Ring$, the respective left adjoint functors assigns to any set~$X$ the respective free algebraic structures on~$X$.

	\item
		For the forgetful functor from~$\Top$ to~$\Set$, its left adjoint assigns to each set~$X$ the discrete topological space whose underlying set is~$X$.

\end{itemize}

Each of categories~$\Set$,~$\Vect{𝕜}$,~$\Grp$,~$\Ab$,~$\Ring$ and~$\Top$ contains an object whose underlying set consists of at least two distinct elements.
These categories hence satisfy the assumption of part~(a) of this exercise.
It follows from part~(b) of this exercise that these categories are not essentially small.
