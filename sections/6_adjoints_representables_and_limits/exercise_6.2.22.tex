\subsection{}



%\subsubsection*{First solution}

We denote the unique object of the category~$\One$ by~$1$.
Let~$\{ \ast \}$ be a singleton set, and let~$I$ be the unique functor from~$\One$ to~$\Set$ with~$I(1) = \{ \ast \}$.
For every functor~$X$ from~$\cat{A}$ to~$\Set$, we can consider the comma category~$I \comma X$.

An object of~$I \comma X$ is a triple~$(1, e, A)$ consisting of the unique object~$1$ of~$\One$, an object~$A$ of~$\cat{A}$, and a map~$e$ from~$I(1)$ to~$X(A)$.
We have~$I(1) = \{ \ast \}$.
Maps~$e$ from~$F(1)$ to~$X(A)$ are therefore in one-to-one correspondence to elements of~$X(A)$ (which is a set) via
\[
	e \mapsto e(\ast) \,.
\]
The map
\[
	F
	\colon
	\Ob( I \comma X )
	\to
	\Ob( \Elements(X) ) \,,
	\quad
	(1, e, A)
	\mapsto
	(A, e(\ast))
\]
is therefore bijective.

Morphism in the comma category~$I \comma X$ of the form
\[
	(1, e, A) \to (1, e', A')
\]
are pairs~$(i, f)$ consisting of morphisms
\[
	i \colon 1 \to 1 \,,
	\quad
	f \colon A \to A'
\]
such that the following diagram commutes:
\[
	\begin{tikzcd}[sep = large]
		I(1)
		\arrow{r}[above]{e}
		\arrow{d}[left]{I(i)}
		&
		X(A)
		\arrow{d}[right]{X(f)}
		\\
		I(1)
		\arrow{r}[above]{e'}
		&
		X(A')
	\end{tikzcd}
\]
The morphism~$i$ is necessarily the identity morphism of the unique object~$1$ of~$\One$, and the above diagram can equivalently be written as follows:
\[
	\begin{tikzcd}[sep = large]
		\ast
		\arrow{r}[above]{e}
		\arrow{d}[left]{\id}
		&
		X(A)
		\arrow{d}[right]{X(f)}
		\\
		\ast
		\arrow{r}[above]{e'}
		&
		X(A')
	\end{tikzcd}
\]
The commutativity of this diagram is equivalent to the condition
\[
	X(f)( e(\ast) ) = e'(\ast) \,.
\]
We get therefore a bijection
\[
	F_{A, A'}
	\colon
	(I \comma G)( (1, e, A), (1, e', A') )
	\to
	\Elements(X)( F(A), F(A') ) \,,
	\quad
	(i, f)
	\mapsto
	f \,.
\]
The map~$F$ together with the maps~$F_{A, A}$, where~$A$ and~$A'$ both range through~$\cat{A}$, define a functor from the comma category~$I \comma X$ to the category of elements~$\Elements(X)$.
This functor is both bijective on objects and fully faithful, and therefore an isomorphism of categories.

In the notation of Example~2.3.4, the comma category~$I \comma X$ can also be written as $\{ \ast \} \comma X$.
If we denote the singleton set~$\{ \ast \}$ by~$1$, then we thus arrive at the notation~$1 \comma X$.
So while the elements of a set~$S$ are given by maps~$1 \to X$, the (category of) elements of a presheaf~$X$ is given by~$1 \comma X$.



%\subsubsection*{Second solution}
%
%We have for every object~$A$ of~$\cat{A}$ the bijection
%\[
%	[\cat{A}^{\op}, \Set](\HY_A, X) \to X(A) \,,
%	\quad
%	h \mapsto h_A(\id_A)
%\]
%by Yoneda’s lemma.
%We get thus an induced one-to-one correspondence between element~$(A, x)$ of~$X$, and pairs~$(A, α)$ consisting of an object~$A$ of~$\cat{A}$ and a natural transformation~$α$ from~$\HY_A$ to~$X$.
%This correspondence is given by the mapping
%\[
%	(A, α) \mapsto ( A, α_A(\id_A) ) \,.
%\]
%
%Let~$(A, x)$ and~$(A', x')$ be two elements of~$X$ and let~$(A, α)$ and~$(A', α')$ be the corresponding pairs as before.
%We know from the fully faithfulness of the Yoneda embedding that every natural transformation from~$\HY_{A'}$ to~$\HY_{A'}$ is of the form~$f_*$ for a unique morphism~$f$ from~$A$ to~$A'$.
%The resulting diagram
%\[
%	\begin{tikzcd}[row sep = large]
%		\HY_A
%		\arrow{rr}[above]{\HY_f}
%		\arrow{dr}[below left]{α}
%		&
%		{}
%		&
%		\HY_{A'}
%		\arrow{dl}[below right]{α'}
%		\\
%		{}
%		&
%		X
%		&
%		{}
%	\end{tikzcd}
%\]
%commutes if and only if
%\[
%	α = α' ∘ \HY_f \,,
%\]
%which is by Yoneda’s lemma equivalent to the equality
%\[
%	α_A(\id_A)
%	=
%	(α' ∘ \HY_f)_A(\id_A) \,.
%\]
%The left-hand side of this equation is~$x$, and its right-hand side can be rewritten as
%\begin{align*}
%	(α' ∘ \HY_f)_A(\id_A)
%	&=
%	(α'_A ∘ \HY_f )(\id_A)
%	\\
%	&=
%	α'_A( \HY_f(\id_A) )
%	\\
%	&=
%	α'_A( f )
%	\\
%	&=
%	α'_A( \HY_{A'}(f)( \id_{A'} ) )
%	\\
%	&=
%	X(f)( α_{A'}( \id_{A'} ) )
%	\\
%	&=
%	X(f)(x') \,.
%\end{align*}
%The commutativity of the above diagram is therefore equivalent to the equality
%\[
%	x = X(f)(x') \,.
%\]
