\subsection{}

We denote the unique object of the category~$\One$ by~$1$.
Let~$\{ \ast \}$ be a singleton set, and let~$I$ be the unique functor from~$\One$ to~$\Set$ with~$I(1) = \{ \ast \}$.
For every functor~$X$ from~$\cat{A}$ to~$\Set$, we can consider the comma category~$I \comma X$.

An object of~$I \comma X$ is a triple~$(1, e, A)$ consisting of the unique object~$1$ of~$\One$, an object~$A$ of~$\cat{A}$, and a map~$e$ from~$I(1)$ to~$X(A)$.
We have~$I(1) = \{ \ast \}$.
Maps~$e$ from~$F(1)$ to~$X(A)$ are therefore in one-to-one correspondence to elements of~$X(A)$ (which is a set) via
\[
	e \mapsto e(\ast) \,.
\]
The map
\[
	F
	\colon
	\Ob( I \comma X )
	\to
	\Ob( \Elements(X) ) \,,
	\quad
	(1, e, A)
	\mapsto
	(A, e(\ast))
\]
is therefore bijective.

Morphism in the comma category~$I \comma X$ of the form
\[
	(1, e, A) \to (1, e', A')
\]
are pairs~$(i, f)$ consisting of morphisms
\[
	i \colon 1 \to 1 \,,
	\quad
	f \colon A \to A'
\]
such that the following diagram commutes:
\[
	\begin{tikzcd}[sep = large]
		I(1)
		\arrow{r}[above]{e}
		\arrow{d}[left]{I(i)}
		&
		X(A)
		\arrow{d}[right]{X(f)}
		\\
		I(1)
		\arrow{r}[above]{e'}
		&
		X(A')
	\end{tikzcd}
\]
The morphism~$i$ is necessarily the identity morphism of the unique object~$1$ of~$\One$, and the above diagram can equivalently be written as follows:
\[
	\begin{tikzcd}[sep = large]
		\ast
		\arrow{r}[above]{e}
		\arrow{d}[left]{\id}
		&
		X(A)
		\arrow{d}[right]{X(f)}
		\\
		\ast
		\arrow{r}[above]{e'}
		&
		X(A')
	\end{tikzcd}
\]
The commutativity of this diagram is equivalent to the condition
\[
	X(f)( e(\ast) ) = e'(\ast) \,.
\]
We get therefore a bijection
\[
	F_{A, A'}
	\colon
	(I \comma G)( (1, e, A), (1, e', A') )
	\to
	\Elements(X)( F(A), F(A') ) \,,
	\quad
	(i, f)
	\mapsto
	f \,.
\]
The map~$F$ together with the maps~$F_{A, A}$, where~$A$ and~$A'$ both range through~$\cat{A}$, define a functor from the comma category~$I \comma X$ to the category of elements~$\Elements(X)$.
This functor is both bijective on objects and fully faithful, and therefore an isomorphism of categories.
