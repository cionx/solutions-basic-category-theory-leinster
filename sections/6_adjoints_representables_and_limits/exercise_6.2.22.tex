\subsection{}


\begin{warning}
	For a covariant functor
	\[
		X \colon \cat{A} \to \Set \,,
	\]
	its category of elements, denoted by~$\Elements(X)$, is typically defined as follows.
	\begin{itemize}

		\item
			The objects of~$\Elements(X)$ are pairs~$(A, x)$ consisting of an object~$A$ of~$\cat{A}$ and an element~$x$ of~$X(A)$.

		\item
			A morphism in~$\Elements(X)$ from an object~$(A, x)$ to an object~$(A', x')$ is a morphism~$f$ from~$A$ to~$A'$ in~$\cat{A}$ with~$X(f)(x) = x'$.

	\end{itemize}
	A presheaf~$X$ on a category~$\cat{A}$ is a contravariant functor from~$\cat{A}$ to~$\Set$, and thus a covariant functor from~$\cat{A}^{\op}$ to~$\Set$.
	The above definition of \enquote{category of elements} can therefore be applied to~$X$.
	This results in the following definition of~$\Elements(X)$.
	\begin{itemize}

		\item
			The objects of~$\Elements(X)$ are pairs~$(A, x)$ consisting of an object~$A$ of~$\cat{A}$ and an element~$x$ of~$X(A)$.

		\item
			A morphism in~$\Elements(X)$ from an object~$(A, x)$ to an object~$(A', x')$ is a morphism~$f$ from~$A'$ to~$A$ in~$\cat{A}$ with~$X(f)(x) = x'$.

	\end{itemize}
	This definition of~$\Elements(X)$ does \emph{not} agree with Definition~6.2.16:
	the two definitions result in opposite categories.%
	\footnote{
		The author is extremely annoyed by this inconsistency.
	}

	In the following, we will use Definition~6.2.16, as intended in the book.
\end{warning}



%\subsubsection*{First solution}
%
%We regard the contravariant functor~$X$ as a covariant functor~$X'$ from~$\cat{A}^{\op}$ to~$\Set$.
%
%We consider the singleton set~$\{ \ast \}$ and the comma category~$\{ \ast \} \comma X'$, in the sense of Example~2.3.4.
%We show in the following that this comma category is contravariantly isomorphic to the category of elements of~$X$.
%
%An object of the category~$\{ \ast \} \comma X$ is a pair~$(A^{\op}, e)$ consisting of an object~$A^{\op}$ of~$\cat{A}^{\op}$, and a map~$e$ from the singleton set~$\{ \ast \}$ to the set~$X(A)$.
%Such maps~$e$ from~$\{ \ast \}$ to~$X(A)$ are in one-to-one correspondence to elements of the set~$X(A)$ via the evaluation map
%\[
%	e \mapsto e(\ast) \,.
%\]
%We have therefore a bijection between the objects of~$\{ \ast \} \comma X$ and the objects of~$\Elements(X)$, given by
%\begin{align*}
%	F
%	\colon
%	\Ob( \{ \ast \} \comma X )
%	&\to
%	\Ob( \Elements(X) ) \,,
%	\\
%	(A^{\op}, e)
%	&\mapsto
%	(A, e(\ast)) \,.
%\end{align*}
%
%Let us now describe morphisms in the comma category~$\{ \ast \} \comma X$.
%Such a morphism
%\[
%	(A^{\op}, e) \to ((A')^{\op}, e')
%\]
%is a morphism
%\[
%	f^{\op} \colon A^{\op} \to (A')^{\op}
%\]
%in~$\cat{A}^{\op}$, such that the following diagram commutes:
%\[
%	\begin{tikzcd}[row sep = normal]
%		{}
%		&
%		X(A)
%		\arrow{dd}[right]{X(f)}
%		\\
%		\{ \ast \}
%		\arrow{ur}[above left]{e}
%		\arrow{dr}[below left]{e'}
%		&
%		{}
%		\\
%		{}
%		&
%		X(A')
%	\end{tikzcd}
%\]
%The commutativity of this diagram is equivalent to the condition
%\[
%	X(f)( e(\ast) ) = e'(\ast) \,.
%\]
%We may also regard~$f$ is a morphism from~$A'$ to~$A$ in~$\cat{A}$.
%The commutativity of this diagram thus means that~$f$ is a morphism from~$(A', e'(\ast))$ to~$(A, e(\ast))$ in~$\Elements(X)$.
%We get therefore a bijection
%\begin{align*}
%	F_{A, A'}
%	\colon
%	(\{\ast\} \comma X)\bigl( (A, e), (A', e') \bigr)
%	&\to
%	\Elements(X)\bigl( F((A', e')), F((A, e)) \bigr) \,,
%	\\
%	(i, f)
%	&\mapsto
%	f \,.
%\end{align*}
%The map~$F$ together with the maps~$F_{A, A}$, where~$A$ and~$A'$ both range through~$\cat{A}$, define a contravariant~(!) functor from the comma category~$\{ \ast \} \comma X$ to the category of elements~$\Elements(X)$.
%This functor is both bijective on objects and fully faithful, and therefore an isomorphism of categories from~$(\{ \ast \} \comma X)^{\op}$ to~$\Elements(X)$.
%
%
%
%\subsubsection*{Second solution}

In the following, let us refer to the objects of the category~$\Elements(X)$ as the elements of~$X$.
By Yoneda’s lemma, we have for every object~$A$ of~$\cat{A}$ the bijection
\[
	[\cat{A}^{\op}, \Set](\HY_A, X) \to X(A) \,,
	\quad
	h \mapsto h_A(\id_A) \,.
\]
We get from this bijection an induced one-to-one correspondence between elements of~$X$ and pairs~$(A, α)$ consisting of an object~$A$ of~$\cat{A}$ and a natural transformation~$α$ from~$\HY_A$ to~$X$.
This correspondence is given by the mapping
\[
	(A, α) \mapsto ( A, α_A(\id_A) ) \,.
\]

Let~$(A, x)$ and~$(A', x')$ be two elements of~$X$ and let~$(A, α)$ and~$(A', α')$ be the corresponding pairs as above.
Every natural transformation from~$\HY_{A'}$ to~$\HY_{A'}$ is of the form~$\HY_f$ for a unique morphism~$f$ from~$A$ to~$A'$ because the Yoneda embedding is fully faithful.
The resulting diagram
\[
	\begin{tikzcd}[column sep = normal]
		\HY_A
		\arrow[Rightarrow]{rr}[above]{\HY_f}
		\arrow[Rightarrow]{dr}[below left]{α}
		&
		{}
		&
		\HY_{A'}
		\arrow[Rightarrow]{dl}[below right]{α'}
		\\
		{}
		&
		X
		&
		{}
	\end{tikzcd}
\]
commutes if and only if
\[
	α = α' ∘ \HY_f \,.
\]
By Yoneda’s lemma, this equality of natural transformations is equivalent to the equality of elements
\[
	α_A(\id_A)
	=
	(α' ∘ \HY_f)_A(\id_A)
\]
The left-hand side of this equation is the element~$x$, and the right-hand side can be rewritten as
\begin{align}
	(α' ∘ \HY_f)_A(\id_A)
	\notag
	&=
	(α'_A ∘ (\HY_f)_A )(\id_A)
	\notag
	\\
	&=
	α'_A( (\HY_f)_A(\id_A) )
	\notag
	\\
	&=
	α'_A( \HY_{A'}(f)( \id_{A'} ) )
	\label{shuffeling nonsense}
	\\
	&=
	X(f)( α_{A'}( \id_{A'} ) )
	\label{naturality of alpha dash}
	\\
	&=
	X(f)(x') \,.
	\notag
\end{align}
We use for \eqref{shuffeling nonsense} the chain of equalities
\[
	(\HY_f)_A(\id_A)
	=
	f_*( \id_A )
	=
	f ∘ \id_A
	=
	f
	=
	\id_{A'} ∘ f
	=
	f^*( \id_{A'} )
	=
	\HY_{A'}(f)( \id_{A'} ) \,,
\]
and~\eqref{naturality of alpha dash} holds because~$α$ is a natural transformation from~$\HY_{A'}$ to~$X$.
The commutativity of the above diagram is therefore equivalent to the equality
\[
	x = X(f)(x') \,.
\]
This equality expresses precisely that~$f$ is a morphism from~$(A, x)$ to~$(A', x')$ in~$\Elements(X)$.

We have altogether an isomorphism
\[
	Y \comma X
\]
where~$Y$ is the Yoneda embedding from~$\cat{A}$ to~$[\cat{A}^{\op}, \Set]$, and where we use (the dual of) the notation from Example~2.3.4.%
\footnote{
	More explicitly, we consider the comma category of the following situation:
	\[
		\begin{tikzcd}[ampersand replacement = \&]
			{}
			\&
			\One
			\arrow{d}[right]{X}
			\\
			\cat{A}
			\arrow{r}[below]{Y}
			\&{}
			[\cat{A}^{\op}, \Set]
		\end{tikzcd}
	\]
}
