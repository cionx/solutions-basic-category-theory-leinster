\subsection{}



\subsubsection{}

\subsubsection*{The implication $\text{(A)} \implies \text{(R)}$}

Suppose that the functor~$U$ admits a left adjoint~$F$.
We have for every object~$A$ of~$\cat{A}$ the chain of isomorphisms
\[
	U(A)
	≅
	\Set( \{ \ast \}, U(A) )
	≅
	\cat{A}( F(\{ \ast \}), A ) \,,
\]
which is natural in~$A$.
This tells us that the functor~$U$ is represented by the object~$F( \{ * \} )$ of~$\cat{A}$.

\subsubsection*{The implication $\text{(R)} \implies \text{(L)}$}

This follows from Proposition~6.2.2.



\subsubsection{}

Suppose that the functor~$U$ is represented by an object~$A$ of~$\cat{A}$.
We may assume for simplicity that~$U = \cat{A}(A, \ph)$.

The desired left adjoint~$F$ of~$U$ needs to preserve coproducts, and therefore need to satisfy
\[
	F(B)
	≅
	F\biggl( ∑_{b ∈ B} \{ b \} \biggr)
	≅
	∑_{b ∈ B} F(\{ b \})
	≅
	∑_{b ∈ B} F(\{ \ast \})
\]
for every set~$B$.
We also find that
\[
	\cat{A}( F(\{ \ast \}), A' )
	≅
	\Set( \{ \ast \}, U(A') )
	≅
	U(A')
	=
	\cat{A}(A, A')
\]
for every object~$A'$ of~$\cat{A}$, natural in~$A'$, and therefore
\[
	F(\{ \ast \}) ≅ A
\]
by Yoneda’s lemma.

Motivated by this thought experiment on the behaviour of~$F$, we define the functor~$F$ from~$\Set$ to~$\cat{A}$ as follows.
\begin{itemize}

	\item
		For every set~$B$ let~$F(B)$ be the~\fold{$B$} sum of~$A$, i.e.,
		\[
			F(B) ≔ ∑_{b ∈ B} A \,.
		\]
		We further denote for every element~$b$ of~$B$ the canonical morphism belonging to the~\nth{$b$} summand of~$F(B)$ by~$i^B_b$.
		This is a morphism from~$A$ to~$F(B)$.

	\item
		For every map
		\[
			g \colon B \to B' \,,
		\]
		let~$F(g)$ be the unique morphism from~$F(B)$ to~$F(B')$ given by
		\[
			F(g) ∘ i^B_b
			=
			i^{B'}_{g(b)}
		\]
		for every element~$b$ of~$B$.
		(Intuitively speaking, the morphism~$F(g)$ maps the~\nth{$b$} summand of~$F(B)$ into the~\nth{$g(b)$} summand of~$F(B')$ -- both these summands are~$A$ -- via the identity morphism of~$A$.)

\end{itemize}

We have for every set~$B$ and every object~$A'$ of~$\cat{A}$ the chain of bijections
\begin{align*}
	\cat{A}(F(B), A')
	&≅
	\cat{A}\biggl( ∑_{b ∈ B} A, A' \biggr)
	\\
	&≅
	∏_{b ∈ B} \cat{A}(A, A')
	\\
	&=
	∏_{b ∈ B} U(A')
	\\
	&≅
	\Set(B, U(A')) \,.
\end{align*}
The overall bijection is given by
\[
	α_{B, A'}
	\colon
	\cat{A}(F(B), A') \to \Set(B, U(A')) \,,
	\quad
	f \mapsto [b \mapsto f ∘ i^B_b] \,.%
	\footnote{
		If we only assume that~$U$ is isomorphic to~$\cat{A}(A, \ph)$, then this formula becomes slightly messier.
		Indeed, we need to fix one such isomorphism, and then include its components in this formula.
		The author doesn’t want to bother with this additional notational ballast, and therefore choose~$U$ to be~$\cat{A}(A, \ph)$.
	}
\]
These bijections are natural in both~$B$ and~$A'$.
Let us check this.
\begin{itemize}

	\item
		Let
		\[
			g \colon B \to B'
		\]
		be a map of sets.
		We need to show that the square diagram
		\[
			\begin{tikzcd}
				\cat{A}(F(B'), A')
				\arrow{r}[above]{α_{B', A'}}
				\arrow{d}[left]{F(g)^*}
				&
				\Set(B', U(A'))
				\arrow{d}[right]{g^*}
				\\
				\cat{A}(F(B), A')
				\arrow{r}[above]{α_{B, A'}}
				&
				\Set(B, U(A'))
			\end{tikzcd}
		\]
		commutes.
		This commutativity holds because we have the chain of equalities
		\begin{align*}
			\SwapAboveDisplaySkip
			g^*( α_{B', A'}( f ) )(b)
			&=
			(α_{B', A'}(f) ∘ g)(b)
			\\
			&=
			α_{B', A'}(f)( g(b) )
			\\
			&=
			f ∘ i^{B'}_{g(b)}
			\\
			&=
			f ∘ F(g) ∘ i^B_b
			\\
			&=
			α_{B, A'}(f ∘ F(g))(b)
			\\
			&=
			α_{B, A'}( F(g)^*( f ) )(b)
		\end{align*}
		for every element~$f$ of~$\cat{A}(F(B'), A')$ and every element~$b$ of~$B$.

		\item
			Let
			\[
				h \colon A' \to A''
			\]
			be a morphism in~$\cat{A}$.
			We need to show that the square diagram
			\[
				\begin{tikzcd}
					\cat{A}(F(B), A')
					\arrow{r}[above]{α_{B, A'}}
					\arrow{d}[left]{h_*}
					&
					\Set(B, U(A'))
					\arrow{d}[right]{U(h)_*}
					\\
					\cat{A}(F(B), A'')
					\arrow{r}[above]{α_{B, A''}}
					&
					\Set(B, U(A''))
				\end{tikzcd}
			\]
			commutes.
			This commutativity holds because we have the chain of equalities
			\begin{align*}
				\SwapAboveDisplaySkip
				U(h)_*( α_{B, A'}(f) )(b)
				&=
				( U(h) ∘ α_{B, A'}(f) )(b)
				\\
				&=
				U(h)( α_{B, A'}(f)(b) )
				\\
				&=
				U(h)( f ∘ i^B_b )
				\\
				&=
				h_*( f ∘ i^B_b )
				\\
				&=
				h ∘ f ∘ i^B_b
				\\
				&=
				h_*(f) ∘ i^B_b
				\\
				&=
				α_{B, A''}( h_*(f) )
			\end{align*}
			for every element~$f$ of~$\cat{A}(F(B), A')$ and every element~$b$ of~$B$.

\end{itemize}

We have overall shown that~$\cat{A}(F(B), A')$ is bijective to~$\Set(B, U(A'))$, natural in both~$B$ and~$A'$.
This shows that the constructed functor~$F$ is left adjoint to the original functor~$U$.
