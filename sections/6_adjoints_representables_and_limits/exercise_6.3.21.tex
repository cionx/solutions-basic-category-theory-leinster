\subsection{}



\subsubsection{}

If~$U$ were to admit a right adjoint, then it would preserve colimits, and thus in particular initial objects.
This would mean that for the trivial group~$1$, the set~$U(1)$ would be initial in~$\Set$.
But the initial object of~$\Set$ is the empty set, and~$U(1)$ is non-empty.



\subsubsection{}

The functor~$I$ regards every set as an indiscrete category in the following sense:
gives a set~$A$, the category~$I(A)$ has the set~$A$ as its class of objects, and for every two elements~$a$ and~$a'$ of~$A$, there exists a unique morphism from~$a$ to~$a'$ in~$I(A)$.
Given two non-empty sets~$A$ and~$B$, there exist no morphisms between~$I(A)$ and~$I(B)$ inside~$I(A) + I(B)$.
The categories~$I(A + B)$ and~$I(A) + I(B)$ are therefore not isomorphic.
This tells us that the functor~$I$ does not preserve binary coproducts, and therefore cannot be a left adjoint.
In other words,~$I$ doesn’t have a right adjoint.

The functor~$C$ assigns to each small category its set of connected components.
More explicitly, for every small category~$\scat{A}$, the set~$C(\scat{A})$ is the quotient set~$\Ob(\scat{A}) / {∼}$, where~$\sim$ is the equivalence relation generated by
\[
	A ∼ B
	\quad\text{if}\quad
	\textstyle\text{there exists a morphism~$A \to B$} \,.
\]
Let us show that the functor~$C$ does not preserve pullbacks.

Given a small category~$\scat{A}$ and subcategories~$\scat{B}$ and~$\scat{B}'$ of~$\scat{A}$, their intersection~$\scat{B} ∩ \scat{B}'$ is again a subcategory of~$\scat{A}$.
We have the following commutative diagram of small categories and inclusion functors:
\[
	\begin{tikzcd}
		\scat{B} ∩ \scat{B}'
		\arrow{r}
		\arrow{d}
		&
		\scat{B}'
		\arrow{d}
		\\
		\scat{B}
		\arrow{r}
		&
		\scat{A}
	\end{tikzcd}
\]
This diagram is a pullback diagram.
So if~$C$ were to preserve pullbacks, then it would follow that the induced diagram
\[
	\begin{tikzcd}
		C(\scat{B} ∩ \scat{B}')
		\arrow{r}
		\arrow{d}
		&
		C(\scat{B}')
		\arrow{d}
		\\
		C(\scat{B})
		\arrow{r}
		&
		C(\scat{A})
	\end{tikzcd}
\]
would again a pullback diagram.
But this is not always the case!

To see this we consider the category~$\One$ that consists of two objects~$0$ and~$1$ and precisely one non-identity morphism, which goes from~$0$ to~$1$.
Let~$\Zero_0$ and~$\Zero_1$ be the subcategories of~$\One$ consisting of the single objects~$0$ and~$1$ respectively.
We have in~$\Cat$ the following pushout diagram, where all arrows are inclusions functors:
\[
	\begin{tikzcd}
		∅
		\arrow{r}
		\arrow{d}
		&
		\Zero_1
		\arrow{d}
		\\
		\Zero_0
		\arrow{r}
		&
		\One
	\end{tikzcd}
\]
By applying the functor~$C$ to this diagram, we get the following commutative diagram:
\[
	\begin{tikzcd}
		∅
		\arrow{r}
		\arrow{d}
		&
		\{ * \}
		\arrow{d}
		\\
		\{ * \}
		\arrow{r}
		&
		\{ * \}
	\end{tikzcd}
\]
But this is not a pullback diagram!



\subsubsection{}

The functor~$∇$ from~$\Set$ to~$[\Open(X)^{\op}, \Set]$ is defined on objects by~$∇(B)(X) = B$ and~$∇(B)(U) = \{ * \}$ for every proper open subset~$U$ of~$X$.

Suppose that the space~$X$ is non-empty.
(Otherwise, the functor~$∇$ is an isomorphism, and therefore admits both a left adjoint and a right adjoint.)
This assumption ensures that the space~$X$ admits a proper open subset, namely the empty subset~$∅$.
It then follows for any two sets~$B$ and~$B'$ that
\begin{align*}
	(∇(B) + ∇(B'))(∅)
	&=
	∇(B)(∅) + ∇(B')(∅)
	\\
	&=
	\{ * \} + \{ * \}
	\\
	&≇
	\{ * \}
	\\
	&=
	∇(B + B')(∅) \,.
\end{align*}
This non-isomorphism shows that the functor~$∇$ does not preserve binary coproducts, and does therefore not admit a right adjoint.

The functor~$Λ$ is defined dually to the functor~$∇$.
Given a set~$B$, the functor~$Λ(B)$ is given by~$Λ(B)(∅) = B$ and~$Λ(B)(U) = ∅$ for every non-empty open subset of~$X$.
Suppose again that the space~$X$ is non-empty.
(Otherwise, the functor~$Λ$ is an isomorphism, and therefore admits both a left adjoint and a right adjoint.)
This ensures that the space~$X$ admits a non-empty open subset, namely~$X$ itself.
We recall that the terminal object of the presheaf category~$[\Open(X)^{\op}, \Set]$ is the presheaf with constant value~$\{ * \}$.
But for every set~$B$, we have~$Λ(B)(X) = ∅ ≇ \{ * \}$.
We thus find that for every set~$B$, the presheaf~$Λ(B)$ is not terminal in~$[\Open(X)^{\op}, \Set]$.
This observation entails that the functor~$Λ$ does not preserve terminal objects, and therefore does not admit a left adjoint.
