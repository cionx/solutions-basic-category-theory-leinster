\subsection{}



\subsubsection{}

It follows from Corollary~6.2.6 that the functor category~$[\scat{A}, \cat{S}]$ also has pullbacks.
We know from Lemma~5.1.32 that~$α$ is a monomorphism in~$[\scat{A}, \cat{S}]$ if and only if the diagram
\begin{equation}
	\label{pullback in functor category}
	\begin{tikzcd}[sep = large]
		X
		\arrow{r}[above]{\id_X}
		\arrow{d}[left]{\id_X}
		&
		X
		\arrow{d}[right]{α}
		\\
		X
		\arrow{r}[below]{α}
		&
		Y
	\end{tikzcd}
\end{equation}
is a pullback diagram in~$[\scat{A}, \cat{S}]$.
It follows from Corollary~6.2.6 that if the above diagram is a pullback diagram, then for every object~$A$ of~$\scat{A}$, the resulting diagram
\begin{equation}
	\label{pullback in target category}
	\begin{tikzcd}[sep = large]
		X(A)
		\arrow{r}[above]{\id_{X(A)}}
		\arrow{d}[left]{\id_{X(A)}}
		&
		X(A)
		\arrow{d}[right]{α_A}
		\\
		X(A)
		\arrow{r}[below]{α_A}
		&
		Y(A)
	\end{tikzcd}
\end{equation}
is a pullback diagram in~$\cat{S}$.
The converse is also true, by Theorem~6.2.5.
Therefore, the diagram~\eqref{pullback in functor category} is a pullback diagram in~$[\scat{A}, \cat{S}]$ if and only if for every object~$A$ in~$\scat{A}$, the diagram~\eqref{pullback in target category} is a pullback diagram in~$\cat{S}$.
This means by Lemma~5.1.32 that~$α$ is a monomorphism in~$[\scat{A}, \cat{S}]$ if and only if for each object~$A$ of~$\scat{A}$, its component~$α_A$ is a monomorphism in~$\cat{S}$.



\subsubsection{}

We know that the monomorphisms in~$\Set$ are precisely the injective functions.
It follows from the previous part of this exercise that the monomorphisms in the functor category~$[\scat{A}^{\op}, \Set]$ are precisely those natural transformations that are injective in each component.

We find similarly from the dual of part~(a) that the epimorphisms in~$[\scat{A}^{\op}, \Set]$ are precisely those natural transformations that are surjective in each component.



\subsubsection{}


\subsubsection*{Monomorphisms}

Suppose first that for every object~$A$ of~$\scat{A}$, the morphism~$α_A$ is a monomorphism in~$\cat{S}$.
Let
\[
	β, β' \colon W \to X
\]
be two morphisms in~$[\scat{A}, \cat{S}]$ with~$α ∘ β = α ∘ β'$.
In other words,~$β$ and~$β'$ are two natural transformations with~$α ∘ β = α ∘ β'$.
This means that for every object~$A$ of~$\scat{A}$, we have the chain of equalities
\[
	α_A ∘ β_A
	=
	(α ∘ β)_A
	=
	(α ∘ β')_A
	=
	α_A ∘ β'_A \,.
\]
It follows that~$β_A = β'_A$ for every object~$A$ of~$\scat{A}$ because the morphism~$α_A$ is a monomorphism.
This shows that~$β = β'$, which in turn shows that~$α$ is a monomorphism in~$[\scat{A}, \cat{S}]$.%

Suppose on the other hand that~$α$ is a monomorphism, and that~$\cat{S} = \Set$.
We know from Yoneda’s lemma that for every object~$A$ of~$\scat{A}$, the evaluation functor
\[
	\ev_{\!A} \colon [\scat{A}, \Set] \to \Set
\]
is representable by~$A$.
The functor~$\HY_A$ preserves monomorphisms,%
\footnote{
	A morphism~$f \colon A' \to A''$ in a category~$\cat{A}$ is a monomorphism if and only if the induced map~$f_* \colon \cat{A}(A, A') \to \cat{A}(A, A'')$ is injective for every object~$A$ of~$\cat{A}$.
	(This is a direct reformulation of the definition of a monomorphism.)
	The class of monomorphisms is therefore chosen in precisely such a way that each functor~$\HY_A = \cat{A}(A, \ph)$ preserves monomorphisms.
}
whence it follows that the isomorphic functor~$\ev_{\!A}$ also preserves monomorphisms.
Therefore, the component~$α_A = \ev_{\!A}(α)$ is a monomorphism in~$\Set$ for every object~$A$ of~$\scat{A}$.

\subsubsection*{Epimorphisms}

By using the isomorphism
\[
	[\scat{A}^{\op}, \cat{S}]^{\op}
	≅
	[\scat{A}^{\op\op}, \cat{S}^{\op}]
	=
	[\scat{A}, \cat{S}^{\op}] \,,
\]
we can see that dually, a natural transformation is an epimorphism in~$[\scat{A}^{\op}, \cat{A}]$ if and only if it is an epimorphism in each component.
