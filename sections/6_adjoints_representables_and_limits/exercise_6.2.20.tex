\subsection{}



\subsubsection{}

It follows from Corollary~6.2.6 that the functor category~$[\scat{A}, \cat{S}]$ also has pullbacks.
According to Lemma~5.1.32, the natural transformation~$α$ is a mono\-mor\-phism in~$[\scat{A}, \cat{S}]$ if and only if the diagram
\begin{equation}
	\label{pullback in functor category}
	\begin{tikzcd}
		X
		\arrow{r}[above]{\id_X}
		\arrow{d}[left]{\id_X}
		&
		X
		\arrow{d}[right]{α}
		\\
		X
		\arrow{r}[below]{α}
		&
		Y
	\end{tikzcd}
\end{equation}
is a pullback diagram in~$[\scat{A}, \cat{S}]$.

It follows from Corollary~6.2.6 that if the above diagram is a pullback diagram, then for every object~$A$ of~$\scat{A}$, the resulting diagram
\begin{equation}
	\label{pullback in target category}
	\begin{tikzcd}
		X(A)
		\arrow{r}[above]{\id_{X(A)}}
		\arrow{d}[left]{\id_{X(A)}}
		&
		X(A)
		\arrow{d}[right]{α_A}
		\\
		X(A)
		\arrow{r}[below]{α_A}
		&
		Y(A)
	\end{tikzcd}
\end{equation}
is a pullback diagram in~$\cat{S}$.
The converse is also true, by Theorem~6.2.5.
Therefore, the diagram~\eqref{pullback in functor category} is a pullback diagram in~$[\scat{A}, \cat{S}]$ if and only if for every object~$A$ in~$\scat{A}$, the diagram~\eqref{pullback in target category} is a pullback diagram in~$\cat{S}$.

We hence have by Lemma~5.1.32 the following chain of equivalences:
\begin{align*}
	{}&
	\text{$α$ is a natural transformation}
	\\
	\iff{}&
	\text{the diagram~\eqref{pullback in functor category} is a pullback diagram in~$[\scat{A}, \cat{S}]$}
	\\
	\iff{}&
	\text{the diagram~\eqref{pullback in target category} is a pullback diagram in~$\cat{S}$ for every object~$A$ of~$\scat{A}$}
	\\
	\iff{}&
	\text{the morphism~$α_A$ is a monomorphism in~$\cat{S}$ for every object~$A$ of~$\scat{A}$} \,.
\end{align*}
Therefore, the natural transformation~$α$ is a monomorphism in~$[\scat{A}, \cat{S}]$ if and only if each of its components~$α_A$ is a monomorphism in~$\cat{S}$.

We can similarly use the dual of Lemma~5.1.32 to show the following proposition:
\begin{quote}
	\itshape
	Let~$\scat{A}$ be a small category and let~$\cat{S}$ be a locally small category with pushouts.
	The epimorphisms in~$[\scat{A}, \cat{S}]$ are precisely those natural transformations that are an epimorphism in each component.
\end{quote}



\subsubsection{}

The category~$\Set$ admits both pullback and pushouts.
The monomorphisms in~$\Set$ are precisely those maps that are injective, and the epimorphisms are precisely those maps that are surjective.

It follows that the monomorphisms in~$[\scat{A}, \Set]$ are precisely those natural transformations whose every component is injective.
Similarly, the epimorphisms in~$[\scat{A}, \Set]$ are precisely those natural transformations whose every component is surjective.



\subsubsection{}


\subsubsection*{Monomorphisms}

Suppose first that for every object~$A$ of~$\scat{A}$, the morphism~$α_A$ is a monomorphism in~$\cat{S}$.
Let
\[
	β, β' \colon Z \to X
\]
be two morphisms in~$[\scat{A}, \cat{S}]$ (i.e., natural transformations between the functors~$Z$ and~$X$) with~$α ∘ β = α ∘ β'$.
This means that for every object~$A$ of~$\scat{A}$, we have the chain of equalities
\[
	α_A ∘ β_A
	=
	(α ∘ β)_A
	=
	(α ∘ β')_A
	=
	α_A ∘ β'_A \,.
\]
It follows for every object~$A$ of~$\scat{A}$ that~$β_A = β'_A$ because the morphism~$α_A$ is a monomorphism.
This shows that~$β = β'$, which in turn shows that~$α$ is a monomorphism in~$[\scat{A}, \cat{S}]$.

Suppose conversely that~$α$ is a monomorphism.
We restrict ourselves to the case that~$\cat{S} = \Set$.
We know from Yoneda’s lemma that for every object~$A$ of~$\scat{A}$, the evaluation functor
\[
	\ev_{\!A} \colon [\scat{A}, \Set] \to \Set
\]
is representable by~$\HY_A$.
But representable functors always preserve monomorphisms.%
\footnote{
	A morphism~$g \colon B' \to B''$ in a category~$\cat{B}$ is a monomorphism if and only if the induced map~$g_* \colon \cat{B}(B, B') \to \cat{B}(B, B'')$ is injective for every object~$B$ of~$\cat{B}$.
	(This is a direct reformulation of the definition of a monomorphism.)
	The class of monomorphisms is therefore chosen in precisely such a way that each functor~$\HY_B = \cat{B}(B, \ph)$ preserves monomorphisms.
}
It hence follows that the isomorphic functor~$\ev_{\!A}$ also preserves monomorphisms.
Therefore, the component~$α_A = \ev_{\!A}(α)$ is a monomorphism in~$\Set$ for every object~$A$ of~$\scat{A}$.

\subsubsection*{Epimorphisms}

If each component of~$α$ is an epimorphism in~$\cat{S}$, then we can proceed in the same way as above to see that~$α$ is an epimorphism in~$[\scat{A}, \cat{S}]$.

It remains to show that for an epimorphism in~$[\scat{A}, \Set]$, all of its components are epimorphisms in~$\Set$.
However, the author doesn’t know how to do this without using part~(a) of this exercise.

\begin{remark}
	For more information on the missing part of the above solution, we refer to \cite{stackexchange_epimorphisms_in_set_values_functor_categories}.
%	The author wonders if Leinster himself knows a proof that doesn’t use part~(a).
%
%	Indeed, in \cite{mathoverflow_epimorphisms_pointwise_surjective}, Leinster gives only the argument from part~(a).
%
%	Going back further in time, in~2007, Leinster gave a course on Category Theory.
%	The material of this course can still be found online \cite{leinster_category_theory_course_2007}, and part~(b) and part~(c) of this current exercise (i.e., Exercise~6.2.20) already appears in this course material.
%	More explicitly, Problem~Sheet~9 (titled \enquote{Limits and colimits of presheaves}), Problem~1 reads as follows:
%	\begin{quote}
%		\itshape
%		Let~$\mathbb{A}$ be a small category.
%		\begin{enumerate}[label = (\alph*)]
%
%			\item
%				What does it mean to say that limits and colimits are computed pointwise in~$[\mathbb{A}^{\mathrm{op}}, \mathbf{Set}]$?
%				Prove that this is so.
%
%			\item
%				Describe explicitly the monics and epics in~$[\mathbb{A}^{\mathrm{op}}, \mathbf{Set}]$.
%				(Now see if you can do this without the aid of~(a).)
%
%		\end{enumerate}
%	\end{quote}
%
%	In the hints accompanying these problem sheets, Leinster writes the following:
%	\begin{quote}
%		\itshape
%		\begin{enumerate}[label = (\alph*)]
%
%			\item
%				The meaning of ‘computed pointwise’ is the statement of Theorem~5.1.5 (with~$\mathbb{A}$ changed to~$\mathbb{A}^{\mathrm{op}}$ and~$\mathcal{S}$ to~$\mathbf{Set}$).
%
%			\item
%				Applying Lemma~4.1.31, a map~$α$ in~$[\mathbb{A}^{\mathrm{op}}, \mathbf{Set}]$ is monic iff a certain square involving~$α$ is a pullback, iff for each~$A ∈ \mathbb{A}$ the analogous square involving~$α_A$ is a pullback (since pullbacks are computed pointwise), iff for each~$A ∈ \mathbb{A}$ the map~$α_A$ is monic.
%				The monics in~$\mathbf{Set}$ are the injections, so~$α$ is monic iff each~$α_A$ is injective.
%				Similarly, the epics are the pointwise surjections.
%				Without using~(a), \textup[we\textup] can still figure out what the monics are:
%				do a direct proof by considering maps out of representables.
%				But I know of no way of proving the result on epics without~(a).
%
%		\end{enumerate}
%	\end{quote}
%	The notes of the 2007 course are no longer available on the course’s website.
%	But the author assumes that Theorem~5.1.5 of the course notes is akin to Theorem~6.2.5 or Corollary~6.2.6 of the book, and that Lemma~4.1.31 of the course notes is akin to Lemma~5.1.32 of the book.
\end{remark}
