\subsection{}

Let~$G$ be a group.
We may identify the functor category~$[G, \Set]$ with the category of~\sets{$G$},~$\GSet{G}$.
We recall that an element~$x$ of a~\set{$G$} is called \defemph{invariant} if
\[
	g ⋅ x = x
	\qquad
	\text{for every~$g ∈ G$} \,.
\]
A~\set{$G$} is trivial if each element of~$G$ acts by the identity on~$X$.
In other words, each element of~$X$ needs to be~\invariant{$G$}.

For every~\set{$X$} we can consider its set of invariants,
\[
	X^G ≔ \{ x ∈ X \suchthat \text{$x$ is~\invariant{$G$}} \} \,.
\]
This is the largest subset of~$X$ on which~$G$ act trivially.
We can dually consider its set of coinvariants, denoted by~$X_G$, which is the largest quotient of~$X$ on which~$G$ act trivially.
It can be constructed as the set
\[
	X_G ≔ X / {∼}
\]
where the equivalence relation~$∼$ of~$X$ is generated by~$x ∼ g ⋅ x$ with~$x ∈ X$ and~$g ∈ G$.
The set~$X_G$ can equivalently be described as the set of~\orbits{$G$} on~$X$.

Let~$X$ be a~\set{$G$}.
A cone of~$X$ is a set~$S$ together with a map
\[
	f \colon S \to X
\]
such that~$g ⋅ f(s) = f(s)$ for every element~$s$ of~$S$.
In other words, the image of the map~$f$ must be contained in the set of invariants~$X^G$.

It follows that the set~$X^G$ together with the inclusion map from~$X^G$ to~$X$ is a limit cone over~$X$.
We find dually that the set~$X_G$ together with the quotient map from~$X$ to~$X_G$ is a colimit cocone over~$X$.

The diagonal functor
\[
	Δ \colon \Set \to [G, \Set]
\]
corresponds to the functor
\[
	T \colon \Set \to \GSet{G}
\]
that regards any set as a trivial~\set{$G$}.
We get from Proposition~6.1.4 the adjunctions
\[
	(\ph)_G ⊣ T ⊣ (\ph)^G \,.
\]
