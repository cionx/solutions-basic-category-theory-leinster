\subsection{}



\subsubsection{}

\begin{lemma}
	Let~$\cat{A}$ be a category.
	Let~$\scat{I}$ and~$\scat{I'}$ be two small categories, and let~$D$ and~$D'$ be diagrams in~$\cat{A}$ of shapes~$\scat{I}$ and~$\scat{I}'$ respectively.
	Suppose that these diagrams admit colimits~$(C, (q_I)_I)$ and~$(C', (q'_I)_I)$ respectively.
	Let
	\[
		F \colon \scat{I} \to \scat{I}'
	\]
	be a functor and let
	\[
		α \colon D \To D' ∘ F
	\]
	be a natural transformation.
	Then there exists a unique morphism~$f$ from~$C$ to~$C'$ with
	\[
		f ∘ q_I = q'_{F(I)} ∘ α_I
	\]
	for every object~$I$ of~$\scat{I}$.
\end{lemma}

\begin{proof}
	We have for every object~$I$ of~$\scat{I}$ the morphism
	\[
		f_I
		\colon
		D(I)
		\xto{α_I}
		D'(F(I))
		\xto{q'_I}
		C' \,.
	\]
	We have for every morphism
	\[
		u \colon I \to J
	\]
	in~$\scat{I}$ the chain of equalities
	\begin{align*}
		f_J ∘ D(u)
		&=
		q'_{F(J)} ∘ α_J ∘ D(u)
		\\
		&=
		q'_{F(J)} ∘ (D' ∘ F)(u) ∘ α_I
		\\
		&=
		q'_{F(J)} ∘ D'( F(u) ) ∘ α_I
		\\
		&=
		q'_{F(I)} ∘ α_I
		\\
		&=
		f_I \,.
	\end{align*}
	It follows from the universal property of the colimit~$(C, (q_I)_I)$ that the morphisms~$f_I$ assemble into a morphism
	\[
		f \colon C \to C'
	\]
	with~$f ∘ q_I = f_I$ for every object~$I$ of~$\scat{I}$.
\end{proof}

\begin{proposition}
	\label{induced morphism between colimits of diagrams of different shapes}
	Let~$\cat{A}$ be a category.
	Let~$\scat{I}$ and~$\scat{I'}$ be two small categories, and let~$D$ and~$D'$ be diagrams in~$\cat{A}$ of shapes~$\scat{I}$ and~$\scat{I}'$ respectively.
	Suppose that these diagrams admit colimits~$(C, (q_I)_I)$ and~$(C', (q'_I)_I)$ respectively.
	Let
	\[
		F \colon \scat{I} \to \scat{I}'
	\]
	be a functor with~$D' ∘ F = D$.
	There exists a unique morphism~$f$ from~$C$ to~$C'$ with
	\[
		f ∘ q_I = q'_{F(I)}
	\]
	for every object~$I$ of~$\scat{I}$.
	\qed
\end{proposition}

\subsubsection*{Construction of $\Lan_F(X)$ on objects}

We have for every object~$B$ of~$\scat{B}$ the diagram~$X ∘ P_B$ of shape~$F \comma B$ in~$\cat{S}$.
We choose for every object~$B$ of~$\scat{B}$ a colimit of the associated diagram~$X ∘ P_B$.
We denote this colimit by~$\Lan_F(X)(B)$, and denote for every object~$M$ of the category~$F \comma B$ ~$(X ∘ P_B)(M)$ to~$\Lan_F(X)(B)$ by~$i^B_M$.

Any object of~$F \comma B$ is a pair~$(A, h)$ consisting of an object~$A$ of~$\scat{A}$ and a morphism~$h$ from~$F(A)$ to~$B$, and the morphism~$i^B_{(A, h)}$ is of the form
\[
	i^B_{(A, h)} \colon X(A) \to \Lan_F(X)(B) \,.
\]
That~$(\Lan_F(X)(B), (i^B_M)_M)$ is a colimit of the diagram~$X ∘ P_B$ entails that for every morphism
\[
	f \colon M \to M'
\]
in~$F \comma B$, we have the equality
\begin{equation}
	\label{naturality for inclusion via objects}
	i^B_{M'} ∘ (X ∘ P_B)(f) = i^B_M \,.
\end{equation}
The objects~$M$ and~$M'$ are of the forms~$M = (A, h)$ and~$(A', h')$ and~$f$ is a morphism from~$A$ to~$A'$ such that~$h' ∘ f = h$.
We can re-express the equality~\eqref{naturality for inclusion via objects} as
\begin{equation}
	\label{naturality for inclusion via objects explicit}
	i^B_{(A', h')} ∘ X(f) = i^B_{(A, h)} \,.
\end{equation}
In a more diagrammatic formulation, we have the following:
\[
	\begin{tikzcd}[column sep = normal]
		F(A)
		\arrow{rr}[above]{F(f)}
		\arrow{dr}[below left]{h}
		&
		{}
		&
		F(A')
		\arrow{dl}[below right]{h'}
		\\
		{}
		&
		B
		&
		{}
	\end{tikzcd}
	\leadsto
	\begin{tikzcd}[column sep = small]
		X(A)
		\arrow{rr}[above]{X(f)}
		\arrow{dr}[below left]{i^B_{(A, h)}}
		&
		{}
		&
		X(A')
		\arrow{dl}[below right]{i^B_{(A', h')}}
		\\
		{}
		&
		\Lan_F(X)(B)
		&
		{}
	\end{tikzcd}
\]

\subsubsection*{Construction of $\Lan_F(X)$ on morphisms}

Let
\[
	g \colon B \to B'
\]
be a morphism in~$\scat{B}$.
This morphism induces a functor
\[
	g_* \colon (F \comma B) \to (F \comma B') \,,
\]
given by
\[
	g_*( (A, h) ) = (A, g ∘ h) \,,
	\quad
	g_*( f ) = f
\]
on objects and morphisms respectively.
The action of~$g_*$ may be depicted as follows:
\[
	\begin{tikzcd}[column sep = normal]
		F(A)
		\arrow{rr}[above]{F(f)}
		\arrow{dr}[below left]{h}
		&
		{}
		&
		F(A')
		\arrow{dl}[below right]{h'}
		\\
		{}
		&
		B
		&
		{}
	\end{tikzcd}
	\quad\leadsto
	\quad
	\begin{tikzcd}[column sep = normal]
		F(A)
		\arrow{rr}[above]{F(f)}
		\arrow[dashed]{dr}[below left]{h}
		\arrow[bend right]{ddr}[below left]{g ∘ h}
		&
		{}
		&
		F(A')
		\arrow[dashed]{dl}[below right]{h'}
		\arrow[bend left]{ddl}[below right]{g ∘ h'}
		\\
		{}
		&
		B
		\arrow[dashed]{d}[right]{g}
		&
		{}
		\\
		{}
		&
		B'
		&
		{}
	\end{tikzcd}
\]
The action of the functor~$g_*$ on objects of~$F \comma B$ doesn’t change the first entry, whence~$P_{B'} ∘ g_* = P_B$.
The following diagram therefore commutes:
\[
	\begin{tikzcd}[row sep = normal]
		(F \comma B)
		\arrow[bend left = 20]{drr}[above right]{X ∘ P_B}
		\arrow{dr}[above right]{P_B}
		\arrow{dd}[left]{g_*}
		&
		{}
		&[1em]
		{}
		\\
		{}
		&
		\scat{A}
		\arrow{r}[above, pos = 0.4]{X}
		&
		\cat{S}
		\\
		(F \comma B')
		\arrow{ur}[below right]{P_{B'}}
		\arrow[bend right = 20]{urr}[below right]{X ∘ P_{B'}}
		&
		{}
		&
		{}
	\end{tikzcd}
\]
This entails that the outer triangle
\[
	\begin{tikzcd}[row sep = small]
		(F \comma B)
		\arrow{dr}[above right, pos = 0.35]{X ∘ P_B}
		\arrow{dd}[left]{g_*}
		&
		{}
		\\
		{}
		&
		\cat{S}
		\\
		(F \comma B')
		\arrow{ur}[below right, pos = 0.35]{X ∘ P_{B'}}
		&
		{}
	\end{tikzcd}
\]
commutes.
It follows from \cref{induced morphism between colimits of diagrams of different shapes} that the functor~$g_*$ induces a morphism
\[
	\Lan_F(X)(g) \colon \Lan_F(X)(B) \to \Lan_F(X)(B') \,.
\]
This induced morphism is unique with the property
\begin{equation}
	\label{action of left kan extensions on morphisms}
	\Lan_F(X)(g) ∘ i^B_M = i^{B'}_{g_*(M)}
\end{equation}
for every object~$M$ of~$F \comma B$.

\subsubsection*{Checking the functoriality of $\Lan_F(X)$}

Let us shows that the assignment~$\Lan_F(X)$ from~$\scat{B}$ to~$\cat{S}$ is functorial.
\begin{itemize}

	\item
		Let~$B$ be an object of~$\scat{B}$.
		We need to check that~$\Lan_F(X)(\id_B) = \id_{\Lan_F(X)(B)}$.
		For this, it suffices to check that
		\[
			\Lan_F(X)(\id_B) ∘ i^B_M = \id_{\Lan_F(X)(B)} ∘ i^B_M
		\]
		for every object~$M$ of~$F \comma B$.
		This equality holds because the functor~$(\id_B)_*$ from~$F \comma B$ to~$F \comma B$ is the identity functor, and therefore
		\[
			\Lan_F(X)(\id_B) ∘ i^B_M
			=
			i^B_{(\id_B)_*(M)}
			=
			i^B_M
			=
			\id_{\Lan_F(X)(\id_B)} ∘ i^B_M \,.
		\]

	\item
		Let
		\[
			g \colon B \to B' \,,
			\quad
			g' \colon B' \to B''
		\]
		be two composable morphisms in~$\scat{B}$.
		The induced functors
		\begin{gather*}
			g_* \colon (F \comma B) \to (F \comma B') \,,
			\quad
			(g')_* \colon (F \comma B') \to (F \comma B'') \,,
			\\
			(g' ∘ g)_* \colon (F \comma B) \to (F \comma B'')
		\end{gather*}
		satisfy the identity~$g'_* ∘ g_* = (g' ∘ g)_*$.
		It follows that
		\begin{align*}
			\Lan_F(X)(g') ∘ \Lan_F(X)(g) ∘ i^B_M
			=
			\Lan_F(X)(g') ∘ i^{B'}_{g_*(M)}
			&=
			i^{B''}_{g'_*(g_*(M))}
			\\
			&=
			i^{B''}_{(g'_* ∘ g_*)(M)}
			\\
			&=
			i^{B''}_{(g' ∘ g)_*(M)}
		\end{align*}
		for every object~$M$ of~$F \comma B$.
		The composite~$\Lan_F(X)(g') ∘ \Lan_F(X)(g)$ therefore satisfies the defining property of the morphism~$\Lan_F(X)(g' ∘ g)$, which tells us that
		\[
			\Lan_F(X)(g' ∘ g) = \Lan_F(X)(g') ∘ \Lan_F(X)(g) \,.
		\]

\end{itemize}
We have thus proven the functoriality of~$\Lan_F(X)$ from~$\scat{B}$ to~$\cat{S}$.

\subsubsection*{The natural transformation $η \colon X \To \Lan_F(X) ∘ F$}

We note that every object~$A$ of~$\scat{A}$ results in an object~$F(A)$ of~$\scat{B}$, which furthermore results in the object~$(A, \id_{F(A)})$ of the category~$F \comma F(A)$.
We have therefore for every object~$A$ of~$\scat{A}$ the morphism
\[
	i^{F(A)}_{(A, \id_{F(A)})}
	\colon
	X(A)
	\to
	\Lan_F(X)(F(A)) \,.
\]
Let us abbreviate this morphism by~$η_A$.
It seems reasonable to assume that this transformation~$η$ from~$X$ to~$\Lan_F(X) ∘ F$ is natural.
To check this naturality, we need to verify that for every morphism
\[
	f \colon A \to A'
\]
in~$\scat{A}$, the following square diagram commutes:
\[
	\begin{tikzcd}[column sep = 7em]
		X(A)
		\arrow{r}[above]{X(f)}
		\arrow{d}[left]{η_A}
		&
		X(A')
		\arrow{d}[right]{η_{A'}}
		\\
		\Lan_F(X)(F(A))
		\arrow{r}[above]{\Lan_F(X)(F(f))}
		&
		\Lan_F(X)(F(A'))
	\end{tikzcd}
\]
The morphism~$F(f)$ goes from~$F(A)$ to~$F(A')$.
By construction of the morphism~$\Lan_F(X)(F(f))$ (see~\eqref{action of left kan extensions on morphisms}), we have therefore the chain of equalities
\[
	\Lan_F(X)(F(f)) ∘ η_A
	=
	\Lan_F(X)(F(f)) ∘ i^{F(A)}_{(A, \id_{F(A)})}
	=
	i^{F(A')}_{F(f)_*( (A, \id_{F(A)}) )}
%	=
%	i^{F(A')}_{(A, F(f) ∘ \id_{F(A)})}
	=
	i^{F(A')}_{(A, F(f))} \,.
\]
We have on the other hand the equality
\[
	η_{A'} ∘ X(f)
	=
	i^{F(A')}_{(A', \id_{F(A')})} ∘ X(f) \,.
\]
The desired commutativity is therefore equivalent to the equality
\[
	i^{F(A')}_{(A', \id_{F(A')})} ∘ X(f)
	=
	i^{F(A')}_{(A, F(f))} \,.
\]
This equality holds because~$f$ is a morphism from~$(A, F(f))$ to~$(A', \id_{F(A')})$ in the category~$F \comma F(A')$ (see \eqref{naturality for inclusion via objects explicit}).

\subsubsection*{From $\Lan_F(X) \To Y$ to $X \To Y ∘ F$}

Let from now~$Y$ be any functor from~$\scat{B}$ to~$\cat{S}$.
If~$β$ is a natural transformation from~$\Lan_F(X)$ to~$Y$, then~$β F$ is a natural transformation from~$\Lan_F(X) ∘ F$ to~$Y ∘ F$, whence
\[
	β F ∘ η
\]
is a natural transformation from~$X$ to~$Y ∘ F$.

\subsubsection*{From $X \To Y ∘ F$ to $\Lan_F(X) \To Y$}

Let conversely~$α$ be a natural transformation from~$X$ to~$Y ∘ F$.
Let~$B$ be an object of~$\scat{B}$.
For every object~$(A, h)$ of~$F \comma B$, we have the resulting morphism
\[
	\tilde{β}_{(A, h)} \colon X(A) \xto{α_A} Y(F(A)) \xto{Y(h)} Y(B) \,.
\]
This is a morphism from the object~$X(A) = (X ∘ P_B)( (A, h) )$ to the object~$Y(B)$.

We claim that these morphisms, where~$(A, h)$ ranges through the category~$F \comma B$, form a cocone for the diagram~$X ∘ P_B$.
To prove this claim, we consider a morphism
\[
	f \colon (A, h) \to (A', h')
\]
in~$F \comma B$, and need to show that
\[
	\tilde{β}_{(A', h')} ∘ (X ∘ P_B)(f)
	=
	\tilde{β}_{(A, h)} \,.
\]
This equality holds because
\begin{align*}
	\tilde{β}_{(A', h')} ∘ (X ∘ P_B)(f)
	&=
	Y(h') ∘ α_{A'} ∘ X(f)
	\\
	&=
	Y(h') ∘ (Y ∘ F)(f) ∘ α_A
	\\
	&=
	Y(h') ∘ Y(F(f)) ∘ α_A
	\\
	&=
	Y(h' ∘ F(f)) ∘ α_A
	\\
	&=
	Y(h) ∘ α_A
	\\
	&=
	\tilde{β}_{(A, h)} \,.
\end{align*}

We know that~$(\Lan_F(X), (i^B_M)_M)$ is a colimit of the diagram~$X ∘ P_B$, by construction.
It follows that the morphisms~$\tilde{β}_M$, where~$M$ ranges through~$F \comma B$, induce a morphism~$β_B$ from~$\Lan_F(X)(B)$ to~$Y(B)$, which is unique with the property that
\[
	β_B ∘ i^B_M = \tilde{β}_M
\]
for every object~$M$ of~$F \comma B$.

These morphisms~$β_B$, where~$B$ ranges through the objects of~$\scat{B}$, assemble into a transformation~$β$ from~$\Lan_F(X)$ to~$Y$.
Let us check that the transformation~$β$ is natural.
To show this, we need to check that for every morphism
\[
	g \colon B \to B'
\]
in~$\scat{B}$ the following square diagram commutes:
\[
	\begin{tikzcd}[column sep = 6em]
		\Lan_F(X)(B)
		\arrow{r}[above]{\Lan_F(X)(g)}
		\arrow{d}[left]{β_B}
		&
		\Lan_F(X)(B')
		\arrow{d}[right]{β_{B'}}
		\\
		Y(B)
		\arrow{r}[above]{Y(g)}
		&
		Y(B')
	\end{tikzcd}
\]
In other words, we need to prove the equality
\[
	β_{B'} ∘ \Lan_F(X)(g) = Y(g) ∘ β_B \,.
\]
Both sides of this desired equation are morphisms with domain~$\Lan_F(X)(B)$.
We thus need to show that
\[
	β_{B'} ∘ \Lan_F(X)(g) ∘ i^B_M
	=
	Y(g) ∘ β_B ∘ i^B_M
\]
for every object~$M$ of~$F \comma B$.
Such an object~$M$ is of the form~$(A, h)$, and we have the chain of equalities
\begin{align*}
	β_{B'} ∘ \Lan_F(X)(g) ∘ i^B_{(A, h)}
	&=
	β_{B'} ∘ i^{B'}_{g_*((A, h))}
	\\[0.3em]
	&=
	β_{B'} ∘ i^{B'}_{(A, g ∘ h)}
	\\
	&=
	\tilde{β}_{(A, g ∘ h)}
	\\
	&=
	Y(g ∘ h) ∘ α_A
	\\
	&=
	Y(g) ∘ Y(h) ∘ α_A
	\\
	&=
	Y(g) ∘ \tilde{β}_{(A, h)}
	\\[0.3em]
	&=
	Y(g) ∘ β_B ∘ i^B_{(A, h)} \,.
\end{align*}
We have thus constructed a natural transformation~$β$ from~$\Lan_F(X)$ to~$Y$.

\subsubsection*{Correspondence between $X \To Y ∘ F$ and $\Lan_F(X) \To Y$}

The above two constructions are mutually inverse.

Let first~$α$ be a natural transformation from~$X$ to~$Y ∘ F$.
Let~$β$ be the resulting natural transformation from~$\Lan_F(X)$ to~$Y$, which is uniquely determined by
\[
	β_B ∘ i^B_{(A, h)} = Y(h) ∘ α_A
\]
for every object~$B$ of~$\scat{B}$ and every object~$(A, h)$ of~$F \comma B$.
Let~$α'$ be the natural transformation from~$X$ to~$Y ∘ F$ resulting from~$β$, given by
\[
	α' = β F ∘ η \,.
\]
For every object~$A$ of~$\scat{A}$ we have the chain of equalities
\begin{align*}
	α'_A
	=
	(β F ∘ η)_A
	=
	β_{F(A)} ∘ η_A
	=
	β_{F(A)} ∘ i^{F(A)}_{(A, \id_{F(A)})}
	=
	Y(\id_{F(A)}) ∘ α_A
	&=
	\id_{Y(F(A))} ∘ α_A
	\\
	&=
	α_A \,.
\end{align*}
This shows that~$α' = α$.

Let conversely~$β$ be a natural transformation from~$\Lan_F(X)$ to~$Y$.
Let~$α$ be the resulting natural transformation from~$X$ to~$Y ∘ F$ given by
\[
	α = β F ∘ η \,.
\]
Let~$β'$ be the resulting natural transformation from~$\Lan_F(X)$ to~$Y$, which is uniquely determined by
\[
	β'_B ∘ i^B_{(A, h)} = Y(h) ∘ α_A
\]
for every object~$B$ of~$\scat{B}$ and every object~$(A, h)$ of~$F \comma B$.
We have the chain of equalities
\begin{align*}
	\SwapAboveDisplaySkip
	β'_B ∘ i^B_{(A, h)}
	&=
	Y(h) ∘ α_A
	\\
	&=
	Y(h) ∘ (β F ∘ η)_A
	\\
	&=
	Y(h) ∘ β_{F(A)} ∘ η_A
	\\
	&=
	Y(h) ∘ β_{F(A)} ∘ i^{F(A)}_{(A, \id_{F(A)})}
	\\
	&=
	β_B ∘ \Lan_F(X)(h) ∘ i^{F(A)}_{(A, \id_{F(A)})}
	\\
	&=
	β_B ∘ i^B_{h_*((A, \id_{F(A)}))}
	\\
	&=
	β_B ∘ i^B_{(A, h ∘ \id_{F(A)})}
	\\
	&=
	β_B ∘ i^B_{(A, h)}
\end{align*}
for every object~$B$ of~$\scat{B}$ and every object~$(A, h)$ of~$F \comma B$, therefore
\[
	β'_B = β_B
\]
for every object~$B$ of~$\scat{B}$, and thus~$β' = β$.

This shows that the two constructions are indeed mutually inverse.



\subsubsection{}

We start off by extending~$\Lan_F$ to a fully-fledged functor from~$[\scat{A}, \cat{S}]$ to~$[\scat{B}, \cat{S}]$.
We then show that~$\Lan_F$ is left adjoint to the functor~$F^*$ from~$[\scat{B}, \cat{S}]$ to~$[\scat{A}, \cat{S}]$.

\subsubsection*{Action of $\Lan_F$ on morphisms}

So far, we have only explained how~$\Lan_F$ acts on objects.
Let us construct an action of~$\Lan_F$ on morphisms too.

Let~$X$ and~$X'$ be two functors from~$\scat{A}$ to~$\cat{S}$ and let
\[
	γ \colon X \To X'
\]
be a natural transformation, i.e., a morphism in~$[\scat{A}, \cat{S}]$.
We have for every object~$B$ of~$\scat{B}$ the induced natural transformation
\[
	γ P_B \colon X ∘ P_B \To X' ∘ P_B \,.
\]
The two functors~$X ∘ P_B$ and~$X' ∘ P_B$ are diagrams of the same shape~$F \comma B$ in~$\cat{S}$, with colimits~$\Lan_F(X)(B)$ and~$\Lan_F(X')(B)$ respectively.
The natural transformation~$γ P_B$ therefore induces a morphism
\[
	\Lan_F(γ)_B \colon \Lan_F(X)(B) \to \Lan_F(X')(B)
\]
that is unique with the property
\[
	\Lan_F(γ)_B ∘ i^{X, B}_M
	=
	i^{X',\, B}_M ∘ (γ P_B)_M
\]
for every object~$M$ of~$F \comma B$.
(We use now the slightly more expressive notations~$i^{X, B}_M$ and~$i^{X', B}_M$ instead of just~$i^B_M$, to track which functor we are dealing with.)
This equation can somewhat more explicitly be rewritten as
\[
	\Lan_F(γ)_B ∘ i^{X, B}_{(A, h)}
	=
	i^{X',\, B}_{(A, h)} ∘ γ_A
\]
for every object~$(A, h)$ of~$F \comma B$.

The morphisms~$\Lan_F(γ)_B$, where~$B$ ranges through the objects of~$\scat{B}$, assemble into a transformation~$\Lan_F(γ)$ from~$\Lan_F(X)$ to~$\Lan_F(X')$.
Let us check that this transformation~$\Lan_F(γ)$ is natural.
To this end, we have to check that for every morphism
\[
	g \colon X \to X'
\]
in~$\scat{B}$ the following diagram commutes:
\[
	\begin{tikzcd}[row sep = large, column sep = huge]
		\Lan_F(X)(B)
		\arrow{r}[above]{\Lan_F(X)(g)}
		\arrow{d}[left]{\Lan_F(γ)_B}
		&
		\Lan_F(X)(B')
		\arrow{d}[right]{\Lan_F(γ)_{B'}}
		\\
		\Lan_F(X')(B)
		\arrow{r}[above]{\Lan_F(X')(g)}
		&
		\Lan_F(X')(B')
	\end{tikzcd}
\]
We thus need to prove the equality
\[
	\Lan_F(γ)_{B'} ∘ \Lan_F(X)(g)
	=
	\Lan_F(X')(g) ∘ \Lan_F(γ)_B \,.
\]
Both sides of this desired equality have the object~$\Lan_F(X)(B)$ as its domain, whence we need to show check that
\[
	\Lan_F(γ)_{B'} ∘ \Lan_F(X)(g) ∘ i^{X, B}_M
	=
	\Lan_F(X')(g) ∘ \Lan_F(γ)_B ∘ i^{X, B}_M \,.
\]
for every object~$M$ of~$F \comma B$.
Each such object~$M$ is of the form~$(A, h)$, and we observe the chain of equalities
\begin{align*}
	{}&
	\Lan_F(γ)_{B'} ∘ \Lan_F(X)(g) ∘ i^{X, B}_{(A, h)}
	\\
	={}&
	\Lan_F(γ)_{B'} ∘ i^{X, B'}_{(A, g ∘ h)}
	\\
	={}&
	i^{X',\, B'}_{(A, g ∘ h)} ∘ γ_A
	\\
	={}&
	\Lan_F(X')(g) ∘ i^{X', B}_{(A, h)} ∘ γ_A
	\\
	={}&
	\Lan_F(X')(g) ∘ \Lan_F(γ)_B ∘ i^{X, B}_{(A, h)} \,.
\end{align*}

We have thus shown the naturality of~$\Lan_F(γ)$, and therefore constructed the action of~$\Lan_F$ on morphisms of~$[\scat{A}, \cat{S}]$.

\subsubsection*{Functoriality of $\Lan_F$}

Let us check that the assignment~$\Lan_F$ is functorial.
\begin{itemize}

	\item
		Let~$X$ be a functor from~$\scat{A}$ to~$\cat{S}$.
		We have
		\[
			( \id_{\Lan_F(X)} )_B ∘ i^{X, B}_{(A, h)}
			=
			\id_{\Lan_F(X)(B)} ∘ i^{X, B}_{(A, h)}
			=
			i^{X, B}_{(A, h)}
			=
			i^{X, B}_{(A, h)} ∘ \id_{X(A)}
			=
			i^{X, B}_{(A, h)} ∘ (\id_X)_A
		\]
		for every object~$B$ of~$\scat{B}$ and every object~$(A, h)$ of~$F \comma B$.
		This shows that the natural transformation~$\id_{\Lan_F(X)}$ satisfies the defining property of~$\Lan_F(\id_X)$, whence
		\[
			\Lan_F(\id_X) = \id_{\Lan_F(X)} \,.
		\]

	\item
		Let~$X$,~$X'$ and~$X''$ be functors from~$\scat{A}$ to~$\cat{S}$ and let
		\[
			γ \colon X \To X' \,,
			\quad
			γ' \colon X' \To X''
		\]
		be two composable natural transformations.
		We have thi chain of equalities
		\begin{align*}
			( \Lan_F(γ') ∘ \Lan_F(γ) )_B ∘ i^{X, B}_{(A, h)}
			&=
			\Lan_F(γ')_B ∘ \Lan_F(γ)_B ∘ i^{X, B}_{(A, h)}
			\\
			&=
			\Lan_F(γ)'_B ∘ i^{X',\, B}_{(A, h)} ∘ γ_A
			\\
			&=
			i^{X'',\, B}_{(A, h)} ∘ γ'_A ∘ γ_A
			\\[0.3em]
			&=
			i^{X'',\, B}_{(A, h)} ∘ (γ' ∘ γ)_A
			\\
			&=
			\Lan_F(γ' ∘ γ)_B ∘ i^{X, B}_{(A, h)}
		\end{align*}
		for every object~$B$ of~$\scat{B}$ and every object~$(A, h)$ of~$F \comma B$.
		This shows that the composite~$\Lan_F(γ') ∘ \Lan_F(γ)$ satisfies the defining property of the natural transformation~$\Lan_F(γ' ∘ γ)$, whence
		\[
			\Lan_F(γ' ∘ γ) = \Lan_F(γ') ∘ \Lan_F(γ) \,.
		\]

\end{itemize}
We have thus shown the functoriality of~$\Lan_F$ from~$[\scat{A}, \cat{S}]$ to~$[\scat{B}, \cat{S}]$.

\subsubsection*{Adjointness of $\Lan_F$ and $F^*$}

For every functor~$X$ from~$\scat{A}$ to~$\cat{S}$ we have constructed in part~(a) of this exercise a natural transformation
\[
	η_X \colon X \To \Lan_F(X) ∘ F \,,
\]
so that for every functor~$Y$ from~$\scat{B}$ to~$\cat{S}$, the map
\begin{equation}
	\label{bijectivity for adjointness}
	[\scat{B}, \cat{S}](\Lan_F(X), Y)
	\to
	[\scat{A}, \cat{S}](X, Y ∘ F) \,,
	\quad
	β
	\mapsto
	β F ∘ η_X
\end{equation}
is bijective.

The natural transformations~$η_X$, where~$X$ ranges through the objects of the category~$[\scat{A}, \cat{S}]$, are morphisms in~$[\scat{A}, \cat{S}]$.
These morphisms assemble into a transformation~$η$ from~$\Id_{[\scat{A}, \cat{S}]}$ to~$\Lan_F(\ph) ∘ F = F^* ∘ \Lan_F$.

To see this, we have to check that for every two functors~$X$ and~$X'$ from~$\scat{A}$ to~$\cat{S}$ and every natural transformation
\[
	γ \colon X \to X'
\]
the following diagram commutes:
\[
	\begin{tikzcd}[row sep = large, column sep = huge]
		X
		\arrow[Rightarrow]{r}[above]{γ}
		\arrow[Rightarrow]{d}[left]{η_X}
		&
		X'
		\arrow[Rightarrow]{d}[right]{η_{X'}}
		\\
		\Lan_F(X) ∘ F
		\arrow[Rightarrow]{r}[above]{\Lan_F(γ) F}
		&
		\Lan_F(X') ∘ F
	\end{tikzcd}
\]
This diagram takes place inside a functor category, whence it’s commutativity can be checked pointwise.
We thus have to show that for every object~$A$ of~$\scat{A}$ the following diagram commutes:
\[
	\begin{tikzcd}[row sep = large, column sep = 6em]
		X(A)
		\arrow{r}[above]{γ_A}
		\arrow{d}[left]{(η_X)_A}
		&
		X'(A)
		\arrow{d}[right]{(η_{X'})_A}
		\\
		\Lan_F(X)(F(A))
		\arrow{r}[above]{(\Lan_F(γ) F)_A}
		&
		\Lan_F(X')(F(A))
	\end{tikzcd}
\]
This commutativity hold because
\begin{align*}
	(\Lan_F(γ) F)_A ∘ (η_X)_A
	&=
	\Lan_F(γ)_{F(A)} ∘ i^{X, F(X)}_{(A, \id_{F(X)})}
	\\
	&=
	i^{X', F(X)}_{(A, \id_{F(X)})} ∘ γ_A
	\\
	&=
	(η_{X'})_A ∘ γ_A \,.
\end{align*}

To summarize our findings:
We have overall constructed a natural transformation~$η$ from~$\Id_{[\scat{A}, \cat{S}]}$ to~$F^* ∘ \Lan_F$.
The bijectivity of~\eqref{bijectivity for adjointness} tells us that for every object~$X$ of~$[\scat{A}, \cat{S}]$, the morphism~$η_X$ is an initial object of the comma category~$X \comma F^*$.

It follows from Theorem~2.3.6 and its proof that~$\Lan_F$ is left adjoint to~$F^*$, with~$η$ being the unit of one such adjunction.



\subsubsection{}

\begin{proposition}
	\label{kan adjoints for set valued functor categories}
	Let~$\scat{A}$ and~$\scat{B}$ be two small categories and let~$F$ be a functor from~$\scat{A}$ to~$\scat{B}$.
	The induced functor
	\[
		F_* \colon [\scat{B}, \Set] \to [\scat{A}, \Set]
	\]
	admits both a left adjoint and a right adjoint.
\end{proposition}

\begin{proof}
	The category~$\Set$ admits both small limits and small colimits.
	The assertion therefore follows from part~(b) of this exercise and its dual.
	More explicitly, the left adjoint of~$F^*$ is given by the left Kan extension~$\Lan_F$, and the right adjoint of~$F^*$ is given by the right Kan extension~$\Ran_F$.
\end{proof}

Let~$G$ and~$H$ be two groups and let~$φ$ be a homomorphism of groups from~$G$ to~$H$. We may regard~$G$ and~$H$ as small one-object categories~$\scat{G}$ and~$\scat{H}$ respectively, and the homomorphism~$φ$ as a functor~$Φ$ from~$G$ to~$H$.
The functor categories~$[\scat{G}, \Set]$ and~$[\scat{H}, \Set]$ are isomorphic to the categories of~\sets{$G$} and~\sets{$H$} respectively, and the functor
\[
	Φ^*
	\colon
	[\scat{H}, \Set] \to [\scat{G}, \Set]
\]
corresponds to the pull-back functor
\[
	φ^* \colon \GSet{H} \to \GSet{G}
\]
induced by~$φ$.
We know from \cref{kan adjoints for set valued functor categories} that the functor~$Φ^*$ admits both a left adjoint and a right adjoint.
As a consequence, the functor~$φ^*$ admits both a left adjoint and a right adjoint.

For the trivial group~$1$, the category~$\GSet{1}$ is in turn isomorphic to~$\Set$.
We therefore find the following:
\begin{itemize}

	\item
		By choosing the group~$H$ as trivial, we arrive at the functor
		\[
			\Set \to \GSet{G}
		\]
		that regards every set as a trivial~\set{$G$}.
		This functor admits both a left adjoint and a right adjoint.

	\item
		By choosing the group~$G$ as trivial, we see that the forgetful functor
		\[
			\GSet{H} \to \Set
		\]
		admits both a left adjoint and a right adjoint.

\end{itemize}
