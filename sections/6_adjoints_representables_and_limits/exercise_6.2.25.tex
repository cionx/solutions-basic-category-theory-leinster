\subsection{}



\subsubsection{}

\begin{proposition}
	Let~$\cat{A}$ be a category.
	Let~$\scat{I}$ and~$\scat{I'}$ be two small categories, and let~$D$ and~$D'$ be diagrams in~$\cat{A}$ of shapes~$\scat{I}$ and~$\scat{I}'$ respectively.
	Suppose that these diagrams admit colimits~$(C, (q_I)_I)$ and~$(C', (q'_I)_I)$ respectively.
	Let
	\[
		F \colon \scat{I} \to \scat{I}'
	\]
	be a functor and let
	\[
		α \colon D \To D' ∘ F
	\]
	be a natural transformation.
	There exists a unique morphism~$f$ from~$C$ to~$C'$ with
	\[
		f ∘ q_I = q'_{F(I)} ∘ α_I
	\]
	for every object~$I$ of~$\scat{I}$.
\end{proposition}

\begin{proof}
	We have for every morphism
	\[
		u \colon I \to J
	\]
	in~$\scat{I}$ the equalities
	\[
		q'_{F(J)} ∘ α_J ∘ D(u)
		=
		q'_{F(J)} ∘ (D' ∘ F)(u) ∘ α_I
		=
		q'_{F(J)} ∘ D'( F(u) ) ∘ α_I
		=
		q'_{F(I)} ∘ α_I \,.
	\]
	The assertion thus follows from the universal property of the colimit~$(C, (q_I)_I)$.
\end{proof}

\begin{corollary}
	\label{induced morphism between colimits of diagrams of different shapes}
	Let~$\cat{A}$ be a category.
	Let~$\scat{I}$ and~$\scat{I'}$ be two small categories, and let~$D$ and~$D'$ be diagrams in~$\cat{A}$ of shapes~$\scat{I}$ and~$\scat{I}'$ respectively.
	Suppose that these diagrams admit colimits~$(C, (q_I)_I)$ and~$(C', (q'_I)_I)$ respectively.
	Let
	\[
		F \colon \scat{I} \to \scat{I}'
	\]
	be a functor with~$D' ∘ F = D$.
	There exists a unique morphism~$f$ from~$C$ to~$C'$ with
	\[
		f ∘ q_I = q'_{F(I)}
	\]
	for every object~$I$ of~$\scat{I}$.
	\qed
\end{corollary}

We have for every object~$B$ of~$\scat{B}$ the diagram~$X ∘ P_B$ of shape~$F \comma B$ in~$\cat{S}$.
We choose for every object~$B$ of~$\scat{B}$ a colimit of the associated diagram~$X ∘ P_B$;
we denote this colimit by~$\Lan_F(X)(B)$, and denote for every object~$M$ of~$F \comma B$ the canonical morphism from~$(X ∘ P_B)(M)$ to~$\Lan_F(X)(B)$ by~$i^B_M$.
(Recall that the objects of~$F \comma B$ are pairs~$(A, h)$ consisting of an object~$A$ of~$\scat{A}$ and a morphism~$h$ in~$\scat{B}$ from~$F(A)$ to~$B$.)

Let
\[
	g \colon B \to B'
\]
be a morphism in~$\scat{B}$.
This morphism induces a functor
\[
	g_* \colon (F \comma B) \to (F \comma B') \,,
\]
given by
\[
	g_*( (A, h) ) = (A, g ∘ h) \,,
	\quad
	g_*( f ) = f
\]
on objects and morphisms respectively.
The action of~$g_*$ may be depicted as follows:
\[
	\begin{tikzcd}[row sep = large]
		F(A)
		\arrow{rr}[above]{F(f)}
		\arrow{dr}[below left]{h}
		&
		{}
		&
		F(A')
		\arrow{dl}[below right]{h'}
		\\
		{}
		&
		B
		&
		{}
	\end{tikzcd}
	\quad\leadsto
	\quad
	\begin{tikzcd}[row sep = large]
		F(A)
		\arrow{rr}[above]{F(f)}
		\arrow[dashed]{dr}[below left]{h}
		\arrow[bend right]{ddr}[below left]{g ∘ h}
		&
		{}
		&
		F(A')
		\arrow[dashed]{dl}[below right]{h'}
		\arrow[bend left]{ddl}[below right]{g ∘ h'}
		\\
		{}
		&
		B
		\arrow[dashed]{d}[right]{g}
		&
		{}
		\\
		{}
		&
		B'
		&
		{}
	\end{tikzcd}
\]
We observe that~$P_{B'} ∘ g_* = P_B$, i.e., so that the following diagram commutes:
\[
	\begin{tikzcd}[column sep = large]
		(F \comma B)
		\arrow[bend left = 20]{drr}[above right]{X ∘ P_B}
		\arrow{dr}[above right]{P_B}
		\arrow{dd}[left]{g_*}
		&
		{}
		&[1em]
		{}
		\\
		{}
		&
		\scat{A}
		\arrow{r}[above]{X}
		&
		\cat{S}
		\\
		(F \comma B')
		\arrow{ur}[below right]{P_{B'}}
		\arrow[bend right = 20]{urr}[below right]{X ∘ P_{B'}}
		&
		{}
		&
		{}
	\end{tikzcd}
\]
It follows from \cref{induced morphism between colimits of diagrams of different shapes} that there exists a unique morphism
\[
	\Lan_F(X)(g) \colon \Lan_F(X)(B) \to \Lan_F(X)(B')
\]
such that
\[
	\Lan_F(X)(g) ∘ i^B_M = i^{B'}_{g_*(M)}
\]
for every object~$M$ of~$F \comma B$.

The above assignments define a functor~$\Lan_F(X)$ from~$\scat{B}$ to~$\cat{S}$:
\begin{itemize}

	\item
		Let~$B$ be an object of~$\scat{B}$.
		The functor~$(\id_B)_*$ from~$F \comma B$ to~$F \comma B$ is the identity functor.
		Therefore,
		\[
			\id_{\Lan_F(X)(\id_B)} ∘ i^B_M
			=
			i^B_M
			=
			i^B_{(\id_B)_*(M)}
		\]
		for every object~$M$ of~$F \comma B$.
		This means that~$\id_{\Lan_F(X)(\id_B)}$ satisfies the defining property of~$\Lan_F(X)(\id_B)$, whence
		\[
			\Lan_F(X)(\id_B) = \id_{\Lan_F(X)(\id_B)} \,.
		\]

	\item
		Let
		\[
			g \colon B \to B' \,,
			\quad
			g' \colon B' \to B''
		\]
		be two composable morphisms in~$\scat{B}$.
		We have~$g'_* ∘ g_* = (g' ∘ g)_*$ and therefore
		\begin{align*}
			\Lan_F(X)(g') ∘ \Lan_F(X)(g) ∘ i^B_M
			&=
			\Lan_F(X)(g') ∘ i^{B'}_{g_*(M)}
			\\
			&=
			i^{B''}_{g'_*(g_*(M))}
			\\
			&=
			i^{B''}_{(g' ∘ g)_*(M)}
		\end{align*}
		for every object~$M$ of~$F \comma B$.
		The composite~$\Lan_F(X)(g') ∘ \Lan_F(X)(g)$ hence satisfies the defining property of the morphism~$\Lan_F(X)(g' ∘ g)$, whence
		\[
			\Lan_F(X)(g' ∘ g) = \Lan_F(X)(g') ∘ \Lan_F(X)(g) \,.
		\]

\end{itemize}
This shows that~$\Lan_F(X)$ is indeed a functor from~$\scat{B}$ to~$\cat{S}$.

Any object~$A$ of~$\scat{A}$ results in an object~$F(A)$ of~$\scat{B}$, which furthermore results in the object~$(A, \id_{F(A)})$ of~$F \comma F(A)$.
We have therefore for every object~$A$ of~$\scat{A}$ the morphism
\[
	i^{F(A)}_{(A, \id_{F(A)})}
	\colon
	X(A)
	\to
	\Lan_F(X)(F(A)) \,.
\]
We abbreviate this morphism by~$λ_A$, and claim that~$λ$ is a natural transformation from~$X$ to~$\Lan_F(X) ∘ F$.
To prove this naturality, we have to show that for every morphism
\[
	f \colon A \to A'
\]
in~$\scat{A}$ the following square diagram commutes:
\[
	\begin{tikzcd}[row sep = large, column sep = 7em]
		X(A)
		\arrow{r}[above]{X(f)}
		\arrow{d}[left]{λ_A}
		&
		X(A')
		\arrow{d}[right]{λ_{A'}}
		\\
		\Lan_F(X)(F(A))
		\arrow{r}[above]{\Lan_F(X)(F(f))}
		&
		\Lan_F(X)(F(A'))
	\end{tikzcd}
\]
We note that~$M' ≔ (A', \id_{F(A')})$ and~$M ≔ (A, F(f))$ are two objects of~$F \comma F(A')$, and that the given morphism~$f$ is also a morphism from~$M$ to~$M'$ in~$F \comma F(A')$.
It follows that
\[
	i^{F(A')}_{M'} ∘ (X ∘ P_{F(A')})(f)
	=
	i^{F(A')}_{M} \,.
\]
In other words, we have
\[
	λ_{A'} ∘ X(f) = i^{F(A')}_{M} \,,
\]
because~$i^{F(A')}_{M'} = λ_{A'}$.
We now find that
\begin{align*}
	\Lan_F(X)(F(f)) ∘ λ_A
	&=
	\Lan_F(X)(F(f)) ∘ i^{F(A)}_{(A, \id_{F(A)})}
	\\[0.3em]
	&=
	i^{F(A')}_{(A, F(f))}
	\\[0.3em]
	&=
	i^{F(A')}_{M}
	\\
	&=
	λ_{A'} ∘ X(f) \,.
\end{align*}
This shows the commutativity of the desired diagram.

Let from now~$Y$ be any functor from~$\scat{B}$ to~$\cat{S}$.

If~$β$ is a natural transformation from~$\Lan_F(X)$ to~$Y$, then~$β F$ is a natural transformation from~$\Lan_F(X) ∘ F$ to~$Y ∘ F$, whence
\[
	β F ∘ λ
\]
is a natural transformation from~$X$ to~$Y ∘ F$.

Let on the other hand~$α$ be a natural transformation from~$X$ to~$Y ∘ F$.
Let~$B$ be an object of~$\scat{B}$.
For every object~$(A, h)$ of~$F \comma B$, we have the resulting morphism
\[
	\tilde{β}_{(A, h)} \colon X(A) \xto{α_A} Y(F(A)) \xto{Y(h)} Y(B) \,.
\]
This is a morphism from~$(X ∘ P_B)( (A, h) )$ to~$Y(B)$.
For every morphism
\[
	f \colon (A, h) \to (A', h')
\]
in~$F \comma B$, we have
\begin{align*}
	\tilde{β}_{(A', h')} ∘ (X ∘ P_B)(f)
	&=
	Y(h') ∘ α_{A'} ∘ X(f)
	\\
	&=
	Y(h') ∘ (Y ∘ F)(f) ∘ α_A
	\\
	&=
	Y(h') ∘ Y(F(f)) ∘ α_A
	\\
	&=
	Y(h' ∘ F(f)) ∘ α_A
	\\
	&=
	Y(h) ∘ α_A
	\\
	&=
	\tilde{β}_{(A, h)} \,.
\end{align*}
This shows that the object~$Y(B)$ together with the morphisms~$\tilde{β}_{(A, h)}$, where~$(A, h)$ ranges through the objects of~$F \comma B$, form a cocone of the diagram~$X ∘ P_B$.
There hence exists a unique morphism~$β_B$ from~$\Lan_F(X)(B)$ to~$Y(B)$ with
\[
	β_B ∘ i^B_M = \tilde{β}_M
\]
for every object~$M$ of~$F \comma B$.

The resulting transformation~$β$ from~$\Lan_F(X)$ to~$Y$ is natural.
To see this, we need to show that for every morphism
\[
	g \colon B \to B'
\]
in~$\scat{B}$ the following square diagram commutes:
\[
	\begin{tikzcd}[row sep = large, column sep = 6em]
		\Lan_F(X)(B)
		\arrow{r}[above]{\Lan_F(X)(g)}
		\arrow{d}[left]{β_B}
		&
		\Lan_F(X)(B')
		\arrow{d}[right]{β_{B'}}
		\\
		Y(B)
		\arrow{r}[above]{Y(g)}
		&
		Y(B')
	\end{tikzcd}
\]
In other words, we need to prove the equality
\[
	β_{B'} ∘ \Lan_F(X)(g) = Y(g) ∘ β_B \,.
\]
For this, it suffices to show that for every object~$(A, h)$ of~$F \comma B$ we have
\[
	β_{B'} ∘ \Lan_F(X)(g) ∘ i^B_{(A, h)}
	=
	Y(g) ∘ β_B ∘ i^B_{(A, h)} \,.
\]
This equality holds because
\begin{align*}
	β_{B'} ∘ \Lan_F(X)(g) ∘ i^B_{(A, h)}
	&=
	β_{B'} ∘ i^{B'}_{g_*((A, h))}
	\\
	&=
	β_{B'} ∘ i^{B'}_{(A, g ∘ h)}
	\\
	&=
	\tilde{β}_{(A, g ∘ h)}
	\\
	&=
	Y(g ∘ h) ∘ α_A
	\\
	&=
	Y(g) ∘ Y(h) ∘ α_A
	\\
	&=
	Y(g) ∘ \tilde{β}_{(A, h)}
	\\
	&=
	Y(g) ∘ β_B ∘ i^B_{(A, h)} \,.
\end{align*}
We have thus constructed a natural transformation~$β$ from~$\Lan_F(X)$ to~$Y$.

The above two constructions are mutually inverse.
\begin{itemize}

	\item
		Let~$α$ be a natural transformation from~$X$ to~$Y ∘ F$.
		Let~$β$ be the resulting natural transformation from~$\Lan_F(X)$ to~$Y$, which is uniquely determined by
		\[
			β_B ∘ i^B_{(A, h)} = Y(h) ∘ α_A
		\]
		for every object~$B$ of~$\scat{B}$ and every object~$(A, h)$ of~$F \comma B$.
		Let~$α'$ be the natural transformation from~$X$ to~$Y ∘ F$ resulting from~$β$, given by
		\[
			α' = β F ∘ λ \,.
		\]
		For every object~$A$ of~$\scat{A}$ we have
		\begin{align*}
			α'_A
			=
			(β F ∘ λ)_A
			=
			β_{F(A)} ∘ λ_A
			=
			β_{F(A)} ∘ i^{F(A)}_{(A, \id_{F(A)})}
			=
			Y(\id_{F(A)}) ∘ α_A
			&=
			\id_{Y(F(A))} ∘ α_A
			\\
			&=
			α_A \,,
		\end{align*}
		whence~$α' = α$.

	\item
		Let~$β$ be a natural transformation from~$\Lan_F(X)$ to~$Y$.
		Let~$α$ be the resulting natural transformation from~$X$ to~$Y ∘ F$ given by
		\[
			α = β F ∘ λ \,.
		\]
		Let~$β'$ be the resulting natural transformation from~$\Lan_F(X)$ to~$Y$, which is uniquely determined by
		\[
			β'_B ∘ i^B_{(A, h)} = Y(h) ∘ α_A
		\]
		for every object~$B$ of~$\scat{B}$ and every object~$(A, h)$ of~$F \comma B$.
		It follows that
		\begin{align*}
			β'_B ∘ i^B_{(A, h)}
			&=
			Y(h) ∘ α_A
			\\
			&=
			Y(h) ∘ (β F ∘ λ)_A
			\\
			&=
			Y(h) ∘ β_{F(A)} ∘ λ_A
			\\
			&=
			Y(h) ∘ β_{F(A)} ∘ i^{F(A)}_{(A, \id_{F(A)})}
			\\
			&=
			β_B ∘ \Lan_F(X)(h) ∘ i^{F(A)}_{(A, \id_{F(A)})}
			\\
			&=
			β_B ∘ i^B_{(A, h ∘ \id_{F(A)})}
			\\
			&=
			β_B ∘ i^B_{(A, h)}
		\end{align*}
		for every object~$B$ of~$\scat{B}$ and every object~$(A, h)$ of~$F \comma B$, therefore
		\[
			β'_B = β_B
		\]
		for every object~$B$ of~$\scat{B}$, and thus~$β' = β$.

\end{itemize}
This shows that the above two constructions are indeed mutually inverse.



\subsubsection{}

We start off by extending the assignment~$\Lan_F$ to a functor from~$[\scat{A}, \cat{S}]$ to~$[\scat{B}, \cat{S}]$.
To do this, let~$X$ and~$X'$ be two functors from~$\scat{A}$ to~$\cat{S}$ and let
\[
	γ \colon X \To X'
\]
be a natural transformation.
We have for every object~$B$ of~$\scat{B}$ the induced natural transformation
\[
	γ P_B \colon X ∘ P_B \To X' ∘ P_B \,.
\]
The two functors~$X ∘ P_B$ and~$X' ∘ P_B$ are diagrams of the same shape~$F \comma B$ in~$\cat{S}$.
The natural transformation~$γ P_B$ therefore induces a morphism
\[
	\Lan_F(γ)_B \colon \Lan_F(X)(B) \to \Lan_F(X')(B)
\]
that is unique with the property
\[
	\Lan_F(γ)_B ∘ i^{X, B}_M
	=
	i^{X',\, B}_M ∘ (γ P_B)_M
\]
for every object~$M$ of~$F \comma B$.
This equality can be slightly rewritten as
\[
	\Lan_F(γ)_B ∘ i^{X, B}_{(A, h)}
	=
	i^{X',\, B}_{(A, h)} ∘ γ_A
\]
for every object~$(A, h)$ of~$F \comma B$.

The resulting transformation~$\Lan_F(γ)$ from~$\Lan_F(X)$ to~$\Lan_F(X')$ is again natural.
To show this, we have to prove that for every morphism
\[
	g \colon X \to X'
\]
in~$\scat{B}$ the following diagram commutes:
\[
	\begin{tikzcd}[row sep = large, column sep = huge]
		\Lan_F(X)(B)
		\arrow{r}[above]{\Lan_F(X)(g)}
		\arrow{d}[left]{\Lan_F(γ)_B}
		&
		\Lan_F(X)(B')
		\arrow{d}[right]{\Lan_F(γ)_{B'}}
		\\
		\Lan_F(X')(B)
		\arrow{r}[above]{\Lan_F(X')(g)}
		&
		\Lan_F(X')(B')
	\end{tikzcd}
\]
We thus need to prove the equality
\[
	\Lan_F(γ)_{B'} ∘ \Lan_F(X)(g)
	=
	\Lan_F(X')(g) ∘ \Lan_F(γ)_B \,.
\]
For this, it suffices to show for every object~$(A, h)$ of~$F \comma B$ the equality
\[
	\Lan_F(γ)_{B'} ∘ \Lan_F(X)(g) ∘ i^{X, B}_{(A, h)}
	=
	\Lan_F(X')(g) ∘ \Lan_F(γ)_B ∘ i^{X, B}_{(A, h)} \,.
\]
This equality holds because
\begin{align*}
	{}&
	\Lan_F(γ)_{B'} ∘ \Lan_F(X)(g) ∘ i^{X, B}_{(A, h)}
	\\
	={}&
	\Lan_F(γ)_{B'} ∘ i^{X, B'}_{(A, g ∘ h)}
	\\
	={}&
	i^{X',\, B'}_{(A, g ∘ h)} ∘ γ_A
	\\
	={}&
	\Lan_F(X')(g) ∘ i^{X', B}_{(A, h)} ∘ γ_A
	\\
	={}&
	\Lan_F(X')(g) ∘ \Lan_F(γ)_B ∘ i^{X, B}_{(A, h)} \,.
\end{align*}

We have thus constructed the action of~$\Lan_F$ on both objects and morphisms of~$[\scat{A}, \cat{S}]$.
These actions define a functor from~$[\scat{A}, \cat{S}]$ to~$[\scat{B}, \cat{S}]$, as we will now check:
\begin{itemize}

	\item
		Let~$X$ be a functor from~$\scat{A}$ to~$\cat{S}$.
		We have
		\[
			( \id_{\Lan_F(X)} )_B ∘ i^{X, B}_{(A, h)}
			=
			\id_{\Lan_F(X)(B)} ∘ i^{X, B}_{(A, h)}
			=
			i^{X, B}_{(A, h)}
			=
			i^{X, B} ∘ \id_{X(A)}
			=
			i^{X, B} ∘ (\id_X)_A
		\]
		for every object~$B$ of~$\scat{B}$ and every object~$(A, h)$ of~$F \comma B$.
		This shows that the natural transformation~$\id_{\Lan_F(X)}$ satisfies the defining property of~$\Lan_F(\id_X)$, whence
		\[
			\Lan_F(\id_X) = \id_{\Lan_F(X)} \,.
		\]

	\item
		Let~$X$,~$X'$ and~$X''$ be functors from~$\scat{A}$ to~$\cat{S}$ and let
		\[
			γ \colon X \To X' \,,
			\quad
			γ' \colon X' \To X''
		\]
		be two composable natural transformations.
		We have
		\begin{align*}
			( \Lan_F(γ') ∘ \Lan_F(γ) )_B ∘ i^{X, B}_{(A, h)}
			&=
			\Lan_F(γ')_B ∘ \Lan_F(γ)_B ∘ i^{X, B}_{(A, h)}
			\\
			&=
			\Lan_F(γ)'_B ∘ i^{X',\, B}_{(A, h)} ∘ γ_A
			\\
			&=
			i^{X'',\, B}_{(A, h)} ∘ γ'_A ∘ γ_A
			\\
			&=
			i^{X'',\, B}_{(A, h)} ∘ (γ' ∘ γ)_A
			\\
			&=
			\Lan_F(γ' ∘ γ)_B ∘ i^{X, B}_{(A, h)}
		\end{align*}
		for every object~$B$ of~$\scat{B}$ and every object~$(A, h)$ of~$F \comma B$.
		This shows that the composite~$\Lan_F(γ') ∘ \Lan_F(γ)$ satisfies the defining property of the natural transformation~$\Lan_F(γ' ∘ γ)$, whence
		\[
			\Lan_F(γ' ∘ γ) = \Lan_F(γ') ∘ \Lan_F(γ) \,.
		\]

\end{itemize}
We have thus overall constructed a functor~$\Lan_F$ from~$[\scat{A}, \cat{S}]$ to~$[\scat{B}, \cat{S}]$.

We have seen in the previous part of this exercise that we have for every functor~$X$ from~$\scat{A}$ to~$\cat{S}$ a natural transformation
\[
	λ^X \colon X \To \Lan_F(X) ∘ F \,,
\]
so that for every functor~$Y$ from~$\scat{B}$ to~$\cat{S}$, the map
\begin{equation}
	\label{bijectivity for adjointness}
	[\scat{B}, \cat{S}](\Lan_F(X), Y)
	\to
	[\scat{A}, \cat{S}](X, Y ∘ F) \,,
	\quad
	β
	\mapsto
	β F ∘ λ^X
\end{equation}
is bijective.
We observe that~$λ^X$ is natural in~$X$.
To see this, we have to check that for every two functors~$X$ and~$X'$ from~$\scat{A}$ to~$\cat{S}$ and every natural transformation
\[
	γ \colon X \to X'
\]
the following diagram commutes:
\[
	\begin{tikzcd}[row sep = large, column sep = huge]
		X
		\arrow[Rightarrow]{r}[above]{γ}
		\arrow[Rightarrow]{d}[left]{λ^X}
		&
		X'
		\arrow[Rightarrow]{d}[right]{λ^{X'}}
		\\
		\Lan_F(X) ∘ F
		\arrow[Rightarrow]{r}[above]{\Lan_F(γ) F}
		&
		\Lan_F(X') ∘ F
	\end{tikzcd}
\]
We thus have to show that for every object~$A$ of~$\scat{A}$ the following diagram commutes:
\[
	\begin{tikzcd}[row sep = large, column sep = 6em]
		X(A)
		\arrow{r}[above]{γ_A}
		\arrow{d}[left]{(λ^X)_A}
		&
		X'(A)
		\arrow{d}[right]{(λ^{X'})_A}
		\\
		\Lan_F(X)(F(A))
		\arrow{r}[above]{(\Lan_F(γ) F)_A}
		&
		\Lan_F(X')(F(A))
	\end{tikzcd}
\]
This commutativity hold because
\[
	(\Lan_F(γ) ∘ F)_A ∘ (λ^X)_A
	=
	\Lan_F(γ)_{F(A)} ∘ i^{X, F(X)}_{(A, \id_{F(X)})}
	=
	i^{X', F(X)}_{(A, \id_{F(X)})} ∘ γ_A
	=
	(λ^{X'})_A ∘ γ_A \,.
\]

Let~$η$ be the natural transformation with components~$λ^X$, i.e., the natural transformation from~$\Id_{[\scat{A}, \cat{S}]}$ to~$\Lan_F ∘ F^*$ with~$η_X = λ^X$ for every object~$X$ of~$[\scat{A}, \cat{S}]$.
(We denote here by~$F^*$ the functor induced by~$F$ from~$[\scat{B}, \cat{S}]$ to~$[\scat{A}, \cat{S}]$.)
The bijectivity of~\eqref{bijectivity for adjointness} tells us that for every object~$X$ of~$[\scat{A}, \cat{S}]$, the morphism~$η_X = λ^X$ is an initial object of~$X \comma F^*$.
It follows from Theorem~2.3.6 and its proof, that~$\Lan_F$ is left adjoint to~$F^*$, and that~$η$ is the unit of one such adjunction.



\subsubsection{}

The category~$\Set$ admits both small limits and small colimits.
It follows that for every small category~$\scat{A}$, the functor category~$[\scat{A}, \Set]$ also admits both small limits and small colimits.
We find that for every other small category~$\scat{B}$ and every functor~$F$ from~$\scat{A}$ to~$\scat{B}$, the functor
\[
	F_* \colon [\scat{B}, \Set] \to [\scat{A}, \Set]
\]
admits both a left adjoint and a right adjoint.

Given any two groups~$G$ and~$H$ and a homomorphism of groups~$φ$ from~$G$ to~$H$, we may regard~$G$ and~$H$ as small one-object categories, and the homomorphism~$φ$ as a functor~$Φ$ from~$G$ to~$H$.
The functor categories~$[G, \Set,$ and~$[H, \Set]$ are isomorphic to the categories of~\sets{$G$} and~\sets{$H$} respectively, and the functor
\[
	Φ^*
	\colon
	[H, \Set] \to [G, \Set]
\]
corresponds to the pull-back functor
\[
	φ^* \colon \GSet{H} \to \GSet{G}
\]
induced by~$φ$.
We have seen above that~$Φ^*$ admits both a left adjoint and a right adjoint.
This means that~$φ^*$ admits both a left adjoint and a right adjoint.

For the trivial group~$1$, the category~$\GSet{1}$ is in turn isomorphic to~$\Set$.
By choosing the group~$H$ as trivial, we see that the functor
\[
	\Set \to \GSet{G}
\]
that regards every set as a trivial~\set{$G$} admits both a left adjoint and a right adjoint.
Alternatively, by choosing~$G$ as trivial, we see that the forgetful functor
\[
	\GSet{H} \to \Set
\]
admits both a left adjoint and a right adjoint.
