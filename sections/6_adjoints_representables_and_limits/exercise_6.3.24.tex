\subsection{}



\subsubsection{}

We consider for every natural number~$n$ the set~$W_n ≔ (A × \{1, -1\})^n$, and the evaluation map
\[
	e_n
	\colon
	W_n
	\to
	G \,,
	\quad
	((a_1, ε_1), \dotsc, (a_n, ε_n))
	\mapsto
	a_1^{ε_1} \dotsm a_n^{ε_n} \,.
\]
For the set~$W ≔ ∑_{n ∈ ℕ} W_n$, the maps~$e_n$assemble into a single map
\[
	e \colon W \to G \,.
\]
The subgroup~$H$ of~$G$ generated by~$A$ is precisely the image of the map~$e$.
We therefore find that
\[
	\card{H} ≤ \card{W} \,.
\]
We show in the following that~$\card{W} ≤ \max {} \{ \card{ℕ}, \card{A} \}$.
For this, we distinguish between two cases.
\begin{itemize}

	\item
		Suppose that the set~$A$ is finite.
		It then follows that each set~$W_n$ is again finite.
		The set~$W = ∑_{n ∈ ℕ} W_n$ is therefore countable (either finite or infinite), so that~$\card{W} = \card{ℕ}$.

	\item
		Suppose that the set~$A$ is infinite.
		It then holds that
		\[
			\card{A × \{1, -1\}}
			=
			\card{A} ⋅ \card{\{1, -1\}}
			=
			\card{A} ⋅ 2
			=
			\card{A} + \card{A}
			=
			\card{A} \,,
		\]
		and it follows that~$\card{W_n} = \card{A}^n = \card{A}$.
		It further follows that
		\[
			\card{W}
			=
			\sum_{n ∈ ℕ} {} \card{W_n}
			=
			\sum_{n ∈ ℕ} {} \card{A}
			=
			\card{A} \,.
		\]

\end{itemize}



\subsubsection{}

For every set~$T$ let~$G(T)$ be the set of group structures~$T$.
This collection is indeed a set, since it is a subset of~$\Set(T × T, T)$.

Let~$\mathscr{G}$ be the class of groups who are of cardinality at most~$\card{S}$.
Every subset~$T$ of~$S$ and every element of~$G(T)$ results in a group whose underlying set is~$T$.
We have in this way a map
\[
	\sum_{T ∈ \Power(S)} G(T)
	\to
	\mathscr{G} \,,
\]
which further descends to a map
\[
	\sum_{T ∈ \Power(S)} G(T)
	\to
	\mathscr{G} / {≅} \,.
\]
Every group contained in~$\cat{G}$ is isomorphic to a group whose underlying set is a subset of~$S$, whence this second map is surjective.
The left-hand side is a set, so it follows that the right-hand side is also one.



\subsubsection{}

Let~$(G, g)$ be an object on~$A \comma U$.
This means that~$G$ is a group and~$g$ is a set-theoretic map from~$A$ to~$U(G)$.
Let~$H$ be the subgroup of~$G$ generated by the image of~$g$, i.e., the subgroup of~$G$ generated by the family~$(g(a))_{a ∈ A}$.
The map~$g$ restricts to a set-theoretic map~$h$ from~$A$ to~$U(H)$, resulting in the object~$(H, h)$ of~$A \comma G$.
The inclusion map~$i$ from~$H$ to~$G$ is a homomorphism of groups that makes the diagram
\[
	\begin{tikzcd}[column sep = normal]
		U(H)
		\arrow{rr}[above]{U(i)}
		&
		{}
		&
		U(G)
		\\
		{}
		&
		A
		\arrow{ul}[below left]{h}
		\arrow{ur}[below right]{g}
		&
		{}
	\end{tikzcd}
\]
commute.
In other words,~$i$ is a morphism from~$(H, h)$ to~$(G, g)$ in~$A \comma U$.

Let~$\mathbf{S}''$ be the class of objects~$(H, h)$ of~$A \comma G$ for which the group~$H$ is generated by the image of~$h$.
We have seen above that the class~$\mathbf{S}''$ is weakly initial in~$A \comma U$.

We find from part~(a) of this exercise that for every object~$(H, h)$ of~$\mathbf{S}''$, the cardinality of the group~$H$ is at most~$\max {} \{ \card{ℕ}, \card{A} \}$.

Let~$S$ be a set of cardinality~$\max {} \{ \card{ℕ}, \card{A} \}$, and let~$\mathbf{S}'$ be the class of all those objects~$(G, g)$ of~$A \comma U$ for which the group~$G$ has at most cardinality~$\card{S}$.
The class~$\mathbf{S}'$ contains the class~$\mathbf{S}''$, and is therefore again weakly initial in~$A \comma U$.

Let~$\mathbf{S}$ be the set of all those objects~$(G, g)$ of~$A \comma U$ for which the underlying set of the group~$G$ is a subset of~$S$.
Every object of the class~$\mathbf{S}'$ is isomorphic to some object of the class~$\mathbf{S}$.%
\footnote{
	We use here a similar argument as in part~(b).
	Instead, we could directly use part~(b) as follows:
	\begin{quote}
		By part~(b), the collection of isomorphism classes of objects of~$\mathbf{S}'$ is small.
		We can therefore choose~$\mathbf{S}$ as a set of representatives for the isomorphism classes of objects of~$\mathbf{S}'$.
	\end{quote}
	We haven’t chosen this approach, so that we can avoid choosing representatives.
}
The set~$\mathbf{S}$ is therefore again weakly initial in~$A \comma U$.



\subsubsection{}

The category~$\Grp$ is locally small.
We know from part~(c) of this exercise that for every set~$A$, the comma category~$A \comma G$ admits a weakly initial set.
We know that the category~$\Set$ is complete, and we know from Exercise~5.3.11, part~(a) that the functor~$U$ creates limits.
It follows that~$\Grp$ is complete and that~$U$ preserves limits.

It follows from the general adjoint functor theorem that the functor~$U$ admits a left adjoint.
