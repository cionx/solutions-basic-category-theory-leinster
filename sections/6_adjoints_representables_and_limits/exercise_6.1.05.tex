\subsection{}

The category of diagrams of shape~$\scat{I}$ in~$\cat{A}$ -- i.e., the functor category~$[\scat{I}, \cat{A}]$ -- is isomorphic to the product category~$\cat{A} × \cat{A}$.
For every object~$A$ of~$\cat{A}$, the object of~$\cat{A} × \cat{A}$ corresponding to the constant functor~$Δ(A)$ is the pair~$(A, A)$.

A cone on an object~$(A, B)$ of~$\cat{A} × \cat{A}$ is an object~$Q$ of~$\cat{A}$ together with two morphisms
\[
	q_1 \colon Q \to A \,,
	\quad
	q_2 \colon Q \to B \,.
\]
A limit cone is thus an object~$P$ of~$\cat{A}$ together with two morphisms
\[
	p_1 \colon P \to A \,,
	\quad
	p_2 \colon P \to B
\]
such that for every other cone~$(Q, q_1, q_2)$ as above, there exists a unique morphism~$f$ from~$Q$ to~$P$ with~$p_1 ∘ f = q_1$ and~$p_2 ∘ f = q_2$.
In other words,~$(P, p_1, p_2)$ is precisely a product of the two objects~$A$ and~$B$.

Proposition~6.1.4 gives us the functoriality of the product~$(\ph) × (\ph)$ from Exercise~5.3.8
It also tells us that this product functor
\[
	(\ph) × (\ph) \colon \cat{A} × \cat{A} \to \cat{A}
\]
is right adjoint to the diagonal functor
\[
	Δ \colon \cat{A} \to \cat{A} × \cat{A} \,.
\]
(We had already seen this for~$\cat{A} = \Set$ as part of Exercise~3.1.1.)
