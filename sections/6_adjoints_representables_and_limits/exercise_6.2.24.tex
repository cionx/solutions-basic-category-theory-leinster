\subsection{}

We build up to the case of an arbitrary small category~$\scat{A}$ by considering first some special cases.

\begin{remark}
	This exercise is a massive waste of time and the following solution is not proofread in the slightest.
\end{remark}


\subsubsection*{The category~$\scat{A}$ is the one-object category~$\One$}

We consider first the special case that~$\scat{A}$ is the one-object category~$\One$, whose unique object we denote by~$\ast$.
The category~$[\One^{\op}, \Set]$ is isomorphic to~$\Set$ via the evaluation functor at~$\ast$.
A presheaf~$X$ on~$\One$ corresponds under this isomorphism to the set~$S ≔ X(\ast)$.
This isomorphism between~$[\One^{\op}, \Set]$ and~$\Set$ induces an isomorphism
\[
	[\One^{\op}, \Set] / X ≅ \Set / S \,.
\]
We want to describe the category~$\Set / S$ as a category of presheaves a suitable small category.

We consider first the case that~$S = \{ 0, 1 \}$.
An object of~$\Set / S$ is a pair~$(A, σ)$ consisting of a set~$A$ and a function~$σ$ from~$A$ to~$S$.
We know that maps from~$A$ to~$S$ can be identified with subsets of~$A$, with the function~$σ$ corresponding to the preimage~$σ^{-1}(1)$.
A morphism from~$(A, σ)$ to~$(A', σ')$ in~$\Set / S$ is a map~$f$ from~$A$ to~$A'$ subject to the commutativity of the following diagram:
\[
	\begin{tikzcd}[row sep = large]
		A
		\arrow{rr}[above]{f}
		\arrow{dr}[below left]{σ}
		&
		{}
		&
		A'
		\arrow{dl}[below right]{σ'}
		\\
		{}
		&
		S
		&
		{}
	\end{tikzcd}
\]
The commutativity of this diagram can equivalently be expressed via the sets~$σ^{-1}(1)$ and~$(σ')^{-1}(1)$ as
\[
	f^{-1}( (σ')^{-1}(1) ) = σ^{-1}(1) \,.
\]
We hence find that the category~$\Set / S$ is isomorphic to the following auxiliary category~$\cat{S}$:
Objects of~$\cat{S}$ are pairs~$(A, A_1)$ consisting of a set~$A$ and a subset~$A_1$ of~$A$.
A morphism from~$(A, A_1)$ to~$(A', A'_1)$ in~$\cat{S}$ is a map~$f$ from~$A$ to~$A'$ with~$f^{-1}(A'_1) = A_1$.

We can generalize this description of~$\Set / S$ to the case that~$S$ is an arbitrary set.
We find that~$\Set / S$ is isomorphic to the following auxiliary category~$\cat{P}_S$:
\begin{itemize*}

	\item
		Objects of~$\cat{P}_S$ are pairs~$(A, (A_s)_s)$ consisting of a set~$A$ and a family~$(A_s)_s$ of subsets~$A_s$ of~$A$, indexed over the elements~$s$ of~$S$, such that~$A$ is the disjoint union of the sets~$A_s$.

	\item
		A morphism from~$(A, (A_s)_s)$ to~$(A', (A'_s)_s)$ in~$\cat{P}_S$ is a map~$f$ from~$A$ to~$A'$ such that
		\[
			A_s = f^{-1}(A'_s)
			\qquad
			\text{for every~$s ∈ S$} \,.
		\]
		The set~$A$ is the disjoint union of the sets~$A_s$ but at the same time also the disjoint union of the sets~$f^{-1}(A'_s)$.
		Therefore,
		\begin{align*}
			{}&
			\text{$A_s = f^{-1}(A'_s)$ for every~$s ∈ S$}
			\\
			\iff{}&
			\text{$A_s ⊆ f^{-1}(A'_s)$ for every~$s ∈ S$}
			\\
			\iff{}&
			\text{$f(A_s) ⊆ A'_s$ for every~$s ∈ S$} \,.
		\end{align*}
		This means that the morphisms from~$(A, (A_s)_s)$ to~$(A', (A'_s)_s)$ in~$\cat{P}_S$ are precisely those maps from~$A$ to~$A'$ that restrict for every index~$s$ to a map from~$A_s$ to~$A'_s$.

\end{itemize*}

We have a functor
\[
	\tilde{F} \colon \cat{P}_S \to \Set^S
\]
that splits apart an object~$(A, (A_s)_s)$ into its subsets~$A_s$.
More explicitly, this functor is given on every object~$(A, (A_s)_s)$ of~$\cat{P}_S$ by
\[
	\tilde{F}\Bigl( (A, (A_s)_s) \Bigr) = (A_s)_s \,,
\]
and on every morphism~$f$ from~$(A, (A_s)_s)$ to~$(A', (A'_s)_s)$ by
\[
	\tilde{F}(f) = \left( f_s \right)_s
	\qquad\text{with}\qquad
	f_s = \restrict[\big]{f}{A_s}[B_s] \,.
\]
We have also a functor
\[
	\tilde{G} \colon \Set^S \to \cat{P}_S
\]
given by the disjoint union~$∐_{s ∈ S} (\ph)$.
More explicitly, this functor is given on every object~$(A_s)_s$ of~$\Set^S$ by
\[
	\tilde{G}\Bigl( (A_s)_s \Bigr) = \Biggl( ∐_{s ∈ S} A_s, (\tilde{A}_s)_s \Biggr)
\]
where~$\tilde{A}_t$ is the image of~$A_t$ in~$∐_{s ∈ S} A_s$, and it is given on morphisms by
\[
	\tilde{G}\Bigl( (f_s)_s \Bigr) = ∐_{s ∈ S} f_s \,.
\]
These two functors~$\tilde{F}$ and~$\tilde{G}$ satisfy
\[
	\tilde{G} ∘ \tilde{F} ≅ \Id_{\cat{P}_S} \,,
	\quad
	\tilde{F} ∘ \tilde{G} ≅ \Id_{\Set^S} \,,
\]
and thus form an equivalence between the two categories~$\cat{P}_S$ and~$\Set^S$.

Together with the isomorphism between~$\Set / S$ and~$\cat{P}_S$ we arrive at an equivalence of categories
\[
	\Set / S ≃ \Set^S \,.
\]
We can furthermore identify~$\Set^S$ with the functor category~$[S^{\op}, \Set]$ when considering the set~$S$ as a discrete category.
In this way, we arrive overall at the chain of equivalences
\[
	[\One^{\op}, \Set] / X
	≅
	\Set / S
	≅
	\cat{P}_S
	≃
	\Set^S
	≅
	[S^{\op}, \Set] \,,
\]
where the set~$S$ corresponds to the presheaf~$X$ via~$S = X(\ast)$.

Let~$F$ be the composite of the above equivalences.
We can work out an explicit description of~$F$ as follows.
\begin{enumerate*}

	\item
		An object of~$[\One^{\op}, \Set] / X$ is a pair~$(Y, α)$ consisting of a presheaf~$Y$ on~$\One$ and a natural transformation~$α$ from~$Y$ to~$X$.

	\item
		Under the isomorphism~$[\One^{\op}, \Set] / X ≅ \Set / S$, the pair~$(Y, α)$ corresponds to the pair~$(Y(\ast), α_\ast)$, consisting of the set~$Y(\ast)$ and the map~$α_\ast$ from~$Y(\ast)$ to~$X(\ast) = S$.

	\item
		Under the equivalence~$\Set / S ≃ \cat{P}_S$, this pair~$(Y(\ast), α_\ast)$ further corresponds to the pair~$(Y(\ast), (α_\ast^{-1}(s))_s)$.

	\item
		Under the equivalence~$\cat{P}_S ≃ \Set^S$ we arrive at the object~$(α_\ast^{-1}(s))_s$.

	\item
		Under the isomorphism~$\Set^S ≅ [S^{\op}, \Set]$ we arrive at the presheaf~$F(Y, α)$ on~$S$ (where we view~$S$ as a discrete category).%
		\footnote{
			We image of~$(Y, α)$ under~$F$ ought to be denoted by~$F( (Y, α) )$, but we will use the abbreviated notation~$F(Y, α)$ for better readability.
		}
		It is given by
		\[
			F(Y, α)(s) = α_\ast^{-1}(s)
		\]
		for every element~$s$ of~$S$.
		(We don’t need to worry about the action of~$F(Y, α)$ on morphisms of~$S$ because the category~$S$ is discrete.)

\end{enumerate*}

We can similarly figure out how~$F$ acts on morphisms.
A morphism in~$[\One^{\op}, \Set]$ from~$(Y, α)$ to~$(Y', α')$ is a natural transformation~$β$ from~$Y$ to~$Y'$ with~$α' ∘ β = α$, i.e., such that the following diagram commutes:
\[
	\begin{tikzcd}[row sep = large]
		Y
		\arrow[Rightarrow]{rr}[above]{β}
		\arrow[Rightarrow]{dr}[below left]{α}
		&
		{}
		&
		Y'
		\arrow[Rightarrow]{dl}[below right]{α'}
		\\
		{}
		&
		X
		&
		{}
	\end{tikzcd}
\]
The resulting natural transformation~$F(β)$ from~$F(Y, α)$ to~$F(Y', α')$ is given by
\[
	F(β) = ( F(β)_s )_s
\]
where for every index~$s$, the component~$F(β)_s$ is the restriction of the map~$β_\ast$ to a map from~$F(Y, α)(s) = α_\ast^{-1}(s)$ to~$F(Y', α')(s) = (α'_\ast)^{-1}(s)$.



\subsubsection*{The category $\scat{A}$ is discrete}

We consider now instead of an arbitrary small category~$\scat{A}$ a small category~$\scat{D}$ that is discrete.
We may think about the category~$\scat{D}$ as the disjoint union of copies of~$\One$, with one copy for each object of~$\cat{D}$.
A presheaf~$X$ on~$\scat{D}$ is then a sum of presheaves~$X_d$ on~$\One$, given by~$X_d(\ast) = X(d)$ for each object~$d$ of~$\scat{D}$.
We can see through the power of abstract nonsense that
\begingroup
\allowdisplaybreaks
\begin{align*}
	[\scat{D}^{\op}, \Set] / X
	&=
	[\scat{D}, \Set] / X
	\\[0.5em]
	&≅
	\left[ ∐_{d ∈ \Ob(\scat{D})} \One, \;\Set \right] / X
	\\[0.5em]
	&≅
	\left( ∏_{d ∈ \Ob(\scat{D})} {} [\One, \Set] \right) / (X_d)_d
	\\[0.5em]
	&≅
	∏_{d ∈ \Ob(\scat{D})} {} \Bigl( [\One, \Set] / X_d \Bigr)
	\\[0.5em]
	&=
	∏_{d ∈ \Ob(\scat{D})} {} \Bigl( [\One^{\op}, \Set] / X_d \Bigr)
	\\[0.5em]
	&≃
	∏_{d ∈ \Ob(\scat{D})} {} [(\scat{B}_d)^{\op}, \Set]
	\\[0.5em]
	&≃
	\left[ ∐_{d ∈ \Ob(\scat{D})} (\scat{B}_d)^{\op}, \; \Set \right]
	\\[0.5em]
	&≅
	\left[ \left( ∐_{d ∈ \Ob(\scat{D})} \scat{B}_d \right)^{\op}, \; \Set \right]
\end{align*}
\endgroup
for some suitable small categories~$\scat{B}_d$.
We have seen above that the categories~$\scat{B}_d$ can be chosen as~$\scat{B}_d = X(d)$, viewed as discrete categories.
The desired small category~$\scat{B}$ with
\[
	[ \scat{D}^{\op}, \Set ] / X ≃ [\scat{B}^{\op}, \Set]
\]
should therefore be choosable as the set~$∐_{d ∈ \Ob(\scat{D})} X(d)$ viewed as a discrete category.

Let us make this more explicit.
We consider this set
\[
	B ≔ ∐_{d ∈ \Ob(\scat{D})} X(d)
\]
as a discrete category.
Let~$(Y, α)$ be an object of the category~$[\scat{D}^{\op}, \Set] / X$.
This means that~$Y$ is a presheaf on~$\scat{D}$ and that~$α$ is a natural transformation from~$Y$ to~$X$:
\[
	α \colon Y \To X \,.
\]
We can define a presheaf~$F(Y, α)$ on~$B$ via
\[
	F(Y, α)(d, x)
	=
	α_d^{-1}(x)
\]
for every element~$(d, x)$ of~$B$.%
\footnote{
	Recall that for a family of sets~$(S_i)_{i ∈ I}$, the elements of~$∐_{i ∈ I} S_i$ can be denoted as pairs~$(i, s)$ where~$i$ ranges through the index set~$I$ and~$s$ ranges through the associated set~$S_i$.
}

A morphism from~$(Y, α)$ to~$(Y', α')$ in~$[\scat{D}^{\op}, \Set] / X$ is a natural transformation~$β$ from~$Y$ to~$Y'$ with~$α' ∘ β = α$, i.e., such that the diagram
\[
	\begin{tikzcd}[row sep = large]
		Y
		\arrow[Rightarrow]{rr}[above]{β}
		\arrow[Rightarrow]{dr}[below left]{α}
		&
		{}
		&
		Y'
		\arrow[Rightarrow]{dl}[below right]{α'}
		\\
		{}
		&
		X
		&
		{}
	\end{tikzcd}
\]
commutes.
This means that we have the following commutative diagram for every object~$d$ of~$D$:
\[
	\begin{tikzcd}[row sep = large]
		Y(d)
		\arrow{rr}[above]{β_d}
		\arrow{dr}[below left]{α_d}
		&
		{}
		&
		Y'(d)
		\arrow{dl}[below right]{α'_d}
		\\
		{}
		&
		X(d)
		&
		{}
	\end{tikzcd}
\]
The map~$β_d$ therefore restrict for every element~$x$ of~$X(d)$ to a map between fibres
\[
	F(β)_{(d, x)} \colon α_d^{-1}(x) \to (α'_d)^{-1}(x) \,.
\]
These maps, with~$(d, x)$ ranging through~$B$, define a natural transformation~$F(β)$ from~$F(Y, α)$ to~$F(Y', α')$.
(We don’t need to worry about the naturality of~$F(β)$ because the category~$B$ is discrete.)

Suppose on the other hand that we are given a presheaf~$Z$ of~$B$.
We then construct a corresponding object~$G(B)$ of~$[\scat{D}^{\op}, \Set] / X$.
This object~$G(Z)$ needs to be a pair~$(G_0(Z), G_1(Z))$ consisting of a presheaf~$G_0(Z)$ on~$\scat{D}$ and a natural transformation~$G_1(Z)$ from~$G_0(Z)$ to~$X$.

The presheaf~$G_0$ is constructed as
\[
	G_0(Z)(d) ≔ ∐_{x ∈ X(d)} Z(d, x)
\]
for every object~$d$ of~$\scat{D}$.
We don’t need to worry about the action of~$G_0(Z)$ on morphisms because the category~$\scat{D}$ is discrete.

We will define the natural transformation~$G_1(Z)$ via its components~$G_1(Z)_d$, where the index~$d$ ranges through the objects of~$\scat{D}$.
The component~$G_1(Z)_d$ is the map from~$G_0(Z)(d) = ∐_{x ∈ X(d)} Z(d, x)$ to~$X(d)$ which maps all of~$Z(d, x)$ onto the value~$x$ of~$X(d)$.

These two functors~$F$ and~$G$ satisfy
\[
	G ∘ F ≅ \Id_{[\scat{D}^{\op}, \Set] / X} \,,
	\quad
	F ∘ G ≅ \Id_{[B^{\op}, \Set]} \,.
\]
They hence give an equivalence of categories between~$[\scat{D}^{\op}, \Set] / X$ and~$[B^{\op}, \Set]$.



\subsubsection*{The general case: start}

Let now~$\scat{A}$ be a small category and let~$X$ be a presheaf on~$X$.
To find the desired category~$\scat{B}$, we want a small category whose set of objects is given by~$∐_{A ∈ \Ob(\scat{A})} X(A)$, but who also keeps track of the morphisms in~$\scat{A}$.
We consider for this the category of elements of~$X$, i.e., the category~$\Elements(X)$.

We define first a functor~$F$ from~$[\scat{A}^{\op}, \Set] / X$ to~$[\Elements(X)^{\op}, \Set]$, then a functor~$G$ from~$[\Elements(X)^{\op}, \Set]$ to~$[\scat{A}^{\op}, \Set] / X$, and then we show that these two functors are mutually inverse up to isomorphism.

\subsubsection*{The general case: construction of $F$}

To construct the functor~$F$ on objects, let~$(Y, α)$ be an object of~$[\scat{A}^{\op}, \Set] / X$.
This means that~$Y$ is a presheaf~$Y$ on~$\scat{A}$ and that~$α$ is a natural transformation~$α$ from~$Y$ to~$X$.
The desired object~$F(Y, α)$ needs to be a presheaf on~$\Elements(X)$.
\begin{itemize}

	\item
		An object of~$\Elements(X)$ is a pair~$(A, x)$ consisting of an object~$A$ of~$\scat{A}$ and an element~$x$ of the set~$X(A)$.
		We define the set~$F(Y, α)(A, x)$ as
		\[
			F(Y, α)(A, x) ≔ α_A^{-1}(x) \,.
		\]

	\item
		A morphism in~$\Elements(X)$ from an object~$(A, x)$ to an object~$(A', x')$ is a morphism~$f$ from~$A$ to~$A'$ in~$\scat{A}$ with~$X(f)(x') = x$.
		We know from the naturality of~$α$ that the resulting diagram
		\[
			\begin{tikzcd}[sep = large]
				Y(A')
				\arrow{r}[above]{Y(f)}
				\arrow{d}[left]{α_{A'}}
				&
				Y(A)
				\arrow{d}[right]{α_A}
				\\
				X(A')
				\arrow{r}[above]{X(f)}
				&
				X(A)
			\end{tikzcd}
		\]
		commutes.
		It follows from this commutativity and the identity~$X(f)(x') = x$ that the map~$Y(f)$ restricts to a map from the preimage~$α_{A'}^{-1}(x')$ to the preimage~$α_A^{-1}(x)$.
		We denote this restriction by~$F(Y, α)(f)$, so that
		\[
			F(Y, α)(f) \colon F(Y, α)(A', x') \to F(Y, α)(A, x) \,.
		\]

\end{itemize}

We have now constructed the action of~$F(Y, α)$ on objects of~$\Elements(X)$ and on morphisms of~$\Elements(X)$.
We now check that~$F(Y, α)$ is a contravariant functor from~$\Elements(X)$ to~$\Set$.
\begin{itemize}

	\item
		Let~$(A, x)$ be an object on~$\Elements(X)$.
		The identity morphism of~$(A, x)$ is the identity morphism of~$A$, whence~$F(Y, α)(\id_{(A, x)})$ is the restriction of~$Y(\id_A)$ to a map from~$α_A^{-1}(x)$ to~$α_A^{-1}(x)$.
		But~$Y(\id_A)$ is~$\id_{Y(A)}$, whence this restriction is the identity map of~$α_A^{-1}(x)$, and thus the identity map of~$F(Y, α)(A, x)$.
		This shows that
		\[
			F(Y, α)(\id_{A, x})
			=
			\id_{F(Y, α)(A, x)} \,.
		\]

	\item
		Let
		\[
			f \colon (A, x) \to (A', x') \,,
			\quad
			g \colon (A', x') \to (A'', x'')
		\]
		be two composable morphisms in~$\Elements(X)$.
		The map~$F(Y, α)(f)$ is the restriction of~$Y(f)$ to a map from~$α_{A'}^{-1}(x')$ to~$α_A^{-1}(x)$, and the map~$F(Y, α)(g)$ is similarly the restriction of~$Y(g)$ to a map from~$α_{A''}^{-1}(x'')$ to~$α_{A'}^{-1}(x')$.
		The composite
		\[
			F(Y, α)(f) ∘ F(Y, α)(g)
		\]
		is therefore the restriction of~$Y(f) ∘ Y(g)$ to a map from~$α_{A''}^{-1}(x'')$ to~$α_A^{-1}(x)$.
		But
		\[
			Y(f) ∘ Y(g) = Y(g ∘ f)
		\]
		 by the contravariant functoriality of~$Y$, and the restriction of~$Y(g ∘ f)$ to a map from~$α_{A''}^{-1}(x'')$ to~$α_A^{-1}(x)$ is precisely~$F(Y, α)(g ∘ f)$.
		We have thus shown that
		\[
			F(Y, α)(f) ∘ F(Y, α)(g)
			=
			F(Y, α)(g ∘ f) \,.
		\]

\end{itemize}
This shows that~$F(Y, α)$ is indeed a contravariant functor from~$\Elements(A)$ to~$\Set$.

We have thus constructed the desired functor~$F$ on the level of objects.

We proceed by constructing the action of~$F$ on morphisms.
We consider for this a morphism
\[
	β \colon (Y, α) \to (Y', α')
\]
in~$[\scat{A}^{\op}, \Set] / X$.
This means that~$β$ is a natural transformation from~$Y$ to~$Y'$ that makes the diagram
\[
	\begin{tikzcd}[row sep = large]
		Y
		\arrow[Rightarrow]{rr}[above]{β}
		\arrow[Rightarrow]{dr}[below left]{α}
		&
		{}
		&
		Y'
		\arrow[Rightarrow]{dl}[below right]{α'}
		\\
		{}
		&
		X
		&
		{}
	\end{tikzcd}
\]
commute.
For every object~$(A, x)$ of~$\Elements(X)$ we get the following commutative diagram:
\[
	\begin{tikzcd}[row sep = large]
		Y(A)
		\arrow{rr}[above]{β_A}
		\arrow{dr}[below left]{α_A}
		&
		{}
		&
		Y'(A)
		\arrow{dl}[below right]{α'_A}
		\\
		{}
		&
		X(A)
		&
		{}
	\end{tikzcd}
\]
It follows from the commutativity of this diagram that the map~$β_A$ restricts to a map from~$α_A^{-1}(x)$ to~$α_{A'}^{-1}(x')$.
We denote this restriction by~$F(β)_{(A, α)}$.

We note that~$F(β)$ is a natural transformation from~$F(Y, α)$ to~$F(Y', α')$.
Indeed, let
\[
	f \colon (A, x) \to (A', x')
\]
be a morphism in~$\Elements(X)$.
This means that~$f$ is a morphism from~$A$ to~$A'$ in~$\cat{A}$ with~$X(f)(x') = x$.
%It follows that the following diagram commutes, because~$α$,~$α'$ and~$β$ are natural transformations:
%\[
%	\begin{tikzcd}[row sep = large]
%		Y(A)
%		\arrow{rr}[above]{β_A}
%		\arrow{dr}[below left]{α_A}
%		&
%		{}
%		&
%		Y'(A)
%		\arrow{dl}[below right]{α'_A}
%		\\
%		{}
%		&
%		X(A)
%		&
%		{}
%		\\
%		Y(A')
%		\arrow{rr}[above, near start]{β_{A'}}
%		\arrow{dr}[below left]{α_{A'}}
%		\arrow{uu}[left]{Y(f)}
%		&
%		{}
%		&
%		Y'(A')
%		\arrow{dl}[below right]{α'_{A'}}
%		\arrow{uu}[right]{Y'(f)}
%		\\
%		{}
%		&
%		X(A')
%		\arrow[crossing over]{uu}[right, near end]{X(f)}
%		&
%		{}
%	\end{tikzcd}
%\]
It follows from the naturality of~$β$ that we have the following commutative diagram:
\[
	\begin{tikzcd}[sep = large]
		Y(A)
		\arrow{r}[above]{β_A}
		&
		Y'(A)
		\\
		Y(A')
		\arrow{r}[above]{β_{A'}}
		\arrow{u}[left]{Y(f)}
		&
		Y'(A')
		\arrow{u}[right]{Y'(f)}
	\end{tikzcd}
\]
This commutative diagram restricts to the following commutative diagram:
\[
	\begin{tikzcd}[row sep = large]
		F(Y, α)(A, x)
		\arrow[equal]{r}
		&
		α_A^{-1}(x)
		\arrow{r}[above]{F(β)_{(A, x)}}
		&[3em]
		(α')_A^{-1}(x)
		&
		F(Y', α')(A, x)
		\arrow[equal]{l}
		\\
		F(Y, α)(A', x')
		\arrow[equal]{r}
		&
		α_{A'}^{-1}(x')
		\arrow{r}[above]{F(β)_{(A', x')}}
		\arrow{u}[left]{F(Y, α)(f)}
		&
		(α')_{A'}^{-1}(x')
		\arrow{u}[right]{F(Y', α')(f)}
		&
		F(Y', α')(A', x')
		\arrow[equal]{l}
	\end{tikzcd}
\]
The commutativity of this diagram shows that~$F(β)$ is indeed a natural transformation from~$F(Y, α)$ to~$F(Y', α')$.

We have now constructed the action of~$F$ on morphisms.
We need to check that~$F$ is functorial.
\begin{itemize}

	\item
		Let~$(Y, α)$ be an object of~$[\scat{A}^{\op}, \Set] / X$.
		The identity morphism of this object is the identity natural transformation of~$Y$, i.e.,~$\id_Y$.
		It follows for every object~$(A, x)$ of~$\Elements(X)$ that the map~$(\id_Y)_A$ is given by~$\id_{Y(A)}$.
		The restriction of this map to a map from~$F(Y, α)(A, x)$ to~$F(Y, α)(A, x)$ is therefore~$\id_{F(Y, α)(A, x)}$.
		This shows that
		\[
			F(\id_{(Y, α)})_{(A, x)}
			=
			\id_{F(Y, α)(A, x)}
			=
			( \id_{F(Y, α)} )_{(A, x)}
		\]
		for every object~$(A, x)$ of~$\Elements(X)$, and therefore~$F(\id_{(Y, α)}) = \id_{F(Y, α)}$.

	\item
		Let
		\[
			β \colon (Y, α) \to (Y', α') \,,
			\quad
			β' \colon (Y', α') \to (Y'', α'')
		\]
		be two composable morphisms in~$[\scat{A}^{\op}, \Set] / X$.
		Let~$(A, x)$ be an object of~$\Elements(X)$.
		The map
		\[
			F(β)_{(A, x)}
			\colon
			F(Y, α)(A, x)
			\to
			F(Y', α')(A, x)
		\]
		is a restriction of~$β_A$, and the map
		\[
			F(β')_{(A, x)}
			\colon
			F(Y', α')(A, x)
			\to
			F(Y'', α'')(A, x)
		\]
		is similarly a restriction of~$β'_A$.
		The composite
		\[
			F(β')_{(A, x)} ∘ F(β)_{(A, x)}
			\colon
			F(Y, α)(A, x)
			\to
			F(Y'', α'')(A, x)
		\]
		is therefore the restriction of~$β'_A ∘ β_A$.
		But we have~$β'_A ∘ β_A = (β' ∘ β)_A$, and the restriction of~$(β' ∘ β)_A$ to a map from~$F(Y, α)(A, x)$ to~$F(Y'', α'')(A, x)$ is precisely~$F(β' ∘ β)_{(A, x)}$.
		We have therefore found that
		\[
			( F(β') ∘ F(β) )_{(A, x)}
			=
			F(β')_{(A, x)} ∘ F(β)_{(A, x)}
			=
			F(β' ∘ β)_{(A, x)}
		\]
		for every object~$(A, x)$ of~$\Elements(X)$, and thus altogether
		\[
			F(β') ∘ F(β)
			=
			F(β' ∘ β) \,.
		\]
\end{itemize}
This shows the functoriality of~$G$.

\subsubsection*{The general case: construction of $G$}

Let~$Z$ be an object of~$[\Elements(X)^{\op}, \Set]$.
We start by constructing an object~$(G_0(Z), G_1(Z))$ of~$[\scat{A}^{\op}, \Set] / X$.
This objects needs to consist of an element~$G_0(Z)$ of~$[\scat{A}^{\op}, \Set]$ and a natural transformation~$G_1(Z)$ from~$G_0(Z)$ to~$X$.

We start by constructing~$G_0(Z)$, which needs to be a contravariant functor from~$\scat{A}$ to~$\Set$.
\begin{itemize}

	\item
		Let~$A$ be an object of~$\scat{A}$.
		We define the set~$G_0(Z)(A)$ as
		\[
			G_0(Z)(A) ≔ ∐_{x ∈ X(A)} Z(A, x) \,,
		\]
		and denote for every element~$x$ of~$X(A)$ by
		\[
			j_{A, x} \colon Z(A, x) \to G_0(Z)(A)
		\]
		the canonical inclusion map into the~\nth{$x$} summand.

	\item
		Let
		\[
			f \colon A \to A'
		\]
		be a morphism in~$\scat{A}$.
		For every element~$x'$ of~$X(A')$, this morphism~$f$ is then also a morphism
		\[
			f^{[x']} \colon ( A, X(f)(x') ) \to ( A', x' )
		\]
		in~$\Elements(X)$, and induces therefore a map
		\[
			Z(f^{[x']}) \colon Z(A', x') \to Z(A, X(f)(x')) \,.
		\]
		We have hence for every element~$x'$ of~$X(A')$ the map
		\[
			j_{A, X(f)(x')} \circ Z(f^{[x']})
			\colon
			Z(A', x')
			\to
			G_0(Z)(A) \,.
		\]
		These maps can now be bundled together into a map
		\[
			G_0(Z)(f) \colon G_0(Z)(A') \to G_0(Z)(A)
		\]
		such that
		\[
			G_0(Z)(f) ∘ j_{A',\, x'} = j_{A,\, X(f)(x')} ∘ Z(f^{[x']})
		\]
		for every element~$x'$ of~$X(A')$.

\end{itemize}
These assignments are contravariantly functorial from~$\scat{A}$ to~$\Set$:
\begin{itemize}

	\item
		Let~$A$ be an object of~$\scat{A}$.
		We have
		\[
			X(\id_A)(x)
			=
			\id_{X(A)}(x)
			=
			x
		\]
		for every element~$x$ of~$X(A)$, and therefore~$(\id_A)^{[x]} = \id_{(A, x)}$ for every element~$x$ of~$X(A)$.
		It follows that
		\[
			Z( (\id_A)^{[x]} )
			=
			Z( \id_{(A, x)} )
			=
			\id_{Z(A, x)}
		\]
		for every element~$x$ of~$X(A)$.

	\item
		Let
		\[
			f \colon A \to A' \,,
			\quad
			g \colon A' \to A''
		\]
		be two morphisms in~$\scat{A}$.
		The two morphisms
		\[
			G_0(Z)(g ∘ f) \,,
			\quad
			G_0(Z)(f) ∘ G_0(Z)(g)
		\]
		have the same domain and the same codomain.
		To show that they are the same, it suffices to show that for every element~$x''$ of~$X(A'')$, we have
		\[
			G_0(Z)(g ∘ f) ∘ j_{A'',\, x''}
			=
			G_0(Z)(f) ∘ G_0(Z)(g) ∘ j_{A'',\, x''} \,.
		\]

		The two morphisms
		\[
			g^{[x'']}
			\colon
			(A', X(g)(x'')) \to (A'', x'')
		\]
		and
		\[
			f^{[X(g)(x'')]}
			\colon
			(A, X(f)(X(g)(x''))) \to (A', X(g)(x''))
		\]
		compose into a morphism
		\begin{equation}
			\label{ugly composition for ugly exercise}
			g^{[x'']} ∘ f^{[X(g)(x'')]}
			\colon
			(A, X(f)(X(g)(x''))) \to (A'', x'') \,.
		\end{equation}
		We have
		\[
			X(f)(X(g)(x''))
			=
			( X(f) ∘ X(g) )(x'')
			=
			X(g ∘ f)(x'')
		\]
		because~$X$ is contravariantly functorial.
		The composite~\eqref{ugly composition for ugly exercise} is therefore the morphism~$g ∘ f$ in~$\scat{A}$ regarded as a morphism from~$(A, X(g ∘ f)(x''))$ to~$(A'', x'')$ in~$\Elements(X)$.
		In other words, we have
		\[
			g^{[x'']} ∘ f^{[X(g)(x'')]}
			=
			(g ∘ f)^{[x'']} \,.
		\]
		
		It follows that
		\[
			Z( (g ∘ f)^{[x'']} )
			=
			Z( g^{[x'']} ∘ f^{[X(g)(x'')]} )
			=
			Z( f^{[X(g)(x'')]} ) ∘ Z( g^{[x'']} )
		\]
		by the contravariant functoriality of~$Z$.
		It now further follows that
		\begin{align*}
			{}&
			G_0(Z)(f) ∘ G_0(Z)(g) ∘ j_{A'',\, x''}
			\\
			={}&
			G_0(Z)(f) ∘ j_{A',\, X(g)(x'')} ∘ Z( g^{[x'']} )
			\\
			={}&
			j_{A,\, X(f)(X(g)(x''))} ∘ Z( f^{[X(g)(x'')]} ) ∘ Z( g^{[x'']} )
			\\
			={}&
			j_{A,\, X(g ∘ f)(x'')} ∘ Z( (g ∘ f)^{[x'']} )
			\\
			={}&
			G(Z)(g ∘ f) ∘ j_{A'', x''} \,,
		\end{align*}
		which is the equality that we needed to prove.
\end{itemize}
We have thus proven that~$G_0(Z)$ is a contravariant functor from~$\Elements(X)$ to~$\Set$.

We now have to construct a natural transformation from~$G_0(Z)$ to~$X$.
For this, we need to construct for every object~$A$ of~$\scat{A}$ a map
\[
	G_1(Z)_A \colon G_0(Z)(A) \to X(A) \,,
\]
such that for every morphism
\[
	f \colon A \to A'
\]
in~$\scat{A}$, the following square diagram commutes:
\begin{equation}
	\label{required commutativity for naturality of ugly construction}
	\begin{tikzcd}[row sep = large, column sep = huge]
		G_0(Z)(A')
		\arrow{r}[above]{G_0(Z)(f)}
		\arrow{d}[left]{G_1(Z)_{A'}}
		&
		G_0(Z)(A)
		\arrow{d}[right]{G_1(Z)_A}
		\\
		X(A')
		\arrow{r}[above]{X(f)}
		&
		X(A)
	\end{tikzcd}
\end{equation}
We first construct the transformation~$G_1(Z)$, and then check its naturality.
\begin{itemize}

	\item
		We have~$G_0(Z)(A) = ∐_{x ∈ X(A)} Z(A, x)$.
		There hence exists a unique map~$G_1(Z)_A$ from~$G_0(Z)(A)$ to~$X(A)$ such that for element~$x$ of~$X(A)$ the composite~$G_1(Z)_A ∘ j_{A, x}$ is constant with value~$x$.

	\item
		The diagram~\eqref{required commutativity for naturality of ugly construction} commutes because
		\begin{align*}
			{}&
			G_1(Z)_A ∘ G_0(Z)(f) ∘ j_{A',\, x'}
			\\
			={}&
			G_1(Z)_A ∘ j_{A, X(f)(x')} ∘ Z(f^{[x']}) \,.
			\\
			={}&
			(\text{constant map with value~$X(f)(x')$}) ∘ Z(f^{[x']})
			\\
			={}&
			(\text{constant map with value~$X(f)(x')$})
			\\
			={}&
			X(f) ∘ (\text{constant map with value~$x'$})
			\\
			={}&
			X(f) ∘ G_1(Z)_{A'} ∘ j_{A',\, x'}
		\end{align*}
		for every element~$x'$ of~$X(A')$, and therefore
		\[
			G_1(Z)_A ∘ G_0(Z)(f) = X(f) ∘ G_1(Z)_{A'} \,.
		\]

\end{itemize}
We have thus constructed a natural transformation~$G_1(Z)$ from~$G_0(Z)$ to~$X$.

We have overall constructed the action of~$G$ on objects.
Next, we will construct the action of~$G$ on morphisms.
Let
\[
	γ \colon Z \to Z'
\]
be a morphism in~$[\Elements(X)^{\op}, \Set]$.
We need to construct a morphism
\[
	G(γ) \colon G(Z) \to G(Z')
\]
in~$[\scat{A}^{\op}, \Set] / X$.
In other words~$G(γ)$ needs to be a natural transformation from~$G_0(Z)$ to~$G_0(Z')$ that makes the following diagram commute:
\begin{equation}
	\label{triangular diagram for naturality for ugly construction}
	\begin{tikzcd}[row sep = large]
		G_0(Z)
		\arrow[Rightarrow]{rr}[above]{G(γ)}
		\arrow[Rightarrow]{dr}[below left]{G_1(Z)}
		&
		{}
		&
		G_0(Z')
		\arrow[Rightarrow]{dl}[below right]{G_1(Z')}
		\\
		{}
		&
		X
		&
		{}
	\end{tikzcd}
\end{equation}
\begin{itemize}

	\item
		We have for every object~$A$ of~$\scat{A}$ that
		\[
			G_0(Z)(A) = ∐_{x ∈ X(A)} Z(A, x) \,,
			\quad
			G_0(Z')(A) = ∐_{x ∈ X(A)} Z'(A, x) \,.
		\]
		The morphism~$γ$ is a natural transformation from~$Z$ to~$Z'$, and therefore gives us for every object~$(A, x)$ of~$\Elements(X)$ a map~$γ_{(A, x)}$ from~$Z(A, x)$ to~$Z'(A, x)$.
		This allows us to define the desired transformation~$G(γ)$ via
		\[
			G(γ)_A ≔ ∐_{x ∈ X(A)} γ_{(A, x)}
		\]
		for every object~$A$ of~$\scat{A}$.

	\item
		To check the commutativity of the diagram~\eqref{triangular diagram for naturality for ugly construction}, we need to check that for object~$A$ of~$\scat{A}$ the following diagram commutes:
		\[
			\begin{tikzcd}[row sep = large]
				G_0(Z)(A)
				\arrow{rr}[above]{G(γ)_A}
				\arrow{dr}[below left]{G_1(Z)_A}
				&
				{}
				&
				G_0(Z')(A)
				\arrow{dl}[below right]{G_1(Z')_A}
				\\
				{}
				&
				X(A)
				&
				{}
			\end{tikzcd}
		\]
		For this, it suffices to check that for every element~$x$ of~$X(A)$, we have
		\[
			G_1(Z)_A ∘ G(γ)_A ∘ j^Z_{(A, x)}
			=
			G_1(Z)_A ∘ j^Z_{(A, x)} \,.
		\]
		This equality hold because
		\begin{align*}
			G_1(Z')_A ∘ G(γ)_A ∘ j^Z_{A, x}
			&=
			G_1(Z')_A ∘ j^{Z'}_{A, x} ∘ γ_{(A, x)}
			\\
			&=
			(\text{constant map with value~$x$}) ∘ γ_{(A, x)}
			\\
			&=
			(\text{constant map with value~$x$})
			\\
			&=
			G_1(Z)_A ∘ j^Z_{A, x} \,.
		\end{align*}
\end{itemize}
We have overall constructed an induced morphism~$G(γ)$ from~$G(Z)$ to~$G(Z')$.
We have to check that this construction is functorial.
\begin{itemize}

	\item
		Let~$Z$ be an object of~$[\Elements(X)^{\op}, \Set]$.
		We have
		\begin{align*}
			G(\id_Z)_A
			&=
			∐_{x ∈ X(A)} {} (\id_Z)_{(A, x)}
			\\
			&=
			∐_{x ∈ X(A)} \id_{Z(A, x)}
			\\
			&=
			\id_{∐_{x ∈ X(A)} Z(A, x)}
			\\
			&=
			\id_{G_0(Z)(A)}
			\\
			&=
			( \id_{G_0(Z)} )_A
		\end{align*}
		for every object~$A$ of~$\scat{A}$, and therefore
		\[
			G(\id_Z)
			=
			\id_{G_0(Z)}
			=
			\id_{G(Z)} \,.
		\]

	\item
		Let
		\[
			γ \colon G \to G' \,,
			\quad
			γ' \colon G' \to G''
		\]
		be two composable morphisms in~$[\Elements(X)^{\op}, \Set]$.
		We have for every object~$A$ of~$\scat{A}$ the equalities
		\begin{align*}
			( G(γ') ∘ G(γ) )_A
			&=
			G(γ')_A ∘ G(γ)_A
			\\
			&=
			\left( ∐_{x ∈ X(A)} γ'_{(A, x)} \right) ∘ \left( ∐_{x ∈ X(A)} γ_{(A, x)} \right)
			\\
			&=
			∐_{x ∈ X(A)} {} ( γ'_{(A, x)} ∘ γ_{(A, x)} )
			\\
			&=
			∐_{x ∈ X(A)} {} (γ' ∘ γ)_{(A, x)}
			\\
			&=
			G(γ' ∘ γ) \,.
		\end{align*}

\end{itemize}
We have thus proven the functoriality of~$G$.

\subsubsection*{The general case: the isomorphism $G ∘ F ≅ \Id$}

Let~$(Y, α)$ be an object of~$[\scat{A}^{\op}, \Set] / X$.
We have
\[
	(G ∘ F)(Y, α)(A)
	=
	G(F(Y, α))(A)
	=
	∐_{x ∈ X(A)} F(Y, α)(A, x)
	=
	∐_{x ∈ X(A)} α_A^{-1}(x)
\]
for every object~$A$ of~$\scat{A}$, with~$α_A$ being a map from~$Y(A)$ to~$X(A)$.
The set~$Y(A)$ is the disjoint union of the preimages~$α_A^{-1}(x)$ where~$x$ ranges trough~$X(A)$.
We have therefore a bijection
\[
	ε_{(Y, α), A}
	\colon
	(G, ∘ F)(Y, α)(A)
	\to
	Y(A)
\]
that is given for every element~$x$ of~$X(A)$ on the~\nth{$x$} summand of~$(G ∘ F)(Y, α)(A)$ by the inclusion map from~$α_A^{-1}(x)$ to~$Y(A)$.
We denote this inclusion map by
\[
	i_{(Y, α), A, x} \colon α_A^{-1}(x) \to Y(A) \,.
\]

These bijections~$ε_{(Y, α), A}$ assemble into a transformation~$ε_{(Y, α)}$ from~$G_0(F(Y, α))$ to~$Y$.
This transformation is natural.
To prove this, let
\[
	f \colon A \to A'
\]
be a morphism in~$\scat{A}$.
We need to check the commutativity of the following square diagram:
\[
	\begin{tikzcd}[column sep = 6em, row sep = large]
		G_0(F(Y, α))(A')
		\arrow{r}[above]{G_0(F(Y, α))(f)}
		\arrow{d}[left]{ε_{(Y, α), A'}}
		&
		G_0(F(Y,α))(A)
		\arrow{d}[right]{ε_{(Y, α), A}}
		\\
		Y(A')
		\arrow{r}[above]{Y(f)}
		&
		Y(A)
	\end{tikzcd}
\]
This diagram may be rewritten as follows:
\[
	\begin{tikzcd}[column sep = 6em, row sep = large]
		∐_{x' ∈ X(A')} α_{A'}^{-1}(x')
		\arrow{r}[above]{G_0(F(Y, α))(f)}
		\arrow{d}[left]{ε_{(Y, α), A'}}
		&
		∐_{x ∈ X(A)} α_A^{-1}(x)
		\arrow{d}[right]{ε_{(Y, α), A}}
		\\
		Y(A')
		\arrow{r}[above]{Y(f)}
		&
		Y(A)
	\end{tikzcd}
\]
To check the commutativity of this square diagram, we need to check that
\[
	ε_{(Y, α), A} ∘ G_0(F(Y, α))(f) ∘ j_{A', x'}
	=
	Y(f) ∘ ε_{(Y, α), A'} ∘ j_{A', x'}
\]
for every element~$x'$ of~$X(A')$.
This equality holds because
\begin{align*}
	{}&
	ε_{(Y, α), A} ∘ G_0(F(Y, α))(f) ∘ j_{A', x'}
	\\
	={}&
	ε_{(Y, α), A} ∘ j_{A, X(f)(x')} ∘  F(Y, α)(f^{[x']})
	\\
	={}&
	i_{(Y, α), A, X(f)(x')} ∘ F(Y, α)( f^{[x']} )
	\\
	={}&
	i_{(Y, α), A, X(f)(x')}
	∘
	\restrict[\Big]{ Y(f) }{ α_{A'}^{-1}(x') }[ α_A^{-1}(X(f)(x')) ]
	\\
	={}&
	Y(f) ∘ i_{(Y, α), A', x'}
	\\
	={}&
	Y(f) ∘ ε_{(Y, α), A'} ∘ j_{A', x'} \,.
\end{align*}

We have thus constructed a natural transformation~$ε_{(Y, α)}$ from~$G_0(F(Y, α))$ to~$Y$.
Each component of~$ε_{(Y, α)}$ is bijective, whence~$ε_{(Y, α)}$ is a natural isomorphism.
We claim that this natural isomorphism is a morphism from~$(Y, α)$ to~$G(F(Y, α))$.
For this, we need to prove that the following diagram commutes:
\[
	\begin{tikzcd}[row sep = large, column sep = {5em,between origins}]
		G_0(F(Y, α))
		\arrow[Rightarrow]{rr}[above]{ε_{(Y, α)}}
		\arrow[Rightarrow]{dr}[below left]{G_1(F(Y, α))}
		&
		{}
		&
		Y
		\arrow[Rightarrow]{dl}[below right]{α}
		\\
		{}
		&
		X
		&
		{}
	\end{tikzcd}
\]
We hence need to show that for every object~$A$ of~$\scat{A}$, the following diagram commutes:
\[
	\begin{tikzcd}[row sep = large, column sep = {5em,between origins}]
		G_0(F(Y, α))(A)
		\arrow{rr}[above]{ε_{(Y, α), A}}
		\arrow{dr}[below left]{G_1(F(Y, α))_A}
		&
		{}
		&
		Y(A)
		\arrow{dl}[below right]{α_A}
		\\
		{}
		&
		X(A)
		&
		{}
	\end{tikzcd}
\]
It suffices to show that
\[
	α_A ∘ ε_{(Y, α), A} ∘ j_{(A, x)}
	=
	G_1(F(Y, α))_A ∘ j_{(A, x)}
\]
for every element~$x$ of~$X(A)$.
This equality holds because
\begin{align*}
	{}&
	α_A ∘ ε_{(Y, α), A} ∘ j_{(A, x)}
	\\
	={}&
	α_A ∘ i_{(Y, α), A, x}
	\\
	={}&
	\text{constant map with value~$x$}
	\\
	={}&
	G_1(F(Y, α))_A ∘ j_{(A, x)} \,.
\end{align*}

We have thus constructed for every object~$(Y, α)$ a morphism~$ε_{(Y, α)}$ from~$G(F(Y, α))$ to~$(Y, α)$ in~$[\scat{A}^{\op}, \Set] / X$.
We have also seen that this morphism is an isomorphism in~$[\scat{A}^{\op}, \Set]$;
it is therefore also an isomorphism in~$[\scat{A}^{\op}, \Set] / X$.

The isomorphism~$ε_{(Y, α)}$ is natural in~$(Y, α)$.
To see this, we consider a morphism
\[
	β \colon (Y, α) \to (Y', α')
\]
in~$[\scat{A}^{\op}, \Set] / X$.
We need to show that the following square diagram commutes:
\[
	\begin{tikzcd}[sep = large]
		GF(Y, α)
		\arrow{r}[above]{GF(β)}
		\arrow{d}[left]{ε_{(Y, α)}}
		&
		GF(Y', α')
		\arrow{d}[right]{ε_{(Y', α')}}
		\\
		(Y, α)
		\arrow{r}[above]{β}
		&
		(Y', α')
	\end{tikzcd}
\]
In other words, we need to prove the commutativity of the following square diagram of functors and natural transformations between them:
\[
	\begin{tikzcd}[sep = large]
		G_0(F(Y, α))
		\arrow[Rightarrow]{r}[above]{GF(β)}
		\arrow[Rightarrow]{d}[left]{ε_{(Y, α)}}
		&
		G_0(F(Y', α'))
		\arrow[Rightarrow]{d}[right]{ε_{(Y', α')}}
		\\
		Y
		\arrow[Rightarrow]{r}[above]{β}
		&
		Y'
	\end{tikzcd}
\]
This means that we need to prove for every object~$A$ of~$\scat{A}$ the commutativity of the following square diagram  in~$\Set$:
\[
	\begin{tikzcd}[sep = large]
		G_0(F(Y, α))(A)
		\arrow{r}[above]{G(F(β))_A}
		\arrow{d}[left]{ε_{(Y, α), A}}
		&
		G_0(F(Y', α'))(A)
		\arrow{d}[right]{ε_{(Y', α'), A}}
		\\
		Y(A)
		\arrow{r}[above]{β_A}
		&
		Y'(A)
	\end{tikzcd}
\]
This diagram may be rewritten as follows:
\[
	\begin{tikzcd}[row sep = large, column sep = 7em]
		∐_{x ∈ X(A)} α_A^{-1}(x)
		\arrow{r}[above]{∐_{x ∈ X(A)} F(β)_{(A, x)}}
		\arrow{d}[left]{ε_{(Y, α), A}}
		&
		∐_{x ∈ X(A)} (α')_A^{-1}(x)
		\arrow{d}[right]{ε_{(Y', α'), A}}
		\\
		Y(A)
		\arrow{r}[above]{β_A}
		&
		Y'(A)
	\end{tikzcd}
\]
The map~$F(β)_{(A, x)}$ is the restriction of~$F(β)$ to a map from~$α_A^{-1}(x)$ to~$(α')_A^{-1}(x)$, whence this diagram commutes.

We have thus constructed a natural isomorphism~$ε$ from~$G ∘ F$ to~$\Id_{[\scat{A}^{\op}, \Set] / X}$.
The existence of this isomorphism shows that
\[
	G ∘ F ≅ \id_{ [\scat{A}^{\op}, \Set] / X} \,.
\]

\subsubsection*{The general case: the isomorphism $F ∘ G ≅ \Id$}

Let~$Z$ be an object of~$[\Elements(X)^{\op}, \Set]$.
For every object~$(A, x)$ of~$\Elements(X)$, the set
\[
	F(G(Z))(A, x) = G_1(Z)_A^{-1}(x)
\]
is precisely the image of~$Z(A, x)$ in~$G_0(Z)(A) = ∐_{x' ∈ X(A)} Z(A, x')$.
(Recall that the map ~$G_1(Z)_A$ has the constant value~$x$ on the summand~$Z(A, x)$ of~$G_0(Z)$.)
In other words, we have
\[
	F(G(Z))(A, x) = \{ (x, z) \suchthat z ∈ Z(A, x) \} \,.
\]
We have therefore a bijection
\[
	η_{Z, (A, x)}
	\colon
	Z(A, x)
	\to
	F(G(Z))(A, x) \,,
	\quad
	z \mapsto (x, z) \,.
\]

The bijection~$η_{Z, (A, x)}$ is natural in~$(A, x)$.
To see this, we consider a morphism
\[
	f \colon (A, x) \to (A', x')
\]
in~$\Elements(X)$, and need to show that the following square diagram commutes:
\[
	\begin{tikzcd}[row sep = large, column sep = huge]
		Z(A', x')
		\arrow{r}[above]{Z(f)}
		\arrow{d}[left]{η_{Z, (A', x')}}
		&
		Z(A, x)
		\arrow{d}[right]{η_{Z, (A, x)}}
		\\
		F(G(Z))(A', x')
		\arrow{r}[above]{F(G(Z))(f)}
		&
		F(G(Z))(A, x)
	\end{tikzcd}
\]
The map~$F(G(Z))(f)$ is the restriction of~$G_0(Z)(f)$, and~$G_0(Z)(f)$ maps the summand~$Z(A', x')$ of~$G_0(Z)(A')$ into the summand~$Z(A, x)$ of~$G_0(Z)(A)$ via~$Z(f)$.
This means that the above diagram commutes.

We have thus for every object~$Z$ of~$[\Elements(X)^{\op}, \Set]$ a natural transformation~$η_Z$ from~$Z$ to~$F(G(Z))$, i.e., a morphism from~$Z$ to~$(F ∘ G)(Z)$ in~$[\Elements(X)^{\op}, \Set]$.
Each component of~$η_Z$ is bijective, whence~$η_Z$ is an isomorphism.

We now check that~$η_Z$ is natural in~$Z$.
For this, we need to show that for every morphism
\[
	γ \colon Z \to Z'
\]
in~$[\Elements(X)^{\op}, \Set]$ the following square diagram commutes:
\[
	\begin{tikzcd}[row sep = large, column sep = huge]
		Z
		\arrow{r}[above]{γ}
		\arrow{d}[left]{η_Z}
		&
		Z'
		\arrow{d}[right]{η_{Z'}}
		\\
		F(G(Z))
		\arrow{r}[above]{F(G(γ))}
		&
		F(G(Z'))
	\end{tikzcd}
\]
This is a diagram consisting of functors and natural transformations, whence we need to show that for every object~$(A, x)$ of~$\Elements(X)$ the following square diagram commutes:
\[
	\begin{tikzcd}[row sep = large, column sep = huge]
		Z(A, x)
		\arrow{r}[above]{γ_{(A, x)}}
		\arrow{d}[left]{η_{Z, (A, x)}}
		&
		Z'(A, x)
		\arrow{d}[right]{η_{Z', {(A, x)}}}
		\\
		F(G(Z))(A, x)
		\arrow{r}[above]{F(G(γ))_{(A, x)}}
		&
		F(G(Z'))(A, x)
	\end{tikzcd}
\]
We recall that
\[
	G_0(Z)(A)  = ∐_{x' ∈ X(A)} Z(A, x') \,,
	\qquad
	G_0(Z')(A) = ∐_{x' ∈ X(A)} Z'(A, x')
\]
and that the map~$G(γ)_A$ from~$G_0(Z)(A)$ to~$G_0(Z')(A)$ is given by~$∐_{x' ∈ X(A)} γ_{(A, x')}$.
The map~$F(G(γ))_{(A, x)}$ results from~$G_0(Z)_A$ by restriction, from the copy of~$Z(A, x)$ in~$G_0(Z)(A)$ to the copy of~$Z'(A, x)$ in~$G_0(Z)(A')$.
This means precisely that the above square diagram commutes.

We have thus constructed a natural transformation~$η$ from~$\Id_{[\Elements(X)^{\op}, \Set]}$ to~$F ∘ G$.
Each component of~$η$ is an isomorphism, whence~$η$ is a natural isomorphism.
The existence of such an isomorphism shows that
\[
	\Id_{[\Elements(X)^{\op}, \Set]} ≅ F ∘ G \,.
\]
