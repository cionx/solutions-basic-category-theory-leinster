\subsection{}

The construction of subobjects from Exercise~5.1.40 allows us to generalize the notion of \enquote{subsets} to arbitrary categories.
One important operation that we can do with subsets is taking preimages:
given two sets~$A'$ and~$A$ and a map~$f$ from~$A'$ to~$A$, we can form for every subset~$S$ of~$A$ its preimage~$f^{-1}(S)$, which is a subset of~$A'$.
The following \lcnamecref{preimages as pullbacks} allows us to express preimages in the language of category theory.

\begin{proposition}
	\label{preimages as pullbacks}
	Let~$A$ be set, let~$S$ be a subset of~$A$ and let~$i$ be the inclusion map from~$S$ to~$A$.
	Let~$A'$ be another set and let~$f$ a map from~$A'$ to~$A$.
	Let~$i'$ the inclusion map from~$f^{-1}(S)$ to~$A'$ and let~$f'$ the restriction of~$f$ to a map from~$f^{-1}(S)$ to~$S$.
	The following diagram is a pullback diagram:
	\[
		\begin{tikzcd}
			f^{-1}(S)
			\arrow{r}[above]{f'}
			\arrow{d}[left]{i'}
			&
			S
			\arrow{d}[right]{i}
			\\
			A'
			\arrow{r}[above]{f}
			&
			A
		\end{tikzcd}
	\]
\end{proposition}

In the following, we will use this reformulation of preimages via pullbacks to generalize preimages to arbitrary categories.



\subsubsection{}

We can form for every object~$(X, m)$ of~$\Monic(A)$ the pullback diagram
\begin{equation}
	\label{preimages via pullback diagrams}
	\begin{tikzcd}
		X'
		\arrow{r}
		\arrow{d}[left]{m'}
		&
		X
		\arrow{d}[right]{m}
		\\
		A'
		\arrow{r}[above]{f}
		&
		A
	\end{tikzcd}
\end{equation}
We know from Exercise~5.1.42 that the morphism~$m'$ is again a monomorphism, whence we have constructed an object~$(X', m')$ of~$\Monic(A')$.
By choosing for every object~$(X, m)$ of~$\Monic(A)$ a pullback as above, we have thus constructed a map
\[
	\Monic(f)
	\colon
	\Ob(\Monic(A))
	\to
	\Ob(\Monic(A')) \,.
\]
We show in the following that the map~$\Monic(f)$ descends to a map
\[
	\Sub(f)
	\colon
	\Sub(A) \to \Sub(A') \,.
\]
This then shown that we can pull back subobjects of~$A$ to subobjects of~$A'$ via pullbacks.

To show that the map~$\Monic(f)$ descends to a map~$\Sub(f)$ as desired, we need to show that isomorphic objects of~$\Monic(A)$ lead to isomorphic objects in~$\Monic(A')$.
We will show more conceptually that~$\Monic(f)$ extends to a functor from~$\Monic(A)$ to~$\Monic(A')$.

To construct the action of~$\Monic$ on morphisms, let
\[
	h \colon (X, m) \to (Y, n)
\]
be a morphism in~$\Monic(A)$.
This means that we have the following commutative diagram:
\[
	\begin{tikzcd}[column sep = normal]
		X
		\arrow{rr}[above]{g}
		\arrow{dr}[below left]{m}
		&
		{}
		&
		Y
		\arrow{dl}[below right]{n}
		\\
		{}
		&
		A
		&
		{}
	\end{tikzcd}
\]
Let~$(X', m')$ and~$(Y', n')$ be the images of~$(X, m)$ and~$(Y, n)$ under~$\Monic(f)$.
There exists a unique morphism from~$X'$ to~$Y'$ that makes the following diagram commute:
\[
	\begin{tikzcd}
		X'
		\arrow{rr}
		\arrow{dd}[left]{m'}
		\arrow[dashed]{dr}
		&
		{}
		&
		X
		\arrow{dd}[left, near start]{m}
		\arrow{dr}[above right]{g}
		&
		{}
		\\
		{}
		&
		Y'
		\arrow[crossing over]{rr}
		\arrow{dl}[below right]{n'}
		&
		{}
		&
		Y
		\arrow{dl}[below right]{n}
		\\[0.5em]
		A'
		\arrow{rr}[above, pos = 0.6]{f}
		&
		{}
		&
		A
		&
		{}
	\end{tikzcd}
\]
We denote this induced morphism from~$X'$ to~$Y'$ by~$\Monic(f)(h)$, which is a morphism from~$(X', m')$ to~$(Y', n')$.
This morphism is thus unique with making the following two-dimensional diagram commute:
\[
	\begin{tikzcd}
		A'
		\arrow[equal]{d}
		&
		X'
		\arrow{l}[above]{m'}
		\arrow{d}[right]{\Monic(f)(h)}
		\arrow{r}
		&
		X
		\arrow{d}[right]{h}
		\\
		A'
		&
		Y'
		\arrow{l}[above]{n'}
		\arrow{r}
		&
		Y
	\end{tikzcd}
\]
Let us show that the assignment~$\Monic(f)$ from~$\Monic(A)$ to~$\Monic(A')$ is functorial.
\begin{itemize}

	\item
		Let~$(X, m)$ be an object of~$\Monic(A)$, and let~$(X', m')$ be resulting object of~$\Monic(A')$.
		The diagram
		\[
			\begin{tikzcd}
				A'
				\arrow[equal]{d}
				&
				X'
				\arrow{l}[above]{m'}
				\arrow{d}[right]{\id_{X'}}
				\arrow{r}
				&
				X
				\arrow{d}[right]{\id_X}
				\\
				A'
				&
				X'
				\arrow{l}[above]{n'}
				\arrow{r}
				&
				X
			\end{tikzcd}
		\]
		commutes, whence the morphism~$\id_{X'} = \id_{(X', m')}$ satisfies the defining relationship of the morphism~$\Monic(f)(\id_{(X, n)})$.

	\item
		Let~$(X, m)$,~$(Y, n)$ and~$(Z, p)$ be objects of~$\Monic(A)$, and let~$(X', m')$,~$(Y', n')$ and~$(Z', p')$ be the resulting objects of~$\Monic(A')$.
		Let
		\[
			h \colon (X, m) \to (Y, n) \,,
			\quad
			k \colon (Y, n) \to (Z, p)
		\]
		be two composable morphisms in~$\Monic(A)$.
		We have the following commutative diagram:
		\[
			\begin{tikzcd}
				A'
				\arrow[equal]{d}
				&
				X'
				\arrow{l}[above]{m'}
				\arrow{d}[right]{\Monic(f)(h)}
				\arrow{r}
				&
				X
				\arrow{d}[right]{h}
				\\
				A'
				\arrow[equal]{d}
				&
				Y'
				\arrow{l}[above]{n'}
				\arrow{d}[right]{\Monic(f)(k)}
				\arrow{r}
				&
				Y
				\arrow{d}[right]{k}
				\\
				A'
				&
				Z'
				\arrow{l}[above]{p'}
				\arrow{r}
				&
				Z
			\end{tikzcd}
		\]
		By leaving out the middle row of this diagram, we end up with the following commutative diagram:
		\[
			\begin{tikzcd}
				A'
				\arrow[equal]{d}
				&
				X'
				\arrow{l}[above]{m'}
				\arrow{d}[right]{\Monic(f)(k) ∘ \Monic(f)(h)}
				\arrow{r}
				&[3.5em]
				X
				\arrow{d}[right]{k ∘ h}
				\\
				A'
				&
				Z'
				\arrow{l}[above]{p'}
				\arrow{r}
				&
				Z
			\end{tikzcd}
		\]
		The composite~$\Monic(f)(k) ∘ \Monic(f)(h)$ thus satisfies the defining property of the morphism~$\Monic(f)(k ∘ h)$.
		These morphisms are therefore the same.
		
\end{itemize}

We have altogether constructed a functor
\[
	\Monic(f) \colon \Monic(A) \to \Monic(A') \,.
\]
This functor induces a map
\[
	\Sub(f) \colon \Sub(A) \to \Sub(A')
\]
as desired.
This map~$\Sub(f)$ has the following explicit description:
Given an element~$\class{(X, m)}$ of~$\Sub(A)$, we form the pullback diagram~\eqref{preimages via pullback diagrams}.
The image of~$\class{(X, m)}$ under~$\Sub(f)$ is then the element~$\class{(X', m')}$.

\begin{remark}
	The functor~$\Monic(f)$ that we constructed above is somewhat evil, since it relies on choosing pullbacks.
	However, pullbacks are unique up to isomorphism.
	We can use this uniqueness up to isomorphism to show that the functor~$\Monic(f)$ is again unique up to isomorphism.
	That is, different choices of pullbacks result in isomorphic functors.

	As a consequence, the induced functor~$\Sub(f)$ from~$\Sub(A)$ to~$\Sub(A')$ does not depend on our original choice of pullbacks.
\end{remark}



\subsubsection{}

The collection~$\Sub(A)$ is a set for every object~$A$ of~$\cat{A}$, since the category~$\cat{A}$ is assumed to be well-powered.
We can therefore regard the construction~$\Sub$ as an assignment from~$\cat{A}$ to~$\Set$ (both on objects and on morphisms).
Let us check that this assignment is contravariantly functorial.

\begin{itemize}

	\item
		Let~$A$ be an object of~$\cat{A}$ and consider the identity morphism
		\[
			\id_A \colon A \to A \,.
		\]
		For every object~$(X, m)$ of~$\Monic(A)$, the diagram
		\[
			\begin{tikzcd}
				X
				\arrow{r}[above]{\id_X}
				\arrow{d}[left]{m}
				&
				X
				\arrow{d}[right]{m}
				\\
				A
				\arrow{r}[above]{f}
				&
				A
			\end{tikzcd}
		\]
		is a pullback diagram.
		Therefore,
		\[
			\Sub(\id_A)( \class{(X, m)} ) = \class{(X, m)} \,.
		\]
		This shows that~$\Sub(\id_A) = \id_{\Sub(A)}$.

	\item
		Let
		\[
			f \colon A' \to A \,,
			\quad
			f' \colon A'' \to A'
		\]
		be two composable morphisms in~$\cat{A}$.
		Let~$(X, m)$ be an object of~$\Monic(A)$, let~$(X', m')$ be the image of~$(X, m)$ under~$\Monic(f)$, and let~$(X'', m'')$ be the image of~$(X', m')$ under~$\Monic(f')$.
		This means that in the following commutative diagram, both squares are are pullback diagrams:
		\[
			\begin{tikzcd}
				X''
				\arrow{r}
				\arrow{d}[left]{m''}
				&
				X'
				\arrow{r}
				\arrow{d}[right]{m'}
				&
				X
				\arrow{d}[right]{m}
				\\
				A''
				\arrow{r}[above]{f'}
				&
				A'
				\arrow{r}[above]{f}
				&
				A
			\end{tikzcd}
		\]
		By leaving out the middle column of this diagram, we get the following commutative diagram:
		\[
			\begin{tikzcd}
				X''
				\arrow{r}
				\arrow{d}[left]{m''}
				&
				X
				\arrow{d}[right]{m}
				\\
				A''
				\arrow{r}[above]{f ∘ f'}
				&
				A
			\end{tikzcd}
		\]
		It follows from Exercise~5.1.35 that this diagram is again a pullback diagram.
		We hence find that
		\[
			\Sub(f ∘ f')( \class{(X, m)} ) )
			=
			\class{(X'', m'')} \,.
		\]
		But we also have
		\[
			\class{(X'', m'')}
			=
			\Sub(f')( \class{(X', m')} )
			=
			\Sub(f')( \Sub(f)( \class{(X, m)} ) ) \,.
		\]
		We have therefore shown that~$\Sub(f ∘ f') = \Sub(f') ∘ \Sub(f)$.

\end{itemize}
We have shown that~$\Sub$ is a contravariant functor from~$\cat{A}$ to~$\Set$.

\begin{remark}
	We should actually be able to lift the identities
	\[
		\Sub(\id_A) = \id_{\Sub(A)} \,,
		\quad
		\Sub(f ∘ f') = \Sub(f') ∘ \Sub(A)
	\]
	to isomorphisms
	\[
		\Monic(\id_A) ≅ \id_{\Monic(A)} \,,
		\quad
		\Monic(f ∘ f') ≅ \Monic(f') ∘ \Monic(f) \,.
	\]
	But we cannot expect these isomorphisms to be equalities, since these action of~$\Monic$ on morphisms is only unique up to isomorphism.
\end{remark}



\subsubsection{}

Subobjects in~$\Set$ are the same as subsets.
More precisely, we have for every set~$A$ a bijection
\[
	α_A
	\colon
	\Power(A) \to \Sub(A) \,,
	\quad
	S \mapsto \class{(S, i_S)} \,,
\]
where for every subset~$S$ of~$A$ we denote the inclusion map from~$S$ to~$A$ as~$i_S$.
The existence of these bijections also shows that the category~$\Set$ is well-powered.

Let us check that the bijection~$α_A$ is natural in~$A$, so that~$α$ is a natural isomorphism from~$\Power$ to~$\Sub$.
We need to check that for every map
\[
	f \colon A' \to A
\]
the following diagram commutes:
\[
	\begin{tikzcd}
		\Power(A)
		\arrow{r}[above]{α_A}
		\arrow{d}[left]{\Power(f)}
		&
		\Sub(A)
		\arrow{d}[right]{\Sub(f)}
		\\
		\Power(A')
		\arrow{r}[above]{α_{A'}}
		&
		\Sub(A')
	\end{tikzcd}
\]
In view of our explicit description of~$\Sub(f)$ via pullbacks, this commutativity follows from \cref{preimages as pullbacks}.

Let now~$Ω$ be the set~$\{0, 1\}$.
Given a subset~$S$ of a set~$A$, we can consider the characteristic function of~$S$ on~$A$.
This is the map
\[
	χ_{A, S}
	\colon
	A
	\to
	Ω \,,
	\quad
	a
	\mapsto
	\begin{cases*}
		1 & if~$a ∈ S$, \\
		0 & otherwise.
	\end{cases*}
\]
The resulting map
\[
	χ_A
	\colon
	\Power(A) \to \Set(A, Ω) \,,
	\quad
	S \mapsto χ_{A, S}
\]
is bijective for every set~$A$.
It is also natural in~$A$.
To see this, we need to convince ourselves that for every map
\[
	f \colon A' \to A
\]
between sets, the following diagram commutes:
\[
	\begin{tikzcd}
		\Power(A)
		\arrow{r}[above]{χ_A}
		\arrow{d}[left]{\Power(f)}
		&
		\Set(A, Ω)
		\arrow{d}[right]{f^*}
		\\
		\Power(A')
		\arrow{r}[above]{χ_{A'}}
		&
		\Set(A', Ω)
	\end{tikzcd}
\]
This diagram commutes because we have for every subset~$S$ of~$A$ and every element~$a'$ of~$A'$ the equalities
\[
	f^*( χ_A( S ) )(a')
	=
	f^*( χ_{A, S} )(a')
	=
	(χ_{A, S} ∘ f)(a')
	=
	χ_{A, S}( f(a') )
\]
and
\[
	χ_{A', f^{-1}(S)}(a')
	=
	χ_{A'}( f^{-1}(S) )(a')
	=
	χ_{A'}( \Power(f)(S) )(a') \,,
\]
therefore the chain of equivalences
\begin{align*}
	f^*( χ_A( S ) )(a') = 1
	&\iff
	χ_{A, S}( f(a') ) = 1
	\\
	&\iff
	f(a') ∈ S
	\\
	&\iff
	a' ∈ f^{-1}(S)
	\\
	&\iff
	χ_{A', f^{-1}(S)}(a') = 1
	\\
	&\iff
	χ_{A'}( \Power(f)(S) )(a') = 1 \,,
\end{align*}
and thus the equality~$f^*( χ_A(S) ) = χ_{A'}( \Power(f)(S) )$.

We have overall constructed isomorphisms
\[
	\Sub ≅ \Power ≅ \Set(\ph, Ω) \,.
\]
The existence of the composite isomorphism~$\Sub ≅ \Set(\ph, Ω)$ shows that the functor~$\Sub$ is represented by the set~$Ω = \{ 0, 1 \}$.
