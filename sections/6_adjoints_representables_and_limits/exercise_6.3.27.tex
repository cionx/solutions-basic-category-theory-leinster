\subsection{}

We denote the presheaf category~$[\scat{A}^{\op}, \Set]$ by~$\hat{\scat{A}}$.

To give an explicit description of subobjects in $\hat{\scat{A}}$ we introduce the notion of a subfunctor:
a subfunctor is to a functor what a subset is to a set.
Just as subobjects in~$\Set$ correspond to subsets, we will show that subobjects in~$\hat{\scat{A}}$ correspond to subfunctors.

\subsubsection*{Definition of subfunctors}

Let~$\cat{A}$ be a category and let~$H$ be a functor from~$\cat{A}$ to~$\Set$.
We say that another functor~$S$ from~$\cat{A}$ to~$\Set$ is a \defemph{subfunctor} of~$H$ if it satisfies the following two conditions.
\begin{enumerate*}
	\item
		$S(A)$ is a subset of~$H(A)$ for every object~$A$ of~$\cat{A}$.
	\item
		$S(f)$ is the restriction of~$H(f)$ for every morphism~$f$ in~$\cat{A}$.
\end{enumerate*}
In other words, a subfunctor~$S$ of~$H$ consists of a subset~$S(A)$ of~$H(A)$ for every object~$A$ of~$\cat{A}$, such that~$H(f)( S(A) ) ⊆ S(B)$ for every morphism~$f \colon A \to B$ in~$\cat{A}$.

We write~$S ⊆ H$ if~$S$ is a subfunctor of~$H$.
The relation $⊆$ is a partial order on the class of functors from $\cat{A}$ to $\Set$.
We denote the class of subfunctors of~$H$ by~$\Power(H)$.
Given two subfunctors~$S$ and~$T$ of~$H$, we have~$S ⊆ T$ if and only if~$S(A) ⊆ T(A)$ for every object~$A$ of~$\cat{A}$.

\subsubsection*{The image of a natural transformation}

Given another functor~$H'$ from~$\cat{A}$ to~$\Set$ and a natural transformation~$α$ from~$H'$ to~$H$, we define a subfunctor~$\im(α)$ of~$H$ with
\[
	\im(α)(A) ≔ \im(α_A)
\]
for every object~$A$ of~$\cat{A}$.
This is indeed a subfunctor of~$H$ because
\begin{align*}
	H(f)\bigl( \im(α)(A) \bigr)
	&=
	H(f)( \im(α_A) ) \\
	&=
	H(f)( α_A( F(A) ) ) \\
	&=
	α_B\bigl( F(f)( F(A) ) \bigr) \\
	&⊆
	α_B( F(B) ) \\
	&=
	\im(α_B) \\
	&=
	\im(α)(B)
\end{align*}
for every morphism~$f \colon A \to B$ in~$\cat{A}$.
We refer to the subfunctor~$\im(α)$ of~$H$ as the \defemph{image} of~$α$.

\subsubsection*{Correspondence between subobjects and subfunctors}

Let us now restrict our attention to subobjects of~$H$.
We claim that the map
\begin{equation}
	\label{image map from monics to subfunctors}
	\Ob(\Monic(H)) \to \Power(H) \,,
	\quad
	(X, μ) \mapsto \im(μ) \,,
\end{equation}
descends to a well-defined bijection from~$\Sub(H)$ to~$\Power(H)$.
To prove this claim we need to check the following subclaims:
\begin{enumerate}

	\item
		Isomorphic objects of~$\Monic(H)$ have the same image.

	\item
		The map~\eqref{image map from monics to subfunctors} is surjective.

	\item
		If two objects of~$\Monic(H)$ have the same image, then they are already isomorphic in~$\Monic(A)$.

\end{enumerate}

To show the first subclaim, suppose that we have a commutative diagram of functors and natural transformations as follows:
\[
	\begin{tikzcd}[column sep = normal]
		{}
		&
		H
		&
		\\
		F
		\arrow[Rightarrow]{ur}[above left]{α}
		\arrow[Rightarrow]{rr}
		&
		{}
		&
		G
		\arrow[Rightarrow]{ul}[above right]{β}
	\end{tikzcd}
\]
We then have~$\im(α) ⊆ \im(β)$.
For two isomorphic objects~$(X, μ)$ and~$(X, μ')$ of~$\Monic(H)$ we thus have both~$\im(μ) ⊆ \im(μ')$ and~$\im(μ') ⊆ \im(μ)$, and therefore~$\im(μ) = \im(μ')$.

Now we show the second subclaim.
For every subfunctor~$S$ of~$H$ we have a natural transformation~$ι \colon S \To H$ whose component~$ι_A$ is given by the inclusion map from~$S(A)$ to~$H(A)$ for every object~$A$ of~$\cat{A}$.
Each component of~$ι$ is an injective map, whence~$ι$ is a monomorphism.
The object~$(S, ι)$ of~$\Monic(H)$ serves as a preimage of~$S$ for the map~\eqref{image map from monics to subfunctors}, whence this map is surjective.

To show the third subclaim, suppose that~$(X, μ)$ and~$(X', μ')$ are two objects of~$\Monic(H)$ with the same image in~$\Power(H)$, i.e., with~$\im(μ) = \im(μ')$.
We know from Exercise~6.2.20 that the components of~$μ$ and~$μ'$ are injective.
For every object~$A$ of~$\cat{A}$ we have thefore the two injective maps
\[
	μ_A \colon X(A) \to H(A) \,,
	\quad
	μ'_A \colon X'(A) \to H(A) \,.
\]
These two maps have the same image since
\[
	\im(μ_A)
	=
	\im(μ)(A)
	=
	\im(μ')(A)
	=
	\im(μ'_A) \,.
\]
As seen in the solution to Exercise~5.1.40, there exists a unique bijection~$α_A$ from~$X(A)$ to~$X'(A)$ that makes the following diagram commute:
\[
	\begin{tikzcd}[column sep = small]
		X(A)
		\arrow{rr}[above]{α_A}
		\arrow{dr}[below left]{μ_A}
		&
		{}
		&
		X'(A)
		\arrow{dl}[below right]{μ'_A}
		\\
		{}
		&
		H(A)
		&
		{}
	\end{tikzcd}
\]

We claim that the resulting transformation~$α$ from~$X$ to~$X'$ is natural.
To show this, we need to check that for every morphism~$f \colon A \to B$ in~$\cat{A}$ the diagram
\[
	\begin{tikzcd}
		X(A)
		\arrow{r}[above]{α_A}
		\arrow{d}[left]{X(f)}
		&
		X'(A)
		\arrow{d}[right]{X'(f)}
		\\
		X(B)
		\arrow{r}[above]{α_B}
		&
		X'(B)
	\end{tikzcd}
\]
commutes.
To this end, we consider the following extended diagram:
\[
	\begin{tikzcd}
		X(A)
		\arrow{rr}[above]{α_A}
		\arrow{dr}[below left]{μ_A}
		\arrow{dd}[left]{X(f)}
		&
		{}
		&
		X'(A)
		\arrow{dl}[below right]{μ'_A}
		\arrow{dd}[right]{X'(f)}
		\\
		{}
		&
		H(A)
		&
		{}
		\\
		X(B)
		\arrow{rr}[above, near start]{α_B}
		\arrow{dr}[below left]{μ_B}
		&
		{}
		&
		X'(B)
		\arrow{dl}[below right]{μ'_B}
		\\
		{}
		&
		H(B)
		\arrow[crossing over, from=uu, right, near start, "H(f)"]
		&
		{}
	\end{tikzcd}
\]
The top and bottom of this diagram commute by definition of~$α$, and the frontal two sides commute by the naturality of~$μ$ and~$μ'$.
It follows that
\begin{align*}
	μ'_B ∘ X'(f) ∘ α_A
	&=
	H(f) ∘ μ'_A ∘ α_A \\
	&=
	H(f) ∘ μ_A \\
	&=
	μ_B ∘ X(f) \\
	&=
	μ'_B ∘ α_B ∘ X(f) \,.
\end{align*}
The map~$μ'_B$ is injective, so it follows that~$X'(f) ∘ α_A = α_B ∘ X(f)$, as desired.

We have thus constructed a natural isomorphism~$α$ from~$X$ to~$X'$ that makes the diagram
\[
	\begin{tikzcd}[column sep = normal]
		X
		\arrow[Rightarrow]{rr}[above]{α}
		\arrow[Rightarrow]{dr}[below left]{μ}
		&
		{}
		&
		X'
		\arrow[Rightarrow]{dl}[below right]{μ'}
		\\
		{}
		&
		H
		&
		{}
	\end{tikzcd}
\]
commute.
It follows from \cref{isomorphisms in mono categories} (page~\pageref{isomorphisms in mono categories}) that this natural isomorphism~$α$ is an isomorphism in~$\Mon(H)$ from~$(X, μ)$ to~$(X', μ')$.
This shows the third subclaim.

We have thus constructed a bijection from~$\Sub(H)$ to~$\Power(H)$.
In this way, subobjects of~$H$ are the same as subfunctors of~$H$.

\subsubsection*{The category $\hat{\scat{A}}$ is well-powered}

We have for every functor~$H$ from~$\cat{A}$ to~$\Set$ an injective map from~$\Power(H)$ into the product~$∏_{A ∈ \Ob(\cat{A})} H(A)$.
So if the category~$\cat{A}$ is small (or more generally if~$\Ob(\cat{A})$ is a set), then it follows that~$\Power(H)$ is a set, whence~$\Sub(H)$ is a set.
This tells us that the category~$\hat{\scat{A}}$ is well-powered.

\subsubsection*{Functoriality of the power set}

Let~$H$ and~$H'$ be two functors from~$\cat{A}$ to~$\Set$.
Suppose we have a subfunctor~$S'$ of~$H'$ and a natural transformation~$α$ from~$H$ to~$H'$.
We can then consider for every object~$A$ of~$\cat{A}$ the subset~$S(A) ≔ α^{-1}( S'(A) )$ of~$H(A)$.
These sets form a subfunctor~$S$ of~$H$ since for every morphism~$f \colon A \to B$ in~$\cat{A}$ we have
\[
	α_B\bigl( H(f)( S(A) ) \bigr)
	=
	H'(f)( α_A( S(A) ) )
	⊆
	H'(f)( S'(A) )
	⊆
	S'(B) \,,
\]
and thus~$H(f)( S(A) ) ⊆ α_B^{-1}( S'(B) ) = S(B)$.
We refer to the subfunctor~$S$ of~$H$ as the \defemph{preimage} of~$S'$ under~$α$, and denote it by~$α^{-1}(S')$.
We thus have
\[
	α^{-1}(S')(A) = α_A^{-1}( S'(A) )
\]
for every object~$A$ of~$\cat{A}$.
We get in this way a map
\[
	α^{-1} \colon \Power(S') \to \Power(S) \,.
\]
This construction makes~$\Power$ functorial:
\begin{enumerate}

	\item
		Let~$S$ be a subfunctor of~$H$.
		We have the equalities
		\[
			(\id_H)^{-1}(S)(A) = \id_{H(A)}^{-1}( S(A) ) = S(A)
		\]
		for every object~$A$ of~$\cat{A}$, and therefore~$(\id_H)^{-1}(S) = S$.
		Consequently,~$(\id_H)^{-1}$ is the identity map on~$\Power(H)$.

	\item
		Let~$H''$ be yet another functor from~$\cat{A}$ to~$\Set$.
		Let~$S''$ be a subfunctor of~$H''$ and let
		\[
			α \colon H \To H' \,,
			\quad
			α' \colon H' \To H''
		\]
		be natural transformations.
		We have the chain of equalities
		\begin{align*}
			(α' ∘ α)^{-1}(S'')(A)
			&=
			(α' ∘ α)_A^{-1}( S''(A) ) \\
			&=
			(α'_A ∘ α_A)^{-1}( S''(A) ) \\
			&=
			α_A^{-1}\bigl( (α'_A)^{-1}( S''(A) ) \bigr) \\
			&=
			α_A^{-1}\bigl( (α')^{-1}( S'' )(A) \bigr) \\
			&=
			α_A^{-1}( (α')^{-1}( S'' ) )(A)
		\end{align*}
		for every object~$A$ of~$\cat{A}$, and therefore the equality
		\[
			(α' ∘ α)^{-1}(S'') = α^{-1}( (α')^{-1}(S'') ) \,.
		\]
		Consequently,~$(α' ∘ α)^{-1} = α^{-1} ∘ (α')^{-1}$.

\end{enumerate}
These properties tell us that we have constructed a contravariant functor
\[
	\Power \colon [\cat{A}, \Set] \to \Set
\]
if the category~$\cat{A}$ is small.

\subsubsection*{Naturality of the isomorphism $\Sub(H) ≅ \Power(H)$}

We have previously constructed for every functor~$H$ from~$\cat{A}$ to~$\Set$ an isomorphism~$φ_H$ from~$\Power(H)$ to~$\Sub(H)$.
We will now show that the isomorphism~$φ_H$ is natural in $H$.
If the category~$\cat{A}$ is small, then this means that the two functors~$\Sub$ and~$\Power$ from~$\cat{A}$ to~$\Set$ are isomorphic.

We need to check that for every two functors~$H$ and~$H'$ from~$\cat{A}$ to~$\Set$ and every natural transformation~$α$ from~$H$ to~$H'$ the following diagram commutes:
\[
	\begin{tikzcd}
		\Power(H')
		\arrow{r}[above]{φ_{H'}}
		\arrow{d}[left]{α^{-1}}
		&
		\Sub(H')
		\arrow{d}[right]{\Sub(α)}
		\\
		\Power(H)
		\arrow{r}[above]{φ_H}
		&
		\Sub(H)
	\end{tikzcd}
\]
Let~$S'$ be an element of the top-left corner of this diagram, i.e., a subfunctor of~$H'$.
We give descriptions of the two resulting elements of~$\Sub(H)$ and then show that these two elements are equal.

The subobject~$φ_{H'}(S')$ of~$H'$ (an element of the tow-right corner of the diagram) is represented by~$(S', ι')$ where~$ι'$ is the natural transformation from~$S'$ to~$H'$ that is the inclusion map in each component.
Let us denote the resulting element~$\Sub(α)( [(S', ι')] )$ by $[(T, θ)]$.
This is a subobject of~$H$, and it is uniquely determined by the fact that there exists a pullback diagram in~$[\cat{A}, \Set]$ of the following form:
\[
	\begin{tikzcd}
		T
		\arrow[Rightarrow]{r}
		\arrow[Rightarrow]{d}[left]{θ}
		&
		S'
		\arrow[Rightarrow]{d}[right]{ι'}
		\\
		H
		\arrow[Rightarrow]{r}[above]{α}
		&
		H'
	\end{tikzcd}
\]

The subfunctor~$S ≔ α^{-1}(S')$ of~$H$ (an element of the bottom-right corner of the diagram) is given by
\[
	S(A) = α_A^{-1}( S'(A) )
\]
for every object~$A$ of~$\scat{A}$.
The subobject~$φ_H(S)$ of~$H$ (an element of the bottom-right corner of the diagram) is represented by~$(S, ι)$ where~$ι$ denotes the natural transformation from~$S$ to~$X$ that is the inclusion map in each component.

To show the desired equality~$[(S, ι)] = [(T, θ)]$ we need to show that there exists a pullback diagram of the form
\[
	\begin{tikzcd}
		S
		\arrow[Rightarrow]{r}
		\arrow[Rightarrow]{d}[left]{ι}
		&
		S'
		\arrow[Rightarrow]{d}[right]{ι'}
		\\
		H
		\arrow[Rightarrow]{r}[above]{α}
		&
		H'
	\end{tikzcd}
\]
in~$[\cat{A}, \Set]$.
We know that limits in functor categories can be computed pointwise.
It therefore suffices to show that there exists for every object~$A$ of~$\cat{A}$ a pullback diagram of the form
\[
	\begin{tikzcd}
		S(A)
		\arrow{r}
		\arrow{d}[left]{ι_A}
		&
		S'(A)
		\arrow{d}[right]{ι'_A}
		\\
		H(A)
		\arrow{r}[above]{α_A}
		&
		H'_A
	\end{tikzcd}
\]
in~$\Set$.
This desired diagram simplifies as follows:
\[
	\begin{tikzcd}
		α_A^{-1}(S'(A))
		\arrow{r}
		\arrow{d}[left]{ι_A}
		&
		S'(A)
		\arrow{d}[right]{ι'_A}
		\\
		H(A)
		\arrow{r}[above]{α_A}
		&
		H'_A
	\end{tikzcd}
\]
We get this desired pullback diagram by choosing the upper horizontal arrow as the restriction of~$α_A$, as seen in \cref{preimages as pullbacks} (page~\pageref{preimages as pullbacks}).

\subsubsection*{Special case: the category $\scat{A}$ has one object and is discrete}

To better understand the problem at hand, let us first consider the special case that the category~$\scat{A}$ is the one-object discrete category.
The presheaf category~$\hat{\scat{A}}$ is then just~$\Set$, and the functor~$\Power$ is just the usual contravariant power set functor.

Let~$𝟙 = \{ 1 \}$ be a one-element set.
For every set~$X$, elements of~$X$ are the same as maps from~$𝟙$ to~$X$.
For the subobject classifier~$Ω$ of~$\Set$ we have therefore the isomorphisms
\[
	Ω ≅ \Set(𝟙, Ω) ≅ \Power(𝟙) \,.
\]
We turn this observation into motivation to define~$Ω$ as~$\Power(𝟙)$.

We observe that~$Ω = \{ ∅, 𝟙 \}$, which agrees with the usual definition of the subobject classifier of~$\Set$.
But we won’t need this explicit description of~$Ω$, and will deliberately avoid it.
Instead, our argumentation will rely on the following two properties of the set~$𝟙$ and its element~$1$.
\begin{enumerate*}

	\item
		For every set~$X$ and every element~$x$ of~$X$ there exists a unique map~$e_x$ from~$𝟙$ to~$X$ with~$e_x(1) = x$.

	\item
		The only subset of~$𝟙$ containing~$1$ is~$𝟙$.

\end{enumerate*}
In terms of intuition, the second property tells us that \enquote{the subset of~$𝟙$ generated by the element~$1$ is~$𝟙$}.

In the following, let~$X$ be some set.
We will describe how to abstractly construct the usual bijection between~$\Power(X)$ and~$\Set(X, Ω)$, and how to show that this bijection is natural in~$X$.

Suppose first that~$S$ is a subset of~$X$.
For every element~$x$ of~$X$ we can pull back the subset~$S$ of~$X$ along the map~$e_x$ a subset of~$𝟙$.
We get in this way a map
\[
	χ_S \colon X \to Ω \,,
	\quad
	x \mapsto e_x^{-1}(S) \,.
\]
(We observe that~$χ_S(x) = 𝟙$ if~$x ∈ S$ and~$χ_S(x) = ∅$ otherwise, so our abstract construction of the characteristic function~$χ_S$ agrees with the usual explicit definition.)

Suppose now that~$f$ is any map from~$X$ to~$Ω$.
The set~$𝟙$ as a special subset, namely~$𝟙$ itself.
The set~$U ≔ \{ 𝟙 \}$ is then a subset of~$Ω$, which we can pull back along~$f$ to the subset~$f^{-1}(U)$ of~$X$.

We will now check that the above two constructions between subsets of~$X$ and maps from~$X$ to~$Ω$ are mutually inverse, and also natural in~$X$.

Let~$S$ be a subset of~$X$.
We have for every element~$x$ of~$X$ the chain of equivalences
\begin{align}
	\SwapAboveDisplaySkip
	x ∈ S
	&\iff
	e_x(1) ∈ S \notag \\
	&\iff
	1 ∈ e_x^{-1}(S) = χ_S(x) \notag \\
	&\iff
	𝟙 = χ_S(x)
	\label{subset containg 1 is already everything} \\
	&\iff
	x ∈ χ_S^{-1}(U) \notag \,,
\end{align}
and therefore the equality~$S = χ_S^{-1}(U)$.
For the step~\eqref{subset containg 1 is already everything} we used that $χ_S(x)$ is a subset of~$𝟙$ containing~$1$, and that the only such subset is~$𝟙$.

Let now~$f$ be a map from~$X$ to~$Ω$ and let~$S ≔ f^{-1}(U)$.
To show the equality~$f = χ_S$ we need to show that~$f(x) = χ_S(x)$ for every element~$x$ of~$X$.
Both~$f(x)$ and~$χ_S(x)$ are subsets of~$𝟙$, so we need to show that for every element~$p$ of~$𝟙$, we have~$p ∈ f(x)$ if and only if~$p ∈ χ_S(x)$.
The only element of~$𝟙$ is~$1$, and we have the chain of equivalences
\begin{align*}
	\SwapAboveDisplaySkip
	1 ∈ f(x)
	&\iff
	f(x) = 𝟙 \\
	&\iff
	x ∈ S \\
	&\iff
	χ_S(x) = 𝟙 \\
	&\iff
	1 ∈ χ_S(x) \,.
\end{align*}
We have thus shown that indeed~$f = χ_S$.

To show the required naturality, we need to show that for every map
\[
	f \colon X \to Y
\]
the following diagram commutes:
\[
	\begin{tikzcd}
		\Power(Y)
		\arrow{r}[above]{χ_{(\ph)}}
		\arrow{d}[left]{f^{-1}}
		&
		\Set(Y, Ω)
		\arrow{d}[right]{f^*}
		\\
		\Power(X)
		\arrow{r}[above]{χ_{(\ph)}}
		&
		\Set(X, Ω)
	\end{tikzcd}
\]
Let~$S$ be a subset of~$Y$ and let~$x$ be an element of~$X$.
Then~$f ∘ e_x = e_{f(x)}$ because
\[
	(f ∘ e_x)(1) = f( e_x(1) ) = f(x) = e_{f(x)}(1) \,,
\]
and consequently
\begin{align*}
	\SwapAboveDisplaySkip
	f^*(χ_S)(x)
	&=
	(χ_S ∘ f)(x) \\
	&=
	χ_S( f(x) ) \\
	&=
	e_{f(x)}^{-1}(S) \\
	&=
	(f ∘ e_x)^{-1}(S) \\
	&=
	e_x^{-1}( f^{-1}(S) ) \\
	&=
	χ_{f^{-1}(S)}(x) \,.
\end{align*}
This shows that~$f^*(χ_S) = χ_{f^{-1}(S)}$, so that the diagram commutes.

\subsubsection*{Special case: the category $\scat{A}$ has one object}

Suppose now slightly more generally than before that the category~$\scat{A}$ consists of only a single object~$\ast$.
For the monoid~$M ≔ \scat{A}(\ast, \ast)$, the presheaf category~$\hat{\scat{A}}$ is then isomorphic to the category~$\cat{M}$ of right~\sets{$M$}.
For every right~\set{$M$}~$X$ the set~$\Power(X)$ consists of all~\subsets{$M$} of~$X$, and for every homomorphism of right~\sets{$M$}~$f \colon X \to Y$ the induced map~$f^{-1} \colon \Power(Y) \to \Power(X)$ is given by taking preimages in the usual sense.

For every right~\set{$M$}~$X$, the elements of~$X$ correspond to homomorphisms from~$M$ to~$X$.
For the subobject classifier~$Ω$ of~$\cat{M}$ we have therefore the isomorphisms
\[
	\{ \text{elements of~$Ω$} \}
	≅
	\cat{M}(M, Ω)
	≅
	\Power(M) \,.
\]
We therefore define~$Ω$ as follows:
\begin{itemize*}

	\item
		The underlying set of~$Ω$ is~$\Power(M)$.

	\item
		For every element~$m$ of~$M$, left multiplication with~$m$ is a homomorphism~$λ_m \colon M \to M$, and therefore induces a map~$λ_m^{-1} \colon \Power(M) \to \Power(M)$.
		This map is the action of~$m$ on~$Ω$.
		More explicitly, we have
		\begin{equation}
			\label{action of monoids on its right ideals}
			S m
			=
			λ_m^{-1}(S)
			=
			\{ n ∈ M \suchthat λ_m(n) ∈ S \}
			=
			\{ n ∈ N \suchthat m n ∈ S \} \,.
		\end{equation}

\end{itemize*}
We denote the set~\eqref{action of monoids on its right ideals} as~$(S : m)$, in accordance to the notation for quotient ideals in ring theory.
We note that~$m$ is contained in~$S$ if and only if~$1_M$ is contained in~$(S : m)$.

More generally, if~$X$ is any right~\set{$M$},~$S$ is an~\subset{$M$} of~$X$, and~$x$ is an element of~$X$, then we will use the notation~$(S : x)$ for the set~$\{ m ∈ M \suchthat xm \subseteq S \}$, so that
\[
	(S : x) ∋ m \iff S ∋ xm \,.
\]
We note that~$x$ is contained in~$S$ if and only if~$1_M$ is contained in~$(S : x)$, if and only if~$(S : x) = M$.

As in the previous special case, we will try not to work with the explicit elements of~$Ω$.
Instead, we will rely on the following two observations:
\begin{enumerate*}

	\item
		There exists for every right~\set{$M$}~$X$ and every element~$x$ of~$X$ a unique homomorphism~$e_x$ from~$M$ to~$X$ with~$e_x(1_M) = x$.

	\item
		The only~\subset{$M$} of~$M$ that contains~$1_M$ is~$M$ itself.

\end{enumerate*}

Let~$X$ be a right~\set{$M$} and let~$S$ be an~\subset{$M$} of~$X$.
For every element~$x$ of~$X$ we can pull back~$S$ along the homomorphism~$e_x$ to an~\subset{$M$} of~$M$.
We have in this way a set-theoretic map
\[
	χ_S
	\colon
	X \to Ω \,,
	\quad
	x \mapsto e_x^{-1}(S) \,.
\]
This map is more explicitly given by
\[
	χ_S(x)
	=
	e_x^{-1}(S)
	=
	\{ m ∈ M \mid e_x(m) ∈ S \}
	=
	\{ m ∈ M \mid xm ∈ S \}
	=
	(S : x)
\]
for every~$x ∈ X$.
As a by-product of the construction of~$χ_S$, we see that~$(S : x)$ is an~\subset{$M$} of~$M$.

We claim that the map~$χ_S$ is already a homomorphism.
To see this, we observe that~$e_{xm} = e_x ∘ λ_m$, since both~$e_{xm}$ and~$e_x ∘ λ_m$ are homomorphisms and
\[
	e_{xm}(1_M)
	= xm
	= e_x(1_M) m
	= e_x(m)
	= e_x( λ_m(1_M) )
	= (e_x ∘ λ_m)(1_M) \,.
\]
It follows that
\[
	χ_S(xm)
	=
	e_{xm}^{-1}(S)
	=
	(e_x ∘ λ_m)^{-1}(S)
	=
	λ_m^{-1}( e_x^{-1}(S) )
	=
	χ_S(x) m \,.
\]

Suppose conversely that~$f$ is any homomorphism from~$M$ to~$Ω$.
We note that~$M$ contains a very special~\subset{$M$}, namely~$M$ itself.
Every element~$m$ of~$M$ is contained in~$M$ (surprise!), whence~$Mm = (M : m) = M$.
This tells us that~$M$ is a fixed point in~$Ω$.
Equivalently,~$U ≔ \{ M \}$ is an~\subset{$M$} of~$Ω$.
By pulling back~$U$ along~$f$, we arrive at the~\subset{$M$}~$f^{-1}(U)$ of~$X$.

We will now show that the above two constructions between~\subsets{$M$} of~$X$ and homomorphism from~$X$ to~$Ω$ are mutually inverse and natural.

Let~$S$ be an~\subset{$M$} of~$X$.
We have for every element~$x$ of~$X$ the chain of equivalences
\begin{align}
	\SwapAboveDisplaySkip
	x ∈ S
	&\iff
	e_x(1_M) ∈ S \notag \\
	&\iff
	1_M ∈ e_x^{-1}(S) = χ_S(x) \notag \\
	&\iff
	χ_S(x) = M
	\label{subset containg identity is already everything} \\
	&\iff
	x ∈ χ_S^{-1}(U) \notag
\end{align}
and therefore the equality~$S = χ_S^{-1}(U)$.
We have used in the step~\eqref{subset containg identity is already everything} that~$χ_S(x)$ is an~\subset{$M$} of~$X$ containing~$1_M$, and that the only such~subset is~$M$ itself.

Let~$f$ be a homomorphism from~$X$ to~$Ω$.
For the~\subset{$M$}~$S ≔ f^{-1}(U)$ of~$X$ we want to show that~$f = χ_S$.
We need to show that~$f(x) = χ_S(x)$ for every element~$x$ of~$X$.
Both~$f(x)$ and~$χ_S(x)$ are~\subsets{$M$} of~$M$, so we need to show that for every element~$m$ of~$M$ we have~$m ∈ f(x)$ if and only if~$m ∈ χ_S(x)$.
We have
\[
	(f(x) : m) = f(x)m = f(xm) \,,
\]
and therefore the chain of equivalences
\begin{align*}
	m ∈ f(x)
	&\iff
	1_M ∈ (f(x) : m) \\
	&\iff
	1_M ∈ f(xm) \\
	&\iff
	f(xm) = M \\
	&\iff
	xm ∈ S \\
	&\iff
	m ∈ (S : x) \\
	&\iff
	m ∈ χ_S(x) \,.
\end{align*}
This shows that indeed~$f = χ_S$.

It remains to show the naturality of the bijections~$χ_{(\ph)} \colon \Power(X) \to \cat{M}(X, Ω)$.
For this, we need to show that for every homomorphism
\[
	f \colon X \to Y
\]
the following diagram commutes:
\[
	\begin{tikzcd}
		\Power(Y)
		\arrow{r}[above]{χ_{(-)}}
		\arrow{d}[left]{f^{-1}}
		&
		\cat{M}(Y, Ω)
		\arrow{d}[right]{f^*}
		\\
		\Power(X)
		\arrow{r}[above]{χ_{(-)}}
		&
		\cat{M}(X, Ω)
	\end{tikzcd}
\]
Let~$S$ be an~\subset{$M$} of~$Y$ and let~$x$ be an element of~$X$.
Then~$f ∘ e_x = e_{f(x)}$ because
\[
	(f ∘ e_x)(1_M)
	=
	f( e_x(1_M) )
	=
	f(x)
	=
	e_{f(x)}(1_M) \,,
\]
and consequently
\begin{align*}
	\SwapAboveDisplaySkip
	f^*(χ_S)(x)
	&=
	(χ_S ∘ f)(x) \\
	&=
	χ_S(f(x)) \\
	&=
	e_{f(x)}^{-1}(S) \\
	&=
	(f ∘ e_x)^{-1}(S) \\
	&=
	e_x^{-1}( f^{-1}(S) ) \\
	&=
	χ_{f^{-1}(S)}(x) \,.
\end{align*}
This shows that~$f^*(χ_S) = χ_{f^{-1}(S)}$, which shows the required commutativity.



\subsubsection{}

Suppose that~$Ω$ is a subobject classifier for~$\hat{\scat{A}}$.
For every object~$A$ of~$\scat{A}$ we then have the isomorphisms
\[
	Ω(A)
	≅
	\hat{\scat{A}}( \HY_A, Ω )
	≅
	\Sub(\HY_A)
	≅
	\Power(\HY_A) \,.
\]



\subsubsection{}


We define the desired subobject classifier~$Ω$ of~$\hat{\scat{A}}$ as the composite of the Yoneda embedding from~$\scat{A}$ to~$\hat{\scat{A}}$ (which is covariant) and the contravariant power set functor from~$\hat{\scat{A}}$ to~$\Set$.
More explicitly, we have
\[
	Ω(A) = \Power(\HY_A)
\]
for every object~$A$ of~$\scat{A}$,
and for every morphism~$f \colon B \to A$ in~$\scat{A}$ the map
\[
	Ω(f)
	≔
	\HY_f^{-1}
	\colon
	\Power(\HY_A) \to \Power(\HY_B) \,,
	\quad
	S \mapsto \HY_f^{-1}(S) \,.
\]
More explicitly, we have
\begin{align}
	Ω(f)(S)(C)
	&=
	\HY_f^{-1}(S)(C) \notag \\
	&=
	(\HY_f)_C^{-1}( S(C) ) \notag \\
	&=
	(f_*)^{-1}( S(C) ) \notag \\
	&=
	\{ \textstyle g \colon C \to B \suchthat f_*(g) ∈ S(C)  \} \notag \\
	&=
	\{ \textstyle g \colon C \to B \suchthat f ∘ g ∈ S(C)  \}
	\label{explicit formula for action of Omega on morphisms} \,.
\end{align}

We will show in the following that the presheaf~$Ω$ is a subobject classifier for~$\hat{\scat{A}}$, i.e., that~$\hat{\scat{A}}(\ph, Ω) ≅ \Sub$.
We have already seen that~$\Sub ≅ \Power$, so we will show in the following that~$\Power ≅ \hat{\scat{A}}(\ph, Ω)$.
We will use the following two properties of the presheaves~$\HY_A$.
\begin{enumerate*}

	\item
	For every object~$A$ of~$\scat{A}$ and every element~$x$ of~$X(A)$ there exists a unique natural transformation~$ε_x$ from~$\HY_A$ to~$X$ with~$ε_{x, A}(\id_A) = x$.

	\item
		For every object~$A$ of~$\scat{A}$, the only subfunctor of~$\HY_A$ containing~$\id_A$ is~$\HY_A$ itself.

\end{enumerate*}

Let~$X$ be a presheaf on~$\scat{A}$ and let~$S$ be a subfunctor of~$X$.
Let~$A$ be an object of~$\scat{A}$ and let~$x$ be an element of~$X(A)$.
We can use the natural transformation~$ε_x$ to pull back the subfunctor~$S$ of~$X$ to the subfunctor~$ε_x^{-1}(S)$ of~$\HY_A$.
We get in this way a set-theoretic map
\[
	χ_{S, A}
	\colon
	X(A) \to \Power(\HY_A) = Ω(A) \,,
	\quad
	x \mapsto ε_x^{-1}(S) \,.
\]
More explicitly, we have
\begin{align*}
	χ_{S, A}(B)
	&=
	ε_x^{-1}(S)(B) \\
	&=
	ε_{x, B}^{-1}( S(B) ) \\
	&=
	\{ g ∈ \HY_A(B) \mid ε_{x, B}(g) ∈ S(B) \} \\
	&=
	\{ \textstyle g \colon B \to A \mid X(g)(x) ∈ S(B) \} \,.
\end{align*}

The resulting transformation~$χ_S$ from~$X$ to~$Ω$ is natural.
To see this, we need to check that for every morphism
\[
	f \colon B \to A
\]
in~$\scat{A}$ the following diagram commutes:
\[
	\begin{tikzcd}
		X(A)
		\arrow{r}[above]{χ_{S, A}}
		\arrow{d}[left]{X(f)}
		&
		Ω(A)
		\arrow{d}[right]{Ω(f)}
		\\
		X(B)
		\arrow{r}[above]{χ_{S, B}}
		&
		Ω(B)
	\end{tikzcd}
\]
We observe for every element~$x$ of~$X(A)$ that
\[
	ε_{X(f)(x)} = ε_x ∘ \HY_f
\]
since both sides are functors from~$\HY_B$ to~$X$ with
\begin{align*}
	ε_{X(f)(x), B}(\id_B)
	&=
	X(f)(x) \\
	&=
	X(f)( ε_{x, A}(\id_A) ) \\
	&=
	ε_{x, B}\bigl( (\HY_A)(f)(\id_A) \bigr) \\
	&=
	ε_{x, B}( f ) \\
	&=
	ε_{x, B}\bigl( (\HY_f)_B( \id_B ) \bigr) \\
	&=
	\bigl( ε_{x, B} ∘ (\HY_f)_B \bigr)( \id_B ) \\
	&=
	(ε_{x, B} ∘ \HY_f)_B(\id_B) \,.
\end{align*}
It follows for every element~$x$ of~$X(A)$ that
\begin{align*}
	χ_{S, B}( X(f)(x) )
	&=
	ε_{X(f)(x)}^{-1}(S) \\
	&=
	(ε_x ∘ \HY_f)^{-1}(S) \\
	&=
	\HY_f^{-1}( ε_x^{-1}(S) ) \\
	&=
	Ω(f)( χ_S(x) ) \,.
\end{align*}
This shows the required commutativity.

Suppose now that~$α$ is a natural transformation from~$X$ to~$Ω$.
The functor~$\HY_A$ has itself as a subfunctor, and for every morphism~$f \colon B \to A$ in~$\scat{A}$ we have~$\HY_f^{-1}( \HY_A ) = \HY_B$ because
\begin{align*}
	\HY_f^{-1}( \HY_A )(C)
	&=
	(\HY_f)_C^{-1}( \HY_A(C) ) \\
	&=
	\{ g \in \HY_B(C) \mid (\HY_f)_B(g) ∈ \HY_A(C) \} \\
	&=
	\{ g \in \scat{A}(C, B) \mid f ∘ g ∈ \scat{A}(C, A) \} \\
	&=
	\scat{A}(C, B) \\
	&=
	\HY_B(C)
\end{align*}
for every object~$C$ of~$\scat{A}$.
This tells us that the presheaf~$Ω$ has a subfunctor~$U$ given by~$U(A) = \{ \HY_A \}$ for every object~$A$ of~$\scat{A}$.
We can pull back the subfunctor~$U$ of~$Ω$ along the natural transformation~$α$ to the subfunctor~$α^{-1}(U)$ of~$X$.

We will now check that these two constructions between subfunctors of~$X$ and natural transformations from~$X$ to~$Ω$ are mutually inverse, as well as natural.

Let~$S$ be a subfunctor of~$X$.
We observe for every object~$A$ of~$\scat{A}$ and every element~$x$ of~$X(A)$ the chain of equivalences
\begin{align}
	{}&
	x ∈ S(A) \notag \\
	\iff{}&
	ε_{x, A}(\id_A) ∈ S(A) \notag \\
	\iff{}&
	\id_A ∈ ε_{x, A}^{-1}(S(A)) = ε_x^{-1}(S)(A) = χ_{S, A}(x)(A) \notag \\
	\iff{}&
	χ_{S, A}(x) = \HY_A \label{subfunctor containing identity} \\
	\iff{}&
	x ∈ χ_{S, A}^{-1}( \{ \HY_A \} ) = χ_{S, A}^{-1}( U(A) ) = χ_S^{-1}(U)(A) \notag \,.
\end{align}
For the equivalence~\eqref{subfunctor containing identity} we use that~$χ_{S, A}(x)$ is a subfunctor of~$\HY_A$ containing~$\id_A$, and must therefore be all of~$\HY_A$.
This shows that~$S(A) = χ_S^{-1}(U)(A)$ for every object~$A$ of~$\scat{A}$, and therefore~$S = χ_S^{-1}(U)$.

Let now~$α$ be a natural transformation from~$X$ to~$Ω$ and let~$S = α^{-1}(U)$.
We want to check that~$α = χ_S$.
For this, we need to show that for every object~$A$ of~$\scat{A}$ and every element~$x$ of~$X(A)$ we have~$α_A(x) = χ_{S, A}(x)$.
Both~$α_A(x)$ and~$χ_{S, A}(x)$ are elements of~$Ω(A) = \Power(\HY_A)$, and therefore subfunctors of~$\HY_A$.
We hence need to show that~$α_A(x)(B) = χ_{S, A}(x)(B)$ for every object~$B$ of~$\scat{A}$.
We have for every element~$f$ of~$\HY_A(B) = \scat{A}(B, A)$ the chain of equalities
\begin{align}
	{}&
	f ∈ α_A(x)(B) \notag \\
	\iff{}&
	\id_B ∈ Ω(f)( α_A(x) )(B)
	\label{use definition of Omega} \\
	\iff{}&
	Ω(f)( α_A(x) ) = \HY_B
	\label{subfunctor that contains the identity} \\
	\iff{}&
	α_B( X(f)(x) ) = \HY_B
	\label{using naturality of alpha} \\
	\iff{}&
	X(f)(x) ∈ α_B^{-1}(\{\HY_B\}) = α_B^{-1}(U(B)) = α^{-1}(U)(B) = S(B) \notag
	\\
	\iff{}&
	ε_{x, B}(f) ∈ S(B)
	\label{reinserting epsilon}
	\\
	\iff{}&
	f ∈ ε_{x, B}^{-1}(S(B)) = ε_x^{-1}(S)(B) = χ_S(x)(B) \notag \,,
\end{align}and thus~$α_A(x)(B) = χ_S(x)(B)$.
For~\eqref{use definition of Omega} we use the formula~\eqref{explicit formula for action of Omega on morphisms}.
The equality~\eqref{subfunctor containing identity} holds because~$Ω(f)( α_A(x) )$ is a subfunctor of~$\HY_B$ that contains~$\id_B$, and the only such subfunctor is~$\HY_B$ itself.
For~\eqref{using naturality of alpha} we use the naturality of~$α$, and for~\eqref{reinserting epsilon} we use that~$X(f)(x)$ is precisely~$ε_{x, B}(f)$.

To show the required naturality we need to check that for every natural transformation
\[
	α \colon X \to Y
\]
between presheaves on~$\scat{A}$ the following diagram commutes:
\[
	\begin{tikzcd}
		\Power(Y)
		\arrow{r}[above]{χ_{(\ph)}}
		\arrow{d}[left]{α^{-1}}
		&
		\hat{\scat{A}}(Y, Ω)
		\arrow{d}[right]{α^*}
		\\
		\Power(X)
		\arrow{r}[above]{χ_{(\ph)}}
		&
		\hat{\scat{A}}(X, Ω)
	\end{tikzcd}
\]
Let~$S$ be a subfunctor of~$Y$, let~$A$ be an object of~$\scat{A}$ and let~$x ∈ X(A)$, then
\begin{align*}
	α^*( χ_S )_A(x)
	&=
	(χ_S ∘ α)_A(x) \\
	&=
	(χ_{S, A} ∘ α_A)(x) \\
	&=
	χ_{S, A}( α_A(x) ) \\
	&=
	ε_{α_A(x)} ^{-1}(S) \\
	&=
	( α ∘ ε_x )^{-1}(S) \\
	&=
	ε_x^{-1}( α^{-1}(S) ) \\
	&=
	χ_{α^{-1}(S), A}(x) \,,
\end{align*}
therefore~$α^*(χ_S)_A = χ_{α^{-1}(S), A}$, and thus~$α^*( χ_S ) = χ_{α^{-1}(S)}$.
This shows the required commutativity.



\subsubsection{}

We know from Corollary~6.2.11 that~$\hat{\scat{A}}$ has all small limits.
This entails that~$\hat{\scat{A}}$ has finite limits.
We have seen in Theorem~6.3.20 that~$\hat{\scat{A}}$ is cartesian closed.
We have seen in this exercise that~$\hat{\scat{A}}$ is well-powered and has a subobject classifier.

This shows altogether that~$\hat{\scat{A}}$ is a topos.
