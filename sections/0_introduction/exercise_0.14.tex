\subsection{}


\subsubsection{}

Let~$P \defined X \times Y$, let~$p_1$ be the projection map from~$P$ to~$X$ into the first coordinate, and let~$p_2$ be the projection map from~$P$ to~$Y$ into the second coordinate.

For every pair~$(f_1, f_2)$ of set-theoretic maps
\[
	f_1 \colon V \to X \,,
	\quad
	f_2 \colon V \to Y
\]
there exists a unique set-theoretic map~$f$ from~$X \times Y$ such that~$f_1 = p_1 \circ f$ and~$f_2 = p_2 \circ f$, namely the map
\[
	f
	\colon
	V
	\to
	P \,,
	\quad
	v
	\mapsto
	( f_1(v), f_2(v) ) \,.
\]
If both~$f_1$ and~$f_2$ are linear, then it follows that~$f$ is also linear.





\subsubsection{}

There exists by a unique linear map~$i$ from~$P$ to~$P'$ such that~$p'_1 \circ i = p_1$ and~$p'_2 \circ i = p_2$, and a unique linear map~$j$ from~$P'$ to~$P$ such that~$p_1 \circ j = p'_1$ and~$p_2 \circ j = p'_2$.

There exists by assumption a unique linear map~$f$ from~$P$ to~$P$ such that~$p_1 \circ f = p_1$ and~$p_2 \circ f = p_2$.
Both~$\id_P$ and~$j \circ i$ satisfy this condition.
It therefore follows that~$j \circ i = \id_P$.

There exists by assumption a unique linear map~$g$ from~$P'$ to~$P'$ such that~$p'_1 \circ g = p'_1$ and~$p'_2 \circ g = p'_2$.
Both~$\id_{P'}$ and~$i \circ j$ satisfy this condition.
It therefore follows that~$i \circ j = \id_{P'}$.

This shows that the linear maps~$i$ and~$j$ are mutually inverse isomorphisms.





\subsubsection{}

Let~$Q = X \oplus Y$ and let~$q_1$ and~$q_2$ be the linear maps given by
\begin{alignat*}{2}
	q_1
	\colon
	X
	&\to
	Q \,,
	&\quad
	x
	&\mapsto
	(x, 0)
\shortintertext{and}
	q_2
	\colon
	Y
	&\to
	Q \,,
	&\quad
	y
	&\mapsto
	(0, y) \,.
\end{alignat*}

Let~$f$ be a linear map from~$Q$ to~$V$ such that~$f \circ q_1 = f_1$ and~$f \circ q_2 = f_2$.
Then
\begin{align*}
	f( (x,y) )
	&=
	f( (x, 0) + (0, y) )
	\\
	&=
	f( (x, 0) ) + f( (0, y) )
	\\
	&=
	f( q_1(x) ) + f( q_2(y) )
	\\
	&=
	(f \circ q_1)(x) + (f \circ q_2)(y)
	\\
	&=
	f_1(x) + f_2(y)
\end{align*}
for all~$x \in X$ and~$y \in Y$.
This shows that that~$f$ in uniquely determined by~$f_1$ and~$f_2$.
This shows the desired uniqueness of~$f$.

Consider on the other hand the map
\[
	f
	\colon
	Q
	\to
	V \,,
	\quad
	(x, y)
	\mapsto
	f_1(x) + f_2(y) \,.
\]
This map is linear, and it satisfies
\[
	(f \circ q_1)(x)
	=
	f( q_1(x) )
	=
	f( (x, 0) )
	=
	f_1(x) + f_2(0)
	=
	f_1(x) + 0
	=
	f_1(x)
\]
as well as
\[
	(f \circ q_2)(y)
	=
	f( q_2(y) )
	=
	f( (0, y) )
	=
	f_1(0) + f_2(y)
	=
	0 + f_2(y)
	=
	f_2(y) \,.
\]
This shows the desired existence of the linear map~$f$.





\subsubsection{}

Let~$(Q', q'_1, q'_2)$ be another cocone, such that for every vector space~$V$ and every two linear maps
\[
	f_1
	\colon
	X
	\to
	V \,,
	\quad
	f_2
	\colon
	Y
	\to
	V
\]
there exists a unique linear map~$f$ from~$Q'$ to~$V$ such that~$f \circ q'_1 = f_1$ and~$f \circ q'_2 = f_2$.

It follows from the previous part of this exercise that there exists a uniqui linear map~$i$ from~$Q$ to~$Q'$ such that~$i \circ q_1 = q'_1$ and~$i \circ q_2 = q'_2$.
It follows from the above assumption on~$(Q', q'_1, q'_2)$ that there exists a unique linear map~$j$ from~$Q'$ to~$Q$ such that~$j \circ q'_1 = q_1$ and~$j \circ q'_2 = q_2$.

There exists by the previous part of this exercise a unique linear map~$f$ from~$Q$ to~$Q$ such that~$f \circ q_1 = q_1$ and~$f \circ q_2 = q_2$.
Both~$\id_Q$ end~$j \circ i$ satisfy this defining property of~$f$, which shows that~$j \circ i = \id_Q$.

There exists by assumption on the triple~$(Q', q'_1, q'_2)$ a unique linear map~$g$ from~$Q'$ to~$Q'$ such that~$g \circ q'_1 = q'_1$ and~$g \circ q'_2 = q'_2$.
Both~$\id_{Q'}$ and~$i \circ j$ satisfy the defining property of~$g$, which shows that~$i \circ j = \id_{Q'}$.

This shows that the linear maps~$i$ and~$j$ are mutually inverse isomorphisms.








