\subsection{}


\subsubsection{}

Let~$P$ be the vector space~$X × Y$, let~$p_1$ be the projection map from~$P$ to~$X$ into the first coordinate, and let~$p_2$ be the projection map from~$P$ to~$Y$ into the second coordinate.

For every pair~$(f_1, f_2)$ of set-theoretic maps
\[
	f_1 \colon V \to X \,,
	\quad
	f_2 \colon V \to Y \,,
\]
there exists a unique set-theoretic map
\[
	f \colon V \to X × Y
\]
such that both~$f_1 = p_1 ∘ f$ and~$f_2 = p_2 ∘ f$, namely the map
\[
	f
	\colon
	V \to P \,,
	\quad
	v \mapsto ( f_1(v), f_2(v) ) \,.
\]
The map~$f$ is linear if and only if it is linear in both coordinates.
In other words,~$f$ is linear if and only if both~$f_1$ and~$f_2$ are linear.
It follows that the cone~$(P, p_1, p_2)$ satisfies the desired universal property.



\subsubsection{}

By assumption on the cone~$(P', p'_1, p'_2)$, there exists a unique linear map~$i$ from~$P$ to~$P'$ such that both
\[
	p'_1 ∘ i = p_1
	\quad\text{and}\quad
	p'_2 ∘ i = p_2 \,.
\]
There exists similarly a unique linear map~$j$ from~$P'$ to~$P$ such that both
\[
	p_1 ∘ j = p'_1
	\quad\text{and}\quad
	p_2 ∘ j = p'_2 \,.
\]

By the universal property of the cone~$(P, p_1, p_2)$, there exists a unique linear map~$f$ from~$P$ to~$P$ such that both~$p_1 ∘ f = p_1$ and~$p_2 ∘ f = p_2$.
However, both~$\id_P$ and~$j ∘ i$ satisfy this defining condition of the map~$f$.
We have therefore both~$f = \id_P$ and~$f = j ∘ i$, and thus~$j ∘ i = \id_P$.

We find in the same way that also~$i ∘ j = \id_{P'}$.

This shows that the linear maps~$i$ and~$j$ are mutually inverse isomorphisms of vector spaces.



\subsubsection{}

Let~$Q$ be the vector space~$X ⊕ Y$, and let~$q_1$ and~$q_2$ be the two linear maps
\begin{alignat*}{2}
	q_1
	\colon
	X
	&\to
	Q \,,
	&\quad
	x
	&\mapsto
	(x, 0)
\shortintertext{and}
	q_2
	\colon
	Y
	&\to
	Q \,,
	&\quad
	y
	&\mapsto
	(0, y) \,.
\end{alignat*}

Let~$(V, f_1, f_2)$ be a cocone.
We show in the following that there exists a unique linear map~$f$ from~$Q$ to~$V$ with both~$f ∘ q_1 = f_1$ and~$f ∘ q_2 = f_2$.

To show the required uniqueness, let~$f$ be such a linear map.
Then
\begin{align*}
	f( (x,y) )
	&=
	f( (x, 0) + (0, y) )
	\\
	&=
	f( (x, 0) ) + f( (0, y) )
	\\
	&=
	f( q_1(x) ) + f( q_2(y) )
	\\
	&=
	(f ∘ q_1)(x) + (f ∘ q_2)(y)
	\\
	&=
	f_1(x) + f_2(y)
\end{align*}
for all~$x ∈ X$,~$y ∈ Y$.
This shows that the linear map~$f$ is uniquely determined by the composites~$f_1$ and~$f_2$, which shows the desired uniqueness of~$f$.

To show the existence of the linear map~$f$, we define it as
\[
	f
	\colon
	Q
	\to
	V \,,
	\quad
	(x, y)
	\mapsto
	f_1(x) + f_2(y) \,.
\]
The map~$f$ is linear since it can be expressed as
\[
	f = f_1 ∘ p_1 + f_2 ∘ p_2 \,,
\]
with~$p_1$,~$p_2$,~$f_1$ and~$f_2$ being linear.
The map~$f$ also satisfies both
\[
	(f ∘ q_1)(x)
	=
	f( q_1(x) )
	=
	f( (x, 0) )
	=
	f_1(x) + f_2(0)
	=
	f_1(x) + 0
	=
	f_1(x)
\]
and
\[
	(f ∘ q_2)(y)
	=
	f( q_2(y) )
	=
	f( (0, y) )
	=
	f_1(0) + f_2(y)
	=
	0 + f_2(y)
	=
	f_2(y) \,.
\]
This shows the desired existence of the linear map~$f$.



\subsubsection{}

Let~$(Q', q'_1, q'_2)$ be a cocone such that for every cocone~$(V, f_1, f_2)$ there exists a unique linear map~$f$ from~$Q'$ to~$V$ with~$f ∘ q'_1 = f_1$ and~$f ∘ q'_2 = f_2$.

It follows from the previous part of this exercise that there exists a unique linear map~$i$ from~$Q$ to~$Q'$ such that~$i ∘ q_1 = q'_1$ and~$i ∘ q_2 = q'_2$.
It follows from the above assumption on~$(Q', q'_1, q'_2)$ that there exists a unique linear map~$j$ from~$Q'$ to~$Q$ such that~$j ∘ q'_1 = q_1$ and~$j ∘ q'_2 = q_2$.

There exists by the previous part of this exercise a unique linear map~$f$ from~$Q$ to~$Q$ such that~$f ∘ q_1 = q_1$ and~$f ∘ q_2 = q_2$.
Both~$\id_Q$ end~$j ∘ i$ satisfy this defining property of~$f$, which shows that~$j ∘ i = \id_Q$.
We find in the same way that also~$i ∘ j = \id_{Q'}$.

This shows that the linear maps~$i$ and~$j$ are mutually inverse isomorphisms of vector spaces.
