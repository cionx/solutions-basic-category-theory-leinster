\subsection{}

We recall the following statement from topology:

\begin{proposition}
	\label{continuity via open cover}
	Let~$X$ be a topological space and let~$U$ and~$V$ be two open subsets of~$X$ such that~$X = U ∪ V$.
	Let~$Y$ be another topological space and let~$f$ be a set-theoretic map from~$X$ to~$Y$.
	The map~$f$ is continuous if and only if both restrictions~$\restrict{f}{U}$ and~$\restrict{f}{V}$ are continuous.
\end{proposition}

We also recall the following statement from naive set theory:

\begin{proposition}
	\label{glueing of maps}
	Let~$X$ be a set and let~$U$ and~$V$ be two subsets of~$X$ with~$X = U ∪ V$.
	Let~$Y$ be another set and let
	\[
		f \colon U \to Y
		\quad\text{and}\quad
		g \colon V \to Y
	\]
	be two maps that agree on the intersection~$U ∩ V$, i.e., such that~$\restrict{f}{U ∩ V} = \restrict{g}{U ∩ V}$.
	There exists a unique map~$h$ from~$X$ to~$Y$ with~$\restrict{h}{U} = f$ and~$\restrict{h}{V} = g$.
\end{proposition}

We consider now the situation of the diagram~(0.3):
The commutativity of the outer diagram
\[
	\begin{tikzcd}[row sep = large]
		U ∩ V
		\arrow[hook]{r}[above]{i}
		\arrow[hook]{d}[left]{j}
		&
		U
		\arrow[bend left]{ddr}[above right]{f}
		&
		{}
		\\
		V
		\arrow[bend right]{drr}[below left]{g}
		&
		{}
		&
		{}
		\\
		{}
		&
		{}
		&
		Y
	\end{tikzcd}
\]
means precisely that the functions~$f$ and~$g$ agree on the intersection~$U ∩ V$.
It follows from \cref{glueing of maps} that there exists a unique set-theoretic map~$h$ from~$X$ to~$Y$ with~$\restrict{h}{U} = f$ and~$\restrict{h}{V} = g$.
This means precisely that the map~$h$ makes the diagram
\[
	\begin{tikzcd}[row sep = large]
		U ∩ V
		\arrow[hook]{r}[above]{i}
		\arrow[hook]{d}[left]{j}
		&
		U
		\arrow[bend left]{ddr}[above right]{f}
		\arrow{d}[left]{j'}
		&
		{}
		\\
		V
		\arrow[bend right]{drr}[below left]{g}
		\arrow{r}[above]{i'}
		&
		X
		\arrow[dashed]{dr}[above right]{h}
		&
		{}
		\\
		{}
		&
		{}
		&
		Y
	\end{tikzcd}
\]
It follows from \cref{continuity via open cover} that this map~$h$ is continuous.
