\subsection{}



\subsubsection{}

Let~$ϕ$ be a ring homomorphism from~$\Integer[x]$ to~$R$.
Let~$y ≔ ϕ(x)$.
For any element~$p$ of~$\Integer[x]$, which is of the form~$p = ∑_{n ≥ 0} a_n x^n$ for suitable coefficients~$a_n ∈ \Integer$, we have
\[
	ϕ(p)
	=
	ϕ\Biggl( ∑_{n ≥ 0} a_n x^n \Biggr)
	=
	∑_{n ≥ 0} a_n ϕ(x)^n
	=
	∑_{n ≥ 0} a_n y^n \,.
\]
This shows that the homomorphism~$ϕ$ is uniquely determined by its action on the element~$x$ of~$\Integer[x]$.

Let now~$y$ be an element of~$R$.
The map
\[
	ϕ
	\colon
	\Integer[x]
	\to
	R \,,
	\quad
	∑_{n ≥ 0} a_n x^n
	\mapsto
	∑_{n ≥ 0} a_n y^n
\]
is a homomorphism of rings that maps the variable~$x$ onto the element~$y$.
This shows the desired existence.



\subsubsection{}

There exists by the previous part of the exercise a unique homomorphism of rings~$ι$ from~$\Integer[x]$ to~$A$ with~$ι(x) = a$, and there exists by assumption a unique homomorphism of rings~$π$ from~$A$ to~$\Integer[x]$ with~$π(a) = x$.

There exists by the previous part of the exercise a unique homomorphism of rings from~$\Integer[x]$ to~$\Integer[x]$ that maps the element~$x$ onto itself.
But both~$\id_{\Integer[x]}$ and the composite~$π ∘ ι$ are homomorphisms of rings that satisfy this condition.
It follows that~$π ∘ ι = \id_{\Integer[x]}$.

There exist by assumption a unique homomorphism of rings from~$A$ to~$A$ that maps the element~$a$ onto itself.
But both~$\id_A$ and the composite~$ι ∘ π$ are homomorphisms of rings that satisfy this condition.
It follows that~$ι ∘ π = \id_A$.

We have now shown that the two homomorphisms of rings~$ι$ and~$π$ are mutually inverse.
This means that~$ι$ is an isomorphism with inverse~$π$.
