\subsection{}



\subsubsection{}

Let~$\phi$ be a ring homomorphism from~$\Integer[x]$ to~$R$.
Let~$y \defined \phi(x)$.
For any element~$p$ of~$\Integer[x]$, which is of the form~$p = \sum_{n >= 0} a_n x^n$ for suitable coefficients~$a_n \in \Integer$, we have
\[
	\phi(p)
	=
	\phi\Biggl( \sum_{n \geq 0} a_n x^n \Biggr)
	=
	\sum_{n \geq 0} a_n \phi(x)^n
	=
	\sum_{n \geq 0} a_n y^n \,.
\]
This shows that the homomorphism~$\phi$ is uniquely determined by its action on the element~$x$ of~$\Integer[x]$.

Let now~$y$ be an element of~$R$.
The map
\[
	\phi
	\colon
	\Integer[x]
	\to
	R \,,
	\quad
	\sum_{n \geq 0} a_n x^n
	\mapsto
	\sum_{n \geq 0} a_n y^n
\]
is a homomorphism of rings that maps the variable~$x$ onto the element~$y$.
This shows the desired existence.



\subsubsection{}

There exists by the previous part of the exercise a unique homomorphism of rings~$\iota$ from~$\Integer[x]$ to~$A$ with~$\iota(x) = a$.
There exists by assumption a unique homomorphism of rings~$\pi$ from~$A$ to~$\Integer[x]$ with~$\pi(a) = x$.

There exists by the previous part of the exercise a unique homomorphism of rings from~$\Integer[x]$ to~$\Integer[x]$ that maps the element~$x$ onto itself.
But both~$\id_{\Integer[x]}$ and the composite~$\pi \circ \iota$ are homomorphisms of rings that satisfy this condition.
It follows that~$\pi \circ \iota = \id_{\Integer[x]}$.

There exist by assumption a unique homomorphism of rings from~$A$ to~$A$ that maps the element~$a$ onto itself.
But both~$\id_A$ and the composite~$\iota \circ \pi$ are homomorphisms of rings that satisfy this condition.
It follows that~$\iota \circ \pi = \id_A$.

We have now shown that the two homomorphisms of rings~$\iota$ and~$\pi$ are mutually inverse.
This means that~$\iota$ is an isomorphism with inverse~$\pi$.


