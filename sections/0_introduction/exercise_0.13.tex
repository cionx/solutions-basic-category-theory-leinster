\subsection{}



\subsubsection{}

Let~$ϕ$ be a homomorphism of rings from~$ℤ[x]$ to~$R$.
Let~$y$ be~$ϕ(x)$.
For any element~$p$ of~$ℤ[x]$, which is of the form~$p = ∑_{n ≥ 0} a_n x^n$ for suitable coefficients~$a_n$, we have
\[
	ϕ(p)
	=
	ϕ\Biggl( ∑_{n ≥ 0} a_n x^n \Biggr)
	=
	∑_{n ≥ 0} a_n ϕ(x)^n
	=
	∑_{n ≥ 0} a_n y^n \,.
\]
This shows that the homomorphism~$ϕ$ is uniquely determined by its action on the element~$x$ of~$ℤ[x]$.
This in turn shows the desired uniqueness.

Let now~$y$ be an arbitrary element of~$R$.
The map
\[
	ϕ
	\colon
	ℤ[x]
	\to
	R \,,
	\quad
	∑_{n ≥ 0} a_n x^n
	\mapsto
	∑_{n ≥ 0} a_n y^n
\]
is a homomorphism of rings that maps the variable~$x$ onto the element~$y$.
This shows the desired existence.



\subsubsection{}

There exists by the previous part of the exercise a unique homomorphism of rings~$φ$ from~$ℤ[x]$ to~$A$ with~$φ(x) = a$, and there exists similarly a unique homomorphism of rings~$ψ$ from~$A$ to~$ℤ[x]$ with~$ψ(a) = x$.

There exists by the previous part of the exercise a unique homomorphism of rings from~$ℤ[x]$ to~$ℤ[x]$ that maps the element~$x$ onto itself.
But both~$\id_{ℤ[x]}$ and the composite~$ψ ∘ φ$ are homomorphisms of rings that satisfy this condition.
It follows that~$ψ ∘ φ = \id_{ℤ[x]}$.
We find in the same way that also~$φ ∘ ψ = \id_A$.

We have shown that the two homomorphisms~$φ$ and~$ψ$ are mutually inverse.
This means that~$φ$ is an isomorphism whose inverse is given by the homomorphism~$ψ$.
