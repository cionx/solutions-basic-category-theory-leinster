\subsection{}

The author suspects that the result will be conjucation.

Let~$\Gcat$ be the one-object category corresponding to~$G$.
Let~$g$ and~$h$ be two elements of the group~$G$ and let~$φ$ and~$ψ$ be the corresponding homomorphism of groups from~$\Integer$ to~$G$, given by
\[
	φ(n) ≔ g^n
	\quad\text{and}\quad
	ψ(n) ≔ h^n
\]
for every integer~$n$.
The corresponding functors~$Φ$ and~$Ψ$ are isomorphic if and only if there exists a natural transformation from~$Φ$ to~$Ψ$ since every morphism in~$\Gcat$ is already an isomorphism (and every natural transformation from~$Φ$ to~$Ψ$ is therefore a natural isomorphism by Lemma~1.3.11).

There exists a natural transformation from~$Φ$ to~$Ψ$ if and only if there exists a morphism~$k$ in~$\Gcat$, i.e. an element of~$G$, such that the following square diagram commutes for every integer~$n$:
\[
	\begin{tikzcd}
		Φ(\ast)
		\arrow{r}[above]{Φ(n)}
		\arrow{d}[left]{k}
		&
		Φ(\ast)
		\arrow{d}[right]{k}
		\\
		Ψ(\ast)
		\arrow{r}[above]{Ψ(n)}
		&
		Ψ(\ast)
	\end{tikzcd}
\]
We hence need that~$k ∘ Φ(n) = Ψ(n) ∘ k$;
equivalently
\[
	k g^n = h^n k \,.
\]
For~$n = 1$ this gives us the necessary condition~$k g k^{-1} = h$.
This condition is also sufficient because the cases~$n ≠ 1$ follow from this.

We hence find that natural transformations from~$Φ$ to~$Ψ$ correspond bijectively to elements~$k$ of~$G$ such that~$k g k^{-1} = h$.
In follows overall that the two group elements~$g$ and~$h$ define isomorphic functors if and only if they are conjucated.





