\subsection{}

Suppose that such a functor~$\Center$ would exist.
Let~$H$ be the group with two elements and let~$\symm_3$ be the symmetric group on three letters.
Every transposition in~$\symm_3$ results in a nontrivial group homomorphism~$f$ from~$H$ to~$\symm_3$, and there exist a nontrivial group homomorphism~$g$ from~$\symm_3$ to~$H$ that assigns each permutation its sign.
The composite~$g \circ f$ is the identity on~$H$, so that we have the following commutative diagram:
\[
	\begin{tikzcd}
		{}
		&
		\symm_3
		\arrow{dr}[above right]{g}
		&
		{}
		\\
		H
		\arrow{ur}[above left]{f}
		\arrow{rr}[above]{\id_H}
		&
		{}
		&
		H
	\end{tikzcd}
\]
By applying the functor~$\Center$ we arrive at the following commutative diagram:
\[
	\begin{tikzcd}
		{}
		&
		\Center(\symm_3)
		\arrow{dr}[above right]{\Center(g)}
		&
		{}
		\\
		\Center(H)
		\arrow{ur}[above left]{\Center(f)}
		\arrow{rr}[above]{\id_{\Center(H)}}
		&
		{}
		&
		\Center(H)
	\end{tikzcd}
\]
The group~$H$ is abelian and the center of the symmetric group~$\symm_3$ it trivial.
The above commutative diagram can therefore be more explicitely be rewritten as
\[
	\begin{tikzcd}
		{}
		&
		1
		\arrow{dr}
		&
		{}
		\\
		H
		\arrow{ur}
		\arrow{rr}[above]{\id_H}
		&
		{}
		&
		H
	\end{tikzcd}
\]
It follow from the commutativity of this last diagram that the identity of~$H$ is the trivial group homomorphism.
But this is false because the group~$H$ in nontrivial.

