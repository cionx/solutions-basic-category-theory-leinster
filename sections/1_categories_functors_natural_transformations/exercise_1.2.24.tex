\subsection{}

Suppose that such a functor~$\Center$ were to exist.
The symmetric group~$\symm_3$ contains a transposition, which corresponds to a non-trivial homomorphism of groups~$f$ from~$ℤ/2$ to~$\symm_3$.
There also exists a non-trivial homomorphism of groups~$g$ from~$\symm_3$ to~$ℤ/2$, which is given by
\[
	g(σ)
	=
	\begin{cases*}
		0 & \text{if~$σ$ is even}, \\
		1 & \text{if~$σ$ is odd},
	\end{cases*}
\]
for every permutation~$σ$ in~$\symm_3$.
The composite~$g ∘ f$ is the identity on~$ℤ/2$, whence we have the following commutative diagram:
\[
	\begin{tikzcd}
		{}
		&
		\symm_3
		\arrow{dr}[above right]{g}
		&
		{}
		\\
		ℤ/2
		\arrow{ur}[above left]{f}
		\arrow{rr}[above]{\id_{ℤ/2}}
		&
		{}
		&
		ℤ/2
	\end{tikzcd}
\]
By applying the functor~$\Center$ to this diagram, we arrive at the following commutative diagram:
\[
	\begin{tikzcd}
		{}
		&
		\Center(\symm_3)
		\arrow{dr}[above right]{\Center(g)}
		&
		{}
		\\
		\Center(ℤ/2)
		\arrow{ur}[above left]{\Center(f)}
		\arrow{rr}[above]{\id_{\Center(ℤ/2)}}
		&
		{}
		&
		\Center(ℤ/2)
	\end{tikzcd}
\]
The group~$ℤ/2$ is abelian and the center of the symmetric group~$\symm_3$ it trivial.
The above commutative diagram can therefore be rewritten as follows:
\[
	\begin{tikzcd}
		{}
		&
		1
		\arrow{dr}
		&
		{}
		\\
		ℤ/2
		\arrow{ur}
		\arrow{rr}[above]{\id_{ℤ/2}}
		&
		{}
		&
		ℤ/2
	\end{tikzcd}
\]
It follows from the commutativity of this last diagram that the identity homomorphism of~$ℤ/2$ is trivial.
But this is false because the group~$ℤ/2$ is non-trivial.
