\subsection{}



\subsubsection{}

\begin{description}
	
	\item[Example~1.2.3~(a)]
		The functor~$U$ is faithful.
		But it is not full:
		Let~$G$ be the trivial group and let~$H$ be the two-element group~$\Integer / 2$.
		The map~$U(G) \to U(H)$ given by~$1 \mapsto \class{1}$ is not a homomorphism of groups and is therefore not contained in the image of~$U$.

	\item[Example~1.2.3~(b)]
		This functor, that we will denote by~$U$, is faithful.
		But it is not full:
		Let~$R$ be the zero ring and let~$S$ be the ring of integers~$\Integer$.
		Let~$f$ be the map from~$U(R)$ to~$U(\Integer)$ given by~$0 \mapsto 0$.
		This map is not a homomorphism of rings and is therefore not contained in the image of~$U$.

	\item[Example~1.2.3~(c)]
		This functor is faithful.
		But it is not full:
		this can be seen from the same counterexample as for the previous functor.

	\item[Example~1.2.3~(d)]
		This functor is both faithful and full since~$\Ab$ is a full subcategory of~$\Grp$.

	\item[Example~1.2.4~(a)]
		The functor~$F$ is faithful:
		Let~$S$ and~$T$ be two groups and let~$f$ be a map from~$S$ to~$T$.
		The induced homomorphism of groups~$F(f)$ satisfies the condition
		\[
			F(f)(s)
			=
			f(s)
		\]
		for every~$s ∈ S$.
		The original map~$f$ is therefore uniquely determined by its induced homomorphism of groups~$F(f)$.

		But the functor~$F$ is not full:
		Let~$S$ be a non-empty set and let~$T$ be the empty set.
		There exist no map from~$S$ to~$T$, but there exist a homomorphism of groups from~$F(S)$ to~$F(T)$, namely the trivial homomorphism.

	\item[Example~1.2.4~(b)]
		This functor is faithful, but not full.
		This can be argued as in the previous example.

	\item[Example~1.2.4~(c)]
		This functor is faithful, but not full.
		This can be argued as in the previous example.

	\item[Example~1.2.5~(a)]
		The functor~$\fgroup_1$ is not faithful.
		To see this, we consider the pointed topological space~$X ≔ (\Real, 0)$.
		The group~$\fgroup_1(\Real, 0)$ is trivial, so there exists precisely one homomorphism of groups from~$\fgroup_1(\Real, 0)$ to~$\fgroup_1(\Real, 0)$, namely the trivial one.
		But there exist many more continuous maps from~$(\Real, 0)$ to~$(\Real, 0)$.%
		\footnote{
			Namely,~$2^{\aleph_0}$ many.
		}
		
		The functor~$\fgroup_1$ as also not full.
		The author doesn’t know a good (counter)example for this, and refers to \cite{stackexchange_pi_1_not_full} until further notice.

	\item[Example~1.2.5~(b)]
		These functors are neither faithful nor full.

	\item[Example~1.2.7]
		The functor~$F$ is faithful if and only if the corresponding homomorphism of groups is injective, and the functor~$F$ is full if and only if the corresponding homomorphism of groups is surjective.

	\item[Example~1.2.8]
		This functor is faithful if and only if the corresponding~\set{$G$}, respectively representation, of~$G$ is faithful.

	\item[Example~1.2.9]
		The functor is faithful.
		It is full if and only if it follows for any two elements~$a$ and~$a'$ of~$\cat{A}$ from~$f(a) ≤ f(a')$ that also~$a ≤ a'$.

	\item[Example~1.2.11]
		The functor~$C$ is not faithful.
		To see this, let~$X = \{ x \}$ be the one-point topological space and let~$Y= \{ y_1, y_2 \}$ be the two-point codiscrete topology space.
		Both~$C(X)$ and~$C(Y)$ consist only of constant functions, and the two continuous maps from~$X$ to~$Y$ induce the same homomorphism of rings from~$C(Y)$ to~$C(X)$.
		
		The functor~$C$ ought to be non-full, but the author doesn’t know a counterexample.

	\item[Example~1.2.12]

		The functor~$(\ph)^*$ is faithful.
		To see this, let~$V$ and~$W$ be two~\vectorspaces{$\kf$} and let~$f_1$ and~$f_2$ be two distinct linear maps from~$V$ to~$W$.
		There exists by assumption a vector~$v$ in~$V$ for which the two vectors~$w_1 ≔ f_1(v)$ and~$w_2 ≔ f_2(v)$ in~$W$ are distinct.
		It further follows that there exists some linear functional~$w^*$ in~$W^*$ with~$w^*(w_1) ≠ w^*(w_2)$.
		It follows that
		\begin{align*}
			f_1^*(w^*)(v)
			&=
			(w^* ∘ f_1)(v)
			\\
			&=
			w^*( f_1(w) )
			\\
			&=
			w^*(w_1)
			\\
			&≠
			w^*(w_2)
			\\
			&=
			w^*( f_2(w) )
			\\
			&=
			(w^* ∘ f_2)(v)
			\\
			&=
			f_2^*(w^*)(v) \,.
		\end{align*}
		This shows that~$f_1^*(w^*) ≠ f_2^*(w^*)$, and therefore~$f_1^* ≠ f_2^*$.
		
		The functor~$(\ph)^*$ is not full.
		We observe that for any two~\vectorspaces{$\kf$}~$V$ and~$W$ the map
		\[
			D
			\colon
			\Hom_{\kf}(V, W) \to \Hom_{\kf}(W^*, V^*) \,,
			\quad
			f \mapsto f^*
		\]
		is linear.
		For~$V = \kf$ we have~$\Hom_{\kf}(V, W) = \Hom_{\kf}(\kf, W) ≅ W$ but
		\[
			\Hom_{\kf}(W^*, V^*)
			=
			\Hom_{\kf}(W^*, \kf^*)
			≅
			\Hom_{\kf}(W^*, \kf)
			=
			W^{**} \,.
		\]
		If we choose~$W$ to be infinite-dimensional, then it follows that~$D$ cannot be surjective.
\end{description}



\subsubsection{}

Let~$I_1$ be the category given by two objects~$x$ and~$y$ and one non-identity morphism~$f$:
\[
	I_1
	\colon
	\begin{tikzcd}
		x
		\arrow{r}[above]{f}
		&
		y
	\end{tikzcd}
\]
Let similarly~$I_2$ be the category given by two objects~$x$ and~$y$ and two parallel non-identity morphisms~$f$ and~$g$:
\[
	I_2
	\colon
	\begin{tikzcd}
		x
		\arrow[yshift= 0.5ex]{r}[above]{f}
		\arrow[yshift=-0.5ex]{r}[below]{g}
		&
		y
	\end{tikzcd}
\]
The identity functor of~$I_1$ is both full and faithful.
The unique functor~$F$
\[
	F
	\colon
	I_1 \to I_2
	\quad\text{with}\quad
	F(x) = x \,,
	\quad
	F(y) = y \,,
	\quad
	F(f) = f
\]
is faithful but not full.
The unique functor
\[
	G
	\colon
	I_2 \to I_1
	\quad\text{with}\quad
	F(x) = x \,,
	\quad
	F(y) = y \,,
	\quad
	F(f) = f
	\quad
	F(g) = g
\]
is full but not faithful.
The composite~$F ∘ G$ is neither full nor faithful.
