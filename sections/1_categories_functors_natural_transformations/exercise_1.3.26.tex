\subsection{}

Suppose first that~$α$ is a natural isomorphism.
This means that there exists a natural transformation~$β$ from~$G$ to~$F$ such that both~$β ∘ α = \id_F$ and~$α ∘ β = \id_G$.
For every object~$A$ of~$\cat{A}$, we have therefore the equality
\[
	β_A ∘ α_A
	=
	(β ∘ α)_A
	=
	(\id_F)_A
	=
	\id_{F(A)}
\]
and similarly the equality
\[
	α_A ∘ β_A
	=
	(α ∘ β)_A
	=
	(\id_G)_A
	=
	\id_{G(A)} \,.
\]
This shows that the morphism~$α_A$ is an isomorphism whose inverse is given by~$β_A$.

Suppose now that for every object~$A$ of~$\cat{A}$ the morphism~$α_A$ is an isomorphism, and let~$β_A$ be the inverse of~$α_A$, i.e.,~$β_A ≔ α_A^{-1}$.
We have for every morphism
\[
	f \colon A \to A'
\]
in~$\cat{A}$ the commutative square diagram
\[
	\begin{tikzcd}
		F(A)
		\arrow{r}[above]{F(f)}
		\arrow{d}[left]{α_A}
		&
		F(A')
		\arrow{d}[right]{α_{A'}}
		\\
		G(A)
		\arrow{r}[above]{G(f)}
		&
		G(A')
	\end{tikzcd}
\]
by the naturality of~$α$.
The commutativity of this diagram is equivalent to the equality
\[
	α_{A'} ∘ F(f) = G(f) ∘ α_A \,,
\]
which in turn is equivalent to the equality
\[
	F(f) ∘ α_A^{-1} = α_{A'}^{-1} ∘ G(f) \,.
\]
In other words,
\[
	F(f) ∘ β_A = β_{A'} ∘ G(f) \,,
\]
which tells us that the following square diagram commutes:
\[
	\begin{tikzcd}
		F(A)
		\arrow{r}[above]{F(f)}
		&
		F(A')
		\\
		G(A)
		\arrow{r}[above]{G(f)}
		\arrow{u}[left]{β_A}
		&
		G(A')
		\arrow{u}[right]{β_{A'}}
	\end{tikzcd}
\]
This shows that~$β = (β_A)_{A ∈ \Ob(\cat{A})}$ is a natural transformation from~$G$ to~$F$.

We have for every object~$A$ of~$\cat{A}$ the equalities
\[
	(β ∘ α)_A
	=
	β_A ∘ α_A
	=
	α_A^{-1} ∘ α_A
	=
	\id_{F(A)}
	=
	(\id_F)_A \,,
\]
which shows that~$β ∘ α = \id_F$.
We find in the same way that also~$α ∘ β = \id_G$.
This shows that~$α$ is a natural isomorphism whose inverse is given by~$β$.%
\footnote{
	In other words,~$(α^{-1})_A = (α_A)^{-1}$ for every object~$A$ of~$\cat{A}$.
	This allows us to just write~$α_A^{-1}$ for this morphism.
}
