\subsection{}

Suppose first that~$α$ is a natural isomorphism.
This means that there exists a natural transformation~$β$ from~$G$ to~$F$ such that both~$β ∘ α = 1_F$ and~$α ∘ β = 1_G$.
For every object~$A$ of~$\Acat$ we have therefore the equality
\[
	β_A ∘ α_A
	=
	(β ∘ α)_A
	=
	(1_F)_A
	=
	1_{F(A)}
\]
and similarly the equality
\[
	α_A ∘ β_A
	=
	(α ∘ β)_A
	=
	(1_G)_A
	=
	1_{G(A)} \,.
\]
This shows that the morphism~$α_A$ is an isomorphism with inverse~$β_A$.

Suppose now that for every object~$A$ of~$\Acat$ the morphism~$α_A$ is an isomorphism.
For every object~$A$ of~$\Acat$ let~$β_A ≔ α_A^{-1}$.
It follows for any two objects~$A_1$ and~$A_2$ of~$\Acat$ and every morphism~$f$ from~$A_1$ to~$A_2$ from the commutativity of the square diagram
\[
	\begin{tikzcd}
		F(A_1)
		\arrow{r}[above]{F(f)}
		\arrow{d}[left]{α_{A_1}}
		&
		F(A_2)
		\arrow{d}[right]{α_{A_2}}
		\\
		G(A_1)
		\arrow{r}[above]{G(f)}
		&
		G(A_2)
	\end{tikzcd}
\]
that
\[
	α_{A_2} ∘ F(f) = G(f) ∘ α_{A_1} \,,
\]
and therefore
\[
	F(f) ∘ α_{A_1}^{-1} = α_{A_2}^{-1} ∘ G(f) \,.
\]
In other words,
\[
	F(f) ∘ β_{A_1} = β_{A_2} ∘ G(f) \,,
\]
so that the following diagram commutes:
\[
	\begin{tikzcd}
		F(A_1)
		\arrow{r}[above]{F(f)}
		&
		F(A_2)
		\\
		G(A_1)
		\arrow{r}[above]{G(f)}
		\arrow{u}[left]{β_{A_1}}
		&
		G(A_2)
		\arrow{u}[right]{β_{A_2}}
	\end{tikzcd}
\]
This shows that~$β$ is a natural transformation from~$G$ to~$F$.
We have~$α ∘ β = 1_G$ and~$β ∘ α = 1_F$ by choice of~$β$.
This means that~$α$ is a natural isomorphism with inverse~$β$.
