\subsection{}

A functor~$F$ from~$\Acat$ to~$\Bcat$ consists of the following data:
\begin{itemize*}
	\item
		A map~$f$ from~$A$ to~$B$.
	\item
		For every two elements~$a$ and~$a'$ of~$A$ and every morphism~$i$ from~$a$ to~$a'$ in~$\Acat$ a morphism~$F(i)$ from~$F(a)$ to~$F(a')$ in~$\Bcat$, such that the following two properties hold:
		\begin{enumerate*}
			\item
				\label{identity to identity}
				$F(\id_a) = \id_{f(a)}$ for every element~$a$ of~$\Acat$.
			\item
				\label{composite of composite}
				$F(j ∘ i) = F(j) ∘ F(i)$ for every three elements~$a$,~$a'$,~$a''$ of~$A$ and morphisms~$i$ from~$a$ to~$a'$ and~$j$ from~$a'$ to~$a''$ in~$\Acat$.
		\end{enumerate*}
\end{itemize*}

In the category~$\Bcat$ there exist at most one morphism between any two objects.
The two conditions~\ref{identity to identity} and~\ref{composite of composite} are therefore always satisfied.
It also follows that for every morphism~$i$ in~$\Acat$ the morphism~$F(i)$ is uniquely determined by its domain and its codomain, and therefore already uniquely determined by the function~$f$.

The data of a functor~$F$ from~$\Acat$ to~$\Bcat$ does therefore only consist of a function~$f$ from~$A$ to~$B$ satisfying the following condition:
there exist a morphism from~$f(a)$ to~$f(a')$ for every two elements~$a$ and~$a'$ of~$\Acat$ for which there exist a morphism from~$a$ in~$a'$ in~$\Acat$.
In other words, the function~$f$ needs to satisfy the condition~$f(a) ≤ f(a')$ for every two elements~$a$ and~$a'$ of~$A$ with~$a ≤ a'$.
But this is precisely what it means for~$f$ to be order-preserving.
