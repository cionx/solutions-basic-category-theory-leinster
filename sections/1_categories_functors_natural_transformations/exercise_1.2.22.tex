\subsection{}

A functor~$F$ from~$\cat{A}$ to~$\cat{B}$ consists of the following data:
\begin{itemize*}
	\item
		A map~$f$ from~$A$ to~$B$ that gives us the action of~$F$ on the objects on~$\cat{A}$.
	\item
		For every two elements~$a$ and~$a'$ of~$A$ and every morphism~$i$ from~$a$ to~$a'$ in~$\cat{A}$ a morphism~$F(i)$ from~$F(a)$ to~$F(a')$ in~$\cat{B}$, such that the following two properties hold:
		\begin{enumerate*}
			\item
				\label{identity to identity}
				$F(\id_a) = \id_{f(a)}$ for every element~$a$ of~$\cat{A}$.
			\item
				\label{composite of composite}
				For every two composable morphisms
				\[
					i \colon a \to a' \,,
					\quad
					j \colon a' \to a''
				\]
				in~$\cat{A}$, the equality~$F(j ∘ i) = F(j) ∘ F(i)$ holds.
		\end{enumerate*}
\end{itemize*}

In the category~$\cat{B}$, any two morphisms with the same domain and the same co\-domain are automatically equal.
This has the following two consequences:
\begin{itemize*}

	\item
		The two conditions~\ref{identity to identity} and~\ref{composite of composite} are both automatically satisfied.

	\item
		The action of~$F$ on morphisms is uniquely determined by the action of~$F$ on objects, and thus by the map~$f$.

\end{itemize*}

The data of a functor~$F$ from~$\cat{A}$ to~$\cat{B}$ does therefore only consist of a function~$f$ from~$A$ to~$B$ satisfying the following condition:
\begin{quote}
	There exist a morphism from~$f(a)$ to~$f(a')$ in~$\cat{B}$ for every two elements~$a$ and~$a'$ of~$A$ for which there exist a morphism from~$a$ in~$a'$ in~$\cat{A}$.
\end{quote}
In other words, the function~$f$ needs to satisfy the condition~$f(a) ≤ f(a')$ for every two elements~$a$ and~$a'$ of~$A$ with~$a ≤ a'$.
But this is precisely what it means for~$f$ to be order-preserving.
