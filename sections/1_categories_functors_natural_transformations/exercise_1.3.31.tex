\subsection{}



\subsubsection{}

Let~$X$ and~$Y$ be two finite sets and let~$f$ be a bijection from~$X$ to~$Y$ (i.e., a morphism in the given category~$\cat{B}$).
We let~$\Sym(f)$ be the resulting conjugation isomorphism of groups from~$\Sym(X)$ to~$\Sym(Y)$, i.e., the map
\[
	\Sym(f)
	\colon
	\Sym(X) \to \Sym(Y) \,,
	\quad
	s \mapsto f ∘ s ∘ f^{-1} \,.
\]
This construction defines a functor~$\Sym$ from~$\cat{B}$ to~$\Set$:
\begin{itemize}

	\item
		For every set~$X$ we have
		\[
			\Sym(\id_X)(s)
			=
			\id_X ∘ s ∘ \id_X^{-1}
			=
			\id_X ∘ s ∘ \id_X
			=
			s
			=
			\id_{\Sym(X)}(s)
		\]
		for every~$s ∈ \Sym(X)$, and therefore~$\Sym(\id_X) = \id_{\Sym(X)}$.

	\item
		We have for any two composable morphisms
		\[
			f \colon X \to Y,
			\quad
			g \colon Y \to Z
		\]
		in~$\cat{B}$, the equalities
		\begin{align*}
			\Sym(g)( \Sym(f)(s) )
			&=
			g ∘ f ∘ s ∘ f^{-1} ∘ g^{-1}
			\\
			&=
			(g ∘ f) ∘ s ∘ (g ∘ f)^{-1}
			\\
			&=
			\Sym(g ∘ f)(s)
		\end{align*}
		for every~$s ∈ \Sym(X)$, and therefore the equality
		\[
			\Sym(g∘ f) = \Sym(g) ∘ \Sym(f) \,.
		\]
\end{itemize}
We have shown that~$\Sym$ is indeed a functor from~$\cat{B}$ to~$\Set$.%
\footnote{
	The functor~$\Sym$ lifts along the forgetful functor from~$\Grp$ to~$\Set$ to a functor from~$\cat{B}$ to~$\Grp$.
}


Let~$X$ and~$Y$ be two finite sets and let~$f$ be a bijection from~$X$ to~$Y$.
Given a total order~$≤$ on~$X$ we get a total order~$≤_f$ on~$Y$ given by
\[
	f(x_1) ≤_f f(x_2)
	\iff
	x_1 ≤ x_2
\]
for all~$x_1, x_2 ∈ X$.
We define~$\Ord(f)$ as the map
\[
	\Ord(f)
	\colon
	\Ord(X) \to \Ord(Y) \,,
	\quad
	{≤} \mapsto {≤_f} \,.
\]
This defined a functor~$\Ord$ from~$\cat{B}$ to~$\Set$:
\begin{itemize}

	\item
		Let~$X$ be a set.
		For every total order~$≤$ on~$X$, i.e., element of~$\Ord(X)$, we have
		\[
			x_1 ≤_{\id_X} x_2
			\iff
			\id_X(x_1) ≤_{\id_X} \id_X(x_2)
			\iff
			x_1 ≤ x_2
		\]
		for all~$x_1, x_2 ∈ X$, which means that the total order~${≤_{\id_X}} = \Ord(\id_X)(≤)$ coincides with the original order~$≤$.
		This shows that~$\Ord(\id_X) = \id_{\Ord(X)}$.

	\item
		Let
		\[
			f \colon X \to Y \,,
			\quad
			g \colon Y \to Z
		\]
		be two composable morphisms in~$\cat{B}$.
		Given a total order~$≤$ on~$X$, we have the chain of equivalences
		\begin{align*}
			{}&
			(g ∘ f)(x_1) \mathrel{(≤_f)_g} (g ∘ f)(x_2) \\
			\iff{}&
			g(f(x_1)) \mathrel{(≤_f)_g} g(f(x_2)) \\
			\iff{}&
			f(x_1) ≤_f f(x_2) \\
			\iff{}&
			x_1 ≤ x_2 \\
			\iff{}&
			(g ∘ f)(x_1) ≤_{g ∘ f} (g ∘ f)(x_2)
		\end{align*}
		for all~$x_1, x_2 ∈ X$, which shows that~$\Ord(g)(\Ord(f)(≤)) = \Ord(g ∘ f)(≤)$.
		We have thus shown that
		\[
			\Ord(g ∘ f) = \Ord(g) ∘ \Ord(f) \,.
		\]

\end{itemize}
This shows that~$\Ord$ is indeed a functor from~$\cat{B}$ to~$\Set$.



\subsubsection{}

We have for every finite set~$X$ a distinguished element of~$\Sym(X)$, namely the identity function~$\id_X$.
For every morphism
\[
	f \colon X \to Y
\]
in~$\cat{B}$, we have
\[
	\Sym(f)(\id_X)
	=
	f ∘ \id_X ∘ f^{-1}
	=
	f ∘ f^{-1}
	=
	\id_Y \,.
\]
This shows that the map~$\Sym(f)$ carries the distinguished element of~$\Sym(X)$ to the distinguished element of~$\Sym(Y)$.
In the following, we denote the distinguished element~$\id_X$ of~$\Sym(X)$ by~$s_X$.

Suppose that there exists a natural transformation~$α$ from~$\Sym$ to~$\Ord$.
For every finite set~$X$ we can then consider the element~$t_X ≔ α_X(s_X)$ of~$\Ord(X)$.
We have for every morphism
\[
	f \colon X \to Y
\]
in~$\cat{B}$ from the following commutative diagram:
\[
	\begin{tikzcd}[column sep = huge]
		\Sym(X)
		\arrow{r}[above]{\Sym(f)}
		\arrow{d}[left]{α_X}
		&
		\Sym(Y)
		\arrow{d}[right]{α_Y}
		\\
		\Ord(X)
		\arrow{r}[above]{\Ord(f)}
		&
		\Ord(Y)
	\end{tikzcd}
\]
It follows from the commutativity of this diagram that
\[
	\Ord(f)(t_X)
	=
	\Ord(f)( α_X( s_X ) )
	=
	α_Y( \Sym(f)( s_X ) )
	=
	α_Y( s_Y )
	=
	t_Y \,.
\]
This calculation tells us that the map~$\Ord(f)$ carries the distinguished element~$t_X$ of~$\Ord(X)$ to the distinguished element~$t_Y$ of~$\Ord(Y)$.

However, such a choice of distinguished elements~$(t_X)_{X ∈ \Ob(\cat{B})}$ cannot exist:
If~$X$ is a finite set with at least two elements, then there exists a bijection~$f$ of~$X$ to itself that swaps these two elements.
It then follows for every element~$≤$ of~$\Ord(X)$ that~$\Ord(f)(≤)$ is distinct from~$≤$.
Therefore,~$\Ord(f)$ cannot carry~$t_X$ to~$t_X$.



\subsubsection{}

For a finite set~$X$ of cardinality~$n$, both~$\Ord(X)$ and~$\Sym(X)$ have~$n!$ elements.



\subsubsection*{Conclusion}

For every finite set~$X$, the two sets~$\Sym(X)$ and~$\Ord(X)$ have the same cardinality, and are therefore isomorphic as objects of~$\Set$.
This means that the two functors~$\Sym$ and~$\Ord$ are pointwise isomorphic.
However, we have seen above that there exists no natural transformation from~$\Sym$ to~$\Ord$, and therefore in particular no natural isomorphism from~$\Sym$ to~$\Ord$.
