\subsection{}



\subsubsection{}

The group~$G^{\op}$ is the opposite group of~$G$:
to every element~$g$ of~$G$ there is an associated element~$g^{\op}$ of~$G^{\op}$ such that the map
\[
	G \to G^{\op} \,,
	\quad
	g \mapsto g^{\op}
\]
is bijective, and the group structure on~$G^{\op}$ is given by
\[
	g^{\op} · h^{\op} = (h · g)^{\op}
	\qquad
	\text{for all~$g, h, ∈ G$.}
\]

Every group~$G$ is isomorphic to its opposite group~$G^{\op}$ via the inversion map
\[
	i
	\colon
	G
	\to
	G^{\op} \,,
	\quad
	g
	\mapsto
	(g^{\op})^{-1} \,.
\]
Indeed, the map~$i$ bijective because it is the composite of the two bijections
\begin{alignat*}{2}
	G
	&\to
	G^{\op} \,,
	&\quad
	g
	&\mapsto
	g^{\op} \,
\shortintertext{and}
	G^{\op}
	&\to
	G^{\op} \,,
	&\quad
	x
	&\mapsto
	x^{-1} \,.
\end{alignat*}
It is a homomorphism of groups because
\[
	i(g h)
	=
	( (g h)^{\op} )^{-1}
	=
	( h^{\op} g^{\op} )^{-1}
	=
	( g^{\op} )^{-1} ( h^{\op} )^{-1}
	=
	i(g) i(h)
\]
for all~$g, h ∈ G$.



\subsubsection{}

We follow the approach from \cite{stackexchange_monoid_isomorphic_to_its_opposite}.
Let~$X$ be a set that contains at least two elements and let~$M$ be the monoid of maps from~$X$ to itself, i.e., let~$M$ be~$\End_{\Set}(X)$.

Every constant map~$c$ from~$X$ to~$X$ is left-absorbing in~$M$, in the sense that~$cf = c$ for every element~$f$ of~$M$.
It follows that the opposite monoid~$M^{\op}$ contains right-absorbing elements.

But~$M$ itself does not admit any right-absorbing element.
Indeed, there exist for every element~$f$ of~$M$ two elements~$x$ and~$y$ of~$X$ with~$f(x) ≠ y$ (because the set~$X$ contains two distinct elements).
Let~$s$ be the transposition on the set~$X$ that interchanges the two elements~$f(x)$ and~$y$.
Then~$(s f)(x) = y ≠ f(x)$ and therefore~$sf ≠ f$.
