\subsection{}

Every category is equivalent -- even isomorphic -- to itself via its identity functor.
This shows that equivalence of categories is reflexive.

The notion of an \enquote{equivalence of categories} is symmetric in the two categories in question.
This shows that equivalence of categories is symmetric.

Let~$\Acat$,~$\Bcat$ and~$\Ccat$ be three categories such that~$\Acat$ is equivalent to~$\Bcat$ and~$\Bcat$ is equivalent to~$\Ccat$.
This means that there exist functors
\[
	F \colon \Acat \to \Bcat \,,
	\quad
	G \colon \Bcat \to \Acat \,,
	\quad
	F' \colon \Bcat \to \Ccat \,,
	\quad
	G' \colon \Ccat \to \Bcat
\]
such that
\[
	F' F ≅ 1_{\Acat} \,,
	\quad
	F F' ≅ 1_{\Bcat} \,,
	\quad
	G' G ≅ 1_{\Bcat} \,,
	\quad
	G G' ≅ 1_{\Ccat} \,.
\]
It follows that
\[
	GF F'G'
	≅
	G 1_{\Bcat} G'
	=
	G G'
	≅
	1_{\Ccat}
\]
and similarly that
\[
	F'G' GF
	≅
	F' 1_{\Bcat} F
	=
	F' F
	≅
	1_{\Acat} \,.
\]
This shows that the functor~$GF$ is again an equivalence of categories, with quasi-inverse given by~$F' G'$.
We have thus shown that equivalence of categories is transitive.
