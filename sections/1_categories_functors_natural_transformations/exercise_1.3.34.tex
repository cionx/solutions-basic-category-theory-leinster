\subsection{}

Every category is equivalent -- even isomorphic -- to itself via its identity functor.
This shows that equivalence of categories is reflexive.

The notion of an \enquote{equivalence of categories} is symmetric in the two categories in question.
This shows that equivalence of categories is symmetric.

Let~$\cat{A}$,~$\cat{B}$ and~$\cat{C}$ be three categories such that~$\cat{A}$ is equivalent to~$\cat{B}$ and~$\cat{B}$ is equivalent to~$\cat{C}$.
This means that there exist functors
\[
	F \colon \cat{A} \to \cat{B} \,,
	\quad
	G \colon \cat{B} \to \cat{A} \,,
	\quad
	F' \colon \cat{B} \to \cat{C} \,,
	\quad
	G' \colon \cat{C} \to \cat{B}
\]
such that
\[
	F' F ≅ 1_{\cat{A}} \,,
	\quad
	F F' ≅ 1_{\cat{B}} \,,
	\quad
	G' G ≅ 1_{\cat{B}} \,,
	\quad
	G G' ≅ 1_{\cat{C}} \,.
\]
It follows that
\[
	GF F'G'
	≅
	G 1_{\cat{B}} G'
	=
	G G'
	≅
	1_{\cat{C}}
\]
and similarly that
\[
	F'G' GF
	≅
	F' 1_{\cat{B}} F
	=
	F' F
	≅
	1_{\cat{A}} \,.
\]
This shows that the functor~$GF$ is again an equivalence of categories, with quasi-inverse given by~$F' G'$.
We have thus shown that equivalence of categories is transitive.
