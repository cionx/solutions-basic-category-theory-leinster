\subsection{}

The composition of morphisms in~$\Matcat$ is given by matrix multiplication.

We consider an auxiliary category~$\cat{B}$ of \enquote{vector spaces together with bases}.
This category is defined as follows:
The objects of~$\cat{B}$ are pairs~$(V, B)$ consisting of a finite-dimensional~\vectorspace{$𝕜$}~$V$ and a basis~$B$ of~$V$.
For any two objects~$(V, B)$ and~$(W, C)$ of~$\cat{B}$, the respective set of morphism~$\cat{B}( (V, B), (W, C) )$ is simply the set of linear maps from~$V$ to~$W$.
The composition of morphisms in~$\cat{B}$ is the usual composition of linear maps.

We have a forgetful functor~$U$ from~$\cat{B}$ to~$\FDVect$ given by
\[
	U((V, B)) ≔ V
	\quad\text{and}\quad
	U(f) ≔ f
\]
on objects and morphisms respectively.
The functor~$U$ is full, faithful and surjective, and hence an equivalence of categories.

We also have a functor~$F$ from~$\cat{B}$ to~$\Matcat$ as follows:
For every object~$(V, B)$ of~$\cat{B}$, the action of~$F$ on~$(V, B)$ is given by
\[
	F( (V, B) ) ≔ \dim(V) \,.
\]
For every morphism
\[
	f \colon (V, B) \to (W, C)
\]
in~$\cat{B}$, we choose~$F(f)$ as the matrix that represents the linear map~$f$ with respect to the bases~$B$ and~$C$ of~$V$ and~$W$.
The functor~$F$ is full, faithful and surjective, and is therefore an equivalence of categories.

The category~$\cat{B}$ is both equivalent to~$\FDVect$ and to~$\Matcat$.
It follows (from the upcoming Exercise~1.3.34) that~$\FDVect$ and~$\Matcat$ are also equivalent.

An essential inverse~$F'$ of the functor~$F$ can explicitly be constructed on objects by~$F'(n) = (𝕜^n, (e_1, \dotsc, e_n))$ for every natural number~$n$, and on morphisms by~$F'(A)(x) = Ax$ for every matrix~$A$ and column vector~$x$ (of suitable sizes).

An essential inverse~$U'$ of the functor~$U$ together with a natural isomorphism from~$U' ∘ U$ to~$\Id_{\FDVect}$ amounts to choosing a basis for every finite-dimensional~\vectorspace{$𝕜$}.

The composite~$U ∘ F'$ is an equivalence of categories from~$\Matcat$ to~$\FDVect{𝕜}$.
It assigns to each natural number~$n$ the vector space~$𝕜^n$, and regards every matrix of size~$m × n$ as a linear map from~$𝕜^n$ to~$𝕜^m$ via matrix-vector multiplication.

We might regard the functor~$U ∘ F'$ as a \enquote{canonical} functor from~$\Matcat$ to~$\FDVect{𝕜}$.
The functor~$F ∘ U'$, on the other hand, is not \enquote{canonical} since it depends on choosing bases.
