\subsection{}

For every topological space~$X$ let~$C(X)$ be the (commutative) ring of continuous, real-valued functions on~$X$.
For any two topological spaces~$X$ and~$Y$ and every continuous map~$φ$ from~$X$ to~$Y$, let~$C(φ)$ be the induced map
\[
	C(φ)
	\colon
	C(Y)
	\to
	C(X) \,,
	\quad
	f
	\mapsto
	f ∘ φ \,.
\]
This map is well-defined because the composite of two continuous maps is again continuous.

The map~$C(φ)$ is a homomorphism of rings:
\begin{itemize}

	\item
		If we denote by~$⋆$ either addition or multiplication, then we have for every two elements~$f$ and~$g$ of~$C(Y)$ the equalities
		\begin{align*}
			C(φ)(f ⋆ g)(x)
			&=
			((f ⋆ g) ∘ φ)(x)
			\\
			&=
			(f ⋆ g)(φ(x))
			\\
			&=
			f(φ(x)) ⋆ g(φ(x))
			\\
			&=
			(f ∘ φ)(x) ⋆ (g ∘ φ)(x)
			\\
			&=
			C(φ)(f)(x) ⋆ C(φ)(g)(x)
			\\
			&=
			(C(φ)(f) ⋆ C(φ)(g))(x)
		\end{align*}
		for every point~$x$ in~$X$, and thus the equality
		\[
			C(φ)(f ⋆ g) = C(φ)(f) ⋆ C(φ)(g) \,.
		\]

	\item
		We also have
		\[
			C(φ)(1_Y)
			=
			1_Y ∘ φ
			=
			1_X \,,
		\]
		and thus~$C(φ)(1) = 1$.
		(We denote here by~$1_X$ and~$1_Y$ the functions with constant value~$1$ on~$X$ and~$Y$ respectively.)

\end{itemize}
These calculations show that the map~$C(φ)$ is indeed a homomorphism of rings from~$C(Y)$ to~$C(X)$.

We now check that the construction~$C$ is indeed a functor.
\begin{itemize}

	\item
		For every topological space~$X$ we have
		\[
			C(\id_X)(f)
			=
			f ∘ \id_X
			=
			f
		\]
		for every~$f ∈ C(X)$, and therefore~$C(\id_X) = \id_{C(X)}$.
		(Here we denote by~$\id_X$ the identity map on the space~$X$.)

	\item
		For every two composable continuous maps
		\[
			φ \colon X \to Y \,,
			\quad
			ψ \colon Y \to Z \,,
		\]
		we have the equalities
		\[
			C(ψ ∘ φ)(f)
			=
			f ∘ ψ ∘ φ
			=
			C(φ)(f ∘ ψ)
			=
			C(φ)( C(ψ)(f) )
			=
			( C(φ) ∘ C(ψ) )(f)
		\]
		for every~$f ∈ C(Z)$, and therefore~$C(ψ ∘ φ) = C(φ) ∘ C(ψ)$.

\end{itemize}
This shows that~$C$ is indeed a contravariant functor from~$\Top$ to~$\Ring$.
