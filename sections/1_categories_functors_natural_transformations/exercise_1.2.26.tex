\subsection{}

For every topological space~$X$ let~$C(X)$ be the (commutative) ring of continuous, real-valued functions on~$X$.
For any two topological spaces~$X$ and~$Y$ and every continuous map~$φ$ from~$X$ to~$Y$ let~$C(φ)$ be the induced map
\[
	C(φ)
	\colon
	C(Y)
	\to
	C(X) \,,
	\quad
	f
	\mapsto
	f ∘ φ \,.
\]
This map is well-defined because the composite of two continuous maps is again continuous.

The above map~$C(φ)$ is a homomorphism of rings:
For every two elements~$f$ and~$g$ of~$C(Y)$ and every~$⋆ ∈ \{+, ·\}$ we have
\begin{align*}
	C(φ)(f ⋆ g)(x)
	&=
	((f ⋆ g) ∘ φ)(x)
	\\
	&=
	(f ⋆ g)(φ(x))
	\\
	&=
	f(φ(x)) ⋆ g(φ(x))
	\\
	&=
	(f ∘ φ)(x) ⋆ (g ∘ φ)(x)
	\\
	&=
	C(φ)(f)(x) ⋆ C(φ)(g)(x)
	\\
	&=
	(C(φ)(f) ⋆ C(φ)(g))(x)
\end{align*}
for every element~$x$ of~$X$, and thus
\[
	C(φ)(f ⋆ g)
	=
	C(φ)(f) ⋆ C(φ)(g) \,.
\]
We also have
\[
	C(φ)(1_Y)
	=
	1_Y ∘ φ
	=
	1_X \,,
\]
and this~$C(φ)(1) = 1$.
These calculations show that~$C(φ)$ is indeed a homomorphism of rings from~$C(Y)$ to~$C(X)$.

For every topological space~$X$ we have
\[
	C(\id_X)(f)
	=
	f ∘ \id_X
	=
	f
\]
for every~$f ∈ C(X)$ and therefore~$C(\id_X) = \id_{C(X)}$.
For every three topological spaces~$X$,~$Y$,~$Z$ and every two continuous maps~$φ$ from~$X$ to~$Y$ and~$ψ$ from~$Y$ to~$Z$ we have
\[
	C(ψ ∘ φ)(f)
	=
	f ∘ ψ ∘ φ
	=
	C(ψ)(f) ∘ φ
	=
	C(φ)( C(ψ)(f) )
	=
	( C(φ) ∘ C(ψ) )(f)
\]
for every~$f ∈ C(Z)$, and therefore~$C(ψ ∘ φ) = C(φ) ∘ C(ψ)$.
This shows altogether that~$C$ is a contravariant functor from~$\Top$ to~$\Ring$.
