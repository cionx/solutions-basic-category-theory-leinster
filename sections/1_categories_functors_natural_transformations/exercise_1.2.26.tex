\subsection{}

For every topological space~$X$ let~$C(X)$ be the (commutative) ring of continuous, real-valued functions on~$X$.
For any two topological spaces~$X$ and~$Y$ and every continuous map~$\varphi$ from~$X$ to~$Y$ let~$C(\varphi)$ be the induced map
\[
	C(\varphi)
	\colon
	C(Y)
	\to
	C(X) \,,
	\quad
	f
	\mapsto
	f \circ \varphi \,.
\]
This map is well-defined because the composite of two continuous maps is again continuous.

The above map~$C(\varphi)$ is a homomorphism of rings:
For every two elements~$f$ and~$g$ of~$C(Y)$ and every~$\star \in \{+, \cdot\}$ we have
\begin{align*}
	C(\varphi)(f \star g)(x)
	&=
	((f \star g) \circ \varphi)(x)
	\\
	&=
	(f \star g)(\varphi(x))
	\\
	&=
	f(\varphi(x)) \star g(\varphi(x))
	\\
	&=
	(f \circ \varphi)(x) \star (g \circ \varphi)(x)
	\\
	&=
	C(\varphi)(f)(x) \star C(\varphi)(g)(x)
	\\
	&=
	(C(\varphi)(f) \star C(\varphi)(g))(x)
\end{align*}
for every element~$x$ of~$X$, and thus
\[
	C(\varphi)(f \star g)
	=
	C(\varphi)(f) \star C(\varphi)(g) \,.
\]
We also have
\[
	C(\varphi)(1_Y)
	=
	1_Y \circ \varphi
	=
	1_X \,,
\]
and this~$C(\varphi)(1) = 1$.
These calculations show that~$C(\varphi)$ is indeed a homomorphism of rings from~$C(Y)$ to~$C(X)$.

For every topological space~$X$ we have
\[
	C(\id_X)(f)
	=
	f \circ \id_X
	=
	f
\]
for every~$f \in C(X)$ and therefore~$C(\id_X) = \id_{C(X)}$.
For every three topological spaces~$X$,~$Y$,~$Z$ and every two continuous maps~$\varphi$ from~$X$ to~$Y$ and~$\psi$ from~$Y$ to~$Z$ we have
\[
	C(\psi \circ \varphi)(f)
	=
	f \circ \psi \circ \varphi
	=
	C(\psi)(f) \circ \varphi
	=
	C(\varphi)( C(\psi)(f) )
	=
	( C(\varphi) \circ C(\psi) )(f)
\]
for every~$f \in C(Z)$, and therefore~$C(\psi \circ \varphi) = C(\varphi) \circ C(\psi)$.
This shows altogether that~$C$ is a contravariant functor from~$\Top$ to~$\Ring$.
