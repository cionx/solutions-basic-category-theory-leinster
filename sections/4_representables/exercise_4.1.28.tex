\subsection{}

For every group~$G$, the set~$U_p(G)$ may more tactile be expressed at
\[
	U_p(G)
	=
	\{ g \in G \suchthat g^p = 1 \} \,.
\]
We see from this alternative description of~$U_p$ that every homomorphism of groups
\[
	φ \colon G \to H
\]
restricts to a map from~$U_p(G)$ to~$U_p(H)$.
We denote this restriction by~$U_p(φ)$.
This given us the desired functor~$U_p$ from~$\Grp$ to~$\Set$.

Let~$G$ be a group.
By the homomorphism theorem, any homomorphism of groups~$ψ$ from~$ℤ / pℤ$ to~$G$ corresponds uniquely to a homomorphism of groups~$φ$ from~$ℤ$ to~$G$ with~$pℤ ⊆ \ker(φ)$, via the formula
\[
	ψ( \class{n} ) = φ(n)
	\qquad
	\text{for all~$n ∈ ℤ$} \,.
\]
We already know that homomorphisms from~$ℤ$ to~$G$ correspond to elements of~$G$, with the homomorphism~$φ$ corresponding to the element~$φ(1)$ of~$G$.
The condition~$pℤ ⊆ \ker(φ)$ is then equivalent to the condition~$φ(1)^p = 1$.
This means altogether that we have a bijection
\[
	α_G
	\colon
	\Grp(ℤ / pℤ, G) \to U_p(G) \,,
	\quad
	ψ \mapsto ψ(\class{1}) \,.
\]

This bijection is natural in~$G$:
the diagram
\[
	\begin{tikzcd}[sep = large]
		\Grp(ℤ / pℤ, G)
		\arrow{r}[above]{φ_*}
		\arrow{d}[left]{α_G}
		&
		\Grp(ℤ / pℤ, H)
		\arrow{d}[right]{α_H}
		\\
		U_p(G)
		\arrow{r}[above]{U_p(φ)}
		&
		U_p(H)
	\end{tikzcd}
\]
commutes for every homomorphism of groups~$φ$ from~$G$ to~$H$, because we have the chain equalities
\[
	α_H( φ_*( ψ ) )
	=
	α_H( φ ∘ ψ )
	=
	(φ ∘ ψ)( \class{1} )
	=
	φ( ψ( \class{1} ) )
	=
	φ( α_G( ψ ) )
	=
	U_p(φ)( α_G( ψ ) )
\]
for every element~$ψ$ of~$\Grp(ℤ / pℤ, G)$.

We have overall constructed a natural isomorphism~$α$ from the represented functor~$\Grp(ℤ / pℤ, -)$ to the functor~$U_p$.
The existence of such an isomorphism shows that the two functors are isomorphic.
