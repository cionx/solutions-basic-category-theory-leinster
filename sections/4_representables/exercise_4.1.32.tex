\subsection{}

We know from Exercise~2.1.14 that an adjunction between~$F$ and~$G$ (with~$F$ left adjoint to~$G$) can be characterized as a choice of bijection
\[
	Φ_{A, B}
	\colon
	\cat{A}(A, G(B)) \to \cat{B}(F(A), B) \,,
	\quad
	f \mapsto \adjtranspose{f}
\]
where~$A$ and~$B$ range through the objects of~$\cat{A}$ and~$\cat{B}$ respectively, subject to the following condition:
for all morphisms
\[
	p \colon A' \to A \,,
	\quad
	q \colon B \to B'
\]
in~$\cat{A}$ and~$\cat{B}$ respectively, we have the identity
\[
	\adjtranspose{ \biggl( A' \xto{p} A \xto{f} G(B) \xto{G(q)} G(B') \biggr) }
	=
	\biggl( F(A') \xto{F(p)} F(A) \xto{\adjtranspose{f}} B \xto{q} B' \biggr) \,.
\]
This identity can equivalently be expressed as the commutativity of the following square diagram:
\[
	\begin{tikzcd}[sep = huge]
		\cat{A}(A, G(B))
		\arrow{r}[above]{Φ_{A, B}}
		\arrow{d}[left]{G(q) ∘ (\ph) ∘ p}
		&
		\cat{B}(F(A), B)
		\arrow{d}[right]{q ∘ (\ph) ∘ F(p)}
		\\
		\cat{A}(A', G(B'))
		\arrow{r}[above]{Φ_{A', B'}}
		&
		\cat{B}(F(A'), B')
	\end{tikzcd}
\]
We can rewrite this diagram in terms of the two functors~$\cat{A}(\ph, G(\ph))$ and~$\cat{B}(F(\ph), \ph)$ as follows:
\[
	\begin{tikzcd}[sep = huge]
		\cat{A}(\ph, G(\ph))(A, B)
		\arrow{r}[above]{Φ_{A, B}}
		\arrow{d}[left]{\cat{A}(\ph, G(\ph))(p, q)}
		&
		\cat{B}(F(\ph), \ph)(A, B)
		\arrow{d}[right]{\cat{B}(F(\ph), \ph)(p, q)}
		\\
		\cat{A}(\ph, G(\ph))(A', B')
		\arrow{r}[above]{Φ_{A', B'}}
		&
		\cat{B}(F(\ph), \ph)(A', B')
	\end{tikzcd}
\]
That this diagram commutes for any two morphisms~$p$ and~$q$ as above means precisely that the family~$Φ_{A, B}$, where~$(A, B)$ ranges through the objects of the category~$\cat{A} × \cat{B}$, is a natural transformation from~$\cat{A}(\ph, G(\ph))$ to~$\cat{B}(F(\ph), \ph)$.

We find overall that an adjunction between~$F$ and~$G$, with~$F$ left adjoint to~$G$, may equivalently be expressed as a natural isomorphism of functors from~$\cat{A}(\ph, G(\ph))$ to~$\cat{B}(F(\ph), \ph)$.
This entails that~$F$ is left adjoint to~$G$ if and only if such a natural isomorphism exists, i.e., if and only if these two functors are naturally isomorphic.
