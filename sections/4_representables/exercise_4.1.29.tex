\subsection{}

Let~$R$ be a ring.
We know from algebra that there exists for every element~$r$ of~$R$ a unique homomorphism of rings~$ϕ$ from~$ℤ[x]$ to~$R$ with~$ϕ(x) = r$.%
\footnote{
	This homomorphism~$ϕ$ maps any polynomial~$∑_n a_n x^n$ in~$ℤ[x]$ to the element~$∑_n a_n r^n$ of~$R$.
}
This means that the map
\[
	ε_R
	\colon
	\Ring(ℤ[x], R) \to R \,,
	\quad
	ϕ \mapsto ϕ(x)
\]
is a bijection.
For every homomorphism of rings
\[
	f \colon R \to S \,,
\]
the square diagram
\[
	\begin{tikzcd}[sep = large]
		\Ring(ℤ[x], R)
		\arrow{r}[above]{ε_R}
		\arrow{d}[right]{f_*}
		&
		R
		\arrow{d}[left]{f}
		\\
		\Ring(ℤ[x], S)
		\arrow{r}[above]{ε_S}
		&
		S
	\end{tikzcd}
\]
commutes, because for every element~$ϕ$ of~$\Ring(ℤ[x], R)$, the resulting diagram of elements commutes:
\[
	\begin{tikzcd}[sep = large]
		ϕ
		\arrow[mapsto]{r}[above]{ε_R}
		\arrow[mapsto]{d}[left]{f_*}
		&
		ϕ(x)
		\arrow[mapsto]{d}[right]{f}
		\\
		f ∘ ϕ
		\arrow[mapsto]{r}[above]{ε_S}
		&
		f(ϕ(x))
	\end{tikzcd}
\]
This shows that the bijections~$ε_R$, where~$R$ ranges through the class of rings, assemble into a natural isomorphism~$ε$ from~$\Ring(ℤ[x], \ph)$ to the forgetful functor from~$\CRing$ to~$\Set$.
