\subsection{}

Let~$R$ be a ring.
We know from algebra that there exists for every element~$r$ of~$R$ a unique homomorphism of rings~$ϕ$ from~$ℤ[x]$ to~$R$ with~$ϕ(x) = r$.%
\footnote{
	This homomorphism~$ϕ$ maps any polynomial~$∑_n a_n x^n$ in~$ℤ[x]$ to the element~$∑_n a_n r^n$ of~$R$.
}
This means that the map
\[
	ε_R
	\colon
	\Ring(ℤ[x], R) \to R \,,
	\quad
	ϕ \mapsto ϕ(x)
\]
is bijective.
We claim that the bijection~$ε_R$ is natural in~$R$.

To prove this naturality, we need to show that for every homomorphism of rings
\[
	f \colon R \to S \,,
\]
the square diagram
\[
	\begin{tikzcd}
		\Ring(ℤ[x], R)
		\arrow{r}[above]{ε_R}
		\arrow{d}[right]{f_*}
		&
		R
		\arrow{d}[left]{f}
		\\
		\Ring(ℤ[x], S)
		\arrow{r}[above]{ε_S}
		&
		S
	\end{tikzcd}
\]
commutes.
To this end, we have for every element~$ϕ$ of~$\Ring(ℤ[x], R)$ the chain of equalities
\[
	f( ε_R( ϕ ) )
	=
	f( ϕ(x) )
	=
	(f ∘ ϕ)(x)
	=
	ε_R(f ∘ ϕ)
	=
	ε_R(f_*(ϕ)) \,.
\]

This shows that the bijections~$ε_R$, where~$R$ ranges through the class of rings, assemble into a natural isomorphism~$ε$ from the functor~$\Ring(ℤ[x], \ph)$ to the forgetful functor from~$\CRing$ to~$\Set$.
