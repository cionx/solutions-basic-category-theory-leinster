\subsection{}


We denote the unique object of~$M$ by~$\star$.
For every functor~$F$ from~$M^{\op}$ to~$\Set$, we denote by~$X(F)$ the resulting right~\set{$M$}.%
\footnote{
	In other words, we denote by~$X$ the isomorphism of categories from the functor category~$[M^{\op}, \Set]$ to the category of right~\sets{$M$}.
}



\subsubsection{}

Let~$H$ be the unique represented functor from~$M^{\op}$ to~$\Set$.
This functor is simply~$M(\ph, \star)$ because~$\star$ is the unique object of~$M$.

The underlying set of~$X(H)$ is given by the set
\[
	H(\star)
	=
	M(\star, \star)
	=
	M \,.
\]
For every element~$m$ of~$M$, the action of~$m$ on~$X$ is given by the map~$H(m)$, whence
\[
	x ⋅ m
	=
	H(m)(x)
	=
	m^*(x)
	=
	x m
\]
for every element~$x$ of~$X(H)$.

This shows overall that~$X(H)$ is the right regular~\set{$M$}.



\subsubsection{}

We denote the forgetful functor from the category of right~\sets{$M$} to the category~$\Set$ by~$U$.

For every \set{$M$}~$Y$ and every element~$y$ of~$U(Y)$, there exists a unique homomorphism of right~\sets{$M$}~$f_y$ from~$M$ to~$Y$ with~$f_y(1) = y$.
This homomorphism is more explicitly given by
\[
	f_y(m) = ym
\]
for every element~$m$ of~$M$.
The map
\[
	\{ \text{homomorphisms of right~\sets{$M$}~$\textstyle M \to Y$} \} \to U(Y) \,,
	\quad
	g \mapsto g(1)
\]
is therefore a bijection.



\subsubsection{}

Let~$F$ be functor from~$M^{\op}$ to~$\Set$.
We have on the one hand the bijection
\begin{align*}
	\left\{
		\begin{tabular}{c}
			natural transformations \\
			$H \to F$
		\end{tabular}
	\right\}
	&\to
	\left\{
		\begin{tabular}{c}
			homomorphisms of right~\sets{$M$} \\
			$X(H) \to X(F)$
		\end{tabular}
	\right\} \,,
	\\
	α &\mapsto α_\star \,.
\end{align*}
We have on the other hand the bijection
\begin{align*}
	\left\{
		\begin{tabular}{c}
			homomorphisms of right~\sets{$M$} \\
			$X(H) \to X(F)$
		\end{tabular}
	\right\}
	&\to
	F(\star) \,,
	\\
	g &\mapsto g(1) \,,
\end{align*}
because~$X(H)$ is the right regular~\set{$M$} and because
\[
	U(X(F)) = F(\star) = F(\star) \,.
\]
By combining both of these bijections, we end up with the bijection
\[
	[M^{\op}, \Set](H, F) \to F(\star) \,,
	\quad
	α \mapsto α_\star(1) = α_\star(\id_\star) \,.
\]

The naturality of this bijection can be checked as in the general proof of the Yoneda~lemma.
Alernatively, one can check that all of the above bijections were natural to begin with, and then conclude that the final bijection is also natural.
