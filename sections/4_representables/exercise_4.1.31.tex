\subsection{}

Let~$\Two$ be the category with two objects, denoted as~$0$ and~$1$, and precisely one non-identity morphism, which we denote by~$i$, and which goes from~$0$ to~$1$.%
\footnote{
	In other words, the category~$\Two$ corresponds to the partially ordered set~$\{0, 1\}$ with~$0 < 1$.
}
This category may be depicted as follows:
\[
	\begin{tikzcd}
		0
		\arrow{r}[above]{i}
		&
		1
	\end{tikzcd}
\]

Let~$\cat{A}$ be an arbitrary category.
There exists for every morphism
\[
	f \colon A \to A'
\]
in~$\cat{A}$ a unique functor~$F$ from~$\Two$ to~$\cat{A}$ with~$F(i) = f$.
This functor is given by
\[
	F(0) = A \,,
	\quad
	F(1) = A' \,,
	\quad
	F(\id_0) = \id_A \,,
	\quad
	F(\id_1) = \id_{A'} \,,
	\quad
	F(i) = f \,.
\]
If~$\cat{A}$ is small, then we have therefore a bijection of sets
\[
	ε_{\cat{A}}
	\colon
	\Cat(\Two, \cat{A}) \to M(\cat{A}) \,,
	\quad
	F \mapsto F(i) \,.
\]

Let us check that this bijection is natural in~$\cat{A}$.
For this, we need to check that for every functor
\[
	G \colon \cat{A} \to \cat{B}
\]
between small categories~$\cat{A}$ and~$\cat{B}$ the following square diagram commutes:
\[
	\begin{tikzcd}[sep = large]
		\Cat(\Two, \cat{A})
		\arrow{r}[above]{ε_{\cat{A}}}
		\arrow{d}[left]{G_*}
		&
		M(\cat{A})
		\arrow{d}[right]{M(G)}
		\\
		\Cat(\Two, \cat{B})
		\arrow{r}[above]{ε_{\cat{B}}}
		&
		M(\cat{B})
	\end{tikzcd}
\]
To this end, we have for every element~$F$ of~$\Cat(\Two, \cat{A})$, the chain of equalities
\begin{align*}
	M(G)( ε_{\cat{A}}( F ) )
	&=
	G( ε_{\cat{A}}(F) )
	\\
	&=
	G( F(i) )
	\\
	&=
	(G ∘ F)(i)
	\\
	&=
	ε_{\cat{B}}(G ∘ F)
	\\
	&=
	ε_{\cat{B}}( G_*( F ) ) \,.
\end{align*}

We have thus found that the bijections~$ε_{\cat{A}}$, where~$\cat{A}$ ranges through the class of small categories, assemble into a natural isomorphism~$ε$ from the functor~$\Cat(\Two, \ph)$ to the functor~$M$.
The existence of such a natural isomorphism shows that the functor~$M$ is represented by the category~$\Two$.
