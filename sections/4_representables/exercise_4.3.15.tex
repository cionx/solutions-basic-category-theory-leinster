\subsection{}



\subsubsection{}

If~$f$ is an isomorphism in~$\cat{A}$ then~$J(f)$ is an isomorphism in~$\cat{B}$ because functors preserve isomorphisms.

Suppose on the other hand that~$J(f)$ is an isomorphism in~$\cat{B}$.
The morphism~$f$ is of the form
\[
	f \colon A \to A'
\]
for some objects~$A$ and~$A'$ of~$\cat{A}$.
It follows that the morphism~$J(f)$ is of the form
\[
	J(f) \colon J(A) \to J(A') \,.
\]
There exists by assumption a morphism
\[
	g' \colon J(A') \to J(A)
\]
with both
\[
	J(f) ∘ g' = \id_{J(A')}
	\quad\text{and}\quad
	g' ∘ J(f) = \id_{J(A)} \,.
\]
There exists a morphism
\[
	g \colon A' \to A
\]
in~$\cat{A}$ with~$g' = J(g)$ because the functor~$J$ is full.
We find that
\[
	J(g ∘ f)
	=
	J(g) ∘ J(f)
	=
	g' ∘ J(f)
	=
	\id_{J(A)}
	=
	J( \id_A ) \,,
\]
and therefore~$g ∘ f = \id_A$ because the functor~$J$ is faithful.
We find in the same way that also~$f ∘ g = \id_{A'}$.
This shows that the morphism~$f$ is an isomorphism.



\subsubsection{}

There exists a unique morphism
\[
	f \colon A \to A'
\]
in~$\cat{A}$ with~$J(f) = g$ because the functor~$J$ is fully faithful.
We see from part~(a) of this exercise that the morphism~$f$ is an isomorphism because~$J(f) = g$ is an isomorphism.



\subsubsection{}

If the objects~$A$ and~$A'$ are isomorphic then so are the objects~$J(A)$ and~$J(A')$ by the functoriality of~$J$.
Suppose on the other hand that the two objects~$J(A)$ and~$J(A')$ are isomorphic.
This means that there exists an isomorphism from~$J(A)$ to~$J(A')$ in~$\cat{B}$.
It follows from part~(b) of this exercise that this isomorphism stems from an isomorphism from~$A$ to~$A'$.
The existence of this isomorphism entails that the objects~$A$ and~$A'$ are isomorphic.
