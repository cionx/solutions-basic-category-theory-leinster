\subsection{}



\subsubsection{}

Let
\[
	f \colon A \to A'
\]
be a morphism in~$\cat{A}$.
The induces natural transformation
\[
	f_* \colon \HY_A \To \HY_{A'}
\]
satisfies
\[
	(f_*)_{A}(\id_A)
	=
	f ∘ \id_A
	=
	f \,.
\]
Therefore, the morphism~$f$ can be retrieved from its induced natural transformation~$f_*$.
This tells us that the functor~$\HY_\bullet$ is faithful.



\subsubsection{}

Let~$A$ and~$A'$ be two objects of~$\cat{A}$, and let
\[
	α \colon \HY_A \To \HY_{A'}
\]
be a natural transformation.
Let~$f$ be the image of the identity morphism~$\id_A$ under the map~$α_A$, i.e., under the map
\[
	\cat{A}(A, A)
	=
	\HY_A(A)
	\xto{α_A}
	\HY_{A'}(A)
	=
	\cat{A}(A, A') \,.
\]
We note that~$f$ is a morphism from~$A$ to~$A'$ in~$\cat{A}$.

The naturality of~$α$ tells us that we have for every morphism
\[
	g \colon B \to B'
\]
in~$\cat{A}$ the following commutative square diagram:
\[
	\begin{tikzcd}
		\HY_A(B')
		\arrow{r}[above]{\HY_A(g)}
		\arrow{d}[left]{α_{B'}}
		&
		\HY_A(B)
		\arrow{d}[right]{α_B}
		\\
		\HY_{A'}(B')
		\arrow{r}[above]{\HY_{A'}(g)}
		&
		\HY_{A'}(B)
	\end{tikzcd}
\]
This diagram may equivalently be written out as follows:
\[
	\begin{tikzcd}
		\cat{A}(B', A)
		\arrow{r}[above]{g^*}
		\arrow{d}[left]{α_{B'}}
		&
		\cat{A}(B, A)
		\arrow{d}[right]{α_B}
		\\
		\cat{A}(B', A')
		\arrow{r}[above]{g^*}
		&
		\cat{A}(B, A')
	\end{tikzcd}
\]
Let now~$B$ be any object of~$\cat{A}$ and let~$g$ be an element of the set~$\HY_A(B)$.
In light of the equality~$\HY_A(B) = \cat{A}(B, A)$, this means that~$g$ is a morphism from~$B$ to~$A$.
We have therefore the following commutative diagram:
\[
	\begin{tikzcd}
		\cat{A}(A, A)
		\arrow{r}[above]{g^*}
		\arrow{d}[left]{α_A}
		&
		\cat{A}(B, A)
		\arrow{d}[right]{α_B}
		\\
		\cat{A}(A, A')
		\arrow{r}[above]{g^*}
		&
		\cat{A}(B, A')
	\end{tikzcd}
\]
The commutativity of this diagram tells us that
\[
	α_B( g )
	=
	α_B( g^*( \id_A ))
	=
	g^*( α_A( \id_A ) )
	=
	g^*( f )
	=
	f ∘ g
	=
	f_*( g ) \,.
\]
This shows that the map
\[
	α_B \colon \HY_A(B) \to \HY_{A'}(B)
\]
is given by~$f_*$.

We have shown that the natural transformations~$α$ and~$f_*$ from~$\HY_A$ to~$\HY_{A'}$ coincide in each component, whence they are equal.
This shows that each natural transformation from~$\HY_A$ to~$\HY_{A'}$ stems from a morphism from~$A$ to~$A'$.
In other words, the functor~$\HY_{\bullet}$ is full.



\subsubsection{}

Suppose that there exists an object~$A$ of~$\cat{A}$ and an element~$u$ of~$X(A)$ such that the provided universal property holds:
\begin{quote}
	\itshape
	For every object~$B$ of~$\cat{A}$ and every element~$x$ of~$X(B)$, there exists a unique morphism~$f$ from~$B$ to~$A$ in~$\cat{A}$ such that~$x = X(f)(u)$.
\end{quote}
This universal property states that for every object~$B$ of~$\cat{A}$, the map
\[
	α_B
	\colon
	\cat{A}(B, A) \to X(B) \,,
	\quad
	f \mapsto X(f)(u)
\]
is bijective.
These bijections are natural.
Indeed, let
\[
	g \colon B \to B'
\]
be a morphism in~$\cat{A}$.
The resulting diagram
\[
	\begin{tikzcd}
		\cat{A}(B', A)
		\arrow{r}[above]{g^*}
		\arrow{d}[left]{α_{B'}}
		&
		\cat{A}(B, A)
		\arrow{d}[right]{α_B}
		\\
		X(B')
		\arrow{r}[above]{X(g)}
		&
		X(B)
	\end{tikzcd}
\]
commutes, because we have for every element~$f$ of its top-left corner the chain of equalities
\begin{align*}
	\SwapAboveDisplaySkip
	α_B( g^*( f ) )
	&=
	α_B(f ∘ g)
	\\
	&=
	X(f ∘ g)(u)
	\\
	&=
	(X(g) ∘ X(f))(u)
	\\
	&=
	X(g)( X(f)(u) )
	\\
	&=
	X(g)( α_{B'}( u ) ) \,.
\end{align*}
We have thus constructed a natural isomorphism~$α$ the functor~$\HY_A$ to the functor~$X$, showing that these functors are isomorphic.



%\subsubsection*{(c), second solution}
%
%We denote the presheaf category~$[\cat{A}^{\op}, \Set]$ by~$\cat{A}^+$.
%
%We refer to the given property of the presheaf~$X$ and the tuple~$(A, u)$ as its \emph{universal property}.
%This universal property is \enquote{internal} to~$X$:
%it does not involve any other presheaf, and instead makes a statement about the internal structure of~$X$.
%
%We show in the following that as a consequence of its internal universal property, the presheaf~$X$ also satisfies the following \enquote{external} universal property:
%\begin{quote}
%	\itshape
%	For every presheaf~$Y$ on~$\cat{A}$ and every element~$y$ of the set~$Y(A)$, there exists a unique natural transformation~$α$ from~$X$ to~$Y$ such that~$α_A(u) = y$.
%\end{quote}
%
%Suppose first that~$α$ is a natural transformation from~$X$ to~$Y$ with~$α_A(u) = y$.
%For every object~$B$ of~$\cat{A}$ and every element~$x$ of~$X(B)$, there exists a morphism~$f$ from~$B$ to~$A$ in~$\cat{A}$ with~$x = X(f)(u)$ by the internal universal property of~$X$ and~$(A, u)$.
%It follows from the naturality of~$α$ that
%\[
%	α_B(x)
%	=
%	α_B( X(f)(u) )
%	=
%	Y(f)( α_A(u) )
%	=
%	Y(f)(y) \,.
%\]
%This shows that the natural transformation~$α$ is already uniquely determined by its single value~$α_A(u)$.
%This proves the claimed uniqueness.
%
%Let now~$y$ be an arbitrary element of~$Y(A)$.
%We need to construct a natural transformation~$α$ from~$X$ to~$Y$ with~$α_A(u) = y$.
%
%Given an object~$B$ of~$\cat{B}$, every element of the set~$X(B)$ can be expressed as~$X(f)(u)$ for some morphism~$f$ from~$B$ to~$A$, by the internal universal property of~$X$ and~$(A, u)$.
%We can therefore define a map
%\[
%	α_B
%	\colon
%	X(B) \to Y(B) \,,
%\]
%via
%\[
%	α_{B}( X(f)(u) ) ≔ Y(f)(y)
%\]
%for every object~$B$ of~$\cat{A}$ and every element~$f$ of~$\cat{A}(B, A)$.
%These maps define a natural transformation~$α$ from~$X$ to~$Y$:
%for every morphism
%\[
%	g \colon B' \to B
%\]
%in~$\cat{A}$, we have the chain of equalities
%\begin{align*}
%	α_{B'}( X(g)( X(f)(u) ) )
%	&=
%	α_{B'}( X(f ∘ g)(u) )
%	\\
%	&=
%	Y(f ∘ g)(y)
%	\\
%	&=
%	Y(g)( Y(f)(y) )
%	\\
%	&=
%	Y(g)( α_{B}( X(f)(u) ) )
%\end{align*}
%for every element~$f$ of~$\cat{A}(B, A)$, which shows the required commutativity of the following square diagram:
%\[
%	\begin{tikzcd}
%		X(B)
%		\arrow{r}[above]{X(g)}
%		\arrow{d}[left]{α_{B}}
%		&
%		X(B')
%		\arrow{d}[right]{α_{B'}}
%		\\
%		Y(B)
%		\arrow{r}[above]{Y(g)}
%		&
%		Y(B')
%	\end{tikzcd}
%\]
%
%We have thus proven that the functor~$X$ satisfies the universal property.
%We may rephrase this property by saying that the map
%\[
%	Φ_Y
%	\colon
%	\cat{A}^+(X, Y) \to Y(A) \,,
%	\quad
%	α \mapsto α_A(u)
%\]
%is bijective.
%This bijection is natural in~$Y$.
%Indeed, let~$Y$ and~$Z$ be two presheaves on~$\cat{A}$ and let~$β$ be a natural transformation from~$Y$ to~$Z$.
%The resulting diagram
%\[
%	\begin{tikzcd}
%		\cat{A}^+(X, Y)
%		\arrow{r}[above]{Φ_Y}
%		\arrow{d}[left]{β_*}
%		&
%		Y(A)
%		\arrow{d}[right]{β_A}
%		\\
%		\cat{A}^+(X, Z)
%		\arrow{r}[above]{Φ_Z}
%		&
%		Z(A)
%	\end{tikzcd}
%\]
%commutes because
%\[
%	Φ_Z( β_*( α ) )
%	=
%	Φ_Z( β ∘ α )
%	=
%	(β ∘ α)_A(u)
%	=
%	β_A( α_A( u ) )
%	=
%	β_A( Φ_Y( α ) )
%\]
%for every element~$α$ of~$\cat{A}(X, Y)$.
%
%We have overall seen that the internal universal property of~$X$ results in an external universal property, which in turn tells us that the functor~$\cat{A}^+(X, \ph)$ is naturally isomorphic to the evaluation functor at~$A$.
%
%We now observe that~$(\HY_A, \id_A)$ has the same internal universal property as~$(X, A)$.
%Therefore, the functor~$\cat{A}^+(\HY_A, \ph)$ as also naturally isomorphic to the evaluation functor at~$A$.
%It follows that two functors~$\cat{A}^+(X, \ph)$ and~$\cat{A}^+(\HY_A, \ph)$ are naturally isomorphic.
%By Lemma~4.3.8, this implies that~$X$ and~$\HY_A$ are isomorphic in~$\cat{A}^+$.
