\subsection{}



\subsubsection{}

Let
\[
	f \colon A \to A'
\]
be a morphism in~$\cat{A}$.
The induces natural transformation
\[
	f_* \colon \HY_A \To \HY_{A'}
\]
satisfies
\[
	(f_*)_{A}(\id_A)
	=
	f ∘ \id_A
	=
	f \,.
\]
Therefore,~$f$ can be retrieved from its induced natural transformation~$f_*$.
This tells us that the functor~$\HY_\bullet$ is faithful.



\subsubsection{}

Let~$A$ and~$A'$ be two objects of~$\cat{A}$, and let~$α$ be a natural transformation from~$\HY_A$ to~$\HY_{A'}$.
Let~$f$ be the image of~$\id_A$ under the map~$α_A$, i.e., under the map
\[
	\cat{A}(A, A)
	=
	\HY_A(A)
	\xto{α_A}
	\HY_{A'}(A)
	=
	\cat{A}(A, A') \,.
\]
This entails that~$f$ is a morphism in~$\cat{A}$ from~$A$ to~$A$'.

The naturality of~$α$ tells us that we have for every morphism
\[
	g \colon B \to B'
\]
in~$\cat{A}$ the following commutative sequare diagram:
\[
	\begin{tikzcd}[sep = large]
		\HY_A(B')
		\arrow{r}[above]{\HY_A(g)}
		\arrow{d}[left]{α_{B'}}
		&
		\HY_A(B)
		\arrow{d}[right]{α_B}
		\\
		\HY_{A'}(B')
		\arrow{r}[above]{\HY_{A'}(g)}
		&
		\HY_{A'}(B)
	\end{tikzcd}
\]
This diagram may equivalently be written out as follows:
\[
	\begin{tikzcd}[sep = large]
		\cat{A}(B', A)
		\arrow{r}[above]{g^*}
		\arrow{d}[left]{α_{B'}}
		&
		\cat{A}(B, A)
		\arrow{d}[right]{α_B}
		\\
		\cat{A}(B', A')
		\arrow{r}[above]{g^*}
		&
		\cat{A}(B, A')
	\end{tikzcd}
\]
Let now~$B$ be any object~$\cat{B}$ and let~$g$ be an element of the set~$\HY_A(B)$.
In light of the equality~$\HY_A(B) = \cat{A}(B, A)$, this means that~$g$ is a morphism from~$B$ to~$A$.
We have therefore the following commutative diagram:
\[
	\begin{tikzcd}[sep = large]
		\cat{A}(A, A)
		\arrow{r}[above]{g^*}
		\arrow{d}[left]{α_A}
		&
		\cat{A}(B, A)
		\arrow{d}[right]{α_B}
		\\
		\cat{A}(A, A')
		\arrow{r}[above]{g^*}
		&
		\cat{A}(B, A')
	\end{tikzcd}
\]
The commutativity of this diagram tells us that
\[
	α_B( g )
	=
	α_B( g^*( \id_A ))
	=
	g^*( α_A( \id_A ) )
	=
	g^*( f )
	=
	f ∘ g
	=
	f_*( g ) \,.
\]
This shows that the map
\[
	α_B \colon \HY_A(B) \to \HY_{A'}(B)
\]
is given by~$f_*$.

We have thus shown that the natural transformations~$α$ and~$f_*$ from~$\HY_A$ to~$\HY_{A'}$ coincide in each component, whence they are equal.
This shows that each natural transformation from~$\HY_A$ to~$\HY_{A'}$ stems from a morphism from~$A$ to~$A'$, so that the functor~$\HY_{\bullet}$ is full.



\subsubsection{}

We denote the presheaf category of~$\cat{A}$, i.e., the functor category~$[\cat{A}^{\op}, \Set]$, by~$\cat{A}^+$.
We refer to the given property of the pair~$(X, u)$ as its \emph{universal property}.
We might think about it as a universal property \enquote{internal} to~$X$:
it does not involve any other presheaf, and instead makes a statement about the internal structure of~$X$.

We show in the following that the presheaf~$X$ satisfies the following \enquote{external} universal property:
\begin{quote}
	For every presheaf~$Y$ on~$\cat{A}$ and every element~$y$ of the set~$Y(A)$, there exists a unique natural transformation~$α$ from~$X$ to~$Y$ with~$α_A(u) = y$.
\end{quote}

Suppose first that~$α$ is a natural transformation from~$X$ to~$Y$ with~$α_A(u) = y$.
For every object~$A'$ of~$\cat{A}$ and every element~$x$ of~$X(A')$, there exists a morphism~$f$ from~$A'$ to~$A$ with~$x = X(f)(u)$y by the universal property of~$(X, u)$.
It follows from the naturality of~$α$ that
\[
	α_{A'}(x)
	=
	α_{A'}(X(f)(u))
	=
	Y(f)(α_A(u))
	=
	Y(f)(y) \,.
\]
This shows that the natural transformation~$α$ is already uniquely determined by its single value~$α_A(u)$.
This proved the claimed uniqueness.

Let now~$y$ be an arbitrary element of~$Y(A)$.
According to the universal property of~$(X, u)$, every element of~$X(A')$, where~$A'$ is an arbitrary object of~$\cat{A}$, can uniquely be expressed in the form~$X(f)(u)$ for some morphism~$f$ from~$A'$ to~$A$, i.e., some element~$f$ on~$\cat{A}(A', A)$.
This allows us to define the map
\[
	α_{A'}
	\colon
	X(A') \to Y(A') \,,
\]
via
\[
	α_{A'}( X(f)(u) ) ≔ Y(f)(y)
\]
for every object~$A'$ of~$\cat{A}$ and every element~$f$ of~$\cat{A}(A', A)$.
These maps define a natural transformation~$α$ from~$X$ to~$Y$:
for every morphism
\[
	g \colon A'' \to A'
\]
in~$\cat{A}$, we have the chain of equalities
\begin{align*}
	α_{A''}( X(g)( X(f)(u) ) )
	&=
	α_{A''}( X(f ∘ g)(u) )
	\\
	&=
	Y(f ∘ g)(y)
	\\
	&=
	Y(g)( Y(f)(y) )
	\\
	&=
	Y(g)( α_{A'}( X(f)(u) ) )
\end{align*}
for every element~$f$ of~$\cat{A}(A', A)$, which shows the required commutativity of the following square diagram:
\[
	\begin{tikzcd}[sep = large]
		X(A')
		\arrow{r}[above]{X(g)}
		\arrow{d}[left]{α_{A'}}
		&
		X(A'')
		\arrow{d}[right]{α_{A''}}
		\\
		Y(A')
		\arrow{r}[above]{Y(g)}
		&
		Y(A'')
	\end{tikzcd}
\]

We have thus proven that the functor~$X$ satisfies the universal property.
We may rephrase this property by saying that the map
\[
	Φ_Y
	\colon
	\cat{A}^+(X, Y) \to Y(A) \,,
	\quad
	α \mapsto α_A(u)
\]
is bijective.
This bijection is natural in~$Y$.
Indeed, let~$Y$ and~$Z$ be two presheafes on~$\cat{A}$ and let~$β$ be a natural transformation from~$Y$ to~$Z$.
The resulting diagram
\[
	\begin{tikzcd}[sep = large]
		\cat{A}^+(X, Y)
		\arrow{r}[above]{Φ_Y}
		\arrow{d}[left]{β_*}
		&
		Y(A)
		\arrow{d}[right]{β_A}
		\\
		\cat{A}^+(X, Z)
		\arrow{r}[above]{Φ_Z}
		&
		Z(A)
	\end{tikzcd}
\]
commutes because
\[
	Φ_Z( β_*( α ) )
	=
	Φ_Z( β ∘ α )
	=
	(β ∘ α)_A(u)
	=
	β_A( α_A( u ) )
	=
	β_A( Φ_Y( α ) )
\]
for every element~$α$ of~$\cat{A}(X, Y)$.

We have overall seen that the internal universal property of~$X$ results in an external universal property, which in turn tells us that the functor~$\cat{A}^+(X, \ph)$ is naturally isomorphic to the evaluation functor at~$A$.

We now observe that~$(\HY_A, \id_A)$ has the same internal universal property as~$(X, A)$.
Therefore, the functor~$\cat{A}^+(\HY_A, \ph)$ as also naturally isomorphic to the evaluation functor at~$A$.
It follows that that two functors~$\cat{A}^+(X, \ph)$ and~$\cat{A}^+(\HY_A, \ph)$ are naturally isomorphic.
By Lemma~4.3.8, this implies that~$X$ and~$\HY_A$ are isomorphic in~$\cat{A}^+$.
