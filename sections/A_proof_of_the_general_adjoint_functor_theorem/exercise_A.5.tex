\subsection{}



\subsubsection{}

For every object~$I$ of~$\scat{I}$, an object~$D(I)$ of~$(A \comma G)$ is the same as a pair~$(E(I), e_I)$ consisting of an object~$E(I)$ of~$\cat{B}$ and a morphism~$e_I$ from~$A$ to~$G(E(I))$.
For every morphism~$u \colon I \to J$ in~$\scat{I}$, a morphism~$D(u)$ from~$D(I)$ to~$D(J)$ is then the same as a morphism~$E(u)$ from~$E(I)$ to~$E(J)$ such that~$G(E(u)) ∘ e_I = e_J$.

The functoriality of the assignment~$D$ is precisely the functoriality of the assignment~$E$.
If this functoriality is satisfied, then~$(A, (e_I)_)I$ is precisely a cone on~$G ∘ E$ with vertex~$A$.

We hence find that a functor~$D$ from~$\scat{I}$ to~$(A \comma G)$ is the same as a functor~$E$ from~$\scat{I}$ to~$\cat{B}$ together with a cone~$(A, (e_I)_I)$ on~$G ∘ E$.
The two functors~$D$ and~$E$ are related through the projection functor~$P_A$ from~$(A \comma G)$ to~$\cat{B}$ via
\[
	E = P_A ∘ D \,.
\]



\subsubsection{}

Let~$\scat{I}$ be a small category and let~$D$ be a diagram in~$(A \comma G)$ of shape~$\scat{I}$.
The resulting diagram~$E ≔ P_A ∘ D$ in~$\scat{B}$ admits a limit~$(L', (p'_I)_I)$ since the category~$\cat{B}$ is complete.
We need to show the following two assertions:
\begin{enumerate*}

	\item
		There exists a unique cone~$(L, (p_I)_I)$ on the original diagram~$D$ such that~$L' = P_A(L)$ and~$p'_I = P_A(p_I)$ for every object~$I$ of~$\scat{I}$.

	\item
		This cone~$(L, (p_I)_I)$ is a limit cone.

\end{enumerate*}

We know from part~(a) of this exercise that we may regard the diagram~$D$ in~$(A \comma G)$ as the diagram~$E$ together with a cone of the form~$(A, (e_I)_I )$ on~$G ∘ E$.
To prove the first of the above two assertions, we make the following observations:
\begin{itemize*}

	\item
		Lifting the object~$L'$ of~$\cat{B}$ to an object~$L$ of~$(A \comma G)$ means choosing a morphism~$ℓ$ from~$A$ to~$G(L')$ in~$\cat{A}$, so that then~$L = (L', ℓ)$.

	\item
		Let~$I$ be some object of~$\scat{I}$.
		There can be at most one lift of the morphism~$p'_I \colon L' \to E(I)$ to a morphism~$p_I \colon L \to D(I)$ because the projection functor~$P_A$ is faithful.
		More explicitly, the only possible lift is~$p'_I$ itself, but~$p'_I$ is a morphism from~$L = (L', ℓ)$ to~$D(I) = (E(I), e_I)$ if and only if the diagram
		\[
			\begin{tikzcd}[column sep = normal]
				{}
				&
				A
				\arrow{dl}[above left]{ℓ}
				\arrow{dr}[above right]{e_I}
				&
				{}
				\\
				G(L')
				\arrow{rr}[above]{G(p'_I)}
				&
				{}
				&
				G(E(I))
			\end{tikzcd}
		\]
		commutes, i.e., if and only if~$e_I = G(p'_I) ∘ ℓ$.

	\item
		Writing~$p_I = p'_I$, the object~$L$ of~$(A \comma G)$ together with the morphisms~$p_I$ is then automatically a cone for the diagram~$D$.
		Indeed, we need for every morphism~$u \colon I \to J$ the commutativity of the following diagram:
		\[
			\begin{tikzcd}[column sep = normal]
				{}
				&
				L
				\arrow{dl}[above left]{p_I}
				\arrow{dr}[above right]{p_J}
				&
				{}
				\\
				D(I)
				\arrow{rr}[above]{D(u)}
				&
				{}
				&
				D(J)
			\end{tikzcd}
		\]
		It suffices to check that this diagram commutes after applying the forgetful functor~$P_A$, resulting in the following diagram:
		\[
			\begin{tikzcd}[column sep = normal]
				{}
				&
				L'
				\arrow{dl}[above left]{p'_I}
				\arrow{dr}[above right]{p'_J}
				&
				{}
				\\
				E(I)
				\arrow{rr}[above]{E(u)}
				&
				{}
				&
				E(J)
			\end{tikzcd}
		\]
		This diagram commutes because~$(L', (p'_I)_I)$ is a cone on~$E$.
\end{itemize*}

As a consequence of these observations, to prove the first assertion, we need to show that there exists a unique morphism~$ℓ$ from~$A$ to~$G(L')$ such that~$G(p'_I) ∘ ℓ = e_I$ for every object~$I$ of~$\scat{I}$.

We note that the object~$G(L')$ of~$\cat{A}$ together with the morphisms~$G(p'_I)$ forms a cone on~$G ∘ E$ because~$(L', (p'_I)_I)$ is a cone on~$E$.
We note that the cone~$(G(L'), (G(p'_I))_I)$ in question is already a limit cone because the original cone~$(L', (p'_I)_I)$ is a limit cone and the functor~$G$ is continuous.
It follows that there exists a unique morphism~$ℓ$ from~$A$ to~$G(L')$ with
\[
	G(p'_I) ∘ ℓ = e_I
\]
for every object~$I$ of~$\scat{I}$.
The overall situation is depicted in \cref{constructing ell}.
\begin{figure}
	\[
		\begin{array}{ccc}
			\begin{tikzcd}[column sep = normal]
				A
				\arrow[dashed]{rr}[above]{ℓ}
				\arrow{dr}[below left]{e}
				&
				{}
				&
				G(L')
				\arrow{dl}[below right]{G(p')}
				\\
				{}
				&
				G ∘ E
				&
				{}
			\end{tikzcd}
			& \displaystyle \leadsfrom &
			\begin{tikzcd}
				{}
				&
				L'
				\arrow{dl}[below right]{p'}
				\\
				E
				&
				{}
			\end{tikzcd}
			\\[4em]
			\text{in~$\cat{A}$} \hspace{1.5em} % dirty spacing hack
			& {} &
			\text{in~$\cat{B}$}
		\end{array}
	\]
	\caption{Construction of~$ℓ$.}
	\label{constructing ell}
\end{figure}

We have thus proven the first assertion by explicitly construction the required cone~$(L, (p_I)_I)$ on~$D$.

Suppose now that~$(C, (q_I)_I)$ is another cone on~$D$.
To prove the second assertion, we need to show that there exists a unique morphism~$f$ from~$C$ to~$L$ in~$(A \comma G)$ such that~$p_I ∘ f = q_I$ for every object~$I$ of~$\scat{I}$.

The object~$c$ is of the form~$C = (C', c)$ for an object~$C'$ of~$\cat{B}$ and a morphism~$c \colon A \to G(C')$ in~$\cat{A}$, and each morphism~$q_I \colon C \to D(I)$ can we regarded as a morphism~$q'_I \colon C' \to E(I)$ with
\begin{equation}
	\label{property on q prime}
	G(q'_I) ∘ c = e_I \,.
\end{equation}

We need to show that there exists a unique morphism~$f$ from~$C'$ to~$L'$ in~$\cat{B}$ such that the following two properties hold:
\begin{enumerate*}

	\item
		$G(f) ∘ c = ℓ$, where both sides are morphisms from~$G(C')$ to~$G(L')$.

	\item
		$p'_I ∘ f = q'_I$ for every object~$I$ of~$\scat{I}$.

\end{enumerate*}
Indeed, the first property is what it means for the morphism~$f$ from~$C'$ to~$L'$ to also be a morphism from~$C = (C', c)$ to~$L = (L', ℓ)$, whereas the second property means precisely that~$p_I ∘ f = q_I$ for every object~$I$ of~$\scat{I}$.

We know that~$(C, (q_I)_I)$ is a cone on~$D$, so by applying the projection functor~$P_A$ we find that~$(C', (q'_I)_I)$ is a cone on~$E$.
It follows that there exists a unique morphism~$f$ from~$C'$ to~$L'$ satisfying the second property, as~$(L', (p'_I)_I)$ is a limit cone on~$E$.

To show the required equality~$G(f) ∘ c = ℓ$, it suffices to show that
\[
	G(p'_I) ∘ G(f) ∘ c = G(p'_I) ∘ ℓ
\]
for every object~$I$ of~$\scat{I}$, thanks to Exercise~5.1.36, part~(a) and because the cone~$(G(C'), (G(p'_I))_I)$ is a limit cone.
This required equality holds because
\[
	G(p'_I) ∘ G(f) ∘ c
	=
	G(p'_I ∘ f)
	=
	G(q'_I) ∘ c
	=
	e_I
	=
	G(p'_I) ∘ ℓ \,,
\]
where we use the functoriality of~$G$, the definition of~$f$, identity~\eqref{property on q prime}, and the definition of~$ℓ$.
