\subsection{}



\subsubsection{}

The coequalizer of~$f$ and~$\id_X$ is given by the quotient set~$X / {∼}$, where~$∼$ is the equivalence relation on~$X$ generated by
\[
	x ∼ f(x)
	\qquad
	\text{for every~$x ∈ X$} \,.
\]
This equivalence relation can also be described more explicitly:
two elements~$x$ and~$y$ are equivalent with respect to~$∼$ if and only if they are contained in the same~$f$\nobreakdash-orbit, i.e., if and only if there exist natural exponents~$n$ and~$m$ with
\[
	f^n(x) = f^m(y) \,.
\]
We note that if~$f$ is bijective, then the elements of~$X$ that are equivalent to a given element~$x$ are precisely those of the form~$f^n(x)$ with~$n ∈ ℤ$.



\subsubsection{}

The description of the coequalizer stays the same in terms of the underlying sets;
its topology is the quotient topology induces from~$X$.

For~$X = \Sphere^1$ we may consider the rotation map
\[
	f
	\colon
	\Sphere^1 \to \Sphere^1 \,,
	\quad
	x \mapsto \eul^{2 \pi \img α} x
\]
for some irrational number~$α$.
The orbit of any element of~$X$ is a countable, dense subset of~$\Sphere^1$.
The quotient space~$X / {\sim}$ is therefore an uncountable indiscrete topological space.
