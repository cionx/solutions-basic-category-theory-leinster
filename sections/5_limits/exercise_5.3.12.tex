\subsection{}

Let~$D$ be a diagram of shape~$\scat{I}$ in~$\cat{A}$.

The induced diagram~$F ∘ D$ in~$\cat{B}$ is again of shape~$\scat{I}$, and therefore admits a limit cone~$(L', (p'_I)_I)$.
The functor~$F$ creates limits by assumption.
There hence exists a unique cone~$(L, (p_I)_I)$ of~$D$ such that~$L' = F(L)$ and~$p'_I = F(p_I)$ for every object~$I$ of~$\scat{I}$, and this cone~$(L, (p_I)_I)$ is a limit cone of~$D$.
This shows that the category~$\cat{A}$ has limits of shape~$\scat{I}$.

To show that~$F$ preserves limits, we make the following observations.
\begin{itemize}

	\item
		For every category~$\cat{A}$ and every diagram~$D$ of shape~$\scat{I}$ in~$\cat{A}$, we can form the category~$\Cone(D)$ of cones over~$D$.
		\begin{itemize}

			\item
				The objects of~$\Cone(D)$ are cones over~$D$.

			\item
				A morphism from a cone~$(C, (p_I)_I)$ to a cone~$(C', (p'_I)_I)$ in~$\Cone(D)$ is a morphism~$f$ from~$C$ to~$C'$ in~$\cat{A}$ such that~$p'_I ∘ f = p_I$ for every object~$I$ of~$\scat{I}$.

			\item
				The composition of morphisms in~$\Cone(D)$ is the composition of morphisms in~$\cat{A}$.

			\item
				For every cone~$(C, (p_I)_I)$ over~$D$, its identity morphism in~$\Cone(D)$ is given by the identity morphism of~$C$ in~$\cat{A}$.

		\end{itemize}
		(We have thus a forgetful functor from~$\Cone(D)$ to~$\cat{A}$ that assigns to each cone its vertex.)

	\item
		A cone over a diagram~$D$ is a limit cone over~$D$ if and only if it is terminal in~$\Cone(D)$.

	\item
		Let~$\cat{A}$ and~$\cat{B}$ be two categories, let~$F$ be a functor from~$\cat{A}$ to~$\cat{B}$, and let~$D$ be a diagram of shape~$\scat{I}$ in~$\cat{A}$.
		The functor~$F$ induces a functor~$\Cone(F)$ from~$\Cone(D)$ to~$\Cone(F ∘ D)$ as follows:
		\begin{itemize}

			\item
				For every cone~$(C, (p_I)_I)$ over~$D$, its image under the functor~$\Cone(F)$ is the cone~$(F(C), (F(p_I))_I)$ over~$F ∘ D$.

			\item
				Let~$(C, (p_I)_I)$ and~$(C', (p'_I)_I)$ be two cones over~$D$ and let~$f$ be a morphism from~$(C, (p_I)_I)$ to~$(C', (p'_I)_I)$.
				The image of~$f$ under the functor~$\Cone(F)$ is~$F(f)$.

		\end{itemize}
\end{itemize}

Let now~$\tilde{C} ≔ (\tilde{L}, (\tilde{p}_I)_I)$ be a limit cone of the diagram~$D$.
We know that the diagram~$C ≔ (L, (p_I)_I)$ is also a limit cone of~$D$.
It follows that the cones~$\tilde{C}$ and~$C$ are both terminal in~$\Cone(D)$, and therefore isomorphic in~$\Cone(D)$.
The resulting cones~$\Cone(F)(\tilde{C})$ and~$\Cone(F)(C)$ are therefore isomorphic in~$\Cone(F ∘ D)$.
The cone~$\Cone(F)(C)$ is given by~$(L', (p'_I)_I)$, which is a limit cone of~$F ∘ D$.
Therefore,~$\Cone(F)(C)$ is terminal in~$\Cone(F ∘ D)$.
It follows that~$\Cone(F)(\tilde{C})$ is also terminal in~$\Cone(F ∘ D)$, because it is isomorphic to~$\Cone(F)(C)$.
This means that the cone~$\Cone(F)(\tilde{C}) = (F(\tilde{L}), (\tilde{p}_I)_I)$ is a limit cone for the diagram~$F ∘ D$.

This shows that the functor~$F$ preserves limits.

%with~$\tilde{p}_I ∘ f = p_I$ and~$p_I ∘ \tilde{f} = \tilde{p}_I$ for every object~$I$ of~$\scat{I}$.
%We find for the composite~$\tilde{f} ∘ f$ that
%\[
%	p_I ∘ \tilde{f} ∘ f
%	=
%	\tilde{p}_I ∘ f
%	=
%	p_I
%	=
%	p_I ∘ \id_L
%\]
%for every object~$I$ of~$\scat{I}$, and therefore~$\tilde{f} ∘ f = \id_L$.
%We find similarly that~$f ∘ \tilde{f} = \id_{\tilde{L}}$.
%The two morphisms~$f$ and~$\tilde{f}$ are therefore mutually inverse isomorphisms.
%We have~$F(L) = L'$, and the induced morphisms
%\[
%	F(f)
%	\colon
%	L' \to F(\tilde{L}) \,,
%	\quad
%	F(\tilde{f})
%	\colon
%	F(\tilde{L}) \to L'
%\]
%are again mutually inverse isomorphisms.
%They satisfy
%\[
%	F(\tilde{p}_I) ∘ F(f)
%	=
%	F(\tilde{p}_I ∘ F)
%	=
%	F(p_I)
%	=
%	p'_I
%\]
%and similarly
%\[
%	p'_I ∘ F(\tilde{f}) = F(\tilde{p}_I)
%\]
%for every object~$I$ of~$\scat{I}$.
%
%Let~$(L'', (p''_I)_I)$ be a cone of the diagram~$F ∘ D$.
%We know that there exists a unique morphism~$g$ from~$L''$ to~$L'$ with~$p'_I ∘ g = p''_I$ for every object~$I$ of~$\scat{I}$.
%It follows from the above observations that similarly there exists a unique morphism~$h$ from~$L''$ to~$F(\tilde{L})$ with~$F(\tilde{p}_I) ∘ h = p''_I$ for every object~$I$ of~$\scat{I}$, namely the morphism~$h = F(f) ∘ g$.
%This shows that the cone~$( F(\tilde{L}), (\tilde{p}_I)_I )$ of the diagram~$F ∘ D$ is already a limit cone.
%
%We have thus shown that the functor~$F$ preserves limits.
