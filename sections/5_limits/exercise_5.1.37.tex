\subsection{}

The set in question is given by
\[
	L
	=
	\Biggl\{
		(x_I)_I
		∈
		∏_{I ∈ \Ob(\scat{I})} D(I)
	\suchthat[\Bigg]
		\begin{tabular}{@{}l@{}}
			$D(u)(x_I) = x_J$ for every \\
			morphism~$u \colon I \to J$ in~$\scat{I}$
		\end{tabular}
	\Biggr\} \,.
\]
For every object~$I$ of~$\scat{I}$ let~$p_I$ be the projection onto the~\nth{$I$} component from~$L$ to~$D(I)$.
The elements of~$L$ are chosen precisely in such a way that~$L$ together with the family of maps~$(p_I)_I$ is a cone over the diagram~$D$.
It remains to show that the cone~$(L, (p_I)_I)$ is universal.

To this end, let~$(A, (f_I)_I)$ be another cone over~$D$.
We need to show that there exists a unique map~$f$ from~$A$ to~$L$ satisfying the condition~$p_I ∘ f = f_I$ for every object~$I$ of~$\scat{I}$.

We start by showing the uniqueness of the map~$f$.
Given an element~$a$ of~$A$, the element~$f(a)$ needs to satisfy the equalities
\[
	p_I( f(a) )
	=
	(p_I ∘ f)(a)
	=
	f_I(a)
\]
for every object~$I$ of~$\scat{I}$.
This shows that the components of~$f(a)$ are uniquely determined by the functions~$f_I$, whence the value~$f(a)$ is uniquely determined.
This means overall, that the map~$f$ is unique.

We now show the existence of the desired map~$f$.
We start with the map
\[
	\tilde{f}
	\colon
	A \to ∏_{I ∈ \Ob(\scat{I})} D(I) \,,
	\quad
	a \mapsto ( f_I(a) )_I \,.
\]
For every element~$a$ of~$A$, we have the chain of equalities
\[
	D(u)( p_I( \tilde{f}(a) ) )
	=
	D(u)( f_I(a) )
	=
	f_J(a)
	=
	p_J( \tilde{f}(a) )
\]
for every morphism~$u \colon I \to J$ in~$\scat{I}$.
This tells us that the image of the map~$\tilde{f}$ is contained in the subset~$L$ of~$∏_{I ∈ \Ob(\scat{I})} D(I)$.
We can therefore corestrict the map~$\tilde{f}$ to a map
\[
	f \colon A \to L \,.
\]
The auxiliary map~$\tilde{f}$ satisfies the condition~$\pr_I ∘ \tilde{f} = f_I$ for every object~$I$ of~$\scat{I}$, whence the map~$f$ satisfies~$p_I ∘ f = f_I$ for every object~$I$ of~$\scat{I}$.
