\subsection{}



\subsubsection{}

Let~$S$ be an arbitrary set.
We have
\[
	\cat{B}(F(S), \ph)
	≅
	\Set(S, G(\ph))
	≅
	\Set(S, \ph) ∘ G \,.
\]
The functor~$G$ preserves epimorphisms by assumption, so it suffices to show that the functor~$\Set(S, \ph)$ epimorphisms.
But the axiom of choice asserts that every epimorphism in~$\Set$ is a split epimorphism, and every functor preserves split epimorphisms.
(The assertion that~$\Set(S, \ph)$ preserves epimorphisms for every set~$S$, i.e., that every object of~$\Set$ is projective, is in fact equivalent to the axiom of choice.)



\subsubsection{}

The epimorphisms in~$\Ab$ are precisely the surjective homomorphism.
If~$P$ is a projective object of~$\Ab$, then it thus follows that for every surjective homomorphism of abelian groups
\[
	f \colon A \to B \,,
\]
the induced map
\[
	f_* \colon \Ab(P, A) \to \Ab(P, B)
\]
is again surjective.
We consider now the group~$P = ℤ / 2$ and the surjective homomorphism of abelian groups
\[
	f
	\colon
	ℤ \to ℤ / 2 \,,
	\quad
	x \mapsto \class{x} \,.
\]
The induced map
\[
	f_* \colon \Ab(ℤ / 2, ℤ) \to \Ab(ℤ / 2, ℤ / 2)
\]
is not surjective because its domain contains only a single element (namely the zero homomorphism) while its codomain contains to elements (the zero homomorphism and the identity homomorphism).

\begin{remark}
	It is a well-know fact from algebra that in a module category~$\Mod{R}$, where~$R$ is some ring, the following conditions on an object~$P$ are equivalent:
	\begin{equivalenceslist}

		\item
			$P$ is projective in~$\Mod{R}$.

		\item
			For every epimorphism of~\modules{$R$}
			\[
				p \colon A \to B
			\]
			and every homomorphism of~\modules{$R$}~$f$ from~$P$ to~$B$, there exists a homomorphism of~\modules{$R$}~$g$ from~$P$ to~$A$ with~$f = p ∘ g$, i.e., such that the following diagram commutes:
			\[
				\begin{tikzcd}[sep = large]
					{}
					&
					P
					\arrow{d}[right]{f}
					\arrow[dashed]{dl}[above left]{g}
					\\
					A
					\arrow{r}[above]{p}
					&
					B
				\end{tikzcd}
			\]

		\item
			Every epimorphism of~\modules{$R$} with codomain~$P$ splits.

		\item
			Every short exact sequence ending in~$P$ splits.

		\item
			$P$ is (isomorphic to) a direct summand of a free~\module{$R$}.

	\end{equivalenceslist}
\end{remark}



\subsubsection{}

More explicitly, an object~$I$ of a category~$\cat{B}$ is injective if and only if the contravariant functor~$\cat{B}(\ph, I)$ turns monomorphisms (in~$\cat{B}$) into epimorphisms (in~$\Set$).
Even more explicitly:
for every monomorphism
\[
	m \colon A \to A'
\]
in~$\cat{B}$, every morphism~$f$ from~$A$ to~$I$ extends to a morphism~$g$ from~$A'$ to~$I$, in the sense that~$g' ∘ m = f$, i.e., in the sense that the following diagram commutes:
\[
	\begin{tikzcd}[sep = large]
		{}
		&
		I
		\\
		A
		\arrow{ur}[above left]{f}
		\arrow{r}[above]{m}
		&
		A'
		\arrow[dashed]{u}[right]{g}
	\end{tikzcd}
\]

Let~$W$ be an arbitrary~\vectorspace{$𝕜$} and let
\[
	m \colon U \to V
\]
be a monomorphism of~\vectorspaces{$𝕜$}.
This means that~$m$ is an injective,~\linear{$𝕜$} map from~$U$ to~$V$.
Thanks to the axiom of choice, there exists a linear subspace~$V'$ of~$V$ with
\[
	V = \im(m) ⊕ V' \,.
\]
It follows that there exists a linear map~$e$ from~$V$ to~$U$ with~$e ∘ m = \id_U$:
it is given by
\[
	e( m(u) + v' ) = u
\]
for all elements~$u$ and~$v'$ of~$U$ and~$V'$ respectively.
It follows for every linear map~$f$ from~$U$ to~$W$ that the composite~$g ≔ f ∘ e$ is a linear map from~$V$ to~$W$ with
\[
	g ∘ m
	=
	f ∘ e ∘ m
	=
	f ∘ \id_U
	=
	f \,.
\]
We have thus proven that~$W$ is injective.

Let now~$i$ be the inclusion map from~$2ℤ$ to~$ℤ$, and let
\[
	f
	\colon
	2ℤ \to ℤ \,,
	\quad
	ℤ \,,
	\quad
	n \mapsto \frac{n}{2} \,.
\]
There exists no homomorphism of groups~$g$ from~$ℤ$ to~$ℤ$ with~$g ∘ i = f$ because~$g(1)$ would need to be an element of~$ℤ$ with
\[
	2 g(1)
	=
	g(2)
	=
	g(i(2))
	=
	f(2)
	=
	\frac{2}{2}
	=
	1 \,,
\]
and such an element does not exist.
Therefore,~$ℤ$ is not injective as an object of~$\Ab$.

\begin{remark}
	Baer’s criterion asserts that an~\module{$R$}~$I$ is injective if and only if for every ideal~$J$ of~$R$, every homomorphism of~\modules{$R$} from~$J$ to~$I$ extends to a homomorphism of~\modules{$R$} from~$R$ to~$I$.

	If~$R$ is a principal ideal domain, then this means that an~\module{$R$} is injective if and only if it is divisible.
\end{remark}
