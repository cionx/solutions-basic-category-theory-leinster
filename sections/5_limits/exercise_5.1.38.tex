\subsection{}



\subsubsection{}

Let~$A$ be an object of~$\cat{A}$.

The object~$L$ together with the morphism~$p$ from~$∏_{I ∈ \Ob(\scat{I})} D(I)$ is an equalizer for the two morphisms~$s$ and~$t$.
We have therefore a bijection given by
\begin{equation}
	\label{first step in general constructions of limits}
	\cat{A}(A, L)
	\to
	\Biggl\{
		g ∈ \cat{A}\Biggl(A, \prod_{I ∈ \Ob(\scat{I})} D(I)\Biggr)
	\suchthat[\Bigg]
		s ∘ g = t ∘ g
	\Biggr\} \,,
	\quad
	f \mapsto p ∘ f \,.
\end{equation}
To better understand the right-hand side of this bijection, let~$g$ be a morphism from~$A$ to~$∏_{I ∈ \Ob(\scat{I})} D(I)$, with components~$g_I$ where~$I$ ranges through~$\Ob(\scat{I})$.
For every morphism
\[
	u \colon J \to K
\]
in~$\scat{I}$, the~\nth{$u$} component of the composite~$s ∘ g$ is given by~$D(u) ∘ \pr_J ∘ g = D(u) ∘ g_J$, whereas the~\nth{$u$} component of the composite~$t ∘ g$ is given by~$\pr_K ∘ g = g_K$.
We therefore find that~$s ∘ g = t ∘ g$ if and only if~$D(u) ∘ g_J = g_K$ for every morphism~$u \colon J \to K$ in~$\scat{I}$.
It follows that the bijection
\[
	\cat{A}\Biggl( A, \prod_{I ∈ \Ob(\scat{I})} D(I) \Biggr)
	\to
	\prod_{I ∈ \Ob(\scat{I})} \cat{A}(A, D(I)) \,,
	\quad
	g \mapsto (\pr_I ∘ g)_I
\]
restricts to a bijection between the right-hand side of~\eqref{first step in general constructions of limits} and
\[
	\left\{
		(g_I)_I ∈ \prod_{I ∈ \Ob(\scat{I})} \cat{A}(A, D(I))
	\suchthat*
		\begin{tabular}{@{}l@{}}
			$D(u) ∘ g_J = g_K$ for every \\
			morphism~$u \colon J \to K$ in~$\scat{I}$
		\end{tabular}
	\right\} \,.
\]
By combining these bijections, we arrive at a bijection
\begin{align*}
	\cat{A}(A, L)
	&\to
	\left\{
		(g_I)_I ∈ \prod_{I ∈ \Ob(\scat{I})} \cat{A}(A, D(I))
	\suchthat*
		\begin{tabular}{@{}l@{}}
			$D(u) ∘ g_J = g_K$ for every \\
			morphism~$u \colon J \to K$ in~$\scat{I}$
		\end{tabular}
	\right\} \,,
	\\
	g
	&\mapsto
	(\pr_I ∘ p ∘ g)_I
	=
	(p_I ∘ g)_I \,.
\end{align*}
This bijection encapsulates that the object~$L$ together with the given family of morphisms~$(p_I)_I$ is indeed a limit for the given diagram~$D$.



\subsubsection{}

That~$\cat{A}$ admits both binary products and a terminal object means precisely that~$\cat{A}$ admits all finite products.
We can therefore form for every finite category~$\scat{I}$ the products
\[
	\prod_{I ∈ \Ob(\scat{I})} D(I)
	\quad\text{and}\quad
	\prod_{\text{$u \colon J \to K$ in~$\scat{I}$}} D(K) \,.
\]
The general construction of limits from part~(a) can therefore be used without change.
