\subsection{}



\subsubsection{}

Let~$A$ be an object of~$\cat{A}$.

By definition, the object~$L$ together with the morphism~$p$ is an equalizer for the two morphisms~$s$ and~$t$.
We have therefore the bijection
\begin{align}
	\cat{A}(A, L)
	&\to
	\Biggl\{
		g ∈ \cat{A}\Biggl(A, ∏_{I ∈ \Ob(\scat{I})} D(I)\Biggr)
	\suchthat[\Bigg]
		s ∘ g = t ∘ g
	\Biggr\} \,,
	\label{first step in general constructions of limits}
	\\
	f &\mapsto p ∘ f \,.
	\notag
\end{align}
To better understand the right-hand side of this bijection, we use the bijection
\begin{align}
	\cat{A}\Biggl(A, ∏_{I ∈ \Ob(\scat{I})} D(I)\Biggr)
	&\to
	∏_{I ∈ \Ob(\scat{I})} \cat{A}(A, D(I)) \,,
	\label{map into product for construction of limits}
	\\
	g
	&\mapsto
	(\pr_I ∘ g)_I \,.
	\notag
\end{align}
How does the right-hand side of~\eqref{first step in general constructions of limits} look like under this bijection?
To answer this question, we need to reformulate the condition~$s ∘ g = t ∘ g$ in terms of the components of~$g$.

To this end, let~$g$ be a morphism from~$A$ to~$∏_{I ∈ \Ob(\scat{I})} D(I)$.
For every object~$I$ of~$\scat{I}$ let~$g_I$ be the respective component of~$g$, i.e.,
\[
	g_I = \pr_I ∘ g \,.
\]
The two morphisms~$s ∘ g$ and~$t ∘ g$ both go from the object~$A$ to the product~$∏_{\text{$u \colon J \to K$ in~$\scat{I}$}} D(K)$.
These two morphisms hence coincide if and only if they coincide in each component, i.e., if and only if
\begin{equation}
	\label{equality in each coordinate for construction of limits}
	\pr_u ∘ s ∘ g = \pr_u ∘ t ∘ g
\end{equation}
for every morphism~$u$ in~$\scat{I}$.
The morphisms~$s$ and~$t$ are defined via the formulae
\[
	\pr_u ∘ s = D(u) ∘ \pr_J
	\quad\text{and}\quad
	\pr_u ∘ t = \pr_K
\]
for every morphism~$u \colon J \to K$ in~$\scat{I}$.
It follows that~\eqref{equality in each coordinate for construction of limits} can be rewritten as
\[
	D(u) ∘ \pr_J ∘ g = \pr_K ∘ g \,,
\]
and therefore as
\[
	D(u) ∘ g_J = g_K
\]
for every morphism~$u \colon J \to K$ in~$\scat{I}$.

We find from our discussion that~$s ∘ g = t ∘ g$ if and only if~$D(u) ∘ g_J = g_K$ for every morphism~$u \colon J \to K$ in~$\scat{I}$.
In other words:
the bijection~\eqref{map into product for construction of limits} restricts to a bijection between the right-hand side of~\eqref{first step in general constructions of limits} and
\[
	\left\{
		(g_I)_I ∈ ∏_{I ∈ \Ob(\scat{I})} \cat{A}(A, D(I))
	\suchthat*
		\begin{tabular}{@{}l@{}}
			$D(u) ∘ g_J = g_K$ for every \\
			morphism~$u \colon J \to K$ in~$\scat{I}$
		\end{tabular}
	\right\} \,.
\]
By combining these bijections, we arrive at the bijection
\begin{align*}
	\cat{A}(A, L)
	&\to
	\left\{
		(g_I)_I ∈ ∏_{I ∈ \Ob(\scat{I})} \cat{A}(A, D(I))
	\suchthat*
		\begin{tabular}{@{}l@{}}
			$D(u) ∘ g_J = g_K$ for every \\
			morphism~$u \colon J \to K$ in~$\scat{I}$
		\end{tabular}
	\right\} \,,
	\\
	g
	&\mapsto
	(\pr_I ∘ p ∘ g)_I
	=
	(p_I ∘ g)_I \,.
\end{align*}
This bijection encapsulates that the object~$L$ together with the given family of morphisms~$(p_I)_I$ is indeed a limit for the given diagram~$D$.



\subsubsection{}

That~$\cat{A}$ admits both binary products and a terminal object means precisely that~$\cat{A}$ admits all finite products.
We can therefore form for every finite category~$\scat{I}$ the products
\[
	∏_{I ∈ \Ob(\scat{I})} D(I)
	\quad\text{and}\quad
	∏_{\text{$u \colon J \to K$ in~$\scat{I}$}} D(K) \,.
\]
The general construction of limits from part~(a) of this exercise can therefore be used without change.
