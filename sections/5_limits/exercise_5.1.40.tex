\subsection{}



\subsubsection{}

We may regard the power set of~$A$ as a partially ordered set via inclusion of subsets, and therefore as a category.
For every object~$m$ of~$\Monic(A)$, let $I(m)$ denote the image of~$m$ of~$A$.
A morphism
\[
	i \colon m \to m'
\]
in~$\Monic(A)$ between objects
\[
	m \colon X \to A \,,
	\quad
	m' \colon X' \to A
\]
amounts to the following commutative diagram:
\[
	\begin{tikzcd}[row sep = large]
		X
		\arrow{rr}[above]{i}
		\arrow{dr}[below left]{m}
		&
		{}
		&
		X'
		\arrow{dl}[below right]{m'}
		\\
		{}
		&
		A
		&
		{}
	\end{tikzcd}
\]
It follows from the the commutativity of this diagram that the image of~$m$ is contained in the image of~$m'$, i.e., that~$I(m) ⊆ I(m')$.
It further follows that the assignment~$I$ extends to a functor from~$\Monic(A)$ to~$\Power(A)$.
This implies that isomorphic objects of~$\Monic(A)$ have the same image in~$A$.

Let now conversely~$m$ and~$m'$ be two objects of~$\Monic(A)$ with~$I(m) = I(m')$.
We denote this subset of~$A$ by~$A'$, and the inclusion map from~$A'$ to~$A$ by~$i$.
The set~$A'$ is the image of both
\[
	m \colon X \to A
	\quad\text{and}\quad
	m' \colon X' \to A' \,,
\]
whence both of these maps factor through~$i$.
There hence exist maps
\[
	n \colon X \to A' \,,
	\quad
	n' \colon X \to A'
\]
with~$m = i ∘ n$ and~$m' = i ∘ n'$.
These maps~$n$ and~$n'$ are again injective, but they are also surjective by choice of~$A'$.
They are hence bijective, and therefore isomorphisms in~$\Set$.

Let~$f$ be the composite~$(n')^{-1} ∘ n$.
This is a map from~$X$ to~$X'$ that makes the following diagram commute:
\[
	\begin{tikzcd}[row sep = large]
		X
		\arrow{rr}[above]{f}
		\arrow{dr}[below left]{n}
		\arrow[bend right]{ddr}[below left]{m}
		&
		{}
		&
		X'
		\arrow{dl}[below right]{n'}
		\arrow[bend left]{ddl}[below right]{m'}
		\\
		{}
		&
		A'
		\arrow{d}[right]{i}
		&
		{}
		\\
		{}
		&
		A
		&
		{}
	\end{tikzcd}
\]
The commutativity of the outer triangle tells us that~$f$ is a morphism from~$m$ to~$m'$ in~$\Monic(A)$.
The map~$f$ is a bijection since it is a composite of bijections, with its inverse given by~$f^{-1} = n^{-1} ∘ n'$.
We find in the same way as for~$f$ that~$f^{-1}$ is a morphism from~$X'$ to~$X$ in~$\Monic(A)$.
The maps~$f$ and~$f'$ mutually inverse as morphisms of~$\Set$, and therefore also mutually inverse as morphisms of~$\Monic(A)$.
This shows that~$m$ and~$m'$ are isomorphic as objects of~$\Monic(A)$.

We have now seen that the functor~$I$ results in an injective map from the class of subobjects of~$A$ to the porer set of~$A$.
Conversely, every subset~$A'$ of~$A$ stems from a subobjet of~$A$, namely from the inclusion map
\[
	A' \to A
\]
regarded as a subobject of~$A$.



\subsubsection{}

Subobjects in~$\Grp$ are subgroups, subobjects in~$\Ring$ are subrings and subobjects in~$\Vect{\kf}$ are linear subspaces.
(This can be proven with the same argumentation as for~$\Set$ in part~(a).)



\subsubsection{}

We know that the monics in~$\Top$ are precisely the injective continuous maps.
A subobject of a topological space~$X$ in the categorical sense is therefore a subset~$Y$ of~$X$ together with a topology for which the inclusion map from~$Y$ to~$X$ is continuous.
In other words, the topology on~$Y$ needs to be coarser than the subspace topology induced from~$X$.
