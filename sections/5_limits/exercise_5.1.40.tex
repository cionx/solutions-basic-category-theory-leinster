\subsection{}

We denote objects of~$\Monic(A)$ as pairs~$(X, m)$ consisting of an object~$X$ of~$\cat{A}$ and a morphism~$m$ from~$X$ to~$A$.
We denote the class of subobjects of~$A$ by~$\Sub(A)$.



\subsubsection{}

Every subset~$A'$ of~$A$ results in an object of~$\Monic(A)$, namely~$(A', i)$ where~$i$ denotes the inclusion map from~$A'$ to~$A'$.
Conversely, we can consider for every object~$(X, m)$ of~$\Monic(A)$ the image of the map~$m$, which is a subset of~$A'$.
We have thus found a surjective map
\[
	I
	\colon
	\Ob(\Monic(A)) \to \Power(A) \,,
\]
assigning to each object~$(X, m)$ of~$\Monic(A)$ the image of the map~$m$.
We shall check in the following that the map~$I$ descends to a bijection between~$\Sub(A)$ and~$\Power(A)$.
More explicitly, we need to show that two objects of~$\Monic(A)$ are isomorphic if and only if they have the same image under~$I$.

Let us first show that isomorphic objects of~$\Monic(A)$ have the same image under~$I$.
A morphism
\[
	f \colon (X, m) \to (X', m')
\]
in~$\Monic(A)$ amounts to the following commutative diagram:
\[
	\begin{tikzcd}[column sep = normal]
		X
		\arrow{rr}[above]{f}
		\arrow{dr}[below left]{m}
		&
		{}
		&
		X'
		\arrow{dl}[below right]{m'}
		\\
		{}
		&
		A
		&
		{}
	\end{tikzcd}
\]
It follows from the commutativity of this diagram that the image of~$m$ is contained in the image of~$m'$, so that~$I((X, m)) ⊆ I((X', m'))$.
If we view the partially ordered set~$\Power(A)$ as a category, then this means that the assignment~$I$ extends to a functor from~$\Monic(A)$ to~$\Power(A)$.
This entails that isomorphic objects of~$\Monic(A)$ have the same image under~$I$.

Let us now conversely show that objects of~$\Monic(A)$ are isomorphic if they have the same image under~$I$.

Let~$(X, m)$ and~$(X', m')$ be two objects of~$\Monic(A)$ that have the same image under~$I$.
This means that the two maps~$m$ and~$m'$ have the same image in~$A$.
Let~$i$ be the inclusion map from~$A'$ to~$A$.
Both~$m$ and~$m'$ factor through~$i$, in the sense that there exist maps
\[
	n \colon X \to A' \,,
	\quad
	n' \colon X \to A'
\]
with~$m = i ∘ n$ and~$m' = i ∘ n'$.
The maps~$n$ and~$n'$ are injective because~$m$ and~$m'$ are injective, but they are also surjective by choice of~$A'$.
These maps are hence bijective, and thus invertible.

Let~$f$ be the composite~$(n')^{-1} ∘ n$, which is a map from~$X$ to~$X'$.
We have the following commutative diagram:
\[
	\begin{tikzcd}[column sep = normal]
		X
		\arrow{rr}[above]{f}
		\arrow{dr}[below left]{n}
		\arrow[bend right]{ddr}[below left]{m}
		&
		{}
		&
		X'
		\arrow{dl}[below right]{n'}
		\arrow[bend left]{ddl}[below right]{m'}
		\\
		{}
		&
		A'
		\arrow{d}[right]{i}
		&
		{}
		\\
		{}
		&
		A
		&
		{}
	\end{tikzcd}
\]
The commutativity of the outer part of this diagram tells us that~$f$ is a morphism from~$(X, m)$ to~$(X', m')$ in~$\Monic(A)$.
The map~$f$ is bijective since it is a composite of two bijections.
In other words,~$f$ is an isomorphism in~$\Set$.
Let us observe that~$f$ is therefore also an isomorphism in~$\Monic(A)$.

\begin{proposition}
	Let~$\cat{A}$ be a category and let~$A$ be an object of~$\cat{A}$.
	Let~$(X, m)$ and~$(X', m')$ be two objects of~$\Monic(A)$ and let~$f$ be a morphism from~$(X, m)$ to~$(X', m')$.
	Suppose that~$f$ is an isomorphism in~$\cat{A}$ with inverse~$f^{-1}$.
	\begin{enumerate}

		\item
			The inverse~$f^{-1}$ is a morphism from~$(X', m')$ to~$(X, m)$ in~$\Monic(A)$.

		\item
			\label{again mutually inverse in monic category}
			The two morphisms~$f$ and~$f^{-1}$ are mutually inverse in~$\Monic(A)$.

		\item
			The morphism~$f$ is also an isomorphism in~$\Monic(A)$.

	\end{enumerate}
\end{proposition}

\begin{proof}
	That~$f$ is a morphism from~$(X, m)$ to~$(X', m')$ tells us that the following diagram commutes:
	\[
		\begin{tikzcd}[column sep = normal]
			X
			\arrow{rr}[above]{f}
			\arrow{dr}[below left]{m}
			&
			{}
			&
			X'
			\arrow{dl}[below right]{m'}
			\\
			{}
			&
			A
			&
			{}
		\end{tikzcd}
	\]
	\begin{enumerate}

		\item
			The commutativity of the above diagram gives us the equality~$m = m' ∘ f$.
			Rearranging this equality leads to~$m ∘ f^{-1} = m'$, which tells us that~$f^{-1}$ is a morphism from~$(X', m')$ to~$(X, m)$.

		\item
			We have in~$\cat{A}$ the identities
			\begin{equation}
				\label{inverse in the original category}
				f ∘ f^{-1} = \id_{X'} \,,
				\quad
				f^{-1} ∘ f = \id_X \,.
			\end{equation}
			The composition of morphisms in~$\Monic(A)$ is given by the composition of morphisms in~$\cat{A}$, and we have~$\id_X = \id_{(X, m)}$ and~$\id_{X'} = \id_{(X', m')}$.
			We can therefore express the identities~\eqref{inverse in the original category} in~$\cat{A}$ as the identities
			\[
				f ∘ f^{-1} = \id_{(X', m')} \,,
				\quad
				f^{-1} ∘ f = \id_{(X, m)}
			\]
			in~$\Monic(A)$.
			This tells us that the morphisms~$f$ and~$f^{-1}$ are also mutually inverse in~$\Monic(A)$.

		\item
			This is a direct consequence of part~\ref{again mutually inverse in monic category}.
		\qedhere

	\end{enumerate}
\end{proof}

We have seen that the morphism~$f$ from~$(X, m)$ to~$(X', m')$ is an isomorphism in~$\cat{A}$, and therefore also a morphism in~$\Monic(A)$.
The existence of this isomorphism shows that the objects~$(X, m)$ and~$(X', m')$ are isomorphic.



\subsubsection{}

Subobjects in~$\Grp$ are subgroups, subobjects in~$\Ring$ are subrings and subobjects in~$\Vect{𝕜}$ are linear subspaces.
(This can be proven with the same argumentation as for~$\Set$ in part~(a) of this exercise.)



\subsubsection{}

We know that the monomorphisms in~$\Top$ are precisely those continuous maps that are injective.
% TODO: Give a reference to where this was proven.
A subobject of a topological space~$X$ in the categorical sense is therefore a subset~$Y$ of~$X$ together with a topology on~$Y$ for which the inclusion map from~$Y$ to~$X$ is continuous.
In other words, the topology on~$Y$ needs to be coarser than the subspace topology induced from~$X$.

We find in particular that~$X$ typically admits more subobjects than just its subspaces.
