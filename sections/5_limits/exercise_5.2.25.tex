\subsection{}


\subsubsection{}

We denote the ambient category by~$\cat{A}$.


\subsubsection*{split monomorphism $\implies$ regular monomorphism}

Let
\[
	m \colon A \to B
\]
be a split monomorphism in~$\cat{A}$.
This means that there exists a morphism
\[
	e \colon B \to A
\]
such that~$e ∘ m = \id_A$.

We note that the morphism~$m$ is a monomorphism:
for every object~$X$ of~$\cat{A}$ and any two parallel morphisms
\[
	f, g \colon X \to A
\]
with~$m ∘ f = m ∘ g$, we have
\[
	f = e ∘ m ∘ f = e ∘ m ∘ g = g \,.
\]

We show now that~$m$ is an equalizer of the two morphisms
\[
	m ∘ e, \id_B \colon B \to B \,.
\]
We have the equalities
\[
	(m ∘ e) ∘ m
	=
	m ∘ e ∘ m
	=
	m ∘ \id_A
	=
	m
	=
	\id_B ∘ m \,,
\]
which shows that the diagram
\[
	\begin{tikzcd}[sep = large]
		A
		\arrow{r}[above]{m}
		&
		B
		\arrow[shift left]{r}[above]{m ∘ e}
		\arrow[shift right]{r}[below]{\id_B}
		&
		B
	\end{tikzcd}
\]
is a fork.
To see that this fork is universal, let~$X$ be an object of~$\cat{A}$ and let~$f$ be a morphism from~$X$ to~$B$ such that
\[
	(m ∘ e) ∘ f = \id_B ∘ f \,.
\]
This means that~$m ∘ e ∘ f = f$.
It follows for the composite~$f' ≔ e ∘ f$, which is a morphism from~$X$ to~$A$, that
\[
	m ∘ f'
	=
	m ∘ e ∘ f
	=
	f \,.
\]
This shows that the morphism~$f$ factors through~$m$.
We have already shown that~$m$ is a monomorphism, whence this factorization is unique.

\subsubsection*{regular monomorphism $\implies$ monomorphism}

Let~$(E, i)$ be an equalizer of two parallel morphisms
\[
	s, t \colon X \to Y
\]
in~$\cat{A}$.
Let~$A$ be an object of~$\cat{A}$ and let
\[
	f, g \colon A \to E
\]
be two morphisms with~$i ∘ f = i ∘ g$.
We denote this common composite by~$r$.
We have
\[
	s ∘ r
	=
	s ∘ i ∘ f
	=
	t ∘ i ∘ g
	=
	t ∘ r \,,
\]
whence there exists by the universal property of the equalizer~$(E, i)$ a unique morphism~$h$ from~$A$ to~$E$ with~$r = i ∘ h$.
Both~$f$ and~$g$ satisfy the role of~$h$, whence~$h = f$ and~$h = g$ by the uniqueness of~$h$, and thus~$f = g$.



\subsubsection{}

The inclusion homomorphism~$m$ from~$2ℤ$ to~$ℤ$ is injective, and therefore a monomorphism in~$\Ab$.
But it is not a split monomorphism in~$\Ab$.
There would otherwise exist a homomorphism~$e$ from~$ℤ$ to~$2ℤ$ with~$e ∘ m = \id_{2ℤ}$.
This would entail for the element~$x ≔ e(1)$ that
\[
	2 x
	=
	2 e(1)
	=
	e(2)
	=
	e(m(2))
	=
	2 \,.
\]
But no element~$x$ of~$2ℤ$ has this property.

Let now
\[
	m \colon A \to B
\]
be an arbitrary monomorphism in~$\Ab$.
Let~$B'$ be the image of~$m$ and let~$p$ be the canonical quotient map from~$B$ to~$B / B'$.
The morphism~$m$ is the equalizer of~$p$ and the zero homomorphism.
It is therefore a regular monomorphism.



\subsubsection{}

Let
\[
	s, t \colon X \to Y
\]
be two parallel morphisms in~$\Top$ and let~$(E, i)$ be an equalizer of~$r$ and~$s$.
We already know that~$i$ is a monomorphism, and therefore injective.
We show in the following that~$E$ carries the subspace topology induced from~$X$.
To prove this, we show that for every topological space~$A$ and every map~$f$ from~$A$ to~$E$, the map~$f$ is continuous if and only if the composite~$i ∘ f$ is continuous.

We know that the map~$i$ is continuous (since it is a morphism in~$\Top$), so if~$f$ is continuous, then so is the composite~$i ∘ f$.

Suppose conversely that the composite~$i ∘ f$ is continuous.
Let us abbreviate this composite by~$g$.
The map~$g$ is a morphism in~$\Top$ that satisfies
\[
	s ∘ g
	=
	s ∘ i ∘ f
	=
	t ∘ i ∘ f
	=
	t ∘ g \,.
\]
It follows from the universal property of the equalizer~$(E, i)$ that there exists a unique continuous map~$f'$ from~$A$ to~$E$ with~$g = i ∘ f'$.
The two maps~$f$ and~$f'$ satisfy the equalities
\[
	i ∘ f = g = i ∘ f' \,,
\]
and we already know that~$i$ is a monomorphism.
It follows that~$f = f'$, which tells us in particular that the map~$f$ is continuous, since~$f'$ is continuous.

We have not characterized monomorphisms and regular monomorphisms in~$\Top$.
A morphism in~$\Top$ is a monomorphism if and only if it is injective.
A monomorphism is regular if and only if the topology of its domain is the subspace topology induces from its codomain.

We can regard any set~$X$ either as an indiscrete topological space~$I(X)$ or as a discrete topological space~$D(X)$.
The identity map from~$I(X)$ to~$D(X)$ is bijective, therefore injective, and thus a monomorphism.
But it is not a regular monomorphism as long as~$X$ contains at least two distinct elements, because~$I(X)$ does not carry the subspace topology induced from~$D(X)$.
