\subsection{}

Dualizing part~(a) of Exercise~5.2.25 gives us the implications
\[
	\text{split epimorphism}
	\implies
	\text{regular epimorphism}
	\implies
	\text{epimorphism} \,.
\]


\subsubsection{}

An isomorphism is both a split monomorphism and a split epimorphism, and therefore both a monomorphism and a regular epimorphism.

Suppose now conversely that a morphism
\[
	f \colon A \to B
\]
is both a monomorphism and a regular epimorphism.
There exist by assumption two parallel morphisms
\[
	s, t \colon C \to A
\]
such that~$f$ is the coequalizer of~$s$ and~$t$.
This entails that~$f ∘ s = f ∘ t$.
It follows that~$s = t$ because~$f$ is a monomorphism.
It further follows from Exercise~5.2.21 that~$f$ is an isomorphism.



\subsubsection{}

We are only left to show the implication
\[
	\text{epimorphism} \implies \text{split epimorphism} \,,
\]
which is provided by the axiom of choice.



\subsubsection{}

If a category~$\cat{A}$ satisfies the axiom of choice, then every morphism in~$\cat{A}$ that is both a monomorphism and an epimorphism is already an isomorphism:
it is both a monomorphism and a split epimorphism, therefore both a monomorphism and a regular epimorphism, and therefore an isomorphism by part~(a) of this exercise.

Therefore, if~$\Top$ were to satisfy the axiom of choice, then every bijective continuous map would already be a homeomorphism.
But we know that this is not the case.
We thus see that~$\Top$ does not satisfy the axiom of choice.

The homomorphism of groups
\[
	p
	\colon
	ℤ \to ℤ / 2ℤ \,,
	\quad
	x \mapsto \class{x}
\]
is an epimorphism in~$\Grp$ that does not admit a right-inverse.
It is therefore an epimorphism, but not a split epimorphism.
The existence of this example shows that~$\Grp$ does not satisfy the axiom of choice.
