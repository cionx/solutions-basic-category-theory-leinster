\subsection{}

We consider pullbacks, pushouts, and composition.




\subsubsection*{Monomorphisms, pullbacks}

We have seen in Exercise~5.1.42 that the class of monomorphisms is stable under pullbacks.



\subsubsection*{Monomorphisms, pushouts}

The class of monomorphisms is not necessarily stable under pushouts.

To construct a counterexample we consider the category~$\CRing$ of commutative rings.
The coproduct of two commutative rings~$R$ and~$S$ in~$\CRing$ is their tensor product~$R ⊗_ℤ S$, the canonical homomorphism from~$R$ and~$S$ into their coproduct are given by the mappings~$r \mapsto r ⊗ 1$ and~$s \mapsto 1 ⊗ s$.
The initial object of~$\CRing$ is given by the ring of integers,~$ℤ$.
The following diagram is therefore a pushout diagram:
\[
	\begin{tikzcd}[column sep = normal]
		ℤ
		\arrow{r}
		\arrow{d}
		&
		R
		\arrow{d}
		\\
		S
		\arrow{r}
		&
		R ⊗_ℤ S
	\end{tikzcd}
\]
We may consider for~$R$ and~$S$ the two rings~$ℚ$ and~$ℤ / 2$.
We have
\[
	ℚ ⊗_ℤ (ℤ / 2) = 0 \,,
\]
and therefore the following pushout diagram:
\[
	\begin{tikzcd}
		ℤ
		\arrow{r}
		\arrow{d}
		&
		ℚ
		\arrow{d}
		\\
		ℤ / 2
		\arrow{r}
		&
		0
	\end{tikzcd}
\]
The inclusion homomorphism from~$ℤ$ to~$ℚ$ is injective, and therefore a mono\-mor\-phism.
But the homomorphism from~$ℤ / 2$ to~$0$ is not injective, and therefore not a monomorphism.



\subsubsection*{Monomorphisms, composition}

The class of monomorphisms is closed under composition.
To see this, let
\[
	m \colon A \to B \,,
	\quad
	m' \colon B \to C
\]
be two composable monomorphisms in some category~$\cat{A}$.
Let
\[
	f, g \colon X \to A
\]
be two morphisms~$\cat{A}$ with~$m' ∘ m ∘ f = m' ∘ m ∘ g$.
Then~$m ∘ f = m ∘ g$ because~$m'$ is a monomorphism, and thus~$f = g$ because~$m$ is a monomorphism.
This shows that~$m' ∘ m$ is again a monomorphism.



\subsubsection*{Regular monomorphisms, pullbacks}

The class of regular monomorphisms is closed under pullbacks.

Let~$(E, i)$ be an equalizer of two morphisms
\[
	s, t \colon A \to B
\]
is some category~$\cat{B}$.
Suppose that~$i$ is part of a pullback diagram of the following form:
\[
	\begin{tikzcd}
		E'
		\arrow{r}[above]{g}
		\arrow{d}[left]{i'}
		&
		E
		\arrow{d}[right]{i}
		\\
		A'
		\arrow{r}[below]{f}
		&
		A
	\end{tikzcd}
\]
Let~$s'$ and~$t'$ be the composites
\[
	s' ≔ s ∘ f \,,
	\quad
	t' ≔ t ∘ f \,.
\]
We may extend the above diagram as follows:
\[
	\begin{tikzcd}
		E'
		\arrow{r}[above]{g}
		\arrow{d}[left]{i'}
		&
		E
		\arrow{d}[right]{i}
		\\
		A'
		\arrow{r}[above]{f}
		\arrow[shift right]{dr}[below left]{s'}
		\arrow[shift left]{dr}[above right]{t'}
		&
		A
		\arrow[shift right]{d}[left]{s}
		\arrow[shift left]{d}[right]{t}
		\\
		{}
		&
		B
	\end{tikzcd}
\]
We claim that~$(E', i')$ is an equalizer of~$s'$ and~$t'$.

To prove this, let
\[
	h \colon C \to A'
\]
be a morphism in~$\cat{A}$ with~$s' ∘ h = t' ∘ h$.
We need to show that there exists a unique morphism~$k$ from~$C$ to~$E'$ with~$h = i' ∘ k$.
We already know that~$i'$ is again a monomorphism, so it only remains to show the existence of~$k$.

We have by assumption the equalities
\[
	s ∘ f ∘ h
	=
	s' ∘ h
	=
	t' ∘ h
	=
	t ∘ f ∘ h \,.
\]
It follows from the universal property of the equalizer~$(E, i)$ that there exists a unique morphism~$k'$ from~$C$ to~$E$ with
\[
	i ∘ k' = f ∘ h \,.
\]
It further follows from the universal property of the pullback~$(E', g, i')$ that there exist a unique morphism
\[
	k \colon C \to E'
\]
with both~$g ∘ k = k'$ and~$i' ∘ k = h$.
This proves the desired existence of~$k$.

The above argument results in the following diagram:
\[
	\begin{tikzcd}
		C
		\arrow[bend right]{ddr}[left]{h}
		\arrow[bend left]{drr}[above]{k'}
		\arrow{dr}[above right]{k}
		&
		{}
		&
		{}
		\\
		{}
		&
		E'
		\arrow{r}[above]{g}
		\arrow{d}[left]{i'}
		&
		E
		\arrow{d}[right]{i}
		\\
		{}
		&
		A'
		\arrow{r}[above]{f}
		\arrow[shift right]{dr}[below left]{s'}
		\arrow[shift left]{dr}[above right]{t'}
		&
		A
		\arrow[shift right]{d}[left]{s}
		\arrow[shift left]{d}[right]{t}
		\\
		{}
		&
		{}
		&
		B
	\end{tikzcd}
\]



\subsubsection*{Regular monomorphisms, pushouts}

According to \cite{stackexchange_pushout_dont_preserve_regular_monos}, the class of regular monomorphisms is not necessarily closed under pushouts.



\subsubsection*{Regular monomorphisms, composition}

According to \cite[Exercise~7J]{adamek_herrlich_strecker_joy_of_cats}, the class of regular monomorphisms is not necessarily closed under composition.



\subsubsection*{Split monomorphisms, pullbacks}

The class of split monomorphisms is not necessarily closed under pullbacks.

To see this, we consider the category~$\Set$.
We observe that monomorphism in~$\Set$ are split-monomorphisms, except for those whose domain is empty, but codomain is non-empty.
Let~$X$ be a set and let~$A$ and~$B$ be two subsets of~$X$.
Then the diagram
\[
	\begin{tikzcd}
		A ∩ B
		\arrow{r}
		\arrow{d}
		&
		B
		\arrow{d}
		\\
		A
		\arrow{r}
		&
		X
	\end{tikzcd}
\]
is a pullback diagram, where each morphism is the respective inclusion map.
If~$A$ and~$B$ are non-empty and disjoint, then it follows that the inclusion map from~$A$ to~$X$ is a split monomorphism, but the inclusion map from~$A ∩ B = ∅$ to~$B$ is not.



\subsubsection*{Split monomorphisms, pushouts}

The class of split monomorphisms is closed under pushouts.

To see this, let
\[
	m \colon A \to B
\]
be a split monomorphism, and suppose that
\[
	\begin{tikzcd}[sep = large]
		A
		\arrow{r}[above]{f}
		\arrow{d}[left]{m}
		&
		A'
		\arrow{d}[right]{m'}
		\\
		B
		\arrow{r}[above]{g}
		&
		B'
	\end{tikzcd}
\]
is a pushout diagram.
There exists by assumption a morphism
\[
	e \colon B \to A
\]
with~$e ∘ m = \id_A$.
The two morphisms
\[
	\id_{A'} \colon A' \to A' \,,
	\quad
	f ∘ e \colon B \to A'
\]
satisfy
\[
	(f ∘ e) ∘ m
	=
	f ∘ e ∘ m
	=
	f ∘ \id_A
	=
	f
	=
	\id_{A'} ∘ f \,.
\]
By the universal property of the pushout, there hence exists a unique morphism~$e'$ from~$B'$ to~$A'$ with both~$e' ∘ m' = \id_{A'}$ and~$e' ∘ g = f ∘ e$.
This shows that the morphism~$m'$ is again a split monomorphism.



\subsubsection*{Split monomorphisms, composition}

The class of split monomorphisms is closed under composition.

Let
\[
	m \colon A \to B \,,
	\quad
	m' \colon B \to C
\]
be two composable split monomorphisms.
There exist by assumption morphisms
\[
	e \colon B \to A \,,
	\quad
	e' \colon C \to B
\]
with~$e ∘ m = \id_A$ and~$e' ∘ m' = \id_B$.
It follows that
\[
	(e ∘ e') ∘ (m' ∘ m)
	=
	e ∘ e' ∘ m' ∘ m
	=
	e ∘ \id_B ∘ m
	=
	e ∘ m
	=
	\id_A \,.
\]
This shows that the composite~$m' ∘ m$ is again a split monomorphism.



\subsubsection*{Epimorphisms, pullbacks}

We have seen that the class of monomorphisms is not necessarily closed under push\-outs.
It follows by duality that the class of epimorphisms is not necessarily closed under pullbacks.



\subsubsection*{Epimorphisms, pushouts}

We have seen that the class of monomorphisms is closed under pullbacks.
It follows by duality that the class of epimorphisms is closed under pushouts.



\subsubsection*{Epimorphisms, composition}

We have seen that the class of monomorphisms is closed under composition.
It follows by duality that the class of epimorphisms is again closed under composition.



\subsubsection*{Regular epimorphisms, pullbacks}

We have seen that the class of regular monomorphisms in not necessarily closed under pushouts.
It follows by duality that the class of regular epimorphisms is not necessarily closed under pullbacks.



\subsubsection*{Regular epimorphisms, pushouts}

We have seen that the class of regular monomorphisms is closed under pullbacks.
It follows by duality that the class of regular epimorphisms is closed under pushouts.



\subsubsection*{Regular epimorphisms, composition}

We have seen that the class of regular monomorphisms is not necessarily closed under composition.
It follows by duality that the class of regular epimorphisms is not necessarily closed under composition.



\subsubsection*{Split epimorphisms, pullbacks}

We have seen that the class of split monomorphisms is closed under pushouts.
It follows by duality that the class of split epimorphisms is closed under pullback.



\subsubsection*{Split epimorphisms, pushouts}

We have seen that the class of split monomorphisms is not necessarily closed under pullbacks.
It follows by duality that the class of split monomorphisms is not necessarily closed under pushouts.



\subsubsection*{Split epimorphisms, composition}

We have seen that the class of split monomorphisms is closed under composition.
It follows by duality that the class of split epimorphisms is also closed under composition.
