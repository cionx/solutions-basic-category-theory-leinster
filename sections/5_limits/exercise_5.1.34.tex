\subsection{}

If~$(E, i)$ is an equalizer of~$f$ and~$g$, then the given square diagram is not necessarily a pullback diagram.
To see this, we consider in the category~$\Set$ the two maps
\[
	\id_{ℤ}, s \colon ℤ \to ℤ
\]
given by
\[
	s(x) = -x
	\qquad
	\text{for every~$x ∈ ℤ$.}
\]
The equalizer of~$\id_{ℤ}$ and~$ℤ$ is given by the set~$\{ 0 \}$ together with the inclusion map~$i$ from~$\{ 0 \}$ to~$ℤ$.
But the diagram
\[
	\begin{tikzcd}[sep = large]
		\{ 0 \}
		\arrow{r}[above]{i}
		\arrow{d}[left]{i}
		&
		ℤ
		\arrow{d}[right]{s}
		\\
		ℤ
		\arrow{r}[below]{\id}
		&
		ℤ
	\end{tikzcd}
\]
is not a pullback diagram:
the diagram
\[
	\begin{tikzcd}[sep = large]
		ℤ
		\arrow[bend left = 20]{drr}[above right]{s}
		\arrow[bend right = 30]{ddr}[below left]{\id_ℤ}
		&
		{}
		&
		{}
		\\
		{}
		&
		\{ 0 \}
		\arrow{r}[above]{i}
		\arrow{d}[left]{i}
		&
		ℤ
		\arrow{d}[right]{g}
		\\
		{}
		&
		ℤ
		\arrow{r}[below]{f}
		&
		ℤ
	\end{tikzcd}
\]
commutes, but the unique morphism from~$ℤ$ to~$\{ 0 \}$ destroys this commutativity.

Suppose now that the diagram
\[
	\begin{tikzcd}[sep = large]
		E
		\arrow{r}[above]{i}
		\arrow{d}[left]{i}
		&
		X
		\arrow{d}[right]{g}
		\\
		X
		\arrow{r}[below]{f}
		&
		Y
	\end{tikzcd}
\]
is a pullback diagram.
We claim that~$(E, i)$ is an equivalizer of~$f$ and~$g$.
Indeed, we have~$f ∘ i = g ∘ i$ by the commutativiy of the diagram.
Let~$A$ be an object of~$\cat{A}$ and let~$j$ be a morphism from~$A$ to~$X$ with~$f ∘ j = g ∘ j$.
Then there exists by the universal property of the pullback a unique morphism~$h$ from~$A$ to~$E$ with both~$i ∘ h = j$ and~$i ∘ h = j$, i.e., with~$i ∘ h = j$.
This shows that~$(E, i)$ is indeed an equalizer of~$f$ and~$g$.
