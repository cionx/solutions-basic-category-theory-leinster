\subsection{}

Instead of the direct sum~$X_1 ⊕ X_2$, we use the product~$X_1 × X_2$.
Let
\[
	p_1 \colon X_1 × X_2 \to X_1 \,,
	\quad
	p_2 \colon X_1 × X_2 \to X_2
\]
be the canonical projection maps.
Let~$Y$ be another vector space.
We know that~$p_1$ and~$p_2$ induce a bijection
\begin{align*}
	\{ \text{functions~$\textstyle Y \to X_1 × X_2$} \}
	&\to
	\{ \text{functions~$\textstyle Y \to X_1$} \}
	×
	\{ \text{functions~$\textstyle Y \to X_2$} \}
	\\
	f
	&\mapsto
	(p_1 ∘ f, p_2 ∘ f) \,,
\end{align*}
where by \enquote{function} we mean a set-theoretic map between the respective underlying sets.
It therefore suffices to show that a map~$f$ from~$Y$ to~$X_1 × X_2$ is linear if and only if both~$p_1 ∘ f$ and~$p_2 ∘ f$ are linear.

The projection maps~$p_1$ and~$p_2$ are both linear.
So if~$f$ is linear, then it follows that the composites~$p_1 ∘ f$ and~$p_2 ∘ f$ are again linear.
Suppose on the other hand that both~$p_1 ∘ f$ and~$p_2 ∘ f$ are linear.
Then
\begin{align*}
	p_i(f(y + y'))
	&=
	(p_i ∘ f)(y + y')
	\\
	&=
	(p_i ∘ f)(y) + (p_i ∘ f)(y')
	\\
	&=
	p_i(f(y)) + p_i(f(y'))
	\\
	&=
	p_i( f(y) + f(y') )
\end{align*}
for both~$i = 1, 2$, and therefore~$f(y + y') = f(y) + f(y')$.
This shows that the map~$f$ is additive.
We find similarly that~$f$ is homogeneous.
This then shows that~$f$ is linear.
