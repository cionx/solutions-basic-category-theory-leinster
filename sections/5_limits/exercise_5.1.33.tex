\subsection{}

Instead of the direct sum~$X_1 ⊕ X_2$ we are going use the product~$X_1 × X_2$.
Let
\[
	p_1 \colon X_1 × X_2 \to X_1 \,,
	\quad
	p_2 \colon X_1 × X_2 \to X_2
\]
be the canonical projection maps.
Let~$Y$ be another vector space.
We know that the two maps~$p_1$ and~$p_2$ induce a bijection
\begin{align*}
	\{ \text{functions~$\textstyle Y \to X_1 × X_2$} \}
	&\to
	\{ \text{functions~$\textstyle Y \to X_1$} \}
	×
	\{ \text{functions~$\textstyle Y \to X_2$} \} \,,
	\\
	f
	&\mapsto
	(p_1 ∘ f, p_2 ∘ f) \,.
\end{align*}
(By a \enquote{function} we mean a set-theoretic map between the respective underlying sets.)
Thanks to the above bijection, it now suffices to show the following statement:
a map~$f$ from~$Y$ to~$X_1 × X_2$ is linear if and only if both its composites~$p_1 ∘ f$ and~$p_2 ∘ f$ are linear.

Suppose first that the map~$f$ is linear.
We observe that the projection maps~$p_1$ and~$p_2$ are both linear.
It follows that the composites~$p_1 ∘ f$ and~$p_2 ∘ f$ are again linear.

Suppose conversely that both~$p_1 ∘ f$ and~$p_2 ∘ f$ are linear.
To show that the map~$f$ is linear, we need to check that it is both additive and homogeneous.
\begin{itemize}

	\item
		We have for every two elements~$y$ and~$y'$ of~$Y$ the chain of equalities
		\begin{align*}
			p_i(f(y + y'))
			&=
			(p_i ∘ f)(y + y')
			\\
			&=
			(p_i ∘ f)(y) + (p_i ∘ f)(y')
			\\
			&=
			p_i(f(y)) + p_i(f(y'))
			\\
			&=
			p_i( f(y) + f(y') )
		\end{align*}
		for both~$i = 1$ and~$i = 2$.
		This means that the two elements~$f(y + y')$ and~$f(y) + f(y')$ coincide in both components, and are therefore equal.
		In other words, we find that
		\[
			f(y + y') = f(y) + f(y') \,.
		\]
		We have thus shown that the map~$f$ is additive.

	\item
		That~$f$ is homogeneous can be shown in the same way.%
		\footnote{
			That is, we can show that for every scalar~$λ$ and every element~$y$ of~$Y$ that~$p_i(f(λ y)) = p_i(λ f(y))$, and conclude that~$f(λ x) = λ f(y)$.
		}

\end{itemize}
