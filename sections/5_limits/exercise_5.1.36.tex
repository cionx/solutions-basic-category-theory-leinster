\subsection{}



\subsubsection{}

For every object~$I$ of~$\scat{I}$, let~$f_I$ be the composite~$p_I ∘ h$, and thus also the composite~$p_I ∘ h'$.
We have for every morphism
\[
	u \colon I \to J
\]
in~$\scat{I}$ the equality
\[
	D(u) ∘ f_I
	=
	D(U) ∘ p_I ∘ h
	=
	p_J ∘ h
	=
	f_J \,.
\]
This means that the object~$A$ together with the family of morphisms~$(f_I)_{I ∈ \Ob(\scat{I})}$ is a cone for the diagram~$D$.
It follows from the universal property of the limit~$L$, that there exists a unique~(!) morphism~$f$ from~$A$ to~$L$ with~$p_I ∘ f = f_I$ for every object~$I$ of~$\scat{I}$.
Both~$h$ and~$h'$ satisfy this property, whence they must be the same morphism.



\subsubsection{}

If~$\scat{I}$ is the discrete two-object category, then limits of shape~$\scat{I}$ are just products.
A morphism from~$A = 1$ to another set~$X$ is the same as an element of~$X$.
We arrive therefore at the following result:
\begin{quote}
	Let~$D_1$ and~$D_2$ be two sets, and let
	\[
		p_1 \colon D_1 × D_2 \to D_1 \,,
		\quad
		p_2 \colon D_2 × D_2 \to D_2
	\]
	denote the canonical projections.
	Two elements~$x$ and~$y$ of the product~$D_1 × D_2$ are equal if and only if~$p_1(x) = p_1(y)$ and~$p_2(x) = p_2(y)$, i.e., if and only if they have the same components.
\end{quote}
