\subsection{}



\subsubsection{}

For every object~$I$ of~$\scat{I}$, let~$f_I$ be the composite~$p_I ∘ h$, and thus also the composite~$p_I ∘ h'$.
We have for every morphism
\[
	u \colon I \to J
\]
in~$\scat{I}$ the equalities
\[
	D(u) ∘ f_I
	=
	D(U) ∘ p_I ∘ h
	=
	p_J ∘ h
	=
	f_J \,.
\]
This means that the object~$A$ together with the family of morphisms~$(f_I)_{I ∈ \Ob(\scat{I})}$ is a cone for the diagram~$D$.
It follows from the universal property of the limit~$L$ that there exists a unique morphism~$f$ from~$A$ to~$L$ with~$p_I ∘ f = f_I$ for every object~$I$ of~$\scat{I}$.
Both~$h$ and~$h'$ satisfy this defining property of~$f$, whence we must have~$f = h$ and~$f = h'$.
Therefore,~$h = h'$.



\subsubsection{}

Let now~$\scat{I}$ be the discrete two-object category.
Limits of shape~$\scat{I}$ are just products.
A morphism from~$A = 1$ to another set~$X$ is the same as an element of~$X$.
We arrive therefore at the following result:
\begin{quote}
	\itshape
	Let~$D_1$ and~$D_2$ be two sets, and let
	\[
		p_1 \colon D_1 × D_2 \to D_1 \,,
		\quad
		p_2 \colon D_2 × D_2 \to D_2
	\]
	denote the canonical projections.
	Two elements~$x$ and~$y$ of the product~$D_1 × D_2$ are equal if and only if~
	\[
		p_1(x) = p_1(y)
		\quad\text{and}\quad
		p_2(x) = p_2(y) \,.
	\]
	In other words, two elements of~$D_1 × D_2$ are equal if and only if they are equal in both coordinates.
\end{quote}
