\subsection{}



\subsubsection{}

Let~$\scat{I}$ be a small category and let~$D$ be a diagram of shape~$\scat{I}$ in~$\cat{A}$.

An explicit description for the limit of~$U ∘ D$ is given by the set
\[
	L'
	≔
	\left\{
		(x_I)_I
		∈
		∏_{I ∈ \Ob(\scat{I})} U(I)
	\suchthat*
	\begin{tabular}{@{}l@{}}
			$x_K = D(u)(X_J)$ for every \\
			morphism~$u \colon J \to K$ in~$\scat{I}$
		\end{tabular}
	\right\} \,,
\]
with the projections~$p'_I$ from~$L'$ to~$U(D(I))$ given by the restrictions of the projections~$\pr_I$ of the product~$\prod_{I ∈ \Ob(\scat{I})} U(I)$.
We need to show that there exists a unique group structure on~$L'$ that makes each~$p'_I$ into a homomorphism of groups~$p_I$ from the resulting group~$L$ to~$D(I)$.
We then need to show that~$(L, (p_I)_I)$ is a limit cone of the diagram~$D$.

We first want to uniquely endow the set~$L'$ with a group structure -- resulting in a group~$L$ -- such that for each object~$I$ of~$\scat{I}$ the projection map~$p'_I$ is a homomorphism of groups -- then denoted by~$p_I$ -- from~$L$ to~$D(I)$.
If such a group structure exists, then we must have for every two elements~$x$ and~$y$ of~$L$ with~$x = (x_I)_I$ and~$(y_I)_I$ the equality
\[
	p_I( x ⋅ y )
	=
	p_I( x ) ⋅ p( y )
	=
	x_I ⋅ y_I \,.
\]
This means that the group structure of~$L$ needs to be given by
\begin{equation}
	\label{formula for product in limit of groups}
	( x_I )_I ⋅ ( y_I )_I
	=
	( x_I ⋅ y_I )_I \,.
\end{equation}
This shows the uniqueness of the desired group structure.

To show that~\eqref{formula for product in limit of groups} results in a well-defined group structure on~$L'$, we only need to show that~$L'$ is a subgroup of~$∏_{I ∈ \Ob(\scat{I})} D(I)$.
In tother words, we need to show that~$L'$ contain the identity element of this product, and that for any two of its elements~$x$ and~$y$, their product~$x ⋅ y$ is again contained in~$L'$.

The identity element of this product is given by~$1 ≔ ( 1_{D(I)} )_I$.
This element satisfies for every morphism
\[
	u \colon J \to K
\]
of~$\scat{I}$ the chain of equalities
\[
	D(u)( \pr_J( 1 ) )
	=
	D(u)( 1_{D(J)} )
	=
	1_{D(K)}
	=
	\pr_K( 1 ) \,,
\]
and is therefore contained in~$L'$.
For every two elements~$x = (x_I)_I$ and~$y = (y_I)_I$ on~$L'$, we have for every morphism
\[
	u \colon J \to K
\]
of~$\scat{I}$ the chain of equalities
\begin{align*}
	D(u)( \pr_J( x ⋅ y ) )
	&=
	D(u)( \pr_J( x ) ⋅ \pr_J( y ) )
	\\
	&=
	D(u)( x_J ⋅ y_J )
	\\
	&=
	D(u)( x_J ) ⋅ D(u)( y_J )
	\\
	&=
	x_K ⋅ y_K
	\\
	&=
	\pr_K( x ) ⋅ \pr_K( y )
	\\
	&=
	\pr_K( x ⋅ y ) \,.
\end{align*}
This shows that the product~$x ⋅ y$ is again contained in~$L'$.

We have now overall shown that there exists a unique group structure on the set~$L'$ -- making it into a group~$L$ -- such that for every object~$I$ of~$\scat{I}$, the projection map~$p'_I$ from~$L'$ to~$U(D(I))$ is a homomorphism of groups from~$L$ to~$D(I)$ -- which we will then denote by~$p_I$.

To show that~$(L, (p_I)_I)$ is a cone of~$D$, we need to show that
\[
	D(u) ∘ p_J = p_K
\]
for every morphism
\[
	u \colon J \to K
\]
in~$\scat{I}$.
It suffices to show that
\[
	U( D(u) ∘ p_J ) = U( p_K )
\]
because the forgetful functor~$U$ is faithful.
We recall that~$U(D(u)) ∘ p'_J = p'_K$ because~$( L', (p'_I)_I )$ is a (limit) cone of~$D$.
It follows that
\[
	U( D(u) ∘ p_J )
	=
	U(D(u)) ∘ U( p_J )
	=
	U(D(u)) ∘ p'_J
	=
	p'_K
	=
	U( p_K ) \,,
\]
as desired.

It remains to show that the cone~$( L, (p_I)_I )$ of~$D$ is already a limit cone.
To show this, let~$( C, (q_I)_I )$ be another cone of~$D$.
We need to show that there exists a unique homomorphims of groups~$f$ from~$C$ to~$L$ with~$p_I ∘ f = q_I$ for every object~$I$ of~$\scat{I}$.

It follows that~$( U(C), (U(q_I))_I )$ is a cone of~$U ∘ D$.
There hence exists a unique map~$f'$ from~$U(C)$ to~$L'$ with~$p'_I ∘ f' = U(q_I)$ for every object~$I$ of~$\scat{I}$.
It suffices to show in the following that~$f'$ is already a homomorphism of groups from~$C$ to~$L$.

To show this, let~$x$ and~$y$ be two elements of~$C$.
We have for every object~$I$ of~$\scat{I}$ the chain of equalities
\begin{align*}
	p'_I( f'(x ⋅ y) )
	&=
	U(q_I)( x ⋅ y )
	\\
	&=
	U(q_I)(x) ⋅ U(q_I)(y)
	\\
	&=
	p'_I( f'(x) ) ⋅ p'_I( f'(y) )
	\\
	&=
	p'_I( f'(x) ⋅ f'(y) ) \,,
\end{align*}
and therefore altogether the equality
\[
	f'(x ⋅ y) = f'(x) ⋅ f'(y) \,.
\]
This shows that~$f'$ is indeed a homomorphism of groups from~$C$ to~$L$.



\subsubsection{}

The same argumentation goes through for any kind of \enquote{category of algebras}.
