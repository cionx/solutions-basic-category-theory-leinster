\subsection{}



\subsubsection{}

\begin{recall}
	For every map of the form
	\[
		e \colon A \to X \,,
	\]
	the equivalence relation~$∼$ on~$A$ induced by~$e$ is given by
	\[
		a ∼ a'
		\iff
		e(a) = e(a')
		\qquad
		\text{for all~$a, a' ∈ A$} \,.
	\]
\end{recall}

We denote the class of quotient objects of~$A$ by~$\Quot(A)$, and the set of equivalence relations on~$A$ by~$\Equiv(A)$.
We denote the objects of~$\Epic(A)$ as pairs~$(X, e)$, consisting of an object~$X$ and an epimorphism~$e$ from~$A$ to~$X$.
Given such an object, we refer to the equivalence relation induces by~$e$ on~$A$ as the equivalence relation induced by~$(X, e)$.

We have a map
\[
	E \colon \Ob(\Epic(A)) \to \Equiv(A)
\]
that assigns to each object~$(X, e)$ of~$\Epic(A)$ the equivalence relation induced by~$e$.
We shall show in the following that the map~$E$ descends to a bijection from~$\Quot(A)$ to~$\Equiv(A)$.
For this, we need to show that the map~$E$ is surjective, and that two objects of~$\Epic(A)$ induce the same equivalence relation on~$A$ if and only if they are isomorphic.

We can consider for every equivalence relation~$∼$ on~$A$ the quotient set~$A / {∼}$ and the canonical quotient map~$p$ from~$A$ to~$A / {∼}$.
The map~$p$ is surjective, whence~$(A / {∼}, p)$ is an object of~$\Epic(A)$.
This object induces the equivalence relation~$∼$, and is therefore a preimage for~$∼$ with respect to~$E$.
This shows that the map~$E$ is surjective.

Let us now show that isomorphic objects of~$\Epic(A)$ have the same image under~$E$.

Let~$(X, e)$ and~$(X', e')$ be two objects in~$\Epic(A)$, and suppose that there exists a morphism~$f$ from~$(X, e)$ to~$(X', e')$ in~$\Epic(A)$.
This means that~$f$ is a map from~$X$ to~$X'$ that makes the following diagram commute:
\[
	\begin{tikzcd}[column sep = normal]
		{}
		&
		A
		\arrow{dl}[above left]{e}
		\arrow{dr}[above right]{e'}
		&
		{}
		\\
		X
		\arrow{rr}[above]{f}
		&
		{}
		&
		X'
	\end{tikzcd}
\]
It follows for any two elements~$a$ and~$a'$ of~$A$, that
\[
	e(a) = e(a')
	\implies
	f(e(a)) = f(e(a'))
	\implies
	e'(a) = e'(a') \,.
\]
This means that the equivalence relation induced by~$e$ implies the equivalence relation induced by~$e'$.

If~$(X, e)$ and~$(X', e')$ are isomorphic, then there exist morphisms between them in both directions.
It then follows that both~$e$ and~$e'$ induce the same equivalence relation on~$A$.

Let us now show that objects of~$\Epic(A)$ that induce the same equivalence relation are already isomorphic.
To do so, let~$(X, e)$ be an object of~$\Epic(A)$ and let~$∼$ be the equivalence relation induced by~$(X, e)$.
Let furthermore~$p$ be the canonical quotient map from~$A$ to~$A / {∼}$.
We show in the following that
\[
	(X, e) ≅ (A / {∼}, p) \,.
\]
This then entails that the object~$(X, e)$ is determined by the equivalence relation~$∼$ up to isomorphism.

That~$(X, e)$ is an object of~$\Epic(A)$ means that~$X$ is a set and~$e$ is an epimorphism from~$A$ to~$X$.
More specifically,~$e$ is an epimorphism in~$\Set$, and thus a surjective map.
It follows that the map~$e$ factors through a bijection
\[
	f
	\colon
	A / {∼} \to X \,,
	\quad
	\class{a} \mapsto e(a) \,.
\]
This induced bijection makes the diagram
\[
	\begin{tikzcd}[column sep = normal]
		{}
		&
		A
		\arrow{dl}[above left]{p}
		\arrow{dr}[above right]{e}
		&
		{}
		\\
		A / {∼}
		\arrow{rr}[above]{f}
		&
		{}
		&
		X
	\end{tikzcd}
\]
commute, which tells us that~$f$ is a morphism from~$(A / {∼}, p_{∼})$ to~$(X, e)$ in the category~$\Epic(A)$.

We make the following observation:

\begin{proposition}
	\label{isomorphisms in epi categories}
	Let~$\cat{A}$ be a category and let~$A$ be an object of~$\cat{A}$.
	Let~$(X, e)$ and~$(X', e')$ be two objects of~$\Epic(A)$ and let~$f$ be a morphism from~$(X, e)$ to~$(X', e')$.
	Suppose that~$f$ is an isomorphism in~$\cat{A}$ with inverse~$f^{-1}$.
	\begin{enumerate}

		\item
			The inverse~$f^{-1}$ is a morphism from~$(X', m')$ to~$(X, m)$ in~$\Epic(A)$.

		\item
			The two morphisms~$f$ and~$f^{-1}$ are mutually inverse in~$\Epic(A)$.

		\item
			The morphism~$f$ is also an isomorphism in~$\Epic(A)$.

	\end{enumerate}
\end{proposition}

\begin{proof}
	The given \lcnamecref{isomorphisms in epi categories} is the dual of \cref{isomorphisms in mono categories} (page~\pageref{isomorphisms in mono categories}).
\end{proof}

It follows from the above \lcnamecref{isomorphisms in epi categories} that the morphism~$f$ from~$(A / {∼}, p)$ to~$(X, e)$ is an isomorphism.
The two objects~$(A / {∼}, p)$ and~$(X, e)$ are therefore isomorphic.



\subsubsection{}

Given a group~$G$ and object~$(X, e)$ of~$\Epic(G)$, the kernel of~$e$ is a normal subgroup of~$G$.
We can adapt the argumentation from part~(a) of this exercise to show that this assignment yields a bijection between~$\Quot(G)$ and the set of normal subgroups of~$G$.
