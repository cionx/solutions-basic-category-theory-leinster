\subsection{}

Let~$\scat{I}$ be a small category, let~$\cat{A}$ and~$\cat{B}$ be two categories and let~$F$ be a functor from~$\cat{A}$ that creates limits of shape~$\scat{I}$.

Let~$D$ be a diagram in~$\cat{A}$ of shape~$\scat{I}$, let~$C ≔ (L, (p_I)_I)$ be a cone of~$D$ and suppose that the resulting cone~$F(C) ≔ (F(L), (F(p_I))_I)$ of the diagram~$F ∘ D$ is a limit cone of~$F ∘ D$.
We need to show that the cone~$C$ is already a limit cone of~$D$.

There exists by assumption a unique cone~$C' ≔ (L', (p'_I)_I)$ of the diagram~$D$ with~$F(C) = F(C')$, and this cone~$C'$ is a limit cone of~$D$.
The cone~$C'$ must be given by~$C$ by its uniqueness.
Therefore,~$C$ is a limit cone of~$D$.
