\subsection{}

Let~$\scat{I}$ be a small category, let~$\cat{A}$ and~$\cat{B}$ be two categories and let~$F$ be a functor from~$\cat{A}$ that creates limits of shape~$\scat{I}$.

Let~$D$ be a diagram in~$\cat{A}$ of shape~$\scat{I}$.
Let~$(L, (p_I)_I)$ be a cone of~$D$ and suppose that the resulting cone~$(F(L), (F(p_I))_I)$ of the diagram~$F ∘ D$ is a limit cone of~$F ∘ D$.
We need to show that the cone~$(L, (p_I)_I)$ is already a limit cone of~$D$.

There exists by assumption a unique cone~$(L', (p'_I)_I)$ of the~$D$ with~$F(L') = F(L)$ and~$F(p'_I) = F(p_I)$ for every object~$I$ of~$\scat{I}$, and this cone is a limit cone of~$D$.
The cone~$(L', (p'_I)_I)$ must be given by~$(L, (p_I)_I)$ by its uniqueness.
Therefore,~$(L, (p_I)_I)$ is a limit cone of~$D$.
