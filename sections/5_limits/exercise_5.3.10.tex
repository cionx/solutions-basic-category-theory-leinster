\subsection{}

Let~$\cat{A}$ and~$\cat{B}$ be two categories and let~$F$ be a functor from~$\cat{A}$.
Let~$\scat{I}$ be a small category and let~$D$ be a diagram of shape~$\scat{I}$ in~$\cat{A}$.
For every cone~$C = (A, (p_I)_I)$ of the diagram~$D$, we denote the resulting cone~$(F(A), (F(p_I))_I)$ of the diagram~$F ∘ D$ by~$F(C)$.

Suppose now that the functor~$F$ creates limits.
We need to show that the functor~$F$ also reflects limits.
For this, let~$C$ be a cone on~$D$ such that~$F(C)$ is a limit cone on~$F ∘ D$.
We need to show that~$C$ is a limit cone on~$D$.

There exists by assumption a unique cone~$L$ on~$D$ with~$F(L) = F(C)$, and this cone~$L$ is a limit cone on~$D$.
We have~$L = C$ by the uniqueness of~$L$, whence~$C$ is a limit cone on~$D$.
