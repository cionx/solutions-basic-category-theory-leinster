\subsection{}



\subsubsection{}

We denote the subcategories~$\Fix(GF)$ and~$\Fix(FG)$ of~$\cat{A}$ and~$\cat{B}$ by~$\cat{A}'$ and~$\cat{B}'$ respectively.

We start off by showing that the functor~$F$ restrict to a functor from~$\cat{A}'$ to~$\cat{B}'$.
Let~$A$ be an object of~$\cat{A}'$.
The morphisms~$η_A$ from~$A$ to~$GF(A)$ is an isomorphism, whence its image under~$F$ is again an isomorphism, this time from~$F(A)$ to~$FGF(A)$.
We know from the triangle identities that
\[
	\id_{F(A)} = ε_{F(A)} ∘ F(η_A) \,.
\]
It follows that the morphisms~$ε_{F(A)}$ is an isomorphism since both~$\id_{F(A)}$ and~$F(η_A)$ are isomorphisms.
This tells us that object~$F(A)$ is contained in~$\cat{B}'$.
We therefore find that the functor~$F$ restricts to a functor~$F'$ from~$\cat{A}'$ to~$\cat{B}'$.
(We don’t need to worry about the action of~$F$ on morphisms in~$\cat{A}'$ because~$\cat{B}'$ is a full subcategory of~$\cat{B}$.)

We find in the same way (by using the other triangle identity) that the functor~$G$ restricts to functor~$G'$ from~$\cat{B}'$ to~$\cat{A}'$.

The natural transformation~$η$ from~$\Id_{\cat{A}}$ to~$GF$ restricts to a natural transformation from~$\Id_{\cat{A}'}$ to~$G' F'$, and the natural transformation~$ε$ from~$FG$ to~$\Id_{\cat{B}}$ restricts to a natural transformation~$ε'$ from~$F' G'$ to~$\Id_{\cat{B}'}$.
We observe that the subcategories~$\cat{A}'$ and~$\cat{B}'$ of~$\cat{A}$ and~$\cat{B}$ are chosen precisely in such a way that all components of~$η'$ and~$ε'$ are isomorphisms.
The natural transformations~$η'$ and~$ε'$ are therefore natural isomorphisms.

We have now overall the two categories~$\cat{A}'$ and~$\cat{B}'$, the two functors
\[
	F' \colon \cat{A}' \to \cat{B}' \,,
	\quad
	G' \colon \cat{B}' \to \cat{A}' \,,
\]
and the two natural isomorphisms
\[
	η' \colon \Id_{\cat{A}'} \To G' F' \,,
	\quad
	ε' \colon F' G' \To \Id_{\cat{B}'} \,.
\]
We have, in other words, an equivalence of categories~$(F', G', η', ε')$ between the two categories~$\cat{A}'$ and~$\cat{B}'$.



\subsubsection{}

\paragraph{Example~2.1.3,~(a)}
Given any set~$S$, the unit map~$η_S$ from~$S$ to~$UF(S)$ is not surjective because~$UF(S)$ contains the zero vector, which does not lie in the image of~$η_S$.
We therefore find that the subcategory~$\Fix(UF)$ of~$\Set$ is empty.

It follows that the category~$\Fix(FU)$ is also empty, since it is equivalent to~$\Fix(UF)$.

\paragraph{Example~2.1.3,~(b)}
In the same way as in the previous example, we find that both~$\Fix(UF)$ and~$\Fix(FU)$ are empty.

\paragraph{Example~2.1.3,~(c)}
The composite functor~$UF$ assigns to each group~$G$ its abelianization~$G^{\ab}$.
The unit morphism~$η_G$ is the canonical homomorphism of groups from~$G$ to~$G^{\ab}$.
This homomorphism is an isomorphism if and only if the group~$G$ is abelian.
We hence find that the~$\Fix(UF)$ is~$\Ab$, i.e., the full subcategory of~$\Grp$ whose objects are abelian groups.

We find similarly that the subcategory~$\Fix(FU)$ of~$\Ab$ is all of~$\Ab$.

The functor~$U$ restricts to the identity functor of~$\Ab$, while the restriction of the functor~$F$ to an endofunctor of~$\Ab$ assigns to each abelian group~$A$ its abelianization~$A^{\ab}$.

\paragraph{Example~2.1.3,~(d), the functors~$F$ and~$U$}
The counit morphism~$ε_G$ is an isomorphism for every group~$G$, whence the category~$\Fix(FU)$ is all of~$\Grp$.

Given a monoid~$M$, the unit morphism~$η_M$ from~$M$ to~$UF(M)$ can only be an isomorphism if~$M$ was already a group to begin with.
Conversely, if~$M$ is a group, then~$η_M$ will be an isomorphism.
We hence find that the subcategory~$\Fix(UF)$ of~$\Mon$ is~$\Grp$.

The functor~$U$ restricts to the identity functor of~$\Grp$, whereas the restriction of the functor~$F$ will typically change the underlying set of a group, and is therefore not the identity functor.

\paragraph{Example~2.1.3,~(d), the functors~$U$ and~$R$}
The composite~$RU$ as the identity functor, and the unit~$η$ is the identity natural transformation.
The category~$\Fix(RU)$ is therefore all of~$\Grp$.

The composite~$UR$ assigns to each monoid its submonoid~$M^×$ of units, and the counit morphism~$ε_M$ is for every monoid~$M$ given by the inclusion from~$M^×$ into~$M$;
this inclusion is an isomorphism if and only if every element of~$M$ is invertible, i.e., if and only if the monoid~$M$ was already a group to begin with.
The subcategory~$\Fix(UR)$ of~$\Mon$ is therefore given by~$\Grp$.

Both~$U$ and~$R$ restrict to the identity functors of~$\Grp$, and the restrictions of~$η$ and~$ε$ are the identity natural transformation of this identity functor.

\paragraph{Example~2.1.5, the functors~$D$ and~$U$}
The composite~$DU$ assigns to each topological space~$X$ the discrete topological space with the same underlying set as~$X$.
The unit morphism~$η_X$ is the identity map from this discrete space to~$X$.
It is an isomorphism if and only if the space~$X$ is discrete, whence the full subcategory~$\Fix(DU)$ of~$\Top$ consists precisely of the discrete topological spaces.

The composite~$UD$ is the identity functor, and the counit~$ε$ of the adjunction is the identity natural transformation.
The subcategory~$\Fix(UD)$ of~$\Set$ is therefore all of~$\Set$.

The restriction of~$η$ is the identity natural transformation of the identity functor of the category of discrete topological spaces.

\paragraph{Example~2.1.5, the functors~$U$ and~$I$}
The composite~$IU$ assigns to each topological space~$X$ the indiscrete topological space with the same underlying set as~$X$, and the counit morphism~$ε_X$ is the identity map from~$X$ to this indiscrete space.
This map is an isomorphism if and only if the space~$X$ is indiscrete, whence the full subcategory~$\Fix(IU)$ of~$\Top$ consists precisely of the indiscrete topological spaces.

The composite~$UI$ is the identity functor, and the unit~$η$ of the adjunction is the identity natural transformation.
The subcategory~$\Fix(UI)$ of~$\Set$ is therefore all of~$\Set$.

The restriction of~$ε$ is the identity natural transformation of the identity functor of the category of indiscrete topological spaces.

\paragraph{Example~2.1.6}
We denote the given functors by
\[
	F ≔ (\ph) × B \,,
	\quad
	G ≔ (\ph)^B \,.
\]
The composite~$GF$ is given on objects by
\[
	GF(A) = (A × B)^B
\]
for every set~$A$, and the unit natural transformation~$η$ is given by
\[
	η_A
	\colon
	A \to (A × B)^B \,,
	\quad
	a \mapsto [b \mapsto (a, b)]
\]
for every set~$A$.
The composite~$FG$, on the other hand, is given on objects by
\[
	FG(A) = A^B × B \,,
\]
for every set~$A$, and the counit natural transformation~$ε$ is given by the evaluation map
\[
	ε_A
	\colon
	A^B × A \to A \,,
	\quad
	(f, b) \mapsto f(b)
\]
for every set~$A$.
We distinguish in the following between three cases.
\begin{casedistinction}

	\item
		Suppose that the set~$B$ is empty.

		The set~$GF(A)$ is then a singleton for every set~$A$, whence the map~$η_A$ is an isomorphism if and only if the set~$A$ is a singleton.
		The full subcategory~$\Fix(GF)$ of~$\Set$ consists therefore of all the singleton sets.
		The restriction~$η'$ of the unit~$η$ has for every singleton set~$A$ as its component~$η_A$ the unique map from the singleton set~$A$ to the singleton set~$(A × B)^B$.

		The set~$FG(C)$ is empty, whence the counit map~$ε_C$ is an isomorphism if and only if the set~$C$ is empty.
		The subcategory~$\Fix(FG)$ of~$\Set$ therefore consists of a single object, namely the empty set.

		The composite~$FG$ restricts to the identity functor on~$\Fix(FG)$, and the counit~$ε$ to the identity natural transformation of this identity functor.

	\item
		Suppose that the set~$B$ is a singleton.
		Both~$η$ and~$ε$ are natural isomorphisms, whence both~$\Fix(GF)$ and~$\Fix(FG)$ are all of~$\Set$.

	\item
		Suppose that the set~$B$ consists of at least two distinct elements.

		The unit map~$η_A$ is an isomorphism if and only if the set~$A$ is empty.
		The subcategory~$\Fix(GF)$ of~$\Set$ therefore consists of only a single object, namely the empty set.

		The counit map~$ε_C$ is an isomorphism if and only if the set~$C$ is empty.
		The subcategory~$\Fix(FG)$ of~$\Set$ therefore consists of only a single object, namely the empty set.

		The restrictions of~$F$ and~$G$ are the identity functors on the full subcategory of~$\Set$ whose single object is the empty set, and both~$η$ and~$ε$ restrict to the identity natural transformation of this identity functor.

\end{casedistinction}
