\subsection{}



\addtocounter{subsubsection}{1}
\subsubsection{}

We observe that the functor category~$[G, \Vect{𝕜}]$ is isomorphic to the module category~$\Mod{𝕜[G]}$.

Suppose that~$H$ is another group and let~$φ$ be a homomorphism of groups from~$H$ to~$G$.
We may regard~$φ$ as a functor from~$H$ to~$G$.
We get an induced functor
\[
	φ^*
	\colon
	[G, \Vect{𝕜}] \to [H, \Vect{𝕜}]
\]
that is given by pre-composition with~$φ$.
The homomorphism of groups~$φ$ also induces a homomorphism of~\algebras{$𝕜$}
\[
	𝕜[φ]
	\colon
	𝕜[G] \to 𝕜[H] \,.
\]
Under the isomorphism of~$[G, \Vect{𝕜}]$ with~$\Mod{𝕜[G]}$ and the isomorphism of~$[H, \Vect{𝕜}]$ with~$\Mod{𝕜[H]}$, the above functor~$φ^*$ corresponds to the restriction functor
\[
	\Res^G_H
	\colon
	\Mod{𝕜[G]} \to \Mod{𝕜[H]}
\]
induced by~$𝕜[φ]$.
This restriction functor admits both a left adjoint and a right adjoint.
A left adjoint is given by
\[
	\Ind^H_G
	≔
	𝕜[G] ⊗_{𝕜[H]} (\ph)
	\colon
	\Mod{𝕜[H]} \to \Mod{𝕜[G]}
\]
and a right adjoint is given by
\[
	\Coind^H_G
	≔
	\Hom_{𝕜[H]}(𝕜[G], \ph)
	\colon
	\Mod{𝕜[H]} \to \Mod{𝕜[G]} \,.
\]

We may identify the category~$\Vect{𝕜}$ with~$\Mod{𝕜[1]}$ (where~$1$ denotes the trivial group).
We then find from the above discussion that the unique homomorphisms of groups from~$1$ to~$G$ induces adjoint functors
\[
	F ⊣ U ⊣ C \,,
\]
given by the forgetful functor
\[
	U \colon \Mod{𝕜[G]} \to \Vect{𝕜} \,,
\]
the extension of scalars
\[
	F
	≔
	𝕜[G] ⊗_{𝕜} (\ph)
	\colon
	\Vect{𝕜} \to \Mod{𝕜[G]} \,,
\]
and
\[
	C
	≔
	\Hom_{𝕜}(𝕜[G], \ph)
	\colon
	\Vect{𝕜} \to \Mod{𝕜[G]} \,.
\]

Similarly, we can consider the unique homomorphism of groups from~$G$ to~$1$.
This homomorphism induces adjunctions
\[
	C ⊣ T ⊣ I \,.
\]
The functor
\[
	T \colon \Vect{𝕜} \to \Mod{𝕜[G]}
\]
regards a vector space as a trivial~\module{$𝕜[G]$} (i.e., as a trivial representation of~$G$);
the functor
\[
	I \colon \Mod{𝕜[G]} \to \Vect{𝕜}
\]
can be described as
\[
	I
	=
	\Hom_{𝕜[G]}(𝕜, \ph)
	≅
	(\ph)^G \,,
\]
assigning to each~\module{$𝕜[G]$} its linear subspace of invariants;
the functor
\[
	C \colon \Mod{𝕜[G]} \to \Vect{𝕜}
\]
can be described as
\[
	C
	=
	𝕜 ⊗_{𝕜[G]} (\ph)
	≅
	(\ph)_G \,,
\]
assigning to each~\module{$𝕜[G]$}~$M$ its linear quotient of coinvariants, i.e.,
\[
	C(M)
	≅
	M_G
	=
	M / \generate{m - gm \suchthat g ∈ G, m ∈ M}_𝕜 \,.
\]



\addtocounter{subsubsection}{-2}
\subsubsection{}

We can proceed as in part~(b) of this exercise.
For this, we think about the functor category~$[G, \Set]$ as the category of~\sets{$G$}, which we shall denote by~$\GSet{G}$.
We also think about~$\Set$ as~$\GSet{1}$, where~$1$ denotes the trivial group.

Let~$H$ be another group and let~$φ$ be a homomorphism of groups from~$H$ to~$G$.
We regard~$φ$ as a functor from~$H$ to~$G$ and get an induced functor
\[
	φ^* \colon [G, \Set] \to [H, \Set] \,.
\]
This functor corresponds to the restriction functor
\[
	\Res^G_H \colon \GSet{G} \to \GSet{H} \,.
\]
This restriction functor admits both a left adjoint and a right adjoint.
A left adjoint is given by
\[
	\Ind_G^H
	≔
	G ×_H (\ph)
	\colon
	\GSet{H}
	\to
	\GSet{G} \,,%
	\footnote{
		For any~\set{$H$}~$X$, the~\set{$G$}~$G ×_H X$ is given by the set~$(G × X) / {∼}$ where~$∼$ is the equivalence relation generated by~$(gh, x) ∼ (g, hx)$, and the action of~$G$ on~$G ×_H X$ is given by~$g' ⋅ \class{(g, x)} = \class{(g' g, x)}$.
	}
\]
and a right adjoint is given by
\[
	\Coind_G^H
	≔
	\Hom_H(G, \ph) \,.
	\colon
	\GSet{H}
	\to
	\GSet{G} \,.
\]

The unique homomorphism of groups from~$1$ to~$G$ induces adjoint functors
\[
	F ⊣ U ⊣ C \,,
\]
where
\[
	U \colon \GSet{G} \to \Set
\]
is the forgetful functor.
A left adjoint is given by the functor
\[
	F
	≔
	G × (\ph)
	\colon
	\Set
	\to
	\GSet{G} \,,
\]
and a right adjoint is given by the functor
\[
	C
	≔
	\Map(G, \ph)
	\colon
	\Set
	\to
	\GSet{G} \,.
\]
(We can also describe~$F$ and~$C$ as~$F(X) = ∐_{g ∈ G} X$ and~$C(X) = ∏_{g ∈ G} X$.
The action of~$G$ on these sets is then given by permutation of the summands, respectively factors.)

The unique homomorphism of groups from~$G$ to~$1$ does similarly induce adjoint functors
\[
	O ⊣ T ⊣ I \,.
\]
The functor
\[
	T \colon \Set \to \GSet{G}
\]
regard each set as trivial~\sets{$G$};
its right adjoint
\[
	I \colon \GSet{G} \to \Set
\]
can be described as
\[
	I
	=
	\Hom_G(1, \ph)
	≅
	(\ph)^G \,,
\]
assigning to each~\set{$G$} its subset of invariants;
the right adjoint
\[
	O \colon \GSet{G} \to \Set
\]
can be described as
\[
	O
	=
	1 ×_G (\ph)
	≅
	(\ph) / G \,,
\]
assigning to each~\set{$G$} its set of orbits.
