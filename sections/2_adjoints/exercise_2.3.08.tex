\subsection{}

Let~$G$ and~$H$ be two groups, and let~$\cat{G}$ and~$\cat{H}$ be the corresponding one-object categories.
We make the following observations:
\begin{itemize*}

	\item
		A functor from~$\cat{G}$ to~$\cat{H}$ is the same as a homomorphism of groups from~$G$ to~$H$.

	\item
		Given two functors~$F$ and~$F'$ from~$\cat{G}$ to~$\cat{H}$, every natural transformation from~$F$ to~$F'$ is already a natural isomorphism since every morphism in~$\cat{H}$ is an isomorphism.

		Such a natural isomorphism from~$F$ to~$F'$ consists of a single component, which is an element~$h$ of~$H$ with~$h F h^{-1} = F'$.
		This means in particular that there exists a natural transformation between the functors~$F$ and~$F'$ if and only if~$F$ and~$F'$ are conjugated as homomorphisms of groups.

\end{itemize*}


Suppose now that~$(L, R, η, ε)$ is an adjunction between the categories~$\cat{G}$ and~$\cat{H}$.
This adjunction consists of two functors
\[
	L \colon \cat{G} \to \cat{H} \,,
	\quad
	R \colon \cat{H} \to \cat{G} \,,
\]
and two natural transformations
\[
	η \colon \Id_{\cat{G}} \To R L \,,
	\quad
	ε \colon L R \To \Id_{\cat{H}}
\]
that serve as the unit and the counit of the adjunction respectively.
The functors~$L$ and~$R$ correspond to homomorphisms of groups
\[
	l \colon G \to H \,,
	\quad
	r \colon H \to G \,.
\]

Both~$η$ and~$ε$ are actually natural isomorphisms because all morphisms in~$\cat{G}$ and all morphisms in~$\cat{H}$ are isomorphisms.
It follows from Exercise~2.2.11 that both~$L$ and~$R$ are equivalences of categories.
This entails that these functors are fully faithful, which means that~$l$ and~$r$ are bijective, and therefore isomorphisms of groups.

The natural transformation~$η$ consists of only a single component, which is an element~$g$ of~$G$.
That~$η$ is a natural transformation from~$\Id_{\cat{G}}$ to the composite~$R L$ means that the diagram
\[
	\begin{tikzcd}
		\ast_{\cat{G}}
		\arrow{r}[above]{g'}
		\arrow{d}[left]{g}
		&
		\ast_{\cat{G}}
		\arrow{d}[right]{g}
		\\
		\ast_{\cat{G}}
		\arrow{r}[above]{r(l(g'))}
		&
		\ast_{\cat{G}}
	\end{tikzcd}
\]
commutes for every element~$g'$ of~$G$.
In other words, we have
\[
	(r ∘ l)(g) = g g'  g^{-1}
\]
for every element~$g'$ of~$G$, so that the composite~$r ∘ l$ is the inner automorphism of~$G$ given by conjugation with~$g$.

We find similarly that the natural transformation~$ε$ consists of only a single component, which an element~$h$ of~$H$, and that the composite~$l ∘ r$ is the inner automorphism of~$H$ given by conjugation with~$h^{-1}$.

That the natural transformations~$η$ and~$ε$ define an adjunction between the functors~$L$ and~$R$ is equivalent to the triangle identities, i.e., to the commutativity of the following two diagrams:
\[
	\begin{tikzcd}
		L
		\arrow{r}[above]{L η}
		\arrow{dr}[below left]{\id_L}
		&
		LRL
		\arrow{d}[right]{ε L}
		\\
		{}
		&
		L
	\end{tikzcd}
	\qquad
	\begin{tikzcd}
		R
		\arrow{r}[above]{η R}
		\arrow{dr}[below left]{\id_R}
		&
		RLR
		\arrow{d}[right]{R ε}
		\\
		{}
		&
		R
	\end{tikzcd}
\]
By evaluating these diagrams at the single objects of the categories~$\cat{G}$ and~$\cat{H}$, they can equivalently be expressed as follows:
\[
	\begin{tikzcd}
		\ast_{\cat{H}}
		\arrow{r}[above]{l(g)}
		\arrow{dr}[below left]{1_H}
		&
		\ast_{\cat{H}}
		\arrow{d}[right]{h}
		\\
		{}
		&
		\ast_{\cat{H}}
	\end{tikzcd}
	\qquad
	\begin{tikzcd}
		\ast_{\cat{G}}
		\arrow{r}[above]{g}
		\arrow{dr}[below left]{1_G}
		&
		\ast_{\cat{G}}
		\arrow{d}[right]{r(h)}
		\\
		{}
		&
		\ast_{\cat{G}}
	\end{tikzcd}
\]
The commutativity of these diagrams is equivalent two the two conditions
\[
	h = l(g)^{-1} \,,
	\quad
	g = r(h)^{-1} \,.
\]

We find overall that an adjunction between two groups~$G$ and~$H$, when regarded as one-object categories, consists of two isomorphisms of groups
\[
	l \colon G \to H \,,
	\quad
	r \colon H \to G \,,
\]
and elements~$g$ and~$h$ of~$G$ and~$H$ respectively, subject to the following conditions:
the composite~$r ∘ l$ is given by conjugation with~$g$, the composite~$r ∘ l$ is given by composition with~$h^{-1}$, and the two elements~$g$ and~$h$ are related via~$h = l(g)^{-1}$ and~$g = r(h)^{-1}$.%
