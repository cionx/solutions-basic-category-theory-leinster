\subsection{}



\subsubsection*{Adjoint functors}

\begin{itemize}

	\item
		Let~$𝕜$ be a field and let~$V$ be a~\vectorspace{$𝕜$}.
		The two functors
		\[
			(\ph) ⊗_{𝕜} V
			\colon
			\Vect{𝕜} \to \Vect{𝕜}
			\quad\text{and}\quad
			\Hom_{𝕜}(V, \ph)
			\colon
			\Vect{𝕜} \to \Vect{𝕜}
		\]
		are adjoint, with~$(\ph) ⊗_{𝕜} V$ being left adjoint to~$\Hom_{𝕜}(V, \ph)$.
		(This is an example in the vain of Example~2.1.16.)

	\item
		Let more generally~$R$ and~$S$ be two rings and let~$M$ be an~\bimodule{$R$}{$S$}.
		The two functors
		\[
			(\ph) ⊗_R M
			\colon
			\Modr{R} \to \Modr{S}
			\quad\text{and}\quad
			\Hom_S(M, \ph)
			\colon
			\Modr{S} \to \Modr{R}
		\]
		are adjoint, with~$(\ph) ⊗_R M$ being left adjoint to~$\Hom_S(M, \ph)$.

	\item
		For every category~$\cat{A}$ let~$U(\cat{A})$ be its underlying graph, and for every functor~$F$ from a category~$\cat{A}$ to a category~$\cat{B}$ let~$U(F)$ be its induced homomorphism of graphs from~$U(\cat{A})$ to~$U(\cat{B})$.
		This assignment~$U$ is a functor from~$\Cat$ to~$\Graph$, the category of graphs.%
		\footnote{
			By a graph we mean a directed graph, possibly with parallel edges as well as loops.
			We also impose no finiteness conditions on the graphs under consideration.
		}

		For every graph~$Γ$ let~$P(Γ)$ be the following category:
		The objects of~$P(Γ)$ are the vertices of~$Γ$.
		For any two vertices~$x$ and~$y$ of~$Γ$, the morphisms from~$x$ to~$y$ in~$P(Γ)$ are precisely the paths from~$x$ to~$y$ in~$Γ$.%
		\footnote{
			By a path from~$x$ to~$y$ in~$Γ$ we mean a tuple~$p = (x, α_1, \dotsc, α_n, y)$ of the following form:~%
			$α_1, \dotsc, α_n$ are edges in~$Γ$ such that~$α_1$ starts in~$x$, the end vertex of~$α_i$ is the start vertex of~$α_{i+1}$ for all~$i = 1, \dotsc, n - 1$, and~$α_n$ ends in~$y$.
			The vertex~$x$ is the start vertex of~$p$, the vertex~$y$ is the end vertex of~$p$, and the number~$n$ is the length of~$p$.
			This entails in particular that there exists for every vertex~$x$ of~$Γ$ a unique path of length~$0$ from~$x$ to~$x$ in~$Γ$, given by the tuple~$(x, x)$.
			We emphasize that for any two distinct vertices~$x$ and~$y$ of~$Γ$ their associated paths of length~$0$ are again distinct.
		}
		The composition of morphisms of~$P(Γ)$ is the composition of paths in~$Γ$.%
		\footnote{
			The composite~$q ∘ p$ of two paths~$p = (x, α_1, \dotsc, α_n, y)$ and~$q = (y, α_{n+1}, \dotsc, α_m, z)$ is the path~$(x, α_1, \dotsc, α_n, α_{n+1}, \dotsc, α_m, z)$.
		}
		The category~$P(Γ)$ is the \defemph{path category} of~$Γ$.

		Every homomorphism of graphs~$f$ from~$Γ$ to~$Γ'$ induces a functor~$P(f)$ from~$P(Γ)$ to~$P(Γ')$, given on objects by~$P(f)(x) = f(x)$ for every vertex~$x$ of~$Γ$, and on morphisms by
		\[
			P(f)( (x, α_1, \dotsc, α_n, y) )
			=
			( f(x), f(α_1), \dotsc, f(α_n), f(y) )
		\]
		for every path~$(x, α_1, \dotsc, α_n, y)$ in~$Γ$.
		We thus arrive at a functor~$P$ from~$\Graph$ to~$\Cat$.

		The functors~$P$ and~$U$ are adjoint, with~$P$ being left adjoint to~$U$.
		(We may regard~$U$ as a forgetful functor, and~$P(Γ)$ as the \enquote{free category on~$Γ$}.)
\end{itemize}



\subsubsection*{Initial objects}

\begin{itemize}

	\item
		The empty category is initial in~$\CAT$.

	\item
		Let~$R$ be a ring.
		The zero module is initial in~$\Mod{R}$.

	\item
		The initial objects in the category of pointed sets are precisely the one-element pointed sets.

	\item
		Let~$P$ be a partially ordered set and let~$\cat{P}$ be the corresponding category.
		An initial object of~$\cat{P}$ is the same as a least element of~$P$.

\end{itemize}



\subsubsection*{Terminal objects}

\begin{itemize}

	\item
		Let~$R$ be a ring.
		The zero module is terminal in~$\Mod{R}$.

	\item
		Let~$P$ be a partially ordered set and let~$\cat{P}$ be its corresponding category.
		A terminal object of~$\cat{P}$ is the same as a greatest element of~$P$.

	\item
		The terminal objects in~$\Top$ are precisely those topological spaces that consist of only a single point.

\end{itemize}
