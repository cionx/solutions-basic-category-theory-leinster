\subsection{}

Let~$\cat{S}$ be a category.
\begin{enumerate}
	

	\item
		Let~$\cat{A}$ and~$\cat{B}$ be two categories and let~$F$ be a functor from~$\cat{A}$ to~$\cat{B}$.
		We make the following observations:
		\begin{itemize}

			\item
				Let~$K$ be a functor from~$\cat{B}$ to~$\cat{S}$.
				The composite~$KF$ is a functor from~$\cat{A}$ to~$\cat{S}$.

			\item
				Let~$K$ and~$L$ be two functors from~$\cat{B}$ to~$\cat{S}$ and let~$α$ be a natural transformation from~$K$ to~$L$.
				We then have the induced natural transformation~$αF$ from~$KF$ to~$LF$.

			\item
				The above assignments define a functor~$F^*$ from the functor category~$[\cat{B}, \cat{S}]$ to the functor category~$[\cat{A}, \cat{S}]$.
				Let us check the functoriality of~$F^*$:
				\begin{itemize}

					\item
						For every functor~$K$ from~$\cat{B}$ to~$\cat{A}$ we have the equalities
						\[
							F^*(\id_K)_A
							=
							(\id_K F)_A
							=
							(\id_K)_{F(A)}
							=
							\id_{KF(A)}
							=
							( \id_{KF} )_A
							=
							( \id_{F^*(K)} )_A \,,
						\]
						for every object~$A$ of~$\cat{A}$, and therefore the equality of natural transformations
						\[
							F^*(\id_K) = \id_{F^*(K)} \,.
						\]

					\item
						Let~$K$,~$L$ and~$M$ be functors from~$\cat{B}$ to~$\cat{S}$ and let
						\[
							α \colon K \To L \,,
							\quad
							β \colon L \To M
						\]
						be two natural transformations.
						We have the equalities
						\begin{align*}
							( F^*(β) ∘ F^*(α) )_A
							&=
							F^*(β)_A ∘ F^*(α)_A
							\\
							&=
							(βF)_A ∘ (αF)_A
							\\
							&=
							β_{F(A)} ∘ α_{F(A)}
							\\
							&=
							(β ∘ α)_{F(A)}
							\\
							&=
							((β ∘ α)F)_A
							\\
							&=
							F^*(β ∘ α)_A
						\end{align*}
						for every object~$A$ of~$\cat{A}$, and therefore the equality of natural transformations
						\[
							F^*(β) ∘ F^*(α) = F^*(β ∘ α) \,.
						\]

				\end{itemize}
		\end{itemize}
		We have thus constructed an induced functor~$F^*$ from~$[\cat{B}, \cat{S}]$ to~$[\cat{A}, \cat{S}]$.

	\item
		Let~$\cat{A}$,~$\cat{B}$ and~$\cat{C}$ be three categories, and let
		\[
			F \colon \cat{A} \to \cat{B} \,,
			\quad
			G \colon \cat{B} \to \cat{C}
		\]
		be composable functors.
		We then have the equality of functors
		\[
			(G ∘ F)^* = F^* ∘ G^* \,.
		\]
		Let us check this claimed equality in more detail:
		\begin{itemize}

			\item
				Let~$K$ be a functor from~$\cat{C}$ to~$\cat{S}$.
				Then
				\[
					(G ∘ F)^*(K)
					=
					K ∘ G ∘ F
					=
					G^*(K) ∘ F
					=
					F^*( G^*( K ) )
					=
					(F^* ∘ G^*)(K) \,.
				\]

			\item
				Let~$K$ and~$L$ be two functors from~$\cat{C}$ to~$\cat{S}$ and let
				\[
					α \colon K \To L
				\]
				be a natural transformation.
				We have the chain of equalities
				\begin{align*}
					(G ∘ F)^*(α)_A
					&=
					( α (G ∘ F) )_A
					\\
					&=
					α_{(G ∘ F)(A)}
					\\
					&=
					α_{G(F(A))}
					\\
					&=
					(α G)_{F(A)}
					\\
					&=
					((α G) F)_A
					\\
					&=
					(G^*(α) F)_A
					\\
					&=
					F^*( G^*(α) )_A
					\\
					&=
					(F^* ∘ G^*)(α)_A
				\end{align*}
				for every object~$A$ of~$\cat{A}$, and therefore the equality of natural transformations
				\[
					(G ∘ F)^*(α) = (F^* ∘ G^*)(α) \,.
				\]

		\end{itemize}


	\item
		Let~$\cat{A}$ and~$\cat{B}$ be two categories and let~$F$ and~$G$ be two functors from~$\cat{A}$ to~$\cat{B}$, and let~$α$ be a natural transformation from~$F$ to~$G$.
		For every object~$K$ of the functor category~$[\cat{B}, \cat{S}]$, i.e., functor from~$\cat{B}$ to~$\cat{S}$, we get an induced natural transformation~$K α$ from~$K F$ to~$K G$.
		This induced natural transformation is a morphism from~$F^*(K)$ to~$G^*(K)$ in the functor category~$[\cat{A}, \cat{S}]$.
		By setting
		\[
			(α^*)_K ≔ K α
		\]
		for every object~$K$ of~$[\cat{B}, \cat{S}]$, we therefore arrive at an induced transformation~$α^*$ from~$F^*$ to~$G^*$.

		The transformation~$α^*$ is again natural.
		To check this, let
		\[
			β \colon K \to L
		\]
		be a morphism in~$[\cat{B}, \cat{S}]$.
		This means that~$K$ and~$L$ are functors from~$\cat{B}$ to~$\cat{S}$ and that~$β$ is a natural transformation from~$K$ to~$L$.
		We need to check that the diagram
		\[
			\begin{tikzcd}
				F^*(K)
				\arrow{r}[above]{F^*(β)}
				\arrow{d}[left]{(α^*)_K}
				&
				F^*(L)
				\arrow{d}[right]{(α^*)_L}
				\\
				G^*(K)
				\arrow{r}[above]{G^*(β)}
				&
				G^*(L)
			\end{tikzcd}
		\]
		commutes.
		This diagram can be simplified as follows:
		\[
			\begin{tikzcd}
				KF
				\arrow{r}[above]{βF}
				\arrow{d}[left]{Kα}
				&
				LF
				\arrow{d}[right]{Lα}
				\\
				KG
				\arrow{r}[above]{βG}
				&
				LG
			\end{tikzcd}
		\]
		This diagram indeed commutes because, by the interchange law for horizontal and vertical composition, both~$Lα ∘ βF$ and~$βG ∘ Kα$ are given by the horizontal composition~$β * α$.
		Indeed, we have the equalities
		\[
			Lα ∘ βF
			=
			(\id_L * α) ∘ (β * \id_F)
			=
			(\id_L ∘ β) * (α ∘ \id_F)
			=
			β * α
		\]
		and
		\[
			βG ∘ Kα
			=
			(β * \id_G) ∘ (\id_K * α)
			=
			(β ∘ \id_K) * (\id_G ∘ α)
			=
			β * α \,.
		\]

		We have thus extended the construction~$(\ph)^*$ to natural transformations.

	\item
		The induced natural transformation~$α^*$ depends covariantly on~$α$.%
		\footnote{
			Our notation of~$α^*$ is pretty bad in that regard, since it seems to suggest that~$α^*$ depends contravariantly on~$α$.
		}
		Let us check this claim:
		\begin{itemize}

			\item
				Let~$F$ be a functor from~$\cat{A}$ to~$\cat{B}$.
				We have for every object~$K$ of~$[\cat{B}, \cat{S}]$ the chain of equalities
				\[
					( (\id_F)^* )_K
					=
					K \id_F
					=
					\id_{K F}
					=
					\id_{F^*(K)}
					=
					(\id_{F^*})_K \,,
				\]
				and therefore the equality of natural transformations~$(\id_F)^* = \id_{F^*}$.

			\item
				Let now~$F$,~$G$ and~$H$ be three functors from~$\cat{A}$ to~$\cat{B}$, and let
				\[
					α \colon F \To G \,,
					\quad
					β \colon G \To H
				\]
				be two composable natural transformations.
				Then
				\[
					( (β ∘ α)^* )_K
					=
					K (β ∘ α)
					=
					(K β) ∘ (K α)
					=
					( β_* )_K ∘ ( α^* )_K
					=
					( β_* ∘ α_* )_K
				\]
				for every object~$K$ of~$[\cat{B}, \cat{A}]$, and therefore
				\[
					(β ∘ α)^* = β^* ∘ α^* \,.
				\]

	\end{itemize}

	\item
		Let~$\cat{A}$,~$\cat{B}$ and~$\cat{C}$ be three categories.
		Let
		\[
			F, G \colon \cat{A} \to \cat{B} \,,
			\quad
			H \colon \cat{B} \to \cat{C}
		\]
		be functors, and let
		\[
			α \colon F \To G
		\]
		be a natural transformation.

		We have two ways of obtaining from~$α$ and~$H$ an induced natural transformation from the functor~$(H F)^* = F^* H^*$ to the functor~$(H G)^* = G^* H^*$.
		On the one hand, we can form the natural transformation~$H α$ from~$H F$ to~$H G$, which then induces the natural transformation~$(H α)^*$ from~$(H F)^*$ to~$(H G)^*$.
		On the other hand, we can form the induced natural transformation~$α^*$ from~$F^*$ to~$G^*$, and then consider the resulting natural transformation~$α^* H^*$ from~$F^* H^*$ to~$G^* H^*$.

		These natural transformations turn out to be the same, i.e., we have the equality of natural transformations
		\[
			(H α)^* = α^* H^* \,.
		\]
		Indeed, we have for every object~$K$ of~$[\cat{C}, \cat{S}]$ the chain of equalities
		\[
			( (H α)^* )_K
			=
			K (H α)
			=
			(K H) α
			=
			(α^*)_{K H}
			=
			(α^*)_{H^*(K)}
			=
			(α^* H^*)_K \,,
		\]
		and therefore overall the equality~$(H α)^* = α^* H^*$.
\end{enumerate}

We are now well-prepared to prove the statement at hand.
We consider an adjunction between two categories~$\cat{A}$ and~$\cat{B}$ given by two functors
\[
	F \colon \cat{A} \to \cat{B} \,,
	\quad
	G \colon \cat{B} \to \cat{A}
\]
and two natural transformations
\[
	η \colon \Id_{\cat{A}} \To GF \,,
	\quad
	ε \colon FG \To \Id_{\cat{B}}
\]
that serve as the unit and counit of the adjunction respectively.
(The functor~$F$ is left adjoint to the functor~$G$.)
We know that these data satisfy the triangle identities, i.e., that the following two diagrams commute:
\begin{equation}
	\label{triangle identities for adjunction before dualizing}
	\begin{tikzcd}
		F
		\arrow{r}[above]{Fη}
		\arrow{dr}[below left]{\id_F}
		&
		FGF
		\arrow{d}[right]{εF}
		\\
		{}
		&
		F
	\end{tikzcd}
	\qquad
	\begin{tikzcd}
		G
		\arrow{r}[above]{ηG}
		\arrow{dr}[below left]{\id_G}
		&
		GFG
		\arrow{d}[right]{Gε}
		\\
		{}
		&
		G
	\end{tikzcd}
\end{equation}
We have seen in above discussions that the functors~$F$ and~$G$ induce functors
\[
	F^* \colon [\cat{B}, \cat{S}] \to [\cat{A}, \cat{S}] \,,
	\quad
	G^* \colon [\cat{A}, \cat{S}] \to [\cat{B}, \cat{S}] \,.
\]
We have also seen that the natural transformations~$η$ and~$ε$ induce natural transformations
\[
	η^* \colon (\Id_{\cat{A}})^* \To (GF)^* \,,
	\quad
	ε^* \colon (FG)^* \To (\Id_{\cat{B}})^* \,.
\]
By rewriting the domain and codomain of both~$η^*$ and~$ε^*$, we see that these natural transformations are of the forms
\[
	η^* \colon \Id_{[\cat{A}, \cat{S}]} \To F^* G^* \,,
	\quad
	ε^* \colon G^* F^* \To \Id_{[\cat{B}, \cat{S}]} \,.
\]
We can dualize the diagrams~\eqref{triangle identities for adjunction before dualizing} to get the commutative diagrams
\begin{equation}
	\label{triangle identities for adjunction after dualizing}
	\begin{tikzcd}[sep = large]
		F^*
		\arrow{r}[above]{(Fη)^*}
		\arrow{dr}[below left]{(\id_F)^*}
		&
		(FGF)^*
		\arrow{d}[right]{(εF)^*}
		\\
		{}
		&
		F^*
	\end{tikzcd}
	\qquad
	\begin{tikzcd}[sep = large]
		G^*
		\arrow{r}[above]{(ηG)^*}
		\arrow{dr}[below left]{(\id_G)^*}
		&
		(GFG)^*
		\arrow{d}[right]{(Gε)^*}
		\\
		{}
		&
		G*
	\end{tikzcd}
\end{equation}
These diagrams can be simplified as follows:
\begin{equation}
	\begin{tikzcd}[sep = large]
		F^*
		\arrow{r}[above]{η^* F^*}
		\arrow{dr}[below left]{\id_{F^*}}
		&
		F^* G^* F^*
		\arrow{d}[right]{F^* ε^*}
		\\
		{}
		&
		F^*
	\end{tikzcd}
	\qquad
	\begin{tikzcd}[sep = large]
		G^*
		\arrow{r}[above]{G^* η^*}
		\arrow{dr}[below left]{\id_{G^*}}
		&
		G^* F^* G^*
		\arrow{d}[right]{ε^* G^*}
		\\
		{}
		&
		G*
	\end{tikzcd}
\end{equation}
The commutativity of these diagrams tells us that the natural transformations~$η^*$ and~$ε^*$ serve as the unit and counit of an adjunction between the categories~$[\cat{A}, \cat{S}]$ and~$[\cat{B}, \cat{S}]$, with~$F^*$ right adjoint to~$G^*$.
