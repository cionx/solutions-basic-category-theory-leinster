\subsection{}

Let~$\cat{S}$ be a category.
\begin{enumerate}
	

	\item
		Let~$\cat{A}$ and~$\cat{B}$ be two categories and let~$F$ be a functor from~$\cat{B}$ to~$\cat{A}$.
		We make the following observations:
		\begin{itemize}

			\item
				Let~$K$ be a functor from~$\cat{B}$ to~$\cat{S}$.
				The composite~$KF$ is a functor from~$\cat{A}$ to~$\cat{S}$.

			\item
				Let~$K$ and~$L$ be two functors from~$\cat{B}$ to~$\cat{S}$ and let~$α$ be a natural transformation from~$K$ to~$L$.
				We then have the induces natural transformation~$αF$ from~$KF$ to~$LF$.

			\item
				The above assignments define a functor~$F^*$ from the functor category~$[\cat{B}, \cat{S}]$ to the functor category~$[\cat{A}, \cat{S}]$.
				Indeed, for every functor~$K$ from~$\cat{B}$ to~$\cat{A}$ we have the equalities
				\[
					F^*(\id_K)
					=
					\id_K F
					=
					\id_{K F}
					=
					\id_{F^*(K)} \,,
				\]
				and for any three functors~$K$,~$L$ and~$M$ from~$\cat{B}$ to~$\cat{S}$ and every two natural transformations
				\[
					α \colon K \To L \,,
					\quad
					β \colon L \To M
				\]
				we have the equalities
				\[
					F^*(β) ∘ F^*(α)
					=
					βF ∘ αF
					=
					(β ∘ α)F
					=
					F^*(β ∘ α) \,.
				\]
		\end{itemize}
		We have thus constructed an induced functor~$F^*$ from~$[\cat{B}, \cat{S}]$ to~$[\cat{A}, \cat{S}]$.

	\item
		Let~$\cat{A}$,~$\cat{B}$ and~$\cat{C}$ be three categories, and let
		\[
			F \colon \cat{A} \to \cat{B} \,,
			\quad
			G \colon \cat{B} \to \cat{C}
		\]
		be composable functors.
		Then
		\[
			(G ∘ F)^* = F^* ∘ G^*
		\]
		for the induces functors from~$[\cat{C}, \cat{S}]$ to~$[\cat{A}, \cat{S}]$.


	\item
		Let~$\cat{A}$ and~$\cat{B}$ be two categories and let~$F$ and~$G$ be two functors from~$\cat{A}$ to~$\cat{B}$, and let~$α$ be a natural transformation from~$F$ to~$G$.
		For every object~$K$ of the functor category~$[\cat{B}, \cat{S}]$, i.e., functor from~$\cat{B}$ to~$\cat{S}$, we get an induces natural transformation~$K α$ from~$K F$ to~$K G$.
		This induces natural transformation is a morphism from~$F^*(K)$ to~$G^*(K)$ in the functor category~$[\cat{A}, \cat{S}]$.
		By setting
		\[
			(α^*)_K ≔ K α
		\]
		for every object~$K$ of~$[\cat{B}, \cat{S}]$, we therefore arrive at a transformation~$α^*$ from~$F^*$ to~$G^*$.
		
		This transformation~$α^*$ is again natural:
		Let
		\[
			β \colon K \to L
		\]
		be a morphism in~$[\cat{B}, \cat{S}]$.
		This means that~$K$ and~$L$ are functors from~$\cat{B}$ to~$\cat{S}$ and that~$β$ is a natural transformation from~$K$ to~$L$.
		We need to show that the diagram
		\[
			\begin{tikzcd}[sep = large]
				F^*(K)
				\arrow{r}[above]{F^*(β)}
				\arrow{d}[left]{(α^*)_K}
				&
				F^*(L)
				\arrow{d}[right]{(α^*)_L}
				\\
				G^*(K)
				\arrow{r}[above]{G^*(β)}
				&
				G^*(L)
			\end{tikzcd}
		\]
		commutes.
		This diagram can be simplified as follows:
		\[
			\begin{tikzcd}[sep = large]
				KF
				\arrow{r}[above]{βF}
				\arrow{d}[left]{Kα}
				&
				LF
				\arrow{d}[right]{Lα}
				\\
				KG
				\arrow{r}[above]{βG}
				&
				LG
			\end{tikzcd}
		\]
		This diagram indeed commutes because, by the interchange law for horizontal and vertical composition, both composited are given by the horizontal composition~$β * α$.

		We have thus extended the construction~$(\ph)^*$ to natural transformations.

	\item
		The induced natural transformation~$α^*$ depends covariantly~(!) on~$α$:
		\begin{itemize}

			\item
				For every functor~$F$ from~$\cat{B}$ to~$\cat{A}$, we have~$(\id_F)^* = \id_{F^*}$.
				Indeed, we have for every object~$K$ of~$[\cat{A}, \cat{S}]$ the equalities of morphisms
				\[
					( (\id_F)^* )_K
					=
					K \id_F
					=
					\id_{K F}
					=
					\id_{F^*(K)}
					=
					(\id_{F^*})_K \,,
				\]
				and therefore the equality of natural transformations~$(\id_F)^* = \id_{F^*}$.

			\item
				Let now~$F$,~$G$ and~$H$ be three functors from~$\cat{B}$ to~$\cat{A}$, and let
				\[
					α \colon F \To G \,,
					\quad
					β \colon G \To H
				\]
				be two composable natural transformations.
				Then
				\[
					( (β ∘ α)^* )_K
					=
					K (β ∘ α)
					=
					(K β) ∘ (K α)
					=
					( β_* )_K ∘ ( α^* )_K
					=
					( β_* ∘ α_* )_K
				\]
				for every object~$K$ of~$[\cat{B}, \cat{A}]$, and therefore
				\[
					(β ∘ α)^* = α^* ∘ β^* \,.
				\]

	\end{itemize}

	\item
		Let~$\cat{A}$,~$\cat{B}$ and~$\cat{C}$ be three categories.
		Let
		\[
			F, G \colon \cat{A} \to \cat{B} \,,
			\quad
			H \colon \cat{B} \to \cat{C}
		\]
		be functors and let~$α$ be a natural transformation from~$F$ to~$G$.
		We have two ways of obtaining a natural transformation from~$(H F)^* = F^* H^*$ to~$(H G)^* = G^* H^*$.
		On the one hand, we can form the natural transformation~$H α$ from~$H F$ to~$H G$, which then induces the natural transformation~$(H α)^*$ from~$(H F)^*$ to~$(H G)^*$.
		On the other hand, we can form the induces natural transformation~$α^*$ from~$F^*$ to~$G^*$, and then consider the resulting natural transformation~$α^* H^*$ from~$F^* H^*$ to~$G^* H^*$.
		These natural transformations turn out to be the same, i.e.,
		\[
			(H α)^* = α^* H^* \,.
		\]
		Indeed, we have for every object~$K$ of~$[\cat{C}, \cat{S}]$ the equalities of morphisms
		\[
			( (H α)^* )_K
			=
			K (H α)
			=
			(K H) α
			=
			(α^*)_{K H}
			=
			(α^*)_{H^*(K)}
			=
			(α^* H^*)_K \,,
		\]
		and therefore overall the equality of natural transformations~$(H α)^* = α^* H^*$.
\end{enumerate}

We are now well-prepared to prove the statement at hand.

We consider an adjunction between two categories~$\cat{A}$ and~$\cat{B}$ given by two functors
\[
	F \colon \cat{A} \to \cat{B} \,,
	\quad
	G \colon \cat{B} \to \cat{A}
\]
and two natural transformations
\[
	η \colon \Id_{\cat{A}} \To GF \,,
	\quad
	ε \colon FG \To \Id_{\cat{B}}
\]
that serve as the unit and counit of the adjunction respectively.
(The functor~$F$ is left adjoint to the functor~$G$.)
We know that these data satisfy the triangle identities, i.e., that the following two diagrams commute:
\begin{equation}
	\label{triangle identities for adjunction before dualizing}
	\begin{tikzcd}[sep = large]
		F
		\arrow{r}[above]{Fη}
		\arrow{dr}[below left]{\id_F}
		&
		FGF
		\arrow{d}[right]{εF}
		\\
		{}
		&
		F
	\end{tikzcd}
	\qquad
	\begin{tikzcd}[sep = large]
		G
		\arrow{r}[above]{ηG}
		\arrow{dr}[below left]{\id_G}
		&
		GFG
		\arrow{d}[right]{Gε}
		\\
		{}
		&
		G
	\end{tikzcd}
\end{equation}
We have seen in above discussions that the functors~$F$ and~$G$ induce functors
\[
	F^* \colon [\cat{B}, \cat{S}] \to [\cat{A}, \cat{S}] \,,
	\quad
	G^* \colon [\cat{A}, \cat{S}] \to [\cat{B}, \cat{S}] \,.
\]
We have also seen that the natural transformations~$η$ and~$ε$ induce natural transformations
\[
	η^* \colon (\Id_{\cat{A}})^* \To (GF)^* \,,
	\quad
	ε^* \colon (FG)^* \To (\Id_{\cat{B}})^* \,,
\]
which are natural transformations
\[
	η^* \colon \Id_{[\cat{A}, \cat{S}]} \To F^* G^* \,,
	\quad
	ε^* \colon G^* F^* \To \Id_{[\cat{B}, \cat{S}]} \,,
\]
We can dualize the diagrams~\eqref{triangle identities for adjunction before dualizing} to get the commutative diagrams
\begin{equation}
	\label{triangle identities for adjunction after dualizing}
	\begin{tikzcd}[sep = large]
		F^*
		\arrow{r}[above]{(Fη)^*}
		\arrow{dr}[below left]{(\id_F)^*}
		&
		(FGF)^*
		\arrow{d}[right]{(εF)^*}
		\\
		{}
		&
		F^*
	\end{tikzcd}
	\qquad
	\begin{tikzcd}[sep = large]
		G^*
		\arrow{r}[above]{(ηG)^*}
		\arrow{dr}[below left]{(\id_G)^*}
		&
		(GFG)^*
		\arrow{d}[right]{(Gε)^*}
		\\
		{}
		&
		G*
	\end{tikzcd}
\end{equation}
These diagrams can be simplified as follows:
\begin{equation}
	\begin{tikzcd}[sep = large]
		F^*
		\arrow{r}[above]{η^* F^*}
		\arrow{dr}[below left]{\id_{F^*}}
		&
		F^* G^* F^*
		\arrow{d}[right]{F^* ε^*}
		\\
		{}
		&
		F^*
	\end{tikzcd}
	\qquad
	\begin{tikzcd}[sep = large]
		G^*
		\arrow{r}[above]{G^* η^*}
		\arrow{dr}[below left]{\id_{G^*}}
		&
		G^* F^* G^*
		\arrow{d}[right]{ε^* G^*}
		\\
		{}
		&
		G*
	\end{tikzcd}
\end{equation}
The commutativity of these diagrams tells us that the natural transformations~$η^*$ and~$ε^*$ serve as the unit and counit of an adjunction between the categories~$[\cat{A}, \cat{S}]$ and~$[\cat{B}, \cat{S}]$, with~$F^*$ right adjoint to~$G^*$.
