\subsection{}

For every object~$A$ of~$\cat{A}$ and every object~$B$ of~$\cat{B}$, let~$Φ_{A, B}$ be the composite
\[
	\cat{A}(A, G(B))
	\xto{F}
	\cat{B}(F(A), FG(B))
	\xto{(ε_B)_*}
	\cat{B}(F(A), B) \,.
\]
The functor~$F$ is full and faithful and the morphism~$ε_B$ is an isomorphism.
The map~$Φ_{A, B}$ is therefore a composite of two bijections, and thus again a bijection.
We show in the following that the bijection~$Φ_{A, B}$ is natural in both~$A$ and~$B$.

For the naturality of~$A$, we note that for every morphism
\[
	f \colon A \to A'
\]
in~$\cat{A}$, we have the following diagram:
\[
	\begin{tikzcd}
		\cat{A}(A, G(B))
		\arrow{r}[above]{F}
		&
		\cat{B}(F(A), FG(B))
		\arrow{r}[above]{(ε_B)_*}
		&
		\cat{B}(F(A), B)
		\\
		\cat{A}(A', G(B))
		\arrow{u}[left]{f^*}
		\arrow{r}[above]{F}
		&
		\cat{B}(F(A'), FG(B))
		\arrow{u}[right]{F(f)^*}
		\arrow{r}[above]{(ε_B)_*}
		&
		\cat{B}(F(A'), B)
		\arrow{u}[right]{F(f)^*}
	\end{tikzcd}
\]
The left part of this diagram commutes by the functoriality of~$F$.
The right side also commutes.
It follows that the entire diagram commutes, which entails that the following, outer diagram commutes:
\[
	\begin{tikzcd}
		\cat{A}(A, G(B))
		\arrow{r}[above]{Φ_{A, B}}
		&
		\cat{B}(F(A), B)
		\\
		\cat{A}(A', G(B))
		\arrow{u}[left]{f^*}
		\arrow{r}[above]{Φ_{A', B}}
		&
		\cat{B}(F(A'), B)
		\arrow{u}[right]{F(f)^*}
	\end{tikzcd}
\]
This proves the desired naturality.

For the naturality in~$B$, we note that for every morphism
\[
	g \colon B \to B'
\]
in~$\cat{B}$, we have the following diagram:
\[
	\begin{tikzcd}
		\cat{A}(A, G(B))
		\arrow{r}[above]{F}
		\arrow{d}[left]{G(g)_*}
		&
		\cat{B}(F(A), FG(B))
		\arrow{r}[above]{(ε_B)_*}
		\arrow{d}[right]{FG(g)_*}
		&
		\cat{B}(F(A), B)
		\arrow{d}[right]{g_*}
		\\
		\cat{A}(A, G(B'))
		\arrow{r}[above]{F}
		&
		\cat{B}(F(A), FG(B'))
		\arrow{r}[above]{(ε_{B'})_*}
		&
		\cat{B}(F(A), B')
	\end{tikzcd}
\]
The left side of this diagram commutes by the functoriality of~$F$.
The right side commutes by the naturality of~$ε$.
It follows that the entire diagram commutes, which entails that the following outer diagram commutes:
\[
	\begin{tikzcd}
		\cat{A}(A, G(B))
		\arrow{r}[above]{Φ_{A, B}}
		\arrow{d}[left]{G(g)_*}
		&
		\cat{B}(F(A), B)
		\arrow{d}[right]{g_*}
		\\
		\cat{A}(A, G(B'))
		\arrow{r}[above]{Φ_{A, B'}}
		&
		\cat{B}(F(A), B')
	\end{tikzcd}
\]
This shows the desired naturality.

We have thus constructed an adjunction between the two functors~$F$ and~$G$, with~$F$ left-adjoint to~$G$.
(We have constructed this adjunction in precisely such a way that~$ε$ is its counit;
we could have also constructed an adjunction for which~$η$ is its unit.)
