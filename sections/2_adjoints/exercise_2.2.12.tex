\subsection{}



\subsubsection{}

It follows for every morphism
\[
	f \colon A \to A'
\]
in~$\Acat$ from the first formula in Lemma~2.2.4 and the naturality of~$η$ that
\[
	\adjtranspose{ F(f) }
	=
	GF(f) ∘ η_A
	=
	η_{A'} ∘ f \,.
\]
We can therefore express the action of the functor~$F$ on morphisms as
\[
	F(f) = \adjtranspose{ η_{A'} ∘ f } \,.
\]
It follows that the functor~$F$ is faithful, respectively full, if and only if for every two objects~$A$ and~$A'$ of~$\Acat$ the map
\[
	( η_{A'} )_*
	\colon
	\Acat(A, A')
	\to
	\Acat(A, GF(A'))
\]
is injective, respectively surjective.
Putting both of these observations together, we find that~$F$ is fully faithful if and only if the above map~$( η_{A'} )_*$ is bijective for any two objects~$A$ and~$A'$ of~$\Acat$.
This is equivalent to each morphism~$η_{A'}$ being an isomorphism by the upcoming \cref{isomorphism_iff_bijection_on_hom_sets}, and therefore equivalent to~$η$ being a natural isomorphism.

\begin{lemma}
	\label{isomorphism_iff_bijection_on_hom_sets}
	Let~$\Acat$ be a category and let~$f$ be a morphism in~$\Acat$ of the form
	\[
		f \colon A' \to A'' \,.
	\]
	The following conditions on~$f$ are equivalent:
	\begin{equivalenceslist}

		\item
			\label{is_isomorphism}
			The morphism~$f$ is an isomorphism.

		\item
			\label{induced_covariant_bijection}
			The induced map~$f_* \colon \Acat(A, A') \to \Acat(A, A'')$ is a bijection for every object~$A$ of~$\Acat$.

		\item
			\label{induced_contravariant_bijection}
			The induced map~$f^* \colon \Acat(A'', A) \to \Acat(A', A)$ is a bijection for every object~$A$ of~$\Acat$.

	\end{equivalenceslist}
\end{lemma}

\begin{proof}
	It suffices to prove the equivalence of~\ref{is_isomorphism} and~\ref{induced_covariant_bijection}: the equivalence of~\ref{is_isomorphism} and~\ref{induced_contravariant_bijection} then follows by duality.

	If~$f$ is an isomorphism with inverse~$f^{-1}$ then the two resulting maps
	\[
		f_* \colon \Acat(A, A') \to \Acat(A, A'') \,,
		\qquad
		(f^{-1})_* \colon \Acat(A, A'') \to \Acat(A, A')
	\]
	are again mutually inverse.
	This entails that the map~$f_*$ is bijective.
	Thus,~\ref{induced_covariant_bijection} follows from~\ref{is_isomorphism}.

	Suppose conversely that~\ref{induced_covariant_bijection} holds.
	By choosing~$A$ as~$A''$ we see that there exists morphism~$g$ from~$A''$ to~$A'$ with~$\id_{A''} = f_*(g)$, and thus~$\id_{A''} = f ∘ g$.
	It follows that
	\[
		f_*(g ∘ f) = f ∘ g ∘ f = \id_{A''} ∘ f = f = f ∘ \id_{A'} = f_*(\id_{A'})
	\]
	and therefore also~$g ∘ f = \id_{A'}$.
	This shows that the morphism~$g$ is a two-sided inverse to~$f$.
	The existence of such a two-sided inverse means that~$f$ is an isomorphism.
	We have thus shown that~\ref{is_isomorphism} follows from~\ref{induced_covariant_bijection}.
\end{proof}

We find dually the following:
The functor~$G$ is faithful, respectively full, if and only if for every two objects~$B$ and~$B$' of~$\Bcat$ the map
\[
	(ε_B)^*
	\colon
	\Bcat(B, B')
	\to
	\Bcat(FG(B), B')
\]
is injective, respectively surjective.
Therefore,~$G$ is fully faithful if and only if the above map~$(ε_B)^*$ is bijective for any two objects~$B$ and~$B'$ of~$\Bcat$.
By \cref{isomorphism_iff_bijection_on_hom_sets} this is equivalent to~$ε$ being a natural isomorphism.

\begin{remark}
	We have actually shown the following stronger results:
	\begin{enumerate}

		\item
			The left adjoint~$F$ is faithful if and only if the unit~$η$ of the adjunction is a monomorphism in each component.

		\item
			The right adjoint~$G$ is faithful if and only if the counit~$ε$ of the adjunction is an epimorphism in each component.

		\item
			The left adjoint~$F$ is full if and only if the unit~$η$ of the adjunction is a split epimorphism in each component.

		\item
			The right adjoint~$G$ is full if and only if the counit~$ε$ of the adjunction is a split monomorphism in each component.

	\end{enumerate}
	These results can also be found in \cite[IV.3,~Theorem~1]{maclane_working_mathematician}
\end{remark}



\subsubsection{}

\paragraph{Example~2.1.3,~(a)}
The functor~$U$ is not full, and the adjunction therefore not a reflection.

\paragraph{Example~2.1.3,~(b)}
The functor~$U$ is not full, and the adjunction therefore not a reflection.

\paragraph{Example~2.1.3,~(c)}
The functor~$U$ is fully faithful, whence the adjunction is a reflection.

\paragraph{Example~2.1.3,~(d),~$F$ and~$U$}
The functor~$U$ is fully faithful, the adjunction thus a reflection.

\paragraph{Example~2.1.3,~(d),~$U$ and~$R$}
The counit~$ε$ of the adjunction has as its component~$ε_M$ the inclusion map from~$M^×$ to~$M$ for every monoid~$M$.
This map is always injective, but only surjective if~$M$ is a group.
The adjunction is therefore not a reflection.

\paragraph{Example~2.1.5,~$D$ and~$U$}
The functor~$U$ is not full, the adjunction therefore not a reflection.

\paragraph{Example~2.1.5,~$U$ and~$I$}
The functor~$I$ is fully faithful, whence the adjunction is a reflection.

\paragraph{Example~2.1.6}
The right adjoint functor~$(\ph)^B$ is not faithful if~$B$ is empty, and it is not full if the set~$B$ contains at least two distinct elements.
It is fully faithful if and only if the set~$B$ is a singleton, in which case both the left adjoint functor~$(\ph) × B$ and the right adjoint functor~$(\ph)^B$ are mutually inverse equivalences.
