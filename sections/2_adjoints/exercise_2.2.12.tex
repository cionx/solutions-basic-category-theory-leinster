\subsection{}



\subsubsection{}

It follows from the first formula in Lemma~2.2.4 and the naturality of the unit~$η$ that
\[
	\adjtranspose{ F(f) }
	=
	GF(f) ∘ η_A
	=
	η_{A'} ∘ f
\]
for every morphism~$f \colon A \to A'$ in~$\cat{A}$.
We can therefore express the action of the functor~$F$ on such a morphism~$f$ as
\[
	F(f) = \adjtranspose{ η_{A'} ∘ f } \,.
\]
It follows that the functor~$F$ is faithful or full if and only if for every two objects~$A$ and~$A'$ of~$\cat{A}$ the map
\[
	( η_{A'} )_*
	\colon
	\cat{A}(A, A')
	\to
	\cat{A}(A, GF(A'))
\]
is injective, respectively surjective.
By putting both of these observations together we find that~$F$ is full and faithful if and only if the above map~$( η_{A'} )_*$ is bijective for any two objects~$A$ and~$A'$ of~$\cat{A}$.
This is equivalent to each morphism~$η_{A'}$ being an isomorphism by the upcoming \cref{isomorphism_iff_bijection_on_hom_sets}, and therefore equivalent to~$η$ being a natural isomorphism.

\begin{lemma}
	\label{isomorphism_iff_bijection_on_hom_sets}
	Let~$\cat{A}$ be a category and let~$f$ be a morphism in~$\cat{A}$ of the form
	\[
		f \colon A \to A' \,.
	\]
	The following conditions on the morphism~$f$ are equivalent:
	\begin{equivalenceslist}

		\item
			\label{is_isomorphism}
			$f$ is an isomorphism.

		\item
			\label{induced_covariant_bijection}
			The map~$f_* \colon \cat{A}(A'', A) \to \cat{A}(A'', A')$ is bijective for every object~$A''$ of~$\cat{A}$.

		\item
			\label{induced_contravariant_bijection}
			The map~$f^* \colon \cat{A}(A', A'') \to \cat{A}(A, A'')$ is bijective for every object~$A''$ of~$\cat{A}$.

	\end{equivalenceslist}
\end{lemma}

\begin{proof}
	It suffices to prove the equivalence of the conditions~\ref{is_isomorphism} and~\ref{induced_covariant_bijection}.
	The equivalence of the conditions~\ref{is_isomorphism} and~\ref{induced_contravariant_bijection} then follows by duality.

	Suppose first that the morphism~$f$ is an isomorphism.
	The morphisms~$f$ and~$f^{-1}$ are mutually inverse, whence the two induced maps
	\[
		f_* \colon \cat{A}(A'', A) \to \cat{A}(A'', A') \,,
		\qquad
		(f^{-1})_* \colon \cat{A}(A'', A') \to \cat{A}(A'', A)
	\]
	are again mutually inverse.
	This shows that the map~$f_*$ is bijective.
	Thus,~\ref{induced_covariant_bijection} follows from~\ref{is_isomorphism}.

	Suppose conversely that~\ref{induced_covariant_bijection} holds.
	By choosing~$A''$ as~$A'$, we see that there exists a morphism~$g$ from~$A'$ to~$A$ with~$\id_{A'} = f_*(g)$.
	This means that
	\[
		f ∘ g = \id_{A'} \,.
	\]
	We claim that also~$g ∘ f = \id_A$.
	To prove this, we note that both~$g ∘ f$ and~$\id_A$ are morphisms from~$A$ to~$A$ such that
	\[
		f_*(g ∘ f)
		=
		f ∘ g ∘ f
		=
		\id_{A'} ∘ f
		=
		f
		=
		f ∘ \id_A
		=
		f_*( \id_A ) \,.
	\]
	It follows from the injectivity of~$f_*$ (for the case~$A'' = A$) that indeed~$g ∘ f = \id_A$.
\end{proof}

We find dually the following:
The functor~$G$ is faithful, respectively full, if and only if for every two objects~$B$ and~$B$' of~$\cat{B}$ the map
\[
	(ε_B)^*
	\colon
	\cat{B}(B, B')
	\to
	\cat{B}(FG(B), B')
\]
is injective, respectively surjective.
Therefore,~$G$ is full and faithful if and only if the above map~$(ε_B)^*$ is bijective for any two objects~$B$ and~$B'$ of~$\cat{B}$.
By \cref{isomorphism_iff_bijection_on_hom_sets} this is equivalent to~$ε$ being a natural isomorphism.

\begin{remark}
	We have actually shown the following stronger results for the left adjoint~$F$, the right adjoint~$G$, the unit~$η$, and the counit~$ε$.
	\begin{enumerate}

		\item
			$F$ is faithful if and only if~$η$ is a monomorphism in each component.

		\item
			$G$ is faithful if and only if~$ε$ is an epimorphism in each component.

		\item
			$F$ is full if and only if~$η$ is a split epimorphism in each component.

		\item
			$G$ is full if and only if~$ε$ is a split monomorphism in each component.

	\end{enumerate}
	These results can also be found in \cite[IV.3,~Theorem~1]{maclane_working_mathematician}.
\end{remark}



\subsubsection{}

\paragraph{Example~2.1.3,~(a)}
The right adjoint functor~$U$ is not full.
The given adjunction is therefore not a reflection.

\paragraph{Example~2.1.3,~(b)}
The right adjoint functor~$U$ is not full.
The given adjunction is therefore not a reflection.

\paragraph{Example~2.1.3,~(c)}
The right adjoint functor~$U$ is full and faithful.
The given adjunction is therefore a reflection.

\paragraph{Example~2.1.3,~(d), the functors~$F$ and~$U$}
The right adjoint functor~$U$ is full and faithful.
The given adjunction is therefore a reflection.

\paragraph{Example~2.1.3,~(d), the functors~$U$ and~$R$}
The counit~$ε$ of the given adjunction has for every monoid~$M$ as its component~$ε_M$ the inclusion map from~$M^×$ (the group of units of~$M$) to~$M$.
This map is always injective, but only surjective if~$M$ is a group.
The adjunction is therefore not a reflection.

\paragraph{Example~2.1.5, the functors~$D$ and~$U$}
The right adjoint functor~$U$ is not full.
The given adjunction is therefore not a reflection.

\paragraph{Example~2.1.5, the functors~$U$ and~$I$}
The right adjoint functor~$I$ is full and faithful.
The given the adjunction is therefore a reflection.

\paragraph{Example~2.1.6}
The right adjoint functor~$(\ph)^B$ is not faithful if~$B$ is empty, and it is not full if the set~$B$ contains at least two distinct elements.
It is full and faithful if and only if the set~$B$ is a singleton, in which case both the left adjoint functor~$(\ph) × B$ and the right adjoint functor~$(\ph)^B$ are essentially inverse equivalences of categories.
