\subsection{}

The category~$\Open(X)$ of open subsets of~$X$ admits a unique initial object, namely the empty subset~$∅$, as well as a unique terminal object, namely the entire space~$X$.
Instead of~$\Open(X)$ we will consider an arbitrary category~$\cat{A}$ that admits an initial object~$I(\cat{A})$ and a terminal object~$T(\cat{A})$, and which satisfies the following additional properties:
there exist no morphism from~$T(\cat{A})$ to any non-terminal object of~$\cat{A}$, and there exists dually no morphism from any non-initial object of~$\cat{A}$ to~$I(\cat{A})$.%
\footnote{
	We use the notations~$T(\cat{A})$ and~$I(\cat{A})$ instead of the more appropriate~$T_{\cat{A}}$ and~$I_{\cat{A}}$ for better readability.
}

The category~$\Set$ also admits a unique initial object, namely the empty set~$∅$, as well as terminal objects, namely the one-element sets.
Instead of~$\Set$ we will consider an arbitrary category~$\cat{B}$ that admit an initial object~$I(\cat{B})$ and a terminal object~$T(\cat{B})$.



\subsubsection*{The functor~$Δ$}

We start off with the functor
\[
	Δ \colon \cat{B} \to [\cat{A}^{\op}, \cat{B}]
\]
that assigns to each object~$B$ of~$\cat{B}$ the constant functor at~$B$.
To each morphism
\[
	g \colon B \to B'
\]
in~$\cat{B}$ it assigns the natural transformation~$Δ(g)$ from~$Δ(B)$ to~$Δ(B')$ given by the component
\[
	Δ(g)_A ≔ g
\]
for every object~$A$ of~$\cat{A}$.
The assignment~$Δ$ is indeed functorial:
\begin{itemize}

	\item
		We have for every object~$B$ of~$\cat{B}$ the equalities
		\[
			Δ(\id_B)_A
			=
			\id_B
			=
			\id_{Δ(B)(A)}
			=
			( \Id_{Δ(B)} )_A
		\]
		for every object~$A$ of~$\cat{A}$, and therefore the equality
		\[
			Δ(\id_B) = \Id_{Δ(B)} \,.
		\]

	\item
		We have for every two composable morphisms
		\[
			g \colon B \to B' \,,
			\quad
			g' \colon B' \to B''
		\]
		in~$\cat{B}$ the equalities
		\[
			Δ(g' ∘ g)_A
			=
			g' ∘ g
			=
			Δ(g')_A ∘ Δ(g)_A
			=
			( Δ(g') ∘ Δ(g) )_A
		\]
		for every object~$A$ of~$\cat{A}$, and therefore the equality
		\[
			Δ(g' ∘ g) = Δ(g') ∘ Δ(g) \,.
		\]

\end{itemize}



\subsubsection*{The functor~$Γ$}

Let~$F$ be a contravariant functor from~$\cat{A}$ to~$\cat{B}$.
A natural transformation~$β$ from~$Δ(B)$ to~$F$ is, by definition, a family~$(β_A)_{A ∈ \Ob(\cat{A})}$ of morphisms~$β_A$ from~$Δ(B)(A)$ to~$F(A)$ such that for every morphism
\[
	f \colon A \to A'
\]
in~$\cat{A}$, the resulting diagram
\[
	\begin{tikzcd}[sep = large]
		Δ(B)(A')
		\arrow{r}[above]{β_{A'}}
		\arrow{d}[left]{\id_B}
		&
		F(A')
		\arrow{d}[right]{F(f)}
		\\
		Δ(B)(A)
		\arrow{r}[below]{β_A}
		&
		F(A)
	\end{tikzcd}
\]
in~$\cat{B}$ commutes.
By using the definition of~$Δ$, this diagram can equivalently be expressed in the following triangular form:
\[
	\begin{tikzcd}[column sep = large]
		{}
		&
		F(A')
		\arrow{dd}[right]{F(f)}
		\\
		B
		\arrow{ur}[above left]{β_{A'}}
		\arrow{dr}[below left]{β_A}
		&
		{}
		\\
		{}
		&
		F(A)
	\end{tikzcd}
\]
We may consider for the object~$A'$ the terminal object~$T(\cat{A})$, and consequently for the morphism~$f$ the unique morphism from~$A$ to~$T(\cat{A})$.
We then find from the commutativity of the above triangular diagram that the natural transformation~$β$ is uniquely determined by its component at~$T(\cat{A})$, i.e., by the morphism
\[
	β_{T(\cat{A})} \colon B \to F(T(\cat{A})) \,.
\]

In other words, the map
\begin{equation}
	\label{bijection for adjunction between delta and gamma}
	\adjtranspose{(\ph)}
	\colon
	[\cat{A}^{\op}, \cat{B}](Δ(B), F)
	\to
	\cat{B}(B, F(T(\cat{A}))) \,,
	\quad
	β
	\mapsto
	β_{T(\cat{A})}
\end{equation}
is injective.
Let us show that it is also surjective.

Let~$g$ be an arbitrary morphism in~$\cat{B}$ from~$B$ to~$F(T(\cat{A}))$.
For every object~$A$ of~$\cat{A}$ let~$t_A$ be the unique morphisms from~$A$ to the terminal object~$T(\cat{A})$, and let~$β_A$ be the resulting morphisms in~$\cat{B}$ given by
\[
	β_A
	\colon
	B
	\xto{g}
	F(T(\cat{A}))
	\xto{F(t_A)}
	F(A) \,.
\]
The resulting tuple~$β ≔ (β_A)_{A ∈ \Ob(\cat{A})}$ is a natural transformation from~$Δ(B)$ to~$F$.
Indeed, given any morphisms
\[
	f \colon A \to A'
\]
in~$\cat{A}$, we have the commutative diagram
\[
	\begin{tikzcd}[column sep = normal]
		{}
		&
		T(\cat{A})
		&
		{}
		\\
		A
		\arrow{ur}[above left]{t_A}
		\arrow{rr}[above]{f}
		&
		{}
		&
		A'
		\arrow{ul}[above right]{t_{A'}}
	\end{tikzcd}
\]
in~$\cat{A}$ because the object~$T(\cat{A})$ is terminal in~$\cat{A}$.
We conclude that the following diagram in~$\cat{B}$ is again commutative:
\[
	\begin{tikzcd}
		{}
		&
		B
		\arrow{d}[right]{g}
		\arrow[bend right = 35]{ddl}[above left]{β_A}
		\arrow[bend left  = 35]{ddr}[above right]{β_{A'}}
		&
		{}
		\\[1em]
		{}
		&
		F(T(\cat{A}))
		\arrow{dl}[pos = 0.3, above left]{F(t_A)}
		\arrow{dr}[pos = 0.3, above right]{F(t_{A'})}
		&
		{}
		\\
		F(A)
		&
		{}
		&
		F(A')
		\arrow{ll}[above]{F(f)}
	\end{tikzcd}
\]
The commutativity of the outer diagram
\[
	\begin{tikzcd}
		{}
		&
		B
		\arrow{dl}[above left]{β_A}
		\arrow{dr}[above right]{β_{A'}}
		&
		{}
		\\
		F(A)
		&
		{}
		&
		F(A')
		\arrow{ll}[above]{F(f)}
	\end{tikzcd}
\]
shows the claimed naturality of~$β$.

We note that~$t_{T(\cat{A})} = \id_{T(\cat{A})}$, and that therefore
\[
	β_{T(\cat{A})}
	=
	F( t_{T(\cat{A})} ) ∘ g
	=
	F( \id_{T(\cat{A})} ) ∘ g
	=
	\id_{F(T(\cat{A}))} ∘ g
	=
	g \,.
\]
We have therefore shown the surjectivity of~\eqref{bijection for adjunction between delta and gamma}, and thus altogether the bijectivity of~\eqref{bijection for adjunction between delta and gamma}.

Motivated by this bijection, we choose~$Γ$ as the evaluation functor at~$T(\cat{A})$ %
\footnote{
	If we consider for~$\cat{A}$ and~$\cat{B}$ the categories~$\Open(X)$ and~$\Set$, then~$Γ(F)$ is given by~$F(X)$.
	This means that the functor~$Γ$ assigns to each presheaf its global sections.
	The chosen letter~$Γ$ is a reference to these global sections.
}
(We talk about evaluation at~$T(\cat{A})$ because we view~$[\cat{A}^{\op}, \cat{B}]$ as the category of contravariant functors from~$\cat{A}$ to~$\cat{B}$.
If we regard~$[\cat{A}^{\op}, \cat{B}]$ as the category of covariant functor from~$\cat{A}^{\op}$ to~$\cat{B}$, then~$Γ$ is the evaluation functor at~$I(\cat{A}^{\op})$.)
We can then rewrite~\eqref{bijection for adjunction between delta and gamma} as the bijection
\begin{equation}
	\label{bijection for adjunction between delta and gamma rewritten}
	\adjtranspose{(\ph)}
	\colon
	[\cat{A}^{\op}, \cat{B}](Δ(B), F)
	\to
	\cat{B}(B, Γ(F)) \,,
	\quad
	β
	\mapsto
	β_{T(\cat{A})} \,.
\end{equation}
It remains to check that this bijection is natural, since this will then show that the functor~$Γ$ is right adjoint to the functor~$Δ$.

We first check the naturality of~\eqref{bijection for adjunction between delta and gamma rewritten} in the object~$B$ of~$\cat{B}$.
For this, we have to check that for every morphism
\[
	 g \colon B \to B'
\]
in~$\cat{B}$ the following diagram commutes:
\[
	\begin{tikzcd}{}
		[\cat{A}^{\op}, \cat{B}](Δ(B), F)
		\arrow{r}[above]{\adjtranspose{(\ph)}}
		&
		\cat{B}(B, Γ(F))
		\\{}
		[\cat{A}^{\op}, \cat{B}](Δ(B'), F)
		\arrow{u}[left]{Δ(g)^*}
		\arrow{r}[above]{\adjtranspose{(\ph)}}
		&
		\cat{B}(B', Γ(F))
		\arrow{u}[right]{g^*}
	\end{tikzcd}
\]
This diagram indeed commutes, because we have for every element~$β$ of the bottom-left corner the equalities
\begin{align*}
	\adjtranspose{ Δ(g)^*(β) }
	=
	\adjtranspose{ β ∘ Δ(g) }
	=
	( β ∘ Δ(g) )_{T(\cat{A})}
	=
	β_{T(\cat{A})} ∘ Δ(g)_{T(\cat{A})}
	=
	β_{T(\cat{A})} ∘ g
	&=
	\adjtranspose{β} ∘ g
	\\
	&=
	g^*( \adjtranspose{β} ) \,.
\end{align*}

The bijection $\adjtranspose{(\ph)}$ is also natural in~$F$.
To prove this, we need to check that for every morphism
\[
	γ \colon F \To F'
\]
in~$[\cat{A}^{\op}, \cat{B}]$ the following diagram commutes:
\[
	\begin{tikzcd}{}
		[\cat{A}^{\op}, \cat{B}](Δ(B), F)
		\arrow{r}[above]{\adjtranspose{(\ph)}}
		\arrow{d}[left]{γ_*}
		&
		\cat{B}(B, Γ(F))
		\arrow{d}[right]{Γ(γ)_*}
		\\{}
		[\cat{A}^{\op}, \cat{B}](Δ(B), F')
		\arrow{r}[above]{\adjtranspose{(\ph)}}
		&
		\cat{B}(B, Γ(F'))
	\end{tikzcd}
\]
This diagram commutes because for every element~$β$ of the top-left corner we have the equalities
\[
	\adjtranspose{ γ_*( β ) }
	=
	\adjtranspose{ γ ∘ β }
	=
	(γ ∘ β)_{T(\cat{A})}
	=
	γ_{T(\cat{A})} ∘ β_{T(\cat{A})}
	=
	Γ(γ) ∘ \adjtranspose{β}
	=
	Γ(γ)_*( \adjtranspose{β} ) \,.
\]

We have thus altogether constructed a functor~$Γ$ from~$[\cat{A}^{\op}, \cat{B}]$ to~$\cat{B}$ that is right adjoint to the previous functor~$Δ$.



\subsubsection*{The functor~$∇$}

Let us first try to motivate the definition of~$∇$.

Let~$F$ be a contravariant functor from~$\cat{A}$ to~$\cat{B}$, and let~$B$ be an object of~$\cat{B}$.
To construct the desired right adjoint functor~$∇$ of~$Γ$ we need to \enquote{extend} this object~$B$ to a contravariant functor~$∇(B)$ from~$\cat{A}$ to~$\cat{B}$.
This needs to be done in such a way that natural transformations from~$F$ to~$∇(B)$ are \enquote{the same} as morphisms in~$\cat{B}$ from~$Γ(F) = F(T(\cat{A}))$ to~$B$.

In other words, we need to extend morphisms from~$F(T(\cat{A}))$ to~$B$ into natural transformations from~$F$ to a suitable functor~$∇(B)$.

For any such a natural transformation~$β$, its component~$β_{T(\cat{A})}$ is a morphism from~$F(T(\cat{A}))$ to~$∇(B)(T(\cat{A}))$.
We would therefore like to choose~$∇(B)(T(\cat{A}))$ as~$B$, in the hope of making the map
\[
	[\cat{A}^{\op}, \cat{B}](F, ∇(B))
	\to
	\cat{B}(Γ(F), B) \,,
	\quad
	β
	\mapsto
	β_{T(\cat{A})}
\]
a bijection.
To ensure this bijectivity, we then want all other components of~$β$ to be \enquote{trivial} in a suitable sense.
We will be able to achieve this by choosing~$∇(B)(A)$ as the terminal object of~$\cat{B}$ whenever~$A$ is (essentially) distinct from~$T(\cat{A})$.

Motivated by the above discussion, we set
\[
	∇(B)(A)
	≔
	\begin{cases*}
		B          & if~$A$ is terminal in~$\cat{A}$, \\
		T(\cat{B}) & otherwise,
	\end{cases*}
\]
for every object~$A$ of~$\cat{A}$.%
\footnote{
	If we consider for~$\cat{A}$ and~$\cat{B}$ the categories~$\Open(X)$ and~$\Set$ respectively, then~$∇(B)(X) = B$, and~$∇(B)(U) = \{ * \}$ for every proper open subset~$U$ of~$X$.
}
To define the action of~$∇(B)$ on morphisms, we consider an arbitrary morphism
\[
	f \colon A \to A'
\]
in~$\cat{A}$.
We define~$∇(B)(f)$ by case distinction.
\begin{casedistinction}

	\item
		If~$A$ is terminal in~$\cat{A}$, then it follows from the existence of the morphism~$f$ that the object~$A'$ is again terminal in~$\cat{A}$ (by assumption on~$\cat{A}$) whence we can choose the morphism~$∇(B)(f)$ as~$\id_B$.

	\item
		If~$A$ is not terminal in~$\cat{A}$, then we have~$∇(B)(A) = T(\cat{B})$.
		We then let~$∇(B)(f)$ be the unique morphism from~$∇(B)(A')$ to~$T(\cat{B}) = ∇(B)(A)$.

\end{casedistinction}

Let us show that this assignment~$∇(B)$ is a contravariant functor from~$\cat{A}$ to~$\cat{B}$.
For this, we need to check that~$∇(B)$ is compatible with identities and with composition of morphisms.
\begin{itemize}

	\item
		Let~$A$ be on object of~$\cat{A}$.
		To compute the action of~$∇(B)$ on the morphism~$\id_A$ we have two cases to consider.
		\begin{casedistinction}

			\item
				If~$A$ is terminal in~$\cat{A}$, then~$∇(B)(A)$ is defined as~$B$ and~$∇(B)(\id_A)$ is defined as~$\id_B$, whence~$∇(B)(\id_A) = \id_{∇(B)(A)}$.

			\item
				If~$A$ is not terminal in~$\cat{A}$, then the morphisms~$∇(B)(\id_A)$ is defined as the unique morphism from~$∇(B)(A)$ to~$T(\cat{B})$.
				But~$∇(B)(A)$ is defined as~$T(\cat{B})$, so that~$∇(B)(\id_A)$ is the unique morphism from~$T(\cat{B})$ to~$T(\cat{B})$.
				This morphism is precisely~$\id_{T(\cat{B})}$, and thus~$\id_{∇(B)(A)}$.
			
		\end{casedistinction}
		We have in either case that~$∇(B)(\id_A) = \id_{∇(B)(A)}$.

	\item
		Let
		\[
			f \colon A \to A' \,,
			\quad
			f' \colon A' \to A''
		\]
		be two composable morphisms in~$\cat{A}$.
		To compute the action of~$∇(B)$ on the composite~$f' ∘ f$ we have two cases to consider.
		\begin{casedistinction}

			\item
				Suppose that~$A$ is terminal in~$\cat{A}$.
				It then follows from the existence of the morphisms~$f$ and~$f'$ that the objects~$A'$ and~$A''$ are again terminal in~$\cat{A}$ (by assumption on the category~$\cat{A}$).
				It then further follows that the morphisms~$∇(B)(f)$,~$∇(B)(f')$ and~$∇(B)(f' ∘ f)$ are all three given by~$\id_B$.
				Therefore,
				\[
					∇(B)(f) ∘ ∇(B)(f')
					=
					\id_B ∘ \id_B
					=
					\id_B
					=
					∇(B)(f' ∘ f) \,.
				\]

			\item
				Suppose that the object~$A$ is not terminal in~$\cat{A}$.
				The object~$∇(B)(A)$ is then given by the terminal object~$T(\cat{B})$.
				The two morphisms~$∇(B)(f' ∘ f)$ and~$∇(B)(f) ∘ ∇(B)(f')$ have she same domain, namely the object~$∇(B)(A'')$, and their codomain is the terminal object~$T(\cat{B})$.
				These two morphisms must therefore coincide.

		\end{casedistinction}
		We have in either case that~$∇(B)(f' ∘ f) = ∇(B)(f) ∘ ∇(B)(f')$.

\end{itemize}
With this, we have shown that~$∇(B)$ is indeed a contravariant functor from~$\cat{A}$ to~$\cat{B}$.

We will now explain how the functor~$∇(B)$ depends itself functorially on~$B$.
For this, let
\[
	g \colon B \to B'
\]
be a morphism in~$\cat{B}$.
We define a transformation~$∇(g) = (∇(g)_A)_{A ∈ \Ob(\cat{A})}$ from~$∇(B)$ to~$∇(B')$ via the components
\[
	∇(g)_A
	≔
	\begin{cases*}
		g                & if~$A$ is terminal in~$\cat{A}$, \\
		\id_{T(\cat{B})} & otherwise.
	\end{cases*}
\]
This transformation is natural.
To see this, we consider a morphism
\[
	f \colon A \to A'
\]
in~$\cat{A}$, and we need to show that the square diagram
\begin{equation}
	\label{naturality square for nabla}
	\begin{tikzcd}[column sep = huge]
		∇(B)(A')
		\arrow{r}[above]{∇(B)(f)}
		\arrow{d}[left]{∇(g)_{A'}}
		&
		∇(B)(A)
		\arrow{d}[right]{∇(g)_A}
		\\
		∇(B')(A')
		\arrow{r}[above]{∇(B')(f)}
		&
		∇(B')(A)
	\end{tikzcd}
\end{equation}
commutes.
We have two cases to consider.
\begin{casedistinction}

	\item
		Suppose that the object~$A$ is terminal in~$\cat{A}$.
		It then follows from the existence of the morphism~$f$ that the object~$A'$ is also terminal in~$\cat{A}$ (by assumption on the category~$\cat{A}$).
		The square diagram~\eqref{naturality square for nabla} can then be simplified as follows:
		\[
			\begin{tikzcd}
				B
				\arrow{r}[above]{\id_B}
				\arrow{d}[left]{g}
				&
				B
				\arrow{d}[right]{g}
				\\
				B'
				\arrow{r}[above]{\id_{B'}}
				&
				B'
			\end{tikzcd}
		\]
		This diagram commutes.

	\item
		Suppose that the object~$A$ is not terminal in~$\cat{A}$.
		The object~$∇(B')(A)$ is then given by the terminal object~$T(\cat{B})$.
		It follows that the diagram~\eqref{naturality square for nabla} commutes because there exists precisely one morphism from~$∇(B)(A')$ to~$T(\cat{B})$.

\end{casedistinction}
We find in both cases that the diagram~\eqref{naturality square for nabla} commutes.
This shows that~$∇(g)$ is indeed a natural transformation from~$∇(B)$ to~$∇(B')$.

Let us now show that the assignment~$∇$ is functorial.
To this end, we need to check that~$∇$ is compatible with identity morphisms and with composition of morphisms.
\begin{itemize}

	\item
		For every object~$B$ of~$\cat{B}$ we have the equalities
		\begin{align*}
			∇(\id_B)_A
			&=
			\begin{cases*}
				\id_B           & if~$A$ is terminal in~$\cat{A}$, \\
				\id_{T(\cat{B})} & otherwise,
			\end{cases*}
			\\
			&=
			\begin{cases*}
				\id_{∇(B)(A)} & if~$A$ is terminal in~$\cat{A}$, \\
				\id_{∇(B)(A)} & otherwise,
			\end{cases*}
			\\
			&=
			\id_{∇(B)(A)}
			\\
			&=
			( \id_{∇(B)} )_A
		\end{align*}
		for every object~$A$ of~$\cat{A}$, and therefore the equality
		\[
			∇(\id_B)
			=
			\id_{∇(B)} \,.
		\]

	\item
		For every two composable morphisms
		\[
			g \colon B \to B' \,,
			\quad
			g' \colon B' \to B''
		\]
		in~$\cat{B}$ we have the equalities
		\begin{align*}
			{}&
			( ∇(g') ∘ ∇(g) )_A
			\\
			={}&
			∇(g')_A ∘ ∇(g)_A
			\\
			={}&
			\left(
				\begin{cases*}
					g'              & if~$A$ is terminal in~$\cat{A}$, \\
					\id_{T(\cat{B})} & otherwise,
				\end{cases*}
			\right)
			∘
			\left(
				\begin{cases*}
					g               & if~$A$ is terminal in~$\cat{A}$, \\
					\id_{T(\cat{B})} & otherwise,
				\end{cases*}
			\right)
			\\
			={}&
			\begin{cases*}
				g' ∘ g          & if~$A$ is terminal in~$\cat{A}$, \\
				\id_{T(\cat{B})} & otherwise,
			\end{cases*}
			\\
			={}&
			∇(g' ∘ g)_A
		\end{align*}
		for every object~$A$ of~$\cat{A}$, and therefore the equality
		\[
			∇(g') ∘ ∇(g)
			=
			∇(g' ∘ g) \,.
		\]

\end{itemize}
This shows the functoriality of~$∇$.

We will now show that the functor~$∇$ is right adjoint to the functor~$Γ$.
Let~$F$ be a contravariant functor from~$\cat{A}$ to~$\cat{B}$ and let~$B$ be an object of~$\cat{B}$.
A natural transformation~$β$ from~$F$ to~$∇(B)$ has as its component at~$T(\cat{A})$ a morphism from the object~$F(T(\cat{A}))$ to the object~$∇(B)(T(\cat{A}))$.
These objects are given by~$F(T(\cat{A})) = Γ(F)$ and~$∇(B)(T(\cat{A})) = B$ respectively.
We have therefore a well-defined map
\begin{equation}
	\label{map for right adjoint of global sections}
	\adjtranspose{(\ph)}
	\colon
	[\cat{A}^{\op}, \cat{B}](F, ∇(B))
	\to
	\cat{B}(Γ(F), B) \,,
	\quad
	β \mapsto β_{T(\cat{A})} \,.
\end{equation}
We will show in the following that this map is bijective, and natural in both~$F$ and in~$B$.

To show that the map~\eqref{map for right adjoint of global sections} is injective let~$β$ be a natural transformation from~$F$ to~$∇(B)$.
We need to show that~$β$ is uniquely determined by its component~$β_{T(\cat{A})}$, which we shall denote by~$g$.
We show by case distinction that for every object~$A$ of~$\cat{A}$, the morphism~$β_A$ from~$F(A)$ to~$∇(B)(A)$ is uniquely determined by~$g$.
\begin{casedistinction}

	\item
		Suppose the object~$A$ is non-terminal in~$\cat{A}$.
		The object~$∇(B)(A)$ is then given by the terminal object~$T(\cat{B})$.
		There exists precisely one morphism from~$F(A)$ to~$T(\cat{B})$, whence~$β_A$ must be this morphism.

	\item
		Suppose that the object~$A$ is terminal in~$\cat{A}$.
		The object~$∇(B)(A)$ is then given by~$B$.
		Both~$A$ and~$T(\cat{A})$ are terminal objects in~$\cat{A}$, whence the unique morphism~$h$ from~$T(\cat{A})$ to~$A$ is an isomorphism.
		It follows from the naturality of~$β$ that the square diagram
		\[
			\begin{tikzcd}[sep = large]
				F(A)
				\arrow{r}[above]{β_A}
				\arrow{d}[left]{F(h)}
				&
				∇(B)(A)
				\arrow{d}[right]{∇(B)(h)}
				\\
				F(T(\cat{A}))
				\arrow{r}[above]{β_{T(\cat{A})}}
				&
				∇(B)(T(\cat{A}))
			\end{tikzcd}
		\]
		commutes.
		This diagram simplifies as follows:
		\[
			\begin{tikzcd}[sep = large]
				F(A)
				\arrow{r}[above]{β_A}
				\arrow{d}[left]{F(h)}
				&
				B
				\arrow{d}[right]{\id_B}
				\\
				F(T(\cat{A}))
				\arrow{r}[above]{g}
				&
				B
			\end{tikzcd}
		\]
		We find that the morphism~$β_A$ is given by the composite~$g ∘ F(h)$.
		It is therefore uniquely determined by~$g$.

\end{casedistinction}
We have thus shown that the map~\eqref{map for right adjoint of global sections} is injective.

To show that is it surjective, we consider an arbitrary morphism~$g$ from~$Γ(F)$ to~$B$, i.e., a morphism from~$F(T(\cat{A}))$ to~$B$.
We need to construct a natural transformation~$β$ from~$F$ to~$∇(B)$ with~$β_{T(\cat{A})} = g$.
We define the components~$β_A$ via case distinction:
\begin{casedistinction}

	\item
		If~$A$ is a non-terminal object of~$\cat{A}$, then~$∇(B)(A)$ is the terminal object~$T(\cat{B})$, and we let~$β_A$ be the unique morphism from~$F(A)$ to~$∇(B)(A)$.

	\item
		If~$A$ is terminal in~$\cat{A}$, then there exists a unique morphism~$h$ from~$T(\cat{A})$ to~$A$ (and this morphism is an isomorphism), and we let~$β_A$ be the composite~$g ∘ F(h)$.

		(We note that for~$A = T(\cat{A})$ we have~$h = \id_{T(\cat{A})}$, therefore~$F(h) = \id_{F(T(\cat{A}))}$, and thus~$β_{T(\cat{A})} = g ∘ F(h) = g$.)
\end{casedistinction}

To prove the naturality of~$β$ let
\[
	f \colon A \to A'
\]
be an arbitrary morphism in~$\cat{A}$.
We need to show that the square diagram
\[
	\begin{tikzcd}[row sep = large, column sep = huge]
		F(A')
		\arrow{r}[above]{F(f)}
		\arrow{d}[left]{β_{A'}}
		&
		F(A)
		\arrow{d}[right]{β_A}
		\\
		∇(B)(A')
		\arrow{r}[above]{∇(B)(f)}
		&
		∇(B)(A)
	\end{tikzcd}
\]
commutes.
We do so by case distinction.
\begin{casedistinction}

	\item
		Suppose that the object~$A$ is non-terminal in~$\cat{A}$.
		The object~$∇(B)(A)$ is then given by the terminal object~$T(\cat{B})$.
		It follows that the above square diagram commutes because there exists precisely one morphism from~$F(A')$ to~$T(\cat{B})$.

	\item
		Suppose that the object~$A$ is terminal in~$\cat{A}$.
		It then follows from the existence of the morphism~$f$ that the object~$A'$ is again terminal in~$A$ (by assumption on~$\cat{A}$).
		There exist unique morphisms~$h$ and~$h'$ of the forms
		\[
			h \colon T(\cat{A}) \to A \,,
			\quad
			h' \colon T(\cat{A}) \to A' \,.
		\]
		The two morphisms~$h$ and~$h'$ are isomorphisms and fit into the following commutative diagram:
		\begin{equation}
			\label{triagular diagram}
			\begin{tikzcd}[row sep = large]
				{}
				&
				T(\cat{A})
				\arrow{dl}[above left]{h'}
				\arrow{dr}[above right]{h}
				&
				{}
				\\
				A'
				&
				{}
				&
				A
				\arrow{ll}[above]{f}
			\end{tikzcd}
		\end{equation}
		We can now consider the following diagram:
		\begin{equation}
			\label{large diagram for naturality of beta}
			\begin{tikzcd}[sep = large]
				{}
				&
				F(T(\cat{A}))
				\arrow{dd}[right, near start]{g}
				&
				{}
				\\
				F(A')
				\arrow{ur}[above left]{F(h')}
				\arrow[crossing over]{rr}[above, pos = 0.35]{F(f)}
				\arrow{dd}[left]{β_{A'}}
				&
				{}
				&
				F(A)
				\arrow{ul}[above right]{F(h)}
				\arrow{dd}[right]{β_A}
				\\
				{}
				&
				∇(B)(T(\cat{A}))
				&
				{}
				\\
				∇(B)(A')
				\arrow{ur}[above left, pos = 0.65]{∇(B)(h')}
				\arrow{rr}[above]{∇(B)(f)}
				&
				{}
				&
				∇(B)(A)
				\arrow{ul}[above right, pos = 0.65]{∇(B)(h)}
			\end{tikzcd}
		\end{equation}
		The two triangular sides of this diagram commute because the diagram~\eqref{triagular diagram} commutes.
		The background part of this diagram is given by
		\[
			\begin{tikzcd}[sep = large, column sep = huge]
				F(A')
				\arrow{r}[above]{F(h')}
				\arrow{d}[left]{β_{A'}}
				&
				F(T(\cat{A}))
				\arrow{d}[right]{g}
				&
				F(A)
				\arrow{l}[above]{F(h)}
				\arrow{d}[right]{β_A}
				\\
				∇(B)(A')
				\arrow{r}[above]{∇(B)(h')}
				&
				∇(B)(T(\cat{A}))
				&
				∇(B)(A)
				\arrow{l}[above]{∇(B)(h)}
			\end{tikzcd}
		\]
		and can be simplified as follows:
		\[
			\begin{tikzcd}[sep = large]
				F(A')
				\arrow{r}[above]{F(h')}
				\arrow{d}[left]{g ∘ F(h')}
				&
				F(T(\cat{A}))
				\arrow{d}[right]{g}
				&
				F(A)
				\arrow{l}[above]{F(h)}
				\arrow{d}[right]{g ∘ F(h)}
				\\
				B
				\arrow{r}[above]{\id_B}
				&
				B
				&
				B
				\arrow{l}[above]{\id_B}
			\end{tikzcd}
		\]
		This simplified diagram commutes.
		We have thus seen that all non-frontal sides of the diagram~\eqref{large diagram for naturality of beta} commute.
		It follows that its front side also commutes because
		\begin{align*}
			∇(B)(h) ∘ β_A ∘ F(f)
			&=
			g ∘ F(h) ∘ F(f)
			\\
			&=
			g ∘ F(h')
			\\
			&=
			∇(B)(h') ∘ β_{A'}
			\\
			&=
			∇(B)(h) ∘ ∇(B)(f) ∘ β_{A'}
		\end{align*}
		with~$∇(B)(h) = \id_B$ being an isomorphism.

\end{casedistinction}
We have thus proven the naturality of the transformation~$β$, which in turn shows the surjectivity of the map~\eqref{map for right adjoint of global sections}.

To show that the map~\eqref{map for right adjoint of global sections} is natural in~$F$, we consider a natural transformation
\[
	α \colon F \To F'
\]
between two contravariant functors~$F$ and~$F'$ from~$\cat{A}$ to~$\cat{B}$.
We need to show that the diagram
\[
	\begin{tikzcd}[sep = large]
		[\cat{A}^{\op}, \cat{B}](F, ∇(B))
		\arrow{r}[above]{ \adjtranspose{(\ph)} }
		&
		\cat{B}(Γ(F), B)
		\\{}
		[\cat{A}^{\op}, \cat{B}](F', ∇(B))
		\arrow{r}[above]{ \adjtranspose{(\ph)} }
		\arrow{u}[left]{ α^* }
		&
		\cat{B}(Γ(F'), B)
		\arrow{u}[right]{ Γ(α)^* }
	\end{tikzcd}
\]
commutes.
For this, we observe for every element~$β$ of the bottom-left corner of this diagram the equalities
\[
	\adjtranspose{ α^*(β) }
	=
	\adjtranspose{ β ∘ α }
	=
	(β ∘ α)_{T(\cat{A})}
	=
	β_{T(\cat{A})} ∘ α_{T(\cat{A})}
	=
	( α_{T(\cat{A})} )^*( β_{T(\cat{A})} )
	=
	Γ(α)^*( \adjtranspose{β} ) \,.
\]

To show that the map~\eqref{map for right adjoint of global sections} is natural in~$B$, we consider an arbitrary morphism
\[
	g \colon B \to B'
\]
in~$\cat{B}$.
We need to show that the diagram
\[
	\begin{tikzcd}[sep = large]
		[\cat{A}^{\op}, \cat{B}](F, ∇(B))
		\arrow{r}[above]{ \adjtranspose{(\ph)} }
		\arrow{d}[left]{∇(g)_*}
		&
		\cat{B}(Γ(F), B)
		\arrow{d}[right]{g_*}
		\\{}
		[\cat{A}^{\op}, \cat{B}](F, ∇(B'))
		\arrow{r}[above]{ \adjtranspose{(\ph)} }
		&
		\cat{B}(Γ(F), B')
	\end{tikzcd}
\]
commutes.
For this, we observe for every element~$α$ of the top-left corner of this diagram the equalities
\begin{align*}
	g_*( \adjtranspose{α} )
	=
	g ∘ \adjtranspose{α}
	=
	∇(g)_{T(\cat{A})} ∘ α_{T(\cat{A})}
	=
	( ∇(g) ∘ α )_{T(\cat{A})}
	=
	\adjtranspose{ ∇(g) ∘ α }
	=
	\adjtranspose{ ∇(g)_*( α ) } \,.
\end{align*}

We have altogether constructed a functor~$∇$ from~$\cat{B}$ to~$[\cat{A}^{\op}, \cat{B}]$ that is right adjoint to~$Γ$.



\subsubsection*{The functors~$Π$ and~$Λ$}

We have now seen that for suitable categories~$\cat{A}$ and~$\cat{B}$, the diagonal functor
\[
	Δ_{\cat{A}, \cat{B}}
	\colon
	\cat{B} \to [\cat{A}^{\op}, \cat{B}]
\]
admits a right adjoint
\[
	Γ_{\cat{A}, \cat{B}}
	\colon
	[\cat{A}^{\op}, \cat{B}] \to \cat{B} \,,
\]
which in turn admits a right adjoint
\[
	∇_{\cat{A}, \cat{B}}
	\colon
	\cat{B} \to [\cat{A}^{\op}, \cat{B}] \,.
\]
To construct the functors~$Π$ and~$Λ$ we make the following observation.

\begin{proposition}
	\label{adjunctions between opposite categories}
	Let~$\cat{A}$ and~$\cat{B}$ be two categories and let
	\[
		F \colon \cat{A} \to \cat{B} \,,
		\quad
		G \colon \cat{B} \to \cat{A}
	\]
	be two functors.
	We may regard~$F$ and~$G$ as functors
	\[
		F' \colon \cat{A}^{\op} \to \cat{B}^{\op} \,,
		\quad
		G' \colon \cat{B}^{\op} \to \cat{A}^{\op} \,.
	\]
	Then,~$F$ is left adjoint to~$G$ if and only if~$F'$ is right adjoint to~$G'$.
\end{proposition}

\begin{proof}
	Suppose that~$F$ is left adjoint to~$G$.
	This means that we have a bijection
	\[
		Φ_{A, B} \colon \cat{B}(F(A), B) \to \cat{A}(A, G(B))
	\]
	for every object~$A$ of~$\cat{A}$ and every object~$B$ of~$\cat{B}$, such that these bijections are \enquote{natural} in the following sense:
	for all morphisms
	\[
		f \colon A \to A' \,,
		\quad
		g \colon B \to B'
	\]
	in~$\cat{A}$ and~$\cat{B}$ respectively, the following diagram commutes:
	\[
		\begin{tikzcd}[sep = large]
			\cat{B}(F(A'), B)
			\arrow{r}[above]{Φ_{A', B}}
			\arrow{d}[left]{g ∘ (\ph) ∘ F(f)}
			&
			\cat{A}(A', G(B))
			\arrow{d}[right]{G(g) ∘ (\ph) ∘ f}
			\\
			\cat{B}(F(A), B')
			\arrow{r}[above]{Φ_{A, B'}}
			&
			\cat{B}(A, G(B'))
		\end{tikzcd}
	\]

	We may regard the bijections~$Φ_{A, B}$ as bijections
	\[
		Φ'_{B^{\op}, A^{\op}}
		\colon
		\cat{B}^{\op}(B^{\op}, F'(A^{\op}))
		\to
		\cat{A}^{\op}(G'(B^{\op}), A^{\op}) \,.
	\]
	The above commutative diagram can be rewritten as follows:
	\[
		\begin{tikzcd}[row sep = large, column sep = huge]
			\cat{B}^{\op}(B^{\op}, F'((A')^{\op}))
			\arrow{r}[above]{Φ'_{B^{\op},\, (A')^{\op}}}
			\arrow{d}[left]{F'(f^{\op}) ∘ (\ph) ∘ g^{\op}}
			&
			\cat{A}^{\op}(G'(B^{\op}), (A')^{\op})
			\arrow{d}[right]{f^{\op} ∘ (\ph) ∘ G'(g^{\op})}
			\\
			\cat{B}^{\op}((B')^{\op}, F'(A^{\op}))
			\arrow{r}[above]{Φ_{(B')^{\op},\, A^{\op}}}
			&
			\cat{B}^{\op}(G'((B')^{\op}), A^{\op})
		\end{tikzcd}
	\]
	The commutativity of this diagram tells us that the bijections~$Φ_{B^{\op}, A^{\op}}$ are again natural.
	The existence of such natural bijections shows that the functor~$F'$ is right adjoint to the functor~$G'$.

	Suppose conversely that~$F'$ is right adjoint to~$G'$.
	This means that~$G'$ is left adjoint to~$F'$.
	It follows for the functors
	\[
		F'' \colon \cat{A}^{\op\op} \to \cat{B}^{\op\op} \,,
		\quad
		G'' \colon \cat{B}^{\op\op} \to \cat{A}^{\op\op}
	\]
	that~$F''$ is left adjoint to~$G''$.
	Under the equalities of~$\cat{A}^{\op\op}$ and~$\cat{B}^{\op\op}$ with~$\cat{A}$ and~$\cat{B}$ respectively, the functors~$F''$ and~$G''$ correspond to the functors~$F$ and~$G$ respectively.
	Therefore,~$F$ is left adjoint to~$G$.
\end{proof}

We have the chain of adjunctions
\[
	Δ_{\cat{A}^{\op}, \cat{B}^{\op}}
	⊣
	Γ_{\cat{A}^{\op}, \cat{B}^{\op}}
	⊣
	∇_{\cat{A}^{\op}, \cat{B}^{\op}}
\]
between the categories~$\cat{B}^{\op}$ and~$[\cat{A}^{\op\op}, \cat{B}^{\op}]$.
Under the isomorphism
\[
	[\cat{A}^{\op\op}, \cat{B}^{\op}]
	≅
	[\cat{A}^{\op}, \cat{B}]^{\op}
\]
from Exercise~1.3.27, we have a chain of adjunctions between the categories~$\cat{B}^{\op}$ and~$[\cat{A}^{\op}, \cat{B}]^{\op}$.
According to \cref{adjunctions between opposite categories} we get a chain of adjunctions
\[
	∇'_{\cat{A}^{\op}, \cat{B}^{\op}}
	⊣
	Γ'_{\cat{A}^{\op}, \cat{B}^{\op}}
	⊣
	Δ'_{\cat{A}^{\op}, \cat{B}^{\op}}
\]
between~$\cat{B}$ and~$[\cat{A}^{\op}, \cat{B}]$.
Let us derive explicit descriptions of these three functors:
\begin{itemize}

	\item
		Let us abbreviate the functor~$Δ_{\cat{A}^{\op}, \cat{B}^{\op}}$ by~$Δ$.

		Let~$B$ be an object of~$\cat{B}$.
		The functor~$Δ$ assigns to the object~$B^{\op}$ constant functor at~$B^{\op}$.
		The functor~$Δ'$ therefore assigns to the object~$B^{\op\op}$ the constant functor at~$B^{\op\op}$.
		Equivalently, it assigns to the object~$B$ the constant functor at~$B$.

		Let~$g \colon B \to B'$ be a morphism in~$\cat{B}$.
		The functor~$Δ$ assigns to the morphism~$g^{\op}$ the natural transformation~$Δ(g^{\op})$ from the contravariant functor~$Δ(B^{\op})$ to the contravariant functor~$Δ((B')^{\op})$ whose components are given by
		\[
			Δ(g^{\op})_{A^{\op}} = g^{\op}
		\]
		for every object~$A$ of~$\cat{A}$.%
		\footnote{
			The functors~$Δ(B^{\op})$ and~$Δ((B')^{\op})$ are objects of the functor category~$[\cat{A}^{\op\op}, \cat{B}^{\op}]$.
			We view this functor category as the category of contravariant functors from~$\cat{A}^{\op}$ to~$\cat{B}^{\op}$.
			We are therefore indexing the components of~$Δ(g^{\op})$ by the objects of~$\cat{A}^{\op}$, and not by the objects of~$\cat{A}^{\op\op}$.
		}
		The functor~$Δ'$ therefore assigns to the morphism~$g^{\op\op}$ the natural transformation~$Δ'(g^{\op\op})$ whose components are gives by
		\[
			Δ(g^{\op\op})_{A^{\op\op}} = g^{\op\op}
		\]
		for every object~$A$ of~$\cat{A}$.
		Equivalently,
		\[
			Δ(g)_A = g
		\]
		for every object~$A$ of~$\cat{A}$.

		We find from these above descriptions that the functor~$Δ' = Δ'_{\cat{A}^{\op}, \cat{B}^{\op}}$ coincides with the diagonal functor~$Δ_{\cat{A}, \cat{B}}$.

	\item
		Let us abbreviate the functor~$Γ_{\cat{A}^{\op}, \cat{B}^{\op}}$ as~$Γ$.

		The functor~$Γ$, going from the category~$[\cat{A}^{\op\op}, \cat{B}^{\op}]$ to the category~$\cat{B}^{\op}$, is the evaluation functor at~$T(\cat{A}^{\op})$ if we regard~$[\cat{A}^{\op\op}, \cat{B}^{\op}]$ as the category of contravariant functors from~$\cat{A}^{\op}$ to~$\cat{B}^{\op}$.
		As a functor from~$[\cat{A}^{\op}, \cat{B}]^{\op}$ to~$\cat{B}^{\op}$,~$Γ$ is thus given by evaluation at~$T(\cat{A}^{\op})$.

		The functor~$Γ'$ is thus again given by evaluation at~$T(\cat{A}^{\op})$ if we view~$[\cat{A}^{\op}, \cat{B}]$ as the category of covariant functors from~$\cat{A}^{\op}$ to~$\cat{B}$.
		If we view~$[\cat{A}^{\op}, \cat{B}]$ as the category of contravariant functors from~$\cat{A}$ to~$\cat{B}$ instead, then~$Γ'$ is therefore given by evaluation at~$I(\cat{A})$.

	\item
		Let us abbreviate the functor~$∇_{\cat{A}^{\op}, \cat{B}^{\op}}$ by~$∇$.
		This functor goes from the category~$\cat{B}^{\op}$ to the functor category~$[\cat{A}^{\op\op}, \cat{B}^{\op}]$.
		We view this functor category as the category of contravariant functors from~$\cat{A}^{\op}$ to~$\cat{B}^{\op}$.

		Let~$B$ be an object of~$\cat{B}$.
		The contravariant functor~$∇(B^{\op})$ from~$\cat{A}^{\op}$ to~$\cat{B}^{\op}$ is given on objects by
		\[
			∇(B^{\op})(A^{\op})
			=
			\begin{cases*}
				B^{\op}           & if~$A^{\op}$ is terminal in~$\cat{A}^{\op}$, \\
				T(\cat{B}^{\op})  & otherwise,
			\end{cases*}
		\]
		for every object~$A$ of~$\cat{A}$.
		When we regard~$∇(B^{\op})$ as a contravariant functor from~$\cat{A}$ to~$\cat{B}$, then it is given by
		\[
			∇(B^{\op})(A)
			=
			\begin{cases*}
				B           & if~$A$ is initial in~$\cat{A}$, \\
				I(\cat{B})  & otherwise,
			\end{cases*}
		\]
		for every object~$A$ of~$\cat{A}$.
		The functor~$∇'(B)$ from~$\cat{A}$ to~$\cat{B}$ is therefore given by
		\[
			∇'(B)(A)
			=
			\begin{cases*}
				B           & if~$A$ is initial in~$\cat{A}$, \\
				I(\cat{B})  & otherwise,
			\end{cases*}
		\]
		for every object~$A$ of~$\cat{A}$.

		Let~$g \colon B \to B'$ be a morphism in~$\cat{B}$.
		The natural transformation~$∇(g^{\op})$ from the functor~$∇((B')^{\op})$ to the functor~$∇(B^{\op})$ is given by the components
		\[
			∇(g^{\op})_{A^{\op}}
			=
			\begin{cases*}
				g^{\op}                 & if~$A^{\op}$ is terminal in~$\cat{A}^{\op}$, \\
				\id_{T(\cat{B}^{\op})}  & otherwise,
			\end{cases*}
		\]
		for every object~$A$ of~$\cat{A}$.
		If we regard~$∇((B')^{\op})$ and~$∇(B^{\op})$ as functors from~$\cat{A}$ to~$\cat{B}$ instead, then~$∇(g^{\op})$ corresponds to the natural transformation~$α$ from~$∇(B^{\op})$ to~$∇((B')^{\op})$ with components
		\[
			α_A
			=
			\begin{cases*}
				g                 & if~$A$ is initial in~$\cat{A}$, \\
				\id_{I(\cat{B})}  & otherwise,
			\end{cases*}
		\]
		for every object~$A$ of~$\cat{A}$.
		The natural transformation~$∇'(g)$ from~$∇'(B)$ to~$∇'(B')$ is therefore given by the components
		\[
			∇'(g)_A
			=
			\begin{cases*}
				g                 & if~$A$ is initial in~$\cat{A}$, \\
				\id_{I(\cat{B})}  & otherwise,
			\end{cases*}
		\]
		for every object~$A$ of~$\cat{A}$.
\end{itemize}

We have thus derived explicit constructions for the functors~$Λ$ and~$Π$:
\begin{itemize}

	\item
		The functor~$Π$ is the evaluation functor at the initial object of~$\cat{A}$.%
		\footnote{
			If we consider for~$\cat{A}$ and~$\cat{B}$ the categories~$\Open(X)$ and~$\Set$ respectively, then~$Π$ is given by the evaluation functor at~$∅$.
		}

	\item
		Let~$B$ be an object of~$\cat{B}$.
		The contravariant functor~$Λ(B)$ from~$\cat{A}$ to~$\cat{B}$ is given by on objects by
		\[
			Λ(B)(A)
			=
			\begin{cases*}
				B           & if~$A$ is initial in~$\cat{A}$, \\
				I(\cat{B})  & otherwise,
			\end{cases*}
		\]
		for every object~$A$ of~$\cat{A}$,%
		\footnote{
			If we choose for~$\cat{A}$ and~$\cat{B}$ the categories~$\Open(X)$ and~$\Set$ respectively, then the functor~$Λ(B)$ is given on objects by~$Λ(B)(∅) = B$ and~$Λ(B)(U) = ∅$ for every non-empty open subset~$U$ of~$X$.
		}
		and on morphisms by
		\[
			Λ(B)(f)
			=
			\begin{cases*}
				\id_B             & if~$A$ is initial in~$\cat{A}$, \\
				\id_{I(\cat{B})}  & otherwise,
			\end{cases*}
		\]
		for every morphism~$f$ in~$\cat{A}$.
		For every morphism~$g \colon B \to B'$ in~$\cat{B}$, the natural transformation~$Λ(g)$ from~$Λ(B)$ to~$Λ(B')$ is given by the components
		\[
			Λ(g)_A
			=
			\begin{cases*}
				g                 & if~$A$ is initial in~$\cat{A}$, \\
				\id_{I(\cat{B})}  & otherwise,
			\end{cases*}
		\]
		for every object~$A$ of~$\cat{A}$.

\end{itemize}

%The functor~$Π$ can be constructed dually to the functor~$Γ$:
%it is given on objects by
%\[
%	Π(F) ≔ F(I(\cat{A}))
%\]
%for every functor~$F$ from~$\cat{A}^{\op}$ to~$\cat{B}$.
%The functor~$Λ$ can be constructed dually to the functor~$∇$:
%it is given on objects by
%\[
%	Λ(B)(A)
%	=
%	\begin{cases*}
%		B          & if~$A$ is initial in~$\cat{A}$, \\
%		I(\cat{B}) & otherwise,
%	\end{cases*}
%\]
%for every object~$B$ of~$\cat{B}$ and every object~$A$ of~$\cat{A}$.
