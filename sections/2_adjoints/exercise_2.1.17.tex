\subsection{}

The category~$\Open(X)$ of open subsets of~$X$ admits a unique initial object, namely the empty subset~$∅$, as well as a unique terminal object, namely the entire space~$X$.
Instead of~$\Open(X)$ we will consider an arbitrary category~$\Acat$ that admits an initial object~$I_{\Acat}$ and a terminal object~$T_{\Acat}$, and which satisfies the following additional property:
there exist no morphism from a terminal object to any non-terminal object, and there exists no morphism from any non-initial object to an initial object.

The category~$\Set$ also admits a unique initial object, namely the empty set~$∅$, as well as terminal objects, namely the one-element sets.
Instead of~$\Set$ we will consider an arbitrary category~$\Bcat$ that admit an initial object~$I_{\Bcat}$ and a terminal object~$T_{\Bcat}$.



\subsubsection*{The functor~$Δ$}

We start off with the functor
\[
	Δ \colon \Bcat \to [\Acat^{\op}, \Bcat]
\]
that assigns to each object~$B$ of~$\Bcat$ the constant functor at~$B$.
To each morphism
\[
	g \colon B \to B'
\]
in~$\Bcat$ it assigns the natural transformation~$Δ(g)$ from~$Δ(B)$ to~$Δ(B')$ given by the component
\[
	Δ(g)_A
	≔
	g
\]
for every object~$A$ of~$\Acat$.
The assignment~$Δ$ is indeed functorial:
We have for every object~$B$ of~$\Bcat$ the equalities
\[
	Δ(\id_B)_A
	=
	\id_B
	=
	\id_{Δ(B)(A)}
	=
	( \Id_{Δ(B)} )_A
\]
for every object~$A$ of~$\Acat$, and therefore~$Δ(\id_B) = \Id_{Δ(B)}$.
We also have for any two composable morphisms
\[
	g \colon B \to B' \,,
	\quad
	g' \colon B' \to B''
\]
in~$\Bcat$ the equalities
\[
	Δ(g' ∘ g)_A
	=
	g' ∘ g
	=
	Δ(g')_A ∘ Δ(g)_A
	=
	( Δ(g') ∘ Δ(g) )_A
\]
for every object~$A$ of~$\Acat$, and therefore~$Δ(g' ∘ g) = Δ(g') ∘ Δ(g)$.



\subsubsection*{The functor~$Γ$}

Let~$F$ be any functor from~$\Acat^{\op}$ to~$\Bcat$.
A natural transformation~$β$ from~$Δ(B)$ to~$F$ is, by definition, a family~$(β_A)_{A ∈ \Ob(\Acat)}$ of morphisms~$β_A$ from~$Δ(B)(A) = B$ to~$F(A)$ such that for every morphism~$f \colon A \to A'$ in~$\Acat$ the resulting diagram
\[
	\begin{tikzcd}
		B
		\arrow{r}[above]{β_{A'}}
		\arrow{d}[left]{\id_B}
		&
		F(A')
		\arrow{d}[right]{F(f)}
		\\
		B
		\arrow{r}[below]{β_A}
		&
		F(A)
	\end{tikzcd}
\]
in~$\Bcat$ commutes.
We can also express this diagram in the following triangular form:
\[
	\begin{tikzcd}
		{}
		&
		F(A')
		\arrow{dd}[right]{F(f)}
		\\
		B
		\arrow{ur}[above left]{β_{A'}}
		\arrow{dr}[below left]{β_A}
		&
		{}
		\\
		{}
		&
		F(A)
	\end{tikzcd}
\]
We may consider for the object~$A'$ the terminal object~$T_{\Acat}$ of~$\Acat$, and consequently for the morphism~$f$ the unique morphism from~$A$ to~$T_{\Acat}$.
We then find from the above commutative diagram that the natural transformation~$β$ is uniquely determined by its component at~$T_{\Acat}$, i.e., by the morphism
\[
	β_{T_{\Acat}} \colon B \to F(T_{\Acat}) \,.
\]

Let on the other hand~$g$ be an arbitrary morphism from~$B$ to~$F(T_{\Acat})$ in the category~$\Bcat$.
For every object~$A$ of~$\Acat$ let~$t_A$ be the unique morphisms from~$A$ to the terminal object~$T_{\Acat}$ of~$\Acat$, and let~$β_A$ be the resulting morphisms in~$\Bcat$ given by the composite
\[
	β_A ≔ F(t_A) ∘ g \,.
\]
The resulting tuple~$β ≔ (β_A)_{A ∈ \Ob(\Acat)}$ is a natural transformation from~$Δ$ to~$F$.
Indeed, given any morphisms~$f$ in~$\Acat$ from an object~$A$ to an object~$A'$, we have in~$\Acat$ the commutative diagram
\[
	\begin{tikzcd}
		{}
		&
		T_{\Acat}
		&
		{}
		\\
		A
		\arrow{ur}[above left]{t_A}
		\arrow{rr}[above]{f}
		&
		{}
		&
		A'
		\arrow{ul}[above right]{t_{A'}}
	\end{tikzcd}
\]
because the object~$T_{\Acat}$ is terminal in~$\Acat$.
It follows that the following diagram in~$\Bcat$ is again commutative:
\[
	\begin{tikzcd}[sep = large]
		{}
		&
		B
		\arrow{d}[right]{g}
		\arrow[bend right = 35]{ddl}[above left]{β_A}
		\arrow[bend left  = 35]{ddr}[above right]{β_{A'}}
		&
		{}
		\\[1em]
		{}
		&
		F(T_{\Acat})
		\arrow{dl}[pos = 0.3, above left]{F(t_A)}
		\arrow{dr}[pos = 0.3, above right]{F(t_{A'})}
		&
		{}
		\\
		F(A)
		&
		{}
		&
		F(A')
		\arrow{ll}[above]{F(f)}
	\end{tikzcd}
\]
The commutativity of the outer diagram
\[
	\begin{tikzcd}
		{}
		&
		B
		\arrow{dl}[above left]{β_A}
		\arrow{dr}[above right]{β_{A'}}
		&
		{}
		\\
		F(A)
		&
		{}
		&
		F(A')
		\arrow{ll}[above]{F(f)}
	\end{tikzcd}
\]
shows the claimed naturality.
We also note that~$t_{T_{\Acat}} = \id_{T_{\Acat}}$, and therefore
\[
	β_{T_{\Acat}}
	=
	F(\id_{T_{\Acat}}) ∘ g
	=
	\id_{F(T)} ∘ g
	=
	g \,.
\]

Let us summarize our findings:
every natural transformation from~$Δ(B)$ to~$F$ is uniquely determined by its component at the terminal object~$T_{\Acat}$, which is morphism from~$B$ to~$F(T_{\Acat})$; and conversely every such morphism extends to a natural transformation.
We have there found a bijection between the natural transformations from~$Δ(B)$ to~$F$ and the morphisms from~$B$ to~$F(T_{\Acat})$ given by~$β \mapsto β_{T_{\Acat}}$.

Motivated by this observation we construct the right adjoint functor~$Γ$ as follows:
For every functor~$F$ from~$\Acat^{\op}$ to~$\Bcat$ let~$Γ(F)$ be the value of~$Γ$ at~$T_{\Acat}$, i.e., the object~$Γ(T_{\Acat})$ of~$\Bcat$.
For every two functors~$F$ and~$G$ from~$\Acat^{\op}$ to~$\Bcat$ and every natural transformation~$β$ from~$F$ to~$G$ let furthermore~$Γ(β)$ be the component of~$β$ at~$T_{\Acat}$, i.e., the morphism~$β_{T_{\Acat}}$ from~$F(T_{\Acat}) = Γ(F)$ to~$G(T_{\Acat}) = Γ(G)$.

This mapping~$Γ$ is a functor from~$[\Acat^{\op}, \Bcat]$ to~$\Bcat$.
Indeed, for every functor~$F$ from~$\Acat^{\op}$ to~$\Bcat$ we have
\[
	Γ(\id_F)
	=
	(\id_F)_{T_{\Acat}}
	=
	\id_{F(T_{\Acat})}
	=
	\id_{Γ(F)} \,,
\]
and for every two natural transformations
\[
	β \colon F \To F' \,,
	\quad
	β' \colon F' \To F'' \,,
\]
between functors~$F$,~$F'$ and~$F''$ from~$\Acat^{\op}$ to~$\Bcat$ we have
\[
	Γ(β' ∘ β)
	=
	(β' ∘ β)_{T_{\Acat}}
	=
	β'_{T_{\Acat}} ∘ β_{T_{\Acat}}
	=
	Γ(β') ∘ Γ(β) \,.
\]

The functor~$Γ$ is right-adjoint to the functor~$Δ$.
Indeed, let~$B$ be an arbitrary object on~$\Bcat$ and let~$F$ be a functor from~$\Acat^{\op}$ to~$\Bcat$.
We have seen above that the map
\[
	\adjtranspose{(\ph)}
	\colon
	[\Acat^{\op}, \Bcat](Δ(B), F)
	\to
	\Bcat(B, F(T_{\Acat}))
	=
	\Bcat(B, Γ(F)) \,,
	\quad
	β
	\mapsto
	β_{T_{\Acat}}
\]
is bijective.
This bijection is natural in~$B$:
for every morphism
\[
	 g \colon B \to B'
\]
in~$\Bcat$, the resulting diagram
\[
	\begin{tikzcd}[sep = large]
		[\Acat^{\op}, \Bcat](Δ(B), F)
		\arrow{r}[above]{\adjtranspose{(\ph)}}
		&
		\Bcat(B, Γ(F))
		\\{}
		[\Acat^{\op}, \Bcat](Δ(B'), F)
		\arrow{u}[left]{Δ(g)^*}
		\arrow{r}[above]{\adjtranspose{(\ph)}}
		&
		\Bcat(B', Γ(F))
		\arrow{u}[right]{g^*}
	\end{tikzcd}
\]
commutes because for every natural transformation~$β$ from~$Δ(B)$ to~$F$ we have
\[
	\adjtranspose{ Δ(g)^*(β) }
	=
	\adjtranspose{ β ∘ Δ(g) }
	=
	( β ∘ Δ(g) )_{T_{\Acat}}
	=
	β_{T_{\Acat}} ∘ Δ(g)_{T_{\Acat}}
	=
	β_{T_{\Acat}} ∘ g
	=
	\adjtranspose{β} ∘ g
	=
	g^*( \adjtranspose{β} ) \,.
\]
The bijection $\adjtranspose{(\ph)}$ is also natural in~$F$:
for every two functors~$F$ and~$F'$ from~$\Acat^{\op}$ to~$\Bcat$ and every natural transformation
\[
	β \colon F \To F'
\]
the resulting diagram
\[
	\begin{tikzcd}[sep = large]
		[\Acat^{\op}, \Bcat](Δ(B), F)
		\arrow{r}[above]{\adjtranspose{(\ph)}}
		\arrow{d}[left]{β_*}
		&
		\Bcat(B, Γ(F))
		\arrow{d}[right]{Γ(β)_*}
		\\{}
		[\Acat^{\op}, \Bcat](Δ(B), F')
		\arrow{r}[above]{\adjtranspose{(\ph)}}
		&
		\Bcat(B, Γ(F'))
	\end{tikzcd}
\]
commutes because for every natural transformation~$β'$ from~$Δ(B)$ to~$F$ we have
\[
	\adjtranspose{ β_*( β' ) }
	=
	\adjtranspose{ β ∘ β' }
	=
	(β ∘ β')_{T_{\Acat}}
	=
	β_{T_{\Acat}} ∘ β'_{T_{\Acat}}
	=
	Γ(β) ∘ \adjtranspose{β'}
	=
	Γ(β)_*( \adjtranspose{β'} ) \,.
\]

We have thus altogether constructed a functor~$Γ$ that is right adjoint to the previous functor~$Δ$.



\subsubsection*{The functor~$∇$}

Let~$F$ be a functor from~$\Acat^{\op}$ to~$\Bcat$, and let~$B$ be an object of~$\Bcat$.
To construct the desired right adjoint~$∇$ of~$Γ$ we need to \enquote{extend} this object~$B$ to a functor~$∇(B)$ from~$\Acat^{\op}$ to~$\Bcat$.
This needs to be done in such a way that natural transformations from~$F$ to~$∇(B)$ are \enquote{the same} as morphisms from~$Γ(F) = F(T_{\Acat})$ to~$B$.
For any such a natural transformation~$β$, its component~$β_{T_{\Acat}}$ is a morphism from~$F(T_{\Acat})$ to~$∇(B)(T_{\Acat})$.
We would therefore like to choose~$∇(B)(T_{\Acat})$ as~$B$, so that we get a map
\begin{align*}
	\{ \text{natural transformations~$F \to ∇(B)$} \}
	&\to
	\{ \text{morphisms~$Γ(F) \to B$} \} \,,
	\\
	β &\mapsto β_{T_{\Acat}} \,.
\end{align*}
To ensure that this map is a bijection, we then want all other components of~$β$ to be \enquote{trivial} (in a suitable sense).
We will be able to achieve this by choosing~$∇(B)(A)$ as terminal in~$B$ whenever~$A$ is (essentially) distinct from~$T_{\Acat}$.

Motivated by the above discussion, we set
\[
	∇(B)(A)
	≔
	\begin{cases*}
		B         & if~$A$ is terminal in~$\Acat$, \\
		T_{\Bcat} & otherwise,
	\end{cases*}
\]
for every object~$A$ of~$\Acat$.
Let~$f$ be a morphism in~$\Acat$ from an object~$A$ to an object~$A'$.
If~$A$ is terminal in~$\Acat$, then it follows from the existence of the morphism~$f$ that the object~$A'$ is again terminal in~$\Acat$ (by assumption on~$\Acat$) whence we can choose the morphism~$∇(B)(f)$ as~$\id_B$.
If~$A$ is not terminal in~$\Acat$, then we have~$∇(B)(A) = T_{\Bcat}$;
we then let~$∇(B)(f)$ be the unique morphism from~$∇(B)(A')$ to~$T_{\Bcat} = ∇(B)(A)$.

Let us show that this assignment~$∇(B)$ is a functor from~$\Acat^{\op}$ to~$\Bcat$:

Let~$A$ be on object of~$\Acat$.
To compute the action of~$∇(B)$ on the morphism~$\id_A$ we have two cases to consider.
\begin{casedistinction}

	\item
		If~$A$ is terminal in~$\Acat$, then~$∇(B)(A)$ is defined as~$B$ and~$∇(B)(\id_A)$ is defined as~$\id_B$ whence~$∇(B)(\id_A) = \id_{∇(B)(A)}$.

	\item
		If~$A$ is not terminal in~$\Acat$, then~$∇(B)(A)$ is defined as~$T_{\Bcat}$ and the morphisms~$∇(B)(\id_A)$ is defined as the unique morphism from~$T_{\Bcat}$ to~$T_{\Bcat}$, which is~$\id_{T_{\Bcat}}$.
	
\end{casedistinction}
We have in either case that~$∇(B)(\id_A) = \id_{∇(B)(A)}$.

Let now~$f$ and~$f'$ be two composable morphisms in~$\Acat$, given by
\[
	f \colon A \to A' \,,
	\quad
	f' \colon A' \to A''
\]
for some objects~$A$,~$A'$ and~$A''$ of~$\Acat$.
To compute the action of~$∇(B)$ on the composite~$f' ∘ f$ we have again two cases to consider.
\begin{casedistinction}

	\item
		Suppose that~$A$ is terminal in~$\Acat$.
		It then follows from the existence of the morphism~$f$ that the object~$A'$ is also terminal in~$\Acat$ (by assumption on the category~$\Acat$);
		in the same way, the object~$A''$ turns out to be terminal in~$\Acat$.

		It follows that the morphisms~$∇(B)(f)$,~$∇(B)(f')$ and~$∇(B)(f' ∘ f)$ are all three given by~$\id_B$.
		Therefore,
		\[
			∇(B)(f) ∘ ∇(B)(f')
			=
			\id_B ∘ \id_B
			=
			\id_B
			=
			∇(B)(f' ∘ f) \,.
		\]

	\item
		Suppose that the object~$A$ is not terminal in~$\Acat$.
		The object~$∇(B)(A)$ is then given by the terminal object~$T_{\Bcat}$ of~$\Bcat$.
		Both~$∇(B)(f' ∘ f)$ and~$∇(B)(f) ∘ ∇(B)(f')$ are morphisms from~$∇(B)(A'')$ to this terminal object, whence they must coincide.

\end{casedistinction}
We have in either case that~$∇(B)(f' ∘ f) = ∇(B)(f') ∘ ∇(B)(f)$.

With this, we have shown that~$∇(B)$ is indeed a functor from~$\Acat^{\op}$ to~$\Bcat$.
We will now continue by explaining that the functor~$∇(B)$ depends functorially on~$B$.

Let~$B$ and~$B'$ be two objects in~$\Bcat$ and let~$g$ be a morphism from~$B$ to~$B'$.
We define a transformation~$∇(g) = (∇(g)_A)_{A ∈ \Ob(\Acat)}$ from~$∇(B)$ to~$∇(B')$ via the components
\[
	∇(g)_A
	≔
	\begin{cases*}
		g               & if~$A$ is terminal in~$\Acat$, \\
		\id_{T_{\Bcat}} & otherwise.
	\end{cases*}
\]
This transformation is natural:
To see this, we consider a morphism
\[
	f \colon A \to A'
\]
in~$\Acat$, and need to show that the square diagram
\begin{equation}
	\label{naturality_square_for_nabla}
	\begin{tikzcd}[sep = large]
		∇(B)(A')
		\arrow{r}[above]{∇(B)(f)}
		\arrow{d}[left]{∇(g)_{A'}}
		&
		∇(B)(A)
		\arrow{d}[right]{∇(g)_A}
		\\
		∇(B')(A')
		\arrow{r}[above]{∇(B')(f)}
		&
		∇(B')(A)
	\end{tikzcd}
\end{equation}
commutes.
We distinguish between two cases:
\begin{casedistinction}

	\item
		Suppose that the object~$A$ is terminal in~$\Acat$.
		It then follows from the existence of the morphism~$f$ that the object~$A'$ as also terminal in~$\Acat$ (by assumption on the category~$\Acat$).
		The square diagram~\eqref{naturality_square_for_nabla} is then given by the following diagram:
		\[
			\begin{tikzcd}[sep = large]
				B
				\arrow{r}[above]{\id_B}
				\arrow{d}[left]{g}
				&
				B
				\arrow{d}[right]{g}
				\\
				B'
				\arrow{r}[above]{\id_{B'}}
				&
				B'
			\end{tikzcd}
		\]
		This diagram commutes.

	\item
		Suppose that the object~$A$ is not terminal in~$\Acat$.
		The object~$∇(B')(A)$ of~$\Bcat$ is then given by the terminal object~$T_{\Bcat}$.
		It then follows that the diagram~\eqref{naturality_square_for_nabla} commutes because there exists precisely one morphism from~$∇(B)(A')$ to~$T_{\Bcat}$.

\end{casedistinction}
We find in both cases that the diagram~\eqref{naturality_square_for_nabla} commutes.
This shows that~$∇(g)$ is indeed a natural transformation from~$∇(B)$ to~$∇(B')$.

The assignment~$∇$ is itself functorial:
For every object~$B$ of~$\Bcat$ we have
\begin{align*}
	∇(\id_B)_A
	&=
	\begin{cases*}
		\id_B           & if~$A$ is terminal in~$\Acat$, \\
		\id_{T_{\Bcat}} & otherwise,
	\end{cases*}
	\\
	&=
	\begin{cases*}
		\id_{∇(B)(A)} & if~$A$ is terminal in~$\Acat$, \\
		\id_{∇(B)(A)} & otherwise,
	\end{cases*}
	\\
	&=
	\id_{∇(B)(A)}
	\\
	&=
	( \id_{∇(B)} )_A
\end{align*}
for every object~$A$ of~$\Acat$, and therefore
\[
	∇(\id_B)
	=
	\id_{∇(B)} \,.
\]
For every two composable morphisms
\[
	g \colon B \to B' \,,
	\quad
	g' \colon B' \to B''
\]
in~$\Bcat$ we find that
\begin{align*}
	( ∇(g') ∘ ∇(g) )_A
	&=
	∇(g')_A ∘ ∇(g)_A
	\\
	&=
	\left(
		\begin{cases*}
			g'              & if~$A$ is terminal in~$\Acat$, \\
			\id_{T_{\Bcat}} & otherwise,
		\end{cases*}
	\right)
	∘
	\left(
		\begin{cases*}
			g               & if~$A$ is terminal in~$\Acat$, \\
			\id_{T_{\Bcat}} & otherwise,
		\end{cases*}
	\right)
	\\
	&=
	\begin{cases*}
		g' ∘ g          & if~$A$ is terminal in~$\Acat$, \\
		\id_{T_{\Bcat}} & otherwise,
	\end{cases*}
	\\
	&=
	∇(g' ∘ g)_A
\end{align*}
for every object~$A$ of~$\Acat$, and therefore
\[
	∇(g') ∘ ∇(g)
	=
	∇(g' ∘ g) \,.
\]

We have thus constructed a functor~$∇$ from the category~$\Bcat$ to the functor category~$[\Acat^{\op}, \Bcat]$.
We will now show that the functor~$∇$ is right adjoint to the functor~$Γ$.

Let~$F$ be a functor from~$\Acat^{\op}$ to~$\Bcat$ and let~$B$ be an object of~$\Bcat$.
A natural transformation~$β$ from~$F$ to~$∇(B)$ has as its component at~$T_{\Acat}$ a morphism from~$F(T_{\Acat})$ to~$∇(B)(T_{\Acat})$.
These two objects of~$\Bcat$ are given by~$F(T_{\Acat}) = Γ(F)$ and~$∇(B)(T_{\Acat}) = B$.
We have therefore a well-defined map
\begin{equation}
	\label{map_for_right_adjoint_of_global_sections}
	\adjtranspose{(\ph)}
	\colon
	[\Acat^{\op}, \Bcat](F, ∇(B))
	\to
	\Bcat(Γ(F), B) \,,
	\quad
	β \mapsto β_{T_{\Acat}} \,.
\end{equation}
We will show in the following that this map is a well-defined bijection that is natural in both~$F$ and in~$B$.

To show that the map~\eqref{map_for_right_adjoint_of_global_sections} is injective let~$β$ be a natural transformation from~$F$ to~$∇(B)$.
We need to show that~$β$ is uniquely determined by its component~$β_{T_{\Acat}}$, which we will denote by~$g$.
We show by case distinction that for every object~$A$ of~$\Acat$, the morphism~$β_A$ from~$F(A)$ to~$∇(B)(A)$ is uniquely determined by~$g$.
\begin{casedistinction}

	\item
		Suppose the object~$A$ is non-terminal in~$\Acat$.
		The object~$∇(B)(A)$ is then given by the terminal object~$T_{\Bcat}$ of~$\Bcat$.
		There exists precisely one morphism from~$F(A)$ to~$T_{\Bcat}$, whence~$β_A$ must be this morphism.

	\item
		Suppose that the object~$A$ is terminal in~$\Acat$.
		The object~$∇(B)(A)$ is then given by~$B$.
Both~$A$ and~$T_{\Acat}$ are terminal objects in~$\Acat$, whence the unique morphism
		\[
			h \colon T_{\Acat} \to A \,,
		\]
		is an isomorphism.
		It follows from the naturality of~$β$ that the square diagram
		\[
			\begin{tikzcd}[sep = large]
				F(A)
				\arrow{r}[above]{β_A}
				\arrow{d}[left]{F(h)}
				&
				∇(B)(A)
				\arrow{d}[right]{∇(B)(h)}
				\\
				F(T_{\Acat})
				\arrow{r}[above]{β_{T_{\Acat}}}
				&
				∇(B)(T_{\Acat})
			\end{tikzcd}
		\]
		commutes.
		This diagram simplifies as follows:
		\[
			\begin{tikzcd}
				F(A)
				\arrow{r}[above]{β_A}
				\arrow{d}[left]{F(h)}
				&
				B
				\arrow{d}[right]{\id_B}
				\\
				Γ(F)
				\arrow{r}[above]{g}
				&
				B
			\end{tikzcd}
		\]
		We find that the morphism~$β_A$ is given by the composite~$g ∘ F(h)$.
		It is therefore uniquely determined by~$g$.

\end{casedistinction}
We have thus shown that the map~\eqref{map_for_right_adjoint_of_global_sections} is injective.

To show that is it surjective we consider an arbitrary morphism~$g$ from~$Γ(F)$ to~$B$, i.e., a morphism from~$F(T_{\Acat})$ to~$B$.
We need to define a natural transformation~$β$ from~$F$ to~$∇(B)$ with~$β_{T_{\Acat}} = g$.

If~$A$ is a non-terminal object of~$\Acat$, then~$∇(B)(A)$ is the terminal object~$T_{\Bcat}$ of~$\Bcat$, and we let~$β_A$ be the unique morphism from~$F(A)$ to~$∇(B)(A)$.
If~$A$ is terminal in~$\Acat$, then there exists in~$\Acat$ a unique morphism~$h$ from~$T_{\Acat}$ to~$A$ (and this morphism is an isomorphism), and we let~$β_A$ be the composite~$g ∘ F(h)$.
This means in particular for~$A = T_{\Acat}$ that~$h = \id_{T_{\Acat}}$, therefore~$F(h) = \id_{F(T_{\Acat})}$, and thus~$β_{T_{\Acat}} = g ∘ F(h) = g$.

For the surjectivity of~\eqref{map_for_right_adjoint_of_global_sections} it now remains to show that the transformation~$β$ defined in this way is natural.
To prove this naturality let
\[
	f \colon A \to A'
\]
be an arbitrary morphism in~$\Acat$.
We need to show that the square diagram
\[
	\begin{tikzcd}[sep = large]
		F(A')
		\arrow{r}[above]{F(f)}
		\arrow{d}[left]{β_{A'}}
		&
		F(A)
		\arrow{d}[right]{β_A}
		\\
		∇(B)(A')
		\arrow{r}[above]{∇(B)(f)}
		&
		∇(B)(A)
	\end{tikzcd}
\]
commutes.
We do so by case distinction.
\begin{casedistinction}

	\item
		Suppose that the object~$A$ is non-terminal in~$\Acat$.
		Then~$∇(B)(A)$ is given by the terminal object~$T_{\Bcat}$ of~$\Bcat$.
		It then follows that the above square diagram commutes because there exists precisely one morphisms from~$F(A')$ to~$T_{\Bcat}$ in~$\Bcat$.

	\item
		Suppose that the object~$A$ is terminal in~$\Acat$.
		It then follows from the existence of the morphism~$f$ that the object~$A'$ is also terminal in~$A$ (by assumption on~$\Acat$).
		There exist unique morphisms~$h$ and~$h'$ of the forms
		\[
			h \colon T_{\Acat} \to A \,,
			\quad
			h' \colon T_{\Acat} \to A' \,,
		\]
		these morphisms are isomorphisms and form, together with~$f$, the following commutative diagram:
		\begin{equation}
			\label{triagular_diagram}
			\begin{tikzcd}
				{}
				&
				T_{\Acat}
				\arrow{dl}[above left]{h'}
				\arrow{dr}[above right]{h}
				&
				{}
				\\
				A'
				&
				{}
				&
				A
				\arrow{ll}[above]{f}
			\end{tikzcd}
		\end{equation}
		We can consider now the following diagram:
		\begin{equation}
			\label{large_diagram_for_naturality_of_beta}
			\begin{tikzcd}
				{}
				&
				F(T_{\Acat})
				\arrow{dd}[right, near start]{g}
				&
				{}
				\\
				F(A')
				\arrow{ur}[above left]{F(h')}
				\arrow[crossing over]{rr}[below, near start]{F(f)}
				\arrow{dd}[left]{β_{A'}}
				&
				{}
				&
				F(A)
				\arrow{ul}[above right]{F(h)}
				\arrow{dd}[right]{β_A}
				\\
				{}
				&
				∇(B)(T_{\Acat})
				&
				{}
				\\
				∇(B)(A')
				\arrow{ur}[above left]{∇(B)(h')}
				\arrow{rr}[above]{∇(B)(f)}
				&
				{}
				&
				∇(B)(A)
				\arrow{ul}[above right]{∇(B)(h)}
			\end{tikzcd}
		\end{equation}
		The two triangular sides of this diagram commute because the diagram~\eqref{triagular_diagram} commutes
		The background part of this diagram looks as follows:
		\[
			\begin{tikzcd}[sep = large]
				F(A')
				\arrow{r}[above]{F(h')}
				\arrow{d}[left]{β_{A'}}
				&
				F(T_{\Acat})
				\arrow{d}[right]{g}
				&
				F(A)
				\arrow{l}[above]{F(h)}
				\arrow{d}[right]{β_A}
				\\
				∇(B)(A')
				\arrow{r}[above]{∇(B)(h')}
				&
				∇(B)(T_{\Acat})
				&
				∇(B)(A)
				\arrow{l}[above]{∇(B)(h)}
			\end{tikzcd}
		\]
		This diagram simplifies to the following diagram, which commutes:
		\[
			\begin{tikzcd}[sep = large]
				F(A')
				\arrow{r}[above]{F(h')}
				\arrow{d}[left]{g ∘ F(h')}
				&
				F(T_{\Acat})
				\arrow{d}[right]{g}
				&
				F(A)
				\arrow{l}[above]{F(h)}
				\arrow{d}[right]{g ∘ F(h)}
				\\
				B
				\arrow{r}[above]{\id_B}
				&
				B
				&
				B
				\arrow{l}[above]{\id_B}
			\end{tikzcd}
		\]
		This diagram commutes.
		We have thus seen that all non-frontal sides of the diagram~\eqref{large_diagram_for_naturality_of_beta} commute.
		It follows that the front side also commutes because
		\begin{align*}
			∇(B)(h) ∘ β_A ∘ F(f)
			&=
			g ∘ F(h) ∘ F(f)
			\\
			&=
			g ∘ F(h')
			\\
			&=
			∇(B)(h') ∘ β_{A'}
			\\
			&=
			∇(B)(h) ∘ ∇(B)(f) ∘ β_{A'}
		\end{align*}
		and~$∇(B)(h) = \id_B$ is an isomorphism.

\end{casedistinction}
We have thus proven the naturality of the transformation~$β$, which in turn shows the surjectivity of the map~\eqref{map_for_right_adjoint_of_global_sections}.

To show that the map~\eqref{map_for_right_adjoint_of_global_sections} is natural in~$F$, we consider two functors~$F$ and~$G$ from~$\Acat^{\op}$ to~$\Bcat$ and a natural transformation~$α$ from~$F$ to~$G$.
We need to show that the diagram
\[
	\begin{tikzcd}[sep = large]
		[\Acat^{\op}, \Bcat](F, ∇(B))
		\arrow{r}[above]{ \adjtranspose{(\ph)} }
		&
		\Bcat(Γ(F), B)
		\\{}
		[\Acat^{\op}, \Bcat](G, ∇(B))
		\arrow{r}[above]{ \adjtranspose{(\ph)} }
		\arrow{u}[left]{ α^* }
		&
		\Bcat(Γ(G), B)
		\arrow{u}[right]{ (α_{T_{\Acat}})^* }
	\end{tikzcd}
\]
commutes.
For every natural transformation~$β$ from~$G$ to~$∇(B)$ we have
\[
	\adjtranspose{ α^*(β) }
	=
	\adjtranspose{ β ∘ α }
	=
	(β ∘ α)_{T_{\Acat}}
	=
	β_{T_{\Acat}} ∘ α_{T_{\Acat}}
	=
	( α_{T_{\Acat}} )^*( β_{T_{\Acat}} )
	=
	( α_{T_{\Acat}} )^*( \adjtranspose{α} ) \,,
\]
which shows the desired commutativity.

To show that the map~\eqref{map_for_right_adjoint_of_global_sections} is natural in~$B$, we consider two objects~$B$ and~$B'$ of~$\Bcat$ and a morphism~$g$ from~$B$ to~$B'$.
We need to show that the diagram
\[
	\begin{tikzcd}[sep = large]
		[\Acat^{\op}, \Bcat](F, ∇(B))
		\arrow{r}[above]{ \adjtranspose{(\ph)} }
		\arrow{d}[left]{∇(g)_*}
		&
		\Bcat(Γ(F), B)
		\arrow{d}[right]{g_*}
		\\{}
		[\Acat^{\op}, \Bcat](F, ∇(B'))
		\arrow{r}[above]{ \adjtranspose{(\ph)} }
		&
		\Bcat(Γ(F), B')
	\end{tikzcd}
\]
commutes.
For every natural transformation~$α$ from~$F$ to~$∇(B)$ we have
\[
	g_*( \adjtranspose{α} )
	=
	g ∘ \adjtranspose{α}
	=
	g ∘ α_{T_{\Acat}}
	=
	∇(g)_{T_{\Acat}} ∘ α_{T_{\Acat}}
	=
	( ∇(g) ∘ α )_{T_{\Acat}}
	=
	\adjtranspose{ ∇(g) ∘ α }
	=
	\adjtranspose{ ∇(g)_*( α ) } \,,
\]
which shows the desired commutativity.

We have altogether constructed a functor~$∇$ from~$\Bcat$ to~$[\Acat^{\op}, \Bcat]$ that is right adjoint to~$Γ$.



\subsubsection*{The functors~$Π$ and~$Λ$}

The functor~$Π$ can be constructed dually to the functor~$Γ$:
it is given on objects by
\[
	Π(F) ≔ F(I_{\Acat})
\]
for every functor~$F$ from~$\Acat^{\op}$ to~$\Bcat$.
The functor~$Λ$ can be constructed dually to the functor~$∇$:
it is given on objects by
\[
	Λ(B)(A)
	=
	\begin{cases*}
		B         & if~$A$ is initial in~$\Acat$, \\
		I_{\Bcat} & otherwise,
	\end{cases*}
\]
for every object~$B$ of~$\Bcat$ and every object~$A$ of~$\Acat$.

I’m not gonna deal with these constructions in more detail because I won’t waste any more time on this exercise.%
\footnote{
	One ought to be able to conclude the existence, construction and adjointness of the functors~$Π$ and~$Λ$ from that of the functors~$Γ$ and~$∇$ by using the isomorphism~$[\Acat^{\op}, \Bcat]^{\op} ≅ [\Acat^{\op \op}, \Bcat^{\op}]$.
	But this isomorphism hasn’t been introduced yet.
}
