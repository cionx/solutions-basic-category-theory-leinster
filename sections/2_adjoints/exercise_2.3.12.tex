\subsection{}



\subsubsection{Equivalence of categories}

In the following, we will more generally denote a partially defined function on a set~$X$ as a pair~$(S, f)$ where~$f$ is a function defined on some subset~$S'$ of~$X$ with~$S ⊆ S'$.
This is then to be understood as the partially defined function~$(S, \restrict{f}{S})$ as explained in the exercise.

We define a functor~$F$ from~$\Par$ to~$\Set_*$ as follows.
\begin{itemize}

	\item
		For every set~$X$, let~$F(X)$ be the pointed set that arises from~$X$ by adjoining a base point~$x_0$.
		(More explicitly:
		let~$x_0$ be an element not contained in~$X$, and then let~$F(X)$ be the pointed set~$(X ∪ \{ x_0 \}, x_0)$.
		From a set-theoretic perspective, this element~$x_0$ may be choosen to be the set~$X$ itself, since no set can be an element of itself.)

	\item
		For every two sets~$X$ and~$Y$, and every partially defined function~$(S, f)$ from~$X$ to~$Y$, let~$F( (S, f) )$ be the pointed map from~$F(X)$ to~$F(Y)$ given by
		\[
			F( (S, f) )
			\colon
			x
			\mapsto
			\begin{cases*}
				f(x) & if~$x ∈ S$, \\
				y_0  & otherwise.\footnotemark
			\end{cases*}
		\]
		\footnotetext{
			One might think about the newly added base point~$y_0$ of~$Y$ as the value~\enquote{$\mathtt{undefined}$}.
			The map~$F( (S, f) )$ is then an extension of the partially defined function~$(S, f)$ to a map
			\[
				X ∪ \{ \mathtt{undefined} \}
				\to
				Y ∪ \{ \mathtt{undefined} \}
			\]
			that maps all input values outside of~$S$ to~$\mathtt{undefined}$.
		}

\end{itemize}
These assignments define a functor from~$\Par$ to~$\Set_*$:
\begin{itemize}

	\item
		Let~$X$ be a set.
		The identity morphism of~$x$ in the category~$\Par$ is the pair~$(X, \id_X)$, where~$\id_X$ denotes the identity map on the set~$X$.
		The induced map~$F( (X, \id_X) )$ sends every element of~$X$ to itself, and also the basepoint~$x_0$ to itself.
		We therefore have the equality~$F( (X, \id_X) ) = \id_{F(X)}$.

	\item
		Let~$X$,~$Y$ and~$Z$ be three sets, and let
		\[
			(S, f) \colon X \to Y \,,
			\quad
			(T, g) \colon Y \to Z
		\]
		be two partially defined functions.
		The composite~$F( (T, g) ) ∘ F( (S, f) )$ is given on elements of~$X$ as follows:
		\begin{itemize}

			\item
				The basepoint~$x_0$ of~$F(X)$ is mapped to the basepoint~$y_0$ by~$F( (S, f) )$, and then further to the basepoint~$z_0$ by~$F( (T, g) )$.

			\item
				Let~$x$ be an element of~$X$ not contained in~$S$.
				The element~$x$ is first mapped to the basepoint~$y_0$ of~$Y$ by~$F( (S, f) )$, and then further mapped to the basepoint~$z_0$ of~$Z$ by~$F( (T, g) )$.

			\item
				Let~$x$ be an element of~$X$ that is contained in the domain~$S$ of~$f$, but not in the preimage~$f^{-1}(T)$.
				This element~$x$ is first mapped to the point~$f(x)$ in~$Y$ by~$F( (S, f) )$.
				This element~$f(x)$ lies outside of~$T$, whence it it then further mapped to the base point~$z_0$ of~$Z$ by~$F( (T, g) )$.

			\item
				Let~$x$ be an element of~$X$ that is not only contained in~$S$, but already in the preimage~$f^{-1}(T)$.
				This element is first mapped to the point~$f(x)$ in~$T$ by~$F( (S, f) )$, and then further mapped to the point~$g(f(x) )$ in~$Z$ by~$F( (T, g) )$.

		\end{itemize}
		We find overall that the composite~$F( (T, g) ) ∘ F( (S, f) )$ coincides with the induced map~$F( (T, g) ∘ (S, f) )$.%
		\footnote{
			The composite~$(T, g) ∘ (S, f)$ is the partially defined function~$(f^{-1}(T), g ∘ f)$.
		}
\end{itemize}

The required functor~$G$ from~$\Set_*$ to~$\Par$ is easier to construct:
\begin{itemize}

	\item
		For every pointed set~$(X, x_0)$, its value under~$G$ is given by the set~$X ∖ \{ x_0 \}$.

	\item
		For every pointed map
		\[
			f \colon (X, x_0) \to (Y, y_0) \,,
		\]
		the value of~$f$ under~$G$ is given by the restriction of~$f$ to~$X ∖ f^{-1}(y_0)$.
		This domain is a subset of~$X ∖ \{ x_0 \}$ because the basepoint~$x_0$ is contained in the preimage~$f^{-1}(y_0)$.
		Therefore,~$G(f)$ is a well-defined partially defined map from~$G(X)$ to~$G(Y)$.

\end{itemize}
This assignment~$G$ is indeed a functor from~$\Set_*$ to~$\Par$:
\begin{itemize*}

	\item
		Let~$(X, x_0)$ be a pointed set.
		Its identity morphism in the category~$\Set_*$ is simply the identity map of the set~$X$, i.e.,~$\id_{(X, x_0)} = \id_X$.
		The resulting map~$G(\id_{(X, x_0)})$ is the restriction of~$\id_X$ to the domain
		\[
			X ∖ \id_X^{-1}(x_0)
			=
			X ∖ \{ x_0 \} \,.
		\]
		Therefore,
		\[
			G( \id_{(X, x_0)} )
			=
			( X ∖ \{ x_0 \}, \id_X )
			=
			( X ∖ \{ x_0 \}, \id_{X ∖ \{ x_0 \}} )
			=
			\id_{G( (X, x_0) )} \,.
		\]

	\item
		Let
		\[
			f \colon (X, x_0) \to (Y, y_0) \,,
			\quad
			g \colon (Y, y_0) \to (Z, z_0)
		\]
		be two composable maps of pointed sets.
		We have
		\[
			(g ∘ f)^{-1}(z_0)
			=
			f^{-1}( g^{-1}(z_0) ) \,,
		\]
		and therefore
		\begin{align*}
			G(g) ∘ G(f)
			&=
			(Y ∖ g^{-1}(z_0), g) ∘ (X ∖ f^{-1}(y_0), f)
			\\
			&=
			( f^{-1}(Y ∖ g^{-1}(z_0) ), g ∘ f )
			\\
			&=
			( f^{-1}(Y) ∖ f^{-1}(g^{-1}(z_0) ), g ∘ f )
			\\
			&=
			( X ∖ (g ∘ f)^{-1}(z_0), g ∘ f )
			\\
			&=
			G(g ∘ f) \,.
		\end{align*}

\end{itemize*}

In the following, we will show that the two functors~$F$ and~$G$ form an equivalence of categories between~$\Par$ and~$\Set_*$.
We do so by constructing natural isomorphisms
\[
	α \colon \Id_{\Par} \To G F \,,
	\quad
	β \colon \Id_{\Set_*} \To F G \,.
\]

We observe first that~$G F = \Id_{\Par}$.
We can therefore choose~$α$ as the identity natural transformation.

To construct the natural isomorphism~$β$, let~$(X, x_0)$ be a pointed set.
The pointed set~$FG( (X, x_0) )$ is given by
\[
	( X ∖ \{ x_0 \} ∪ \{ x_0' \}, \; x_0' )
\]
with~$x_0' \notin X ∖ \{ x_0 \}$.
In other words: we first remove the base point~$x_0$ of~$X$, and then add a new basepoint~$x_0'$ in its place.
(We can, however, not ensure that the newly added base point~$x_0'$ is set-theoretically equal to the previous base point~$x_0$.
This is why the composite~$F G$ won’t be the identity functor of~$\Set_*$, but will only be isomorphic to it.)
The map
\[
	β_{(X, x_0)}
	\colon
	(X, x_0)
	\to
	( X ∖ \{ x_0 \} ∪ \{ x_0' \}, \; x_0' )
\]
given by
\[
	x
	\mapsto
	\begin{cases*}
		x_0' & if~$x = x_0$, \\
		x    & otherwise,
	\end{cases*}
\]
is therefore an isomorphism of pointed sets from~$(X, x_0)$ to~$FG( (X, x_0) )$.
This isomorphism is natural:
For every pointed map
\[
	f \colon (X, x_0) \to (Y, y_0) \,,
\]
the resulting pointed map value~$GF(f)$ is given by
\[
	GF(f)(x)
	=
	\begin{cases*}
		y_0'  & if~$x = x_0'$, \\
		f(x)  & otherwise,
	\end{cases*}
\]
for every element~$x$ of~$GF( (X, x_0) )$.
The diagram
\[
	\begin{tikzcd}[sep = large]
		(X, x_0)
		\arrow{r}[above]{f}
		\arrow{d}[left]{β_{(X, x_0)}}
		&
		(Y, y_0)
		\arrow{d}[right]{β_{(Y, y_0)}}
		\\
		GF( (X, x_0) )
		\arrow{r}[above]{GF(f)}
		&
		GF( (Y, y_0) )
	\end{tikzcd}
\]
therefore commutes, showing the naturality of~$β$.



\subsubsection{Description of \texorpdfstring{$\Set_*$}{Set\_*} as a coslice category}

We can describe the category~$\Set_*$ as the coslice category~$𝟙 / \Set$, where~$𝟙 ≔ \{ \ast \}$ is a one-element set.
\begin{itemize}

	\item
		An object of the category~$𝟙$ is a pair~$(X, ξ)$ consisting of a set~$X$ and a map~$ξ$ from~$𝟙$ to~$X$.
		This map~$ξ$ amounts to picking out an element~$x_0$ of~$X$, namely the element~$ξ(\ast)$.

	\item
		A morphism
		\[
			f \colon (X, ξ) \to (Y, γ)
		\]
		in~$\Set_*$ is a set-theoretic map~$f$ from~$X$ to~$Y$ that makes the diagram
		\[
			\begin{tikzcd}
				{}
				&
				𝟙
				\arrow{dl}[above left]{ξ}
				\arrow{dr}[above right]{γ}
				&
				{}
				\\
				X
				\arrow{rr}[above]{f}
				&
				{}
				&
				Y
			\end{tikzcd}
		\]
		commute.
		The commutativity of this diagram can be expressed in terms of the points~$x_0 ≔ ξ(\ast)$ and~$y_0 ≔ γ(\ast)$ as the equality
		\[
			f(x_0) = y_0 \,.
		\]
		That means that a morphism from~$(X, ξ)$ to~$(Y, γ)$ is the same as a pointed map from~$(X, x_0)$ to~$(Y, y_0)$.

	\item
		The composition of morphisms is~$\Set_*$ is the usual composition of functions, and the same goes for the coslice category~$𝟙 / \Set$.

\end{itemize}

We can see from these observations that the two categories~$\Set_*$ and~$𝟙 / \Set$ are isomorphic.
