\subsection{}

The dual statement reads as follows:
\begin{quote}
	Let~$\cat{A}$ and~$\cat{B}$ be two categories, and let~$F$ be a functor from~$\cat{A}$ to~$\cat{B}$.
	Then~$F$ has a right adjoint if and only if for every object~$B$ of~$\cat{B}$, the category~$F \To B$ has a terminal object.
\end{quote}

To prove, the statement, we regard~$F$ as a functor~$F'$ from~$\cat{A}^{\op}$ to~$\cat{B}^{\op}$.
A functor from~$\cat{B}$ to~$\cat{A}$ is right adjoint to~$F$ if and only if, as a functor from~$\cat{B}^{\op}$ to~$\cat{A}^{\op}$, it is left adjoint to~$F'$.
(See \cref{adjunctions between opposite categories} (page~\pageref{adjunctions between opposite categories}).)
It follows that the functor~$F$ admits a right adjoint if and only if the functor~$F'$ admits a left adjoint.

According to Corollary~2.3.7, this is the case if and only if for every object~$B$ of~$\cat{B}$, the category~$B^{\op} \To F'$ admits an initial object.
The category~$B^{\op} \To F'$ is isomorphic to the category~$(F \To B)^{\op}$.
Therefore, the category~$B^{\op} \To F'$ admits an initial object if and only if the category~$F \To B$ admits a terminal object.

This shows the claim that the functor~$F$ admits a right adjoint if and only if for every object~$B$ of~$\cat{B}$, the category~$F \To B$ admits a terminal object.
