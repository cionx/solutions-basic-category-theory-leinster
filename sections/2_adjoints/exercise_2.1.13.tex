\subsection{}

Let~$\Acat$ and~$\Bcat$ be two discrete categories with underlying classes of objects~$A$ and~$B$.

We consider first two functors
\begin{align*}
	F \colon \Acat \to \Bcat
	\quad&\text{and}\quad
	G \colon \Bcat \to \Acat
\intertext{
such that~$F$ is left adjoint to~$G$.
We may regard these two functors as functions
}
	f \colon A \to B
	\quad&\text{and}\quad
	g \colon B \to A \,.
\end{align*}
We find for every element~$a$ of~$A$ that
\[
	\Acat(a, g(f(a)))
	=
	\Acat(a, GF(a))
	≅
	\Bcat(F(a), F(a))
	≠
	\emptyset
\]
This means that~$g(f(a)) = a$ because the category~$\Acat$ is discrete.
We find in the same way that~$f(g(b)) = b$ for every element~$b$ of~$B$.
The two functions~$f$ and~$g$ are therefore mutually inverse bijections.

Suppose on the other hand that
\begin{align*}
	f \colon A \to B
	\quad&\text{and}\quad
	g \colon B \to A
\intertext{
	are two mutually inverse bijections.
	We may regard these functions as mutually inverse isomorphisms of categories
}
	F \colon \Acat \to \Bcat
	\quad&\text{and}\quad
	G \colon \Bcat \to \Acat \,.
\end{align*}
We then have
\[
	\Bcat(F(a), b)
	=
	\Bcat(F(a), F(G(b)))
	≅
	\Bcat(a, G(b))
\]
for every element~$a$ of~$A$ and every element~$b$ of~$B$.
This bijection is natural in both~$a$ and~$b$ because the only morphisms in~$\Acat$ and in~$\Bcat$ are the identity morphisms.
This shows that the functors~$F$ and~$G$ are adjoint, with~$F$ being left adjoint to~$G$.

We have thus shown that an adjunction between two discrete categories~$\Acat$ and~$\Bcat$ is the same as a pair of mutually inverse bijections between their underlying classes of objects~$A$ and~$B$.





