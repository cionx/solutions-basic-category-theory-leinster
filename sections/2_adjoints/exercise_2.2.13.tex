\subsection{}



\subsubsection{}

The induced map~$f_* \colon \Power(K) \to \Power(L)$ that assigns to each subset of~$K$ its image under~$f$ is order-preserving, and so can be seen as a functor.
The functor~$f_*$ is left adjoint to the functor~$f^*$ because
\[
	f(S) ⊆ T
	\iff
	\text{$f(s) ∈ T$ for every element~$s$ of~$S$}
	\iff
	S ⊆ f^{-1}(T)
\]
for every subset~$S$ of~$K$ and every subset~$T$ of~$L$.

To figure out the right adjoint~$f_\#$ of~$f^*$ we observe that it needs to satisfy
\[
	t ∈ f_\#(U)
	\iff
	\{ t \} ⊆ f_\#(U)
	\iff
	f^{-1}(t) ⊆ U
\]
for every subset~$U$ of~$K$ and every element~$t$ of~$K$.
We therefore set
\[
	f_\#(U) ≔ \{ t ∈ L \suchthat f^{-1}(t) ⊆ U \} \,.
\]
This defines a map~$f_\#$ from~$\Power(K)$ to~$\Power(L)$ that is order-preserving, and can therefore be regarded as a functor.
We have
\begin{align*}
	T ⊆ f_\#(U)
	&\iff
	\text{$t ∈ f_\#(U)$ for every element~$t$ of~$T$}
	\\
	&\iff
	\text{$f^{-1}(t) ⊆ U$ for every element~$t$ of~$T$}
	\\
	&\iff
	f^{-1}(T) ⊆ U
\end{align*}
for every subset~$T$ of~$L$ and every subset~$U$ of~$L$.
Therefore,~$f_\#$ is right adjoint to~$f^*$.



\subsubsection{}

The projection map~$p$ from~$X × Y$ to~$X$ induces a functor
\[
	p^*
	\colon
	\Power(X)
	\to
	\Power(X × Y) \,.
\]
Logically, this can be expressed as
\begin{align*}
	p^*(S)(x, y)
	\iff
	(x, y) ∈ p^*(S)
	\iff
	(x, y) ∈ p^{-1}(S)
	\iff
	p(x, y) ∈ S
	&\iff
	x ∈ S
	\\
	&\iff
	S(x) \,.
\end{align*}
We have seen in the previous part of this exercise an explicit construction of the left adjoint~$p_*$ of~$p^*$ and the right adjoint~$p_\#$ of~$p^*$.
The left adjoint~$p_*$ can be expressed as
\begin{align*}
	p_*(R)(x)
	&\iff
	x ∈ p(R)
	\\
	&\iff
	\text{there exists an element~$y$ of~$Y$ with~$(x, y) ∈ R$}
	\\
	&\iff
	\text{there exists an element~$y$ of~$Y$ with~$R(x, y)$}
	\\
	&\iff
	\exists y ∈ Y : R(x, y) \,.
\end{align*}
The right adjoint~$p_\#$ can be expressed as
\begin{align*}
	p_\#(R)(x)
	&\iff
	x ∈ p_\#(R)
	\\
	&\iff
	p^{-1}(x) ⊆ R
	\\
	&\iff
	\text{$(x, y) ∈ R$ for every element~$y$ of~$Y$}
	\\
	&\iff
	\text{$R(x, y)$ for every element~$y$ of~$Y$}
	\\
	&\iff
	\forall y ∈ Y: R(x, y) \,.
\end{align*}

For two subsets~$S$ and~$T$ of a finite product~$X_1 × \dotsb × X_n$ we have
\begin{align*}
	{}&
	S ⊆ T
	\\
	\iff{}&
	\Bigl(
		\text{$(x_1, \dotsc, x_n) ∈ S \implies (x_1, \dotsc, x_n) ∈ T$ for all~$(x_1, \dotsc, x_n) ∈ X_1 × \dotsb × X_n$}
	\Bigr)
	\\
	\iff{}&
	\Bigl(
		\text{$S(x_1, \dotsc, x_n) \implies T(x_1, \dotsc, x_n)$ for all~$(x_1, \dotsc, x_n) ∈ X_1 × \dotsb × X_n$}
	\Bigr)
	\\
	\iff{}&
	( S \implies T ) \,.
\end{align*}
The unit~$η$ and counit~$ε$ of the adjunction~$p_* ⊣ p^*$, given by natural transformations
\[
	η \colon \Id_{\Power(X × Y)} \To p^* p_* \,,
	\quad
	ε \colon p_* p^* \To \Id_{\Power(X)} \,,
\]
can therefore be regarded as certain logical implications.
The composite~$p^* p_*$ is given by
\[
	p^*( p_*(R) )(x, y)
	\iff
	p_*(R)(x)
	\iff
	\bigl( \exists y': R(x, y') \bigr) \,,
\]
whence the unit~$η$ of the adjunction~$f_* ⊣ f^*$ can be interpreted as the implication
\[
	R(x, y) \implies \bigl( \exists y' : R(x, y') \bigr) \,.
\]
The composite~$p_* p^*$ is given by
\[
	p_*(p^*(S))(x)
	\iff
	\bigl( \exists y: p^*(S)(x, y) \bigr)
	\iff
	\bigl( \exists y: S(x) \bigr) \,,
\]
whence the counit~$ε$ of the adjunction~$f^* ⊣ f_\#$ can be interpreted as the implication
\[
	\bigl( \exists y: S(x) \bigr) \implies S(x) \,.
\]

We can similarly regard the unit~$η'$ and counit~$ε'$ of the adjunction~$p^* ⊣ p_\#$ as logical implications.
The composite~$p_\# p^*$ is given by
\[
	p_\#( p^*(S) )(x)
	\iff
	\bigl( \forall y ∈ Y: p^*(S)(x, y) \bigr)
	\iff
	\bigl( \forall y ∈ Y: S(x) \bigr) \,,
\]
whence the unit~$η'$ can be interpreted as the implication
\[
	S(x) \implies \bigl( \forall y ∈ Y: S(x) \bigr) \,.
\]
The composite~$p^* p_\#$ is given by
\[
	p^*( p_\#(R) )(x, y)
	\iff
	p_\#(R)(x)
	\iff
	\bigl( \forall y' ∈ Y : R(x, y') \bigr) \,,
\]
whence the counit~$ε'$ can be interpreted as the implication
\[
	\bigl( \forall y' ∈ Y : R(x, y') \bigr)
	\implies
	R(x, y) \,.
\]
